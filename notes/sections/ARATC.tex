% !TEX root =../main/main.tex

\section{MATLANG and product iteration}
We first extend ARA with (some kind of) transitive closure and denote the corresponding fragment by ARA+TC.
More specifically, let $e$ be a ARA+TC expression of schema $\cR(e)$.
Then,
$$
\mathsf{tc}_{X;Y}(e)
$$
is an ARA+TC expression of the same schema. Here, $X$ and $Y$ are sets of attributes in $\cR(e)$,
all distinct, and such that $|X|=|Y|$. The semantics of $\mathbf{tc}_{X;Y}(e)$ 
is defined in terms of the expressions
\begin{align*}
e^{(0)}&=e\\
e^{(k)}&=\pi_{C}\bigl(\rho_{Y\mapsto C}(e)\bowtie \rho_{X\mapsto C}(e^{(i-1)})\bigr).
\end{align*}
(We note that, typically, when introducing TC, the second rule involves also the union with $e$, so 
$e^{(k)}=e\cup \pi_{C}\bigl(\rho_{Y\mapsto C}(e)\bowtie \rho_{X\mapsto C}(e^{(k-1)})\bigr)$ is more
commonly used. This ensures that TC grows in each step, for our product we do not have this, each
additional multiplication may reduce the number of non-zero entries.)

The semantics of $\mathbf{tc}_{X;Y}(e)$ is defined in a straightforward way.
For an instance $r:\Tup_D(Z)\to K$ such that $X$ and $Y$ are subsets of $Z$, $\mathbf{tc}_{X;Y}(r)$ is defined as 
$e^{(k)}(r)$, where $k$ is the smallest number such that $e^{(k)}(r)=e^{(k+1)}(r)$. Of course, restrictions on semirings need to be in place to 
ensure convergence. This has been studied before, they are called $\omega$-continuous semirings. It does not really matter at this point
and as we will see below, to get a translation of ML+$\sum$+$\Pi$ into an extension of ARA, we need a successor relation which can be used to limit the amount of iterations when computing $\mathbf{tc}_{X;Y}(e)$.

Product iteration uses an implicit ordering on the canonical vectors. To accommodate for this ordering we further extend ARA (and also ARA+TC) with a binary successor relation, $\mathbf{succ}(A,B)$ where $A$ and $B$
are compatible attributes whose domains have an order $<$, and unary operators $\mathbf{min}(A)$ and $\mathbf{max+}(A)$.
 On the instance level, we treat $\mathbf{succ}(A,B)$ it as a built-in predicate
which contains pairs $(x,y)$ for $x<y$ and such that no other $z$ lies between $x$ and $y$. Similarly, $\mathbf{min}(A)$ stores the minimum value of attribute $A$, and $\mathbf{max+}(A)$ the successor of the maximal value in attribute $A$. (I know that this asymmetry between min and max is ugly, it is a side effect of squeezing things in a tc-like form. We need to ensure that enough steps are made in the tc computation.)
So, in the following we assume that we use these predicates just as relation in ARA (+TC). We denote the fragment by ARA+TC+succ.


Let's see whether this suffices to simulate ML+$\sum$+$\Pi$ expressions. Take such an expression $e'(v_1,\ldots,v_k)$ and assume by induction that there exists an ARA+TC+succ expression $E'(\row_n,\col_m,\allowbreak A_1,\ldots,A_k)$ such that 
$$(b_i^n)^T\cdot e'(v_1,\ldots,v_k)\left(\cI\left[ v_l\rightarrow b_{i_l}^{n_l}\right]\right)\cdot b_j^m=E'(i,j,i_1,\ldots, i_k).$$
We know already how to deal with operations in ML+$\sum$. We next consider an expression $E=\prod_{v_k} E'$ )(the choice of $v_k$
is arbitrary, it should work for any other vector variable, just a bit more messy perhaps from a notational point of view). We now consider
expressions
\begin{align*}
E_1(\row_n,\col_n,A_1,\ldots,A_k,A_{k+1})&:=E'(\row_n,\col_n,A_1,\ldots,A_k)\bowtie \mathbf{succ}(A_k,A_{k+1})\\
E_2(\row_n,\col_n,A_1,\ldots,A_k,A_{k+1})&=\mathbf{tc}_{\row_n,A_k;\col_n,A_{k+1}}(E_1)\\
E(\row_n,\col_n,A_1,\ldots,A_{k-1})&=E_2(\row_n,\col_n,A_1,\ldots,A_{k-1},\textbf{min}(A_k),\textbf{max+}(A_{k+1}))
\end{align*}
We now claim that 
$$(b_i^n)^T\cdot e(v_1,\ldots,v_{k-1})\left(\cI\left[ v_l\rightarrow b_{i_l}^{n_l}\right]\right)\cdot b_j^m=E(i,j,i_1,\ldots, i_{k-1}).$$

{\bf This needs verification}.


