\section{Constant and variable expressions}

All the expressions so far are valid if they are composed of matrices that are on the instance $I$ (see section 1). Once you have an expression $E$ that is well defined on the given instance, it doesn't change its value unless the current instance is modified. We call this a \textit{constant} expression.

Now, let's turn our attention on the expressions $$E(v)=v^* Av.$$ The value of $E$ depends entirely on $v$, we call this a \textit{variable expression}. This type of expressions have \textit{free} variables ($v$ in this case) and it doesn't mean anything in MATLANG unless the \textit{free} variables are mapped into an actual matrix of the current instance, we denote this $E(v)(I[v\rightarrow B])$, assuming that $B$ is on the domain of the given instance. This mapping will be accomplished trough operators that are introduced in the next section. These operators define how the variable expression $E$ is used and how it's instantiated.
A formal example, let $\lbrace v_i\rbrace_i$ be the canonical vectors. An operator could receive a variable expression $E(v)$ and give the output $E(v_k)=B$, if and only if $[E(v)](I[v\rightarrow v_k])=B$.
We now proceed to define the semantics of this formulas.

\subsection{Extended MATLANG}

A schema $S$ is a set of matrix names $M_1,\ldots, M_p$ and a set of vector names $v_1,\ldots, v_l$.
An instance $I$ is a function that maps matrix names to concrete matrices (of dimention $n\times n$) and vector names to concrete colun vectors (of dimention $n\times 1$). 
In addition, we define $dim(I)=n$ and $can(I)$ as the set of canonical vectors of dimention $n$, indexed as $can(I)[1], \ldots,can(I)[n]$.

Let $I$ be an intance and $n=dim(I)$. Let $X$ be a set of vector variables (matrix variables, respectively). A valuation $\nu$ on instance $I$ is a function from $X$ to column vectors of dimention $n\times 1$ (matrices of dimention $n\times n$, respectively). 

Let $S$ be an schema, $I$ an instance, $F$ a set of functions $f:\mathbb{R}^{k}\rightarrow\mathbb{R}$ (with $1\leq k$) and $X$ a set of vector variables. An \textit{extended} MATLANG expression $E$ is given by

\begin{align*}
E:=& x\in X | M\in S | v\in S | \\
&|E_1+E_2 | E_1\cdot E_2 | E^* | \\
& |f(E_1,\ldots, E_k), f\in F | \\
& |\sum v. E | \prod v. E
\end{align*}

Let $n=dim(I)$. We define $\text{type}(E)=(i,j)$ where $(i,j)\in \lbrace 1, n\rbrace$. The well typeness id defined recursively as follows.

\begin{align*}
\text{type}(x) &= (n,1) \\
\text{type}(M) &= (n,n) \\
\text{type}(v) &= (n,1) 
\end{align*}


