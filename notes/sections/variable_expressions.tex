\section{Constant and variable expressions}

All the expressions so far are valid if they are composed of matrices that are on the instance $I$ (see section 1). Once you have an expression $E$ that is well defined on the given instance, it doesn't change its value unless the current instance is modified. We call this a \textit{constant} expression.

Now, let's turn our attention on the expressions $$E(v)=v^* Av.$$ The value of $E$ depends entirely on $v$, we call this a \textit{variable expression}. This type of expressions have \textit{free} variables ($v$ in this case) and it doesn't mean anything in MATLANG unless the \textit{free} variables are mapped into an actual matrix of the current instance, we denote this $E(v)(I[v\rightarrow B])$, assuming thath $B$ is on the domain of the given instance. This imapping will be accomplished trough operators that are introduced in the next section. These operators define how the variable expression $E$ is used and how it's instantiated.
A formal example, let $\lbrace v_i\rbrace_i$ be the canonical vectors. An operator could receive a variable expression $E(v)$ and give the output $E(v_k)=B$, if and only if $[E(v)](I[v\rightarrow v_k])=B$.