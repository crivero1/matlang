\section{Constant and variable expressions}

All the expressions so far are valid if they are composed of matrices that are on the instance $I$ (see section 1). Once you have an expression $E$ that is well defined on the given instance, it doesn't change its value unless the current instance is modified. We call this a \textit{constant} expression.

Now, let's turn our attention on the expressions $$E(v)=v^* Av.$$ The value of $E$ depends entirely on $v$, we call this a \textit{variable expression}. This type of expressions have \textit{free} variables ($v$ in this case) and it doesn't mean anything in MATLANG unless the \textit{free} variables are mapped into an actual matrix of the current instance, we denote this $E(v)(I[v\rightarrow B])$, assuming that $B$ is on the domain of the given instance. This mapping will be accomplished trough operators that are introduced in the next section. These operators define how the variable expression $E$ is used and how it's instantiated.
A formal example, let $\lbrace v_i\rbrace_i$ be the canonical vectors. An operator could receive a variable expression $E(v)$ and give the output $E(v_k)=B$, if and only if $[E(v)](I[v\rightarrow v_k])=B$.
We now proceed to define the semantics of this formulas.

\subsection{Extended MATLANG}

A schema $S$ is a set of matrix names $M_1,\ldots, M_p$ and a set of vector names $v_1,\ldots, v_l$.
An instance $I$ is a function that maps matrix names to concrete matrices (of dimention $n\times n$) and vector names to concrete colun vectors (of dimention $n\times 1$). 
In addition, we define $dim(I)=n$ and $can(I)$ as the set of canonical vectors of dimention $n$, indexed as $can(I)[1], \ldots,can(I)[n]$.

Let $I$ be an intance and $n=dim(I)$. Let $X$ be a set of vector variables (matrix variables, respectively). A valuation $\nu$ on instance $I$ is a function from $X$ to column vectors of dimention $n\times 1$ (matrices of dimention $n\times n$, respectively). 

Let $S$ be an schema, $I$ an instance, $F$ a set of functions $f:\mathbb{R}^{k}\rightarrow\mathbb{R}$ (with $1\leq k$) and $V$ a set of vector variables. An \textit{extended} MATLANG expression $E$ is given by

\begin{align*}
E:=& x\in V | M\in S | v\in S | \\
&|E^*|E_1+E_2 | E_1\cdot E_2 |\\
& |f(E_1,\ldots, E_k), f\in F | \\
& |\sum x. E | \prod x. E
\end{align*}

Let $n=dim(I)$. We define $\text{type}(E)=(i,j)$ where $(i,j)\in \lbrace 1, n\rbrace$. The well typeness is defined recursively as follows.

\begin{align*}
\text{type}(x) &= (n,1) \\
\text{type}(M) &= (n,n) \\
\text{type}(v) &= (n,1) \\
\text{type}(E^*)&=(j,i) \text{ where } (i,j)=\text{type}(E) \\
\text{type}(E_1 + E_2) &= \begin{cases}
               \text{type}(E_1) \text{ if } \text{type}(E_1)=\text{type}(E_1) \\
               \text{undefined} \text{ otherwise }
            \end{cases} \\
\text{type}(E_1\cdot E_2)&=\begin{cases}
               (i,k) \text{ if } \text{type}(E_1)=(i,j)\text{ and type}(E_2)=(j,k) \\
               \text{undefined} \text{ otherwise }
            \end{cases} \\
\text{type}(f(E_1,\ldots, E_k))&=\begin{cases}
               (1,1) \text{ if } \text{type}(E_1)=\cdots =\text{type}(E_k)=(1,1) \\
               \text{undefined} \text{ otherwise }
            \end{cases} \\
\text{type}\left(\sum x.E\right) &= \text{type}(E) \\
\text{type}\left(\prod x. E\right) &=
\begin{cases}
               (1,1) \text{ if } \text{type}(E)=(1,1) \\
               (n,n) \text{ if } \text{type}(E)=(n,n) \\
               \text{undefined } \text{ otherwise }
            \end{cases}
\end{align*}
We say that $E$ is well typed if it has a defined type. Here, type(undefined)$=$ undefined.


Given $I, \nu, V$ and a well typed expression $E$, we define the semantics for the values $\left[ E\right](I,\nu)$ inductivelly as follows.
\begin{align*}
\left[ x\right](I,\nu)&=\nu (x)\\
\left[ M\right](I,\nu)&=I(M) \\
\left[ v\right]&=I(v) \\
\left[ E^*\right]&=\left[ E\right](I,\nu)^* \\
\left[ E_1+E_2\right]&= \left[ E_1\right](I,\nu)+\left[ E_2\right](I,\nu)\\
\left[ E_1\cdot E_2\right]&= \left[ E_1\right](I,\nu)\times \left[ E_2\right](I,\nu)\\
\left[ f(E_1,\ldots ,E_k)\right] &= f\left(\left[ E_1\right](I,\nu),\ldots , \left[  E_k\right](I,\nu)\right)\\
\left[ \sum x.E\right]&= \sum_{i=1}^{dim(I)}\left[ E\right](I, \nu [x\rightarrow can(I)[i]])\\
\left[ \prod x. E\right]&= \prod_{i=1}^{dim(I)}\left[ E\right](I, \nu [x\rightarrow can(I)[i]])
\end{align*}

Here, $+$ is matrix sum, $\times$ is matrix multiplication and 
\begin{align*}
\sum_{i=1}^n E_i = E_1+\cdots + E_n \\
\prod_{i=1}^n E_i = E_1\times \cdots\times E_n
\end{align*}
Depending on context, we sometimes denote explicitly that an expression $E$ depends on $x$ as $E(x)$.

Note that the definition above is over a set of vector variables $V$. We can naturally extend our language to work with matrix variables as well. Let V be the set of vector a matrix variables, this is $V=\lbrace\lbrace x, y, z, \ldots\rbrace\lbrace X,Y,Z,\ldots\rbrace\rbrace$

