\newpage

\section{Appendix}

\subsection{Recursive iteration}

Given the above problem, we sort of need an operator that iterates over the canonical vectors and compute the same operation over the result of the previous iteration.

Note that in the definition of syntax and semantics of extMATLANG is over a set of vector variables $V$. We can naturally extend our language to work with matrix variables as well. Let V be the set of vector a matrix variables, this is $V=\lbrace\lbrace x, y, z, \ldots\rbrace,\lbrace X,Y,Z,\ldots\rbrace\rbrace$. Also $E:=X\in V$, type($X$)$=(n, n)$ and $\sem{X}(I,\nu)=\nu(X)$.

Now, with matrix variables on the table, we define a new expression $$\mu X, v.E$$ With $$\text{type}(\mu X,v. E) = \text{type}(E)$$ and $\sem{\mu X,v. E}(\cM,\nu) = A_n$ where
\begin{align*}
A_0 &= I \\
A_i &= \sem{E}(\cM,\nu\left[X\rightarrow A_{i-1}, v\rightarrow can(\cM)[i]\right]),\hspace{1ex} i= 1, \ldots, n 
\end{align*}

Here, $I$ is the identity. This operator iterates over the canonical vectors and uses the previous result as an input matrix for the same expression.

\subsubsection{LU factorization in extMATLANG + order}
Let's consider the following setting of the $A=LU$ factorization:

\begin{align*}
A_i&=T_{i-1}A_{i-1},\hspace{1ex}i=2, \cdots, n \\
A_1&=A \\
\end{align*}

Note that $A_n=U$ and 
\begin{align*}
A_i&=T_{i-1}A_{i-1} \\
&=\left(\sprod v_k.\left(I-\dfrac{v_k^*A_{i-1}v_i}{v_i^*A_{i-1}v_i}\cdot v_iv_k^*\cdot\left(v_i^*Zv_k\right)\right)^*\right)^*A_{i-1}.
\end{align*}

Now, if $A=LU$ and $$E(X, v)=\left(\sprod v_k.\left(I-\dfrac{v_k^*(XA)v}{v^*(XA)v}\cdot vv_k^*\cdot\left(v^*Zv_k\right)\right)^*\right)^*X$$
then $U=(\mu X, v.E(X,v))(A)$, this is $L^{-1}=\mu X, v.E(X,v)$.

Note that in the example above we \textit{inserted} the target matrix $A$ in the expression $E(X,v)$. Would it be different if in the definition of $\mu$ we start with some matrix $B$ and not the indentity? We show that we can simulate this using our setting. Let us have an operator $\lambda$ that does exactly what $\mu$ does, except that we have to give another parameter: a starting matrix. This is:

\begin{align*}
\lambda X, v.E(X,v)(B)&=A_n \\
A_i&=E(A_{i-1}, v_{i}),\hspace{1ex}i=1,\cdots, n \\
A_0&=B
\end{align*}
Now, we use the following expression with the operator $\mu$ to simulate $\lambda$: $$E'(X,v)=min(v)\cdot E(B,v_1)+(1-min(v))E(X,v).$$
Note that for $\lambda$ we have 
\begin{align*}
	A^{\lambda}_0&=B \\
	A^{\lambda}_1&=E(A^{\lambda}_0,v_1)=E(B,v_1) \\
	A^{\lambda}_2&=E(A^{\lambda}_1,v_2) \\
	&\vdots \\
	A^{\lambda}_n&=E(A^{\lambda}_{n-1}, v_n)
\end{align*}
and for $\mu$
\begin{align*}
	A^{\mu}_0&=I \\
	A^{\mu}_1&=E'(A^{\mu}_0,v_1)=E(B,v_1) \\
	A^{\mu}_2&=E'(A^{\mu}_1,v_2)=E(A^{\mu}_1,v_2) \\
	&\vdots \\
	A^{\mu}_n&=E'(A^{\mu}_{n-1}, v_n)=E(A^{\mu}_{n-1}, v_n).
\end{align*}
Since $A^{\mu}_1=A^{\lambda}_1$ we have that $A^{\mu}_i=A^{\lambda}_i$ for $i=1,\ldots,n$. So $$\lambda X, v.E(X,v)(B)=\mu X, v.E'(X,v)$$

\subsection{Current main questions}
\begin{itemize}

\item $PA=LU$ factorization: if $A$ needs row interchange, is there some way to compute $P$ beforehand?
\begin{itemize}
\item I am not sure whether you can do this beforehand.
\item But, isn't it possible to simulate \emph{partial pivoting}, i.e., when dealing with canonical vector $v_i$, extract the $i$th column $X[*,i]$ from the current
matrix, find the maximal elements in $X[i-n,i]$ and corresponding indicator vector for this (i.e., ones on positions where the max value occurs), then somehow
reduce this to a vector in which, say only the smallest occurrence $k$ of ones is retained (smallest position), and then turn this into a permutation matrix that exchanges
all rows $X[j,*]$ for $j\geq i$ with the $k$th row, resulting in permuted version of $X$, on which you can then proceed as before for the $i$th step? This will require nested
recursion.
\end{itemize}
\item Study if we can express other factorizations ($LDU, LDL$,
Cholesky, etc.).
\item Starting from $A=LU$, can we compute $A^{-1}$ easily? It's possible that is the gaussian elimination upwards, this is, reduce $A$ to the identity.
\item The importance of $Z$ is that it gives us an order. Can we compute $Z$ with the new operators?
\item Besides function application, are there any other expressions where the sum operator is useful?
\item Can we compute $A^{dim(A)}$?
\item Explore the possibility to define the sum and product operators over invariant expressions, and then extend the definition through $Z$.

\end{itemize}

\subsection{Other examples}

\textbf{Determinant}


Recall that the determinant is define over permutations, this is $$det(A)=\sum_{p}\sigma(p)a_{1p_1}\cdots a_{np_n},$$ where the sum is taken over the $n!$ permutations of the natural order $(1, 2,\ldots, n)$ and 
       \[
  			\sigma(p)=\begin{cases}
               +1 \text{ if p can be restored to natural order in an even number of steps} \\
               -1 \text{ if p can be restored to natural order in an odd number of steps}
            \end{cases}
		\]
In the case of MATLANG, the permutations are represented as permutation matrices, this is, permutations of the identity. In this context, note that, for a permutation matrix $P$, we compute $a_{1p_1}\cdots a_{np_n}$ as $$\prod x. x^*\cdot A\cdot (Px).$$ See that $a_{ip_i}=v_i^*\cdot A\cdot (Pv_i)$, where $v_i$ is the $i$-th canonical vector. 


We also need to compute $\sigma(P)$, to this end, note that $$\sigma(P)=(-1)^{\sum_{i<j}\lbrace p_i>p_j\rbrace}.$$ Also, let $\lbrace v_i\rbrace_{i=1}^{n}$ be the canonical vectors.
Now, note that 

 		\[
  			2\cdot v_i^*Zv_j=\begin{cases}
               2 \text{ if } i < j \\
               0 \text{ if } i \geq j
            \end{cases}
		\]
		
Thus

		\[
  			1-2\cdot v_i^*Zv_j=\begin{cases}
               -1 \text{ if } i < j \\
               +1 \text{ if } i \geq j
            \end{cases}
		\]

So $$\sigma(P)=\prod x_i.\prod x_j:(i<j).\left(1-2\cdot (Px_i)^*Z(Px_j)\right),$$ and since 

		\[
  			v_i^*Zv_j=\begin{cases}
               1 \text{ if } i < j \\
               0 \text{ if } i \geq j
            \end{cases}
		\]

Then we have that $$\sigma(P)=\prod x_i.\prod x_j.\left(1-2\cdot (Px_i)^*Z(Px_j)\cdot x_i^*Zx_j\right)$$

Thus, given we can iterate over permutation matrices, we have $$det(A)=\sum_{P}\left[\left(\prod x_i \prod x_j \left(1-2\cdot (Px_i)^*Z(Px_j)\cdot x_i^*Zx_j\right)\right)\left(\prod y. y^*\cdot A\cdot (Py)\right)\right].$$

\subsubsection{Observations}

\begin{itemize}
	\item Note that if we have
	\[
  			f(x)=\begin{cases}
               1 \text{ if } P(x) \\
               0 \text{ if } Q(x)
            \end{cases}
		\]
		If needed, we can compute a piecewise function with any values, like 
		\[
  			b - (b - a)f(x)=\begin{cases}
               a \text{ if } P(x) \\
               b \text{ if } Q(x)
            \end{cases}
		\]
		
\end{itemize}

\section{Boolean MATLANG}
\section{Expressions as functions}

Another way of looking at expressions in MATLANG is as functions between matrix spaces, this is $$e:(M_1, \ldots, M_k)\rightarrow M,$$ where $M,M_1,\ldots, M_k$ are matrix spaces, i.e., $M,M_1,\ldots, M_k\in\lbrace\mathcal{M}^{n\times m}: n,m\in\mathbb{N}^+\rbrace$. \\

Some examples:

\begin{itemize}
	\item If $A$ is $n\times m$ then 
		\begin{align*}
			e(A)=t(A):\mathcal{M}_{n\times m}&\rightarrow \mathcal{M}_{m\times n} \\
			A&\rightarrow A^*
		\end{align*}
	\item If $A$ is $n\times m$ then 
		\begin{align*}
			e(A)=\mathbf{1}(A):\mathcal{M}_{n\times m}&\rightarrow \mathcal{M}_{n\times 1} \\
			A&\rightarrow n\text{ times}\begin{cases}\begin{bmatrix}
    1 \\
    \vdots \\
    1 \\
    \vdots \\
    1
\end{bmatrix}\end{cases}
		\end{align*}
	\item If $v$ is $n\times 1$ then 
		\begin{align*}
			e(v)=\text{diag}(v):\mathcal{M}_{n\times 1}&\rightarrow \mathcal{M}_{n\times n} \\
			v&\rightarrow \begin{bmatrix}
			    v_{1}       & 0 & 0 & \dots & 0 \\
			    0       & v_{2} & 0 & \dots & 0 \\
			    \hdotsfor{5} \\
			    0       & 0 & 0 & \dots & v_{n}
			\end{bmatrix}
		\end{align*}
	\item If $A$ is $n\times m$ and $B$ is $m\times p$ then 
		\begin{align*}
			e(A, B)=A\cdot B:\mathcal{M}_{n\times m}\times\mathcal{M}_{m\times p}&\rightarrow \mathcal{M}_{n\times p} \\
			(A, B)&\rightarrow A\cdot B
		\end{align*}
	\item If $A^{(1)}, \ldots, A^{(n)}$ are $m\times p$ matrices then $$e(A^{(1)}, \ldots, A^{(n)})=\text{apply}[f](A^{(1)}, \ldots, A^{(n)})$$ has domains
		\begin{align*}
			\mathcal{M}_{m\times p}^n&\rightarrow \mathcal{M}_{m\times p} \\
			(A^{(1)}, \ldots, A^{(n)})&\rightarrow C:C_{ij}=f(A^{(1)}_{ij}, \ldots, A^{(n)}_{ij}).
		\end{align*}
\end{itemize}

We can start to analyze if this functions are increasing, decreasing, boolean, etc. Also, we can study the effects of disturbances on the input in the output of these functions.
\section{Core of Matlab and R}

\subsection*{MATLAB}

The basic operations of MATLAB over matrices are:

\begin{itemize}
	\item \textbf{mldivide}(A, B): returns $x$ such that $Ax=B$.
	\item \textbf{descomposition}(A): returns a decomposition or factorization $LU, LDL, QR,$ Cholesky, etc.
	\item \textbf{inv}(A): returns $A^{-1}$.
	\item \textbf{multiplication}: compute $A\cdot B$.
	\item \textbf{transpose}(A): returns $A^T$.
	\item \textbf{conjugate transpose}(A): returns $A'$.
	\item \textbf{matrix power}(A, k): returns $A^k$.
	\item \textbf{eigen}(A): returns the eigenvectors and the eigenvectors matrices of $A$.
	\item \textbf{funm}(A, f): returns matrix $B$ with elements $b_{ij}=f(a_{ij})$.
	\item \textbf{crossprod}(a, b): vectorial product, returns $c$ such that $c\perp a, b.$
	\item \textbf{dotprod}(a,b): returns $a\cdot b$.
	\item \textbf{diag}(v): $v$ vector. Returns the following matrix: 
	\[
\begin{bmatrix}
    v_{1}       & 0 & 0 & \dots & 0 \\
    0       & v_{2} & 0 & \dots & 0 \\
    \hdotsfor{5} \\
    0       & 0 & 0 & \dots & v_{n}
\end{bmatrix}
\]
	\item \textbf{diag}(A): given matrix $A$, it returns
	\[
\begin{bmatrix}
    a_{11} \\
    a_{22} \\
    \vdots \\
    a_{nn}
\end{bmatrix}
\]
	\item \textbf{det}(A): returns the determinant of $A$.
	\item \textbf{zeros}(n, m): returns a $n\times m$ matrix full of zeros.
	\item \textbf{ones}(n, m): returns a $n\times m$ matrix full of ones.
	\item \textbf{A[i, j]}: you can get $A_{ij}$.
\end{itemize}

\subsection*{R}

The basic operations of the language R over matrices are:

\begin{itemize}
	\item \textbf{A\%*\%B}: matrix multiplication.
	\item \textbf{A*B}: pointwise multiplication.
	\item \textbf{t(A)}: transpose.
	\item \textbf{diag(v)}: returns the matrix 
	\[
\begin{bmatrix}
    v_{1}       & 0 & 0 & \dots & 0 \\
    0       & v_{2} & 0 & \dots & 0 \\
    \hdotsfor{5} \\
    0       & 0 & 0 & \dots & v_{n}
\end{bmatrix}
\]
	\item \textbf{diag(A)}: Returns the vector
	\[
\begin{bmatrix}
    a_{11} \\
    a_{22} \\
    \vdots \\
    a_{nn}
\end{bmatrix}
\]
	\item \textbf{diag(k)}: $k$ scalar. It creates the $k\times k$ identity matrix.
	\item \textbf{matrix(k, n, m)}: returns the $n\times m$ matrix, where every entry is equal to $k$.
	\item \textbf{solve(A, b)}: returns $x$ such that $Ax=b$.
	\item \textbf{solve(A)}: returns $A^{-1}$.
	\item \textbf{det(A)}: determinant of $A$.
	\item \textbf{y}$<$-\textbf{eigen(A)}: stores de eigenvalues of $A$ in y\$val and the eigenvectors in y\$vec.
	\item \textbf{y}$<$-\textbf{svd(A)}: it computes and stores the following:
	\begin{itemize}
		\item y\$d: vector of the singular values of $A$.
		\item y\$u: matrix of the left singular vectors of $A$.
		\item y\$v: matrix of the right singular vectors of $A$.
	\end{itemize}
	\item \textbf{R}$<$-\textbf{chol(A)}: Cholesky fatorization, $R'R=A$.
	\item \textbf{y}$<$-\textbf{qr(A)}: $QR$ decomposition, strored in y\$qr.
	\item \textbf{cbind(A,B, v, ...)}: joins matrices and vector horizontally, returns a matrix.
	\item \textbf{rbind(A,B, v, ...)}: joins matrices and vector vertically, returns a matrix.
	\item \textbf{rowMeans(A)}: returns the vector of the averages over the rows of $A$.
	\item \textbf{colMeans(A)}: returns the vector of the averages over the columns of $A$.
	\item \textbf{rowSums(A)}: returns the vector of the sums over the rows of $A$.
	\item \textbf{colSums(A)}: returns the vector of the sums over the columns of $A$.
	\item \textbf{outer(A, B, f)}: applies $f(\cdot, \cdot)$. Returns matrix $C$ of entries $c_{ij}=f(a_{ij}, b_{ij})$.
	\item \textbf{A[i, j]}: you can get $A_{ij}$.
\end{itemize}

\label{sec:systems}