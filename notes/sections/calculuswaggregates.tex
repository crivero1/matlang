\section{Connection with logic with aggregates}
\newcommand{\ML}{\mathsf{MATLANG}}
\newcommand{\reals}{\mathbf{R}}
\newcommand{\cmplx}{\mathbf{C}}
%\newcommand{\C}{\mathbf{C}}
\newcommand{\nat}{\mathbf{N}}
\newcommand{\pnat}{\nat_{>0}}
\newcommand{\var}{var}
\newcommand{\scm}{\mathcal{S}}
\newcommand{\sizevar}{\mathsf{SizeVars}}
\newcommand{\mname}{\mathsf{MatNames}}
\newcommand{\allinst}[1]{\mathcal{I}_{#1}}
\newcommand{\cmats}{\mathcal{M}}
\newcommand{\false}{\bot}
\newcommand{\true}{\top}
\newcommand{\boolinterp}{\nu}
\newcommand{\Rel}{\mathit{Rel}}
\newcommand{\Mat}{\mathit{Mat}}
\newcommand{\one}{\mathbf{1}}
\newcommand{\Sum}{\mathsf{sum}}
\newcommand{\diag}{\mathsf{diag}}
\newcommand{\Apply}{\mathsf{apply}}
\newcommand{\Tb}{\mathbf{b}}
\newcommand{\Tn}{\mathbf{n}}
\newcommand{\dom}{\mathbf{dom}}
\newcommand{\rname}{\mathsf{RelNames}}
\newtheorem{proposition}{Proposition}
We first recall how MATLANG expressions can be simulated in the calculus with aggregates (see Expressive power of SQL paper by Leonid
for definition of this calculus). I am copying stuff from the ICDT paper (and upcoming TODS submission).

To fix the relational representation of matrices, it is natural to represent an $m \times n$ matrix $A$
by a ternary relation $$
\Rel_2(A) := \{(i,j,A_{i,j}) \mid i \in \{1,\dots,m\}, \ j \in
\{1,\dots,n\}\}. $$  In the special case where $A$ is an
$m \times 1$ matrix (column
vector), $A$ can also be represented by a binary relation
$\Rel_1(A) :=
\{(i,A_{i,1}) \mid i \in \{1,\dots,m\}\}$.  Similarly, a $1
\times n$ matrix (row vector) $A$ can be represented by $\Rel_1(A)
:= \{(j,A_{1,j}) \mid j \in \{1,\dots,n\}\}$.  Finally, a $1
\times 1$ matrix (scalar) $A$ can be represented by the unary
singleton relation $\Rel_0(A) := \{(A_{1,1})\}$. 

More formally,
we assume a supply of \emph{relation variables}, which, for
convenience, we can take to be the same as the matrix variables.
A \emph{relation type} is a tuple of $\Tb$'s and $\Tn$'s.
A \emph{relational schema} $\scm$ is a function, defined on a
nonempty finite set
$\var(\scm)$ of relation variables, that assigns a relation type
to each element of $\var(\scm)$.

To define relational instances, we assume a countably infinite universe
$\dom$ of abstract atomic data elements.  It is convenient to
assume that the natural numbers are contained in $\dom$.  We
stress that this assumption is not essential but simplifies the
presentation.  Alternatively, we would have to work with explicit
embeddings from the natural numbers into $\dom$.

Let $\tau$ be a relation type. A \emph{tuple
of type $\tau$} is a tuple $(t(1),\dots,t(n))$ of the same arity
as $\tau$, such that $t(i) \in \dom$ when $\tau(i) = \Tb$, and
$t(i)$ is a complex number when $\tau(i) = \Tn$.
A \emph{relation of type
$\tau$} is a finite set of tuples of type $\tau$.  
An \emph{instance} of a relational schema $\scm$ is a
function $I$ defined on $\var(\scm)$ so that $I(R)$ is a relation
of type $\scm(R)$ for every $R \in \var(\scm)$.

%In order to translate $\ML$ to the relational algebra with summation, we must connect
The matrix data model can now be formally connected to the relational data
model, as follows. Let $\tau = s_1\times s_2$ be a matrix type.  Let us call $\tau$ a
\emph{general type} if $s_1$ and $s_2$ are both size symbols; a
\emph{vector type} if $s_1$ is a size symbol and $s_2$ is 1, or
vice versa; and the \emph{scalar type} if $\tau$ is $1\times 1$.
To every matrix type $\tau$ we associate a relation type
$$ \Rel(\tau) := \begin{cases}
(\Tb,\Tb,\Tn) & \text{if $\tau$ is general;} \\
(\Tb,\Tn) & \text{if $\tau$ is a vector type;} \\
(\Tn) & \text{if $\tau$ is scalar.} \end{cases} $$
Then to every matrix schema $\scm$ we associate the relational
schema $\Rel(\scm)$ where $\Rel(\scm)(M) = \Rel(\scm(M))$ for
every $M \in \var(\scm)$.  For each instance $I$ of
$\scm$, we define the instance $\Rel(I)$ over
$\Rel(\scm)$ by $$ \Rel(I)(M) = \begin{cases} 
\Rel_2(I(M)) & \text{if $\scm(M)$ is a general type;} \\
\Rel_1(I(M)) & \text{if $\scm(M)$ is a vector type;} \\
\Rel_0(I(M)) & \text{if $\scm(M)$ is the scalar type.}
\end{cases} $$  

\begin{proposition} 
Let $\scm$ be a matrix schema, and let $e$ be a $\ML$ expression
that is well-typed over $\scm$ with output type $\tau$.
Let $\ell=2$, $1$,
or $0$, depending on whether $\tau$ is general, a vector type, or
scalar, respectively.
For every $\ML$ expression $e$ there is a formula
$\varphi_e$ over $\Rel(\scm)$
in the relational calculus with summation, such that
\begin{enumerate}
\item
If\/ $\tau$ is general, $\varphi_e(i,j,z)$ has two free base
variables $i$ and $j$ and one free numerical variable $z$; if\/
$\tau$ is a vector type, we have $\varphi_e(i,z)$; and if\/ $\tau$ is
scalar, we have $\varphi_e(z)$.
\item
For every instance $I$, the relation defined by $\varphi_e$ on
$\Rel(I)$ equals $\Rel_\ell(e(I))$.
\item
The formula $\varphi_e$ uses only {\bf three distinct base variables}. The functions used in pointwise applications in $\varphi_e$ 
 are those used in
pointwise applications in $e$; complex conjugation;
multiplication of two numbers;
and the constant functions $0$ and $1$.
 Furthermore, $\varphi_e$ neither uses equality conditions between numerical variables nor equality conditions on base variables involving constants.
\end{enumerate}
\end{proposition}


Let us assign, to each $\ML$ expression $e$ that is well-typed over $\scm$, an expression $\varphi_e$ in the relational calculus with
summation as follows. 

\begin{itemize}
\item If $e = M$ is a matrix variable of $\scm$, then $\varphi_e(i,j,x):=\Rel_2(M)(i,j,x)$ if $M$ is of general type,
 $\varphi_e(i,x):=\Rel_1(M)(i,x)$ if $M$ is of vector type, and $\varphi_e(x):=\Rel_0(M)(x)$ if $M$ is of scalar type.
\end{itemize}
Let $e'$ be a $\ML$ and let $\tau = s_1 \times s_2$ be the output type of $e'$.
\begin{itemize}
\item If $e = (e')^*$, then  $\varphi_e(i,j,x):=\exists x'\, (\varphi_{e'}(j,i,x')\land x=\overline{x'})$ if $\tau$ is a general type,
$\varphi_e(i,x):=\exists x'\, (\varphi_{e'}(i,x')\land x=\overline{x'})$ if $\tau$ is a vector type, and $\varphi_e(x):=\exists x'\, (\varphi_{e'}(x')\land x=\overline{x'})$ if $\tau$ is the scalar type.
Here, $\overline x$ denotes the complex conjugate operation.

\item If $e = \one(e')$, then $\varphi_e(i,x):=\exists j, x'\, (\varphi_{e'}(i,j,x')\land x=1(x'))$ if $\tau$ is a general type,
$\varphi_e(i,x):=\exists  x'\, (\varphi_{e'}(i,x')\land x=1(x'))$  is a vector type and $s_1\neq 1=s_2$, 
$\varphi_e(x):=\exists  i,x'\, (\varphi_{e'}(i,x')\land x=1(x'))$  is a vector type and $s_1=1\neq s_2$, 
and $\varphi_e(x):=\exists x'\, (\varphi_{e'}(x')\land x=1(x'))$  if $\tau$ is the scalar type. As before,  $1$ in the expression $\varphi_e$ is the constant $1$ function.


\item If $e = \mathrm{diag}(e')$, then $\varphi_e(i,j,x):=(\varphi_{e'}(i,x)\land j=i)\vee (\exists x',x''\, \varphi_{e'}(i,x')\land\varphi_{e'}(j,x'')\land i\neq j \land x=0(x')$
if $s_1 \neq 1 = s_2$ and  $\varphi_e(x):=\varphi_{e'}(x)$ if $\tau$ is the scalar type.


\item If $e = e_1 \cdot e_2$, then 
\[\begin{cases}
\varphi_e(i,j,z):=z = \Sum k,x,y . (\varphi_{e_1}(i,k,x) \land \varphi_{e_2}(k,j,y), x \times y)
 & \text{if } s_1 \neq 1 \neq s_2 \text{ and } s_3 \neq 1
\cr
\varphi_e(i,z):=z = \Sum k,x,y . (\varphi_{e_1}(i,k,x) \land \varphi_{e_2}(k,y), x \times y)
& \text{if } s_1 \neq 1 = s_2 \text{ and } s_3 \neq 1
\cr
\varphi_e(i,z):=z = \Sum k,x,y . (\varphi_{e_1}(k,x) \land \varphi_{e_2}(k,i,y), x \times y)
  & \text{if } s_1 = 1 \neq s_2 \text{ and } s_3 \neq 1
\cr
\varphi_e(z):=z = \Sum k,x,y . (\varphi_{e_1}(k,x) \land \varphi_{e_2}(k,y), x \times y)
 & \text{if } s_1 = 1 = s_2 \text{ and } s_3 \neq 1
\cr
\varphi_e(i,j,z):=\varphi_{e_1}(i,x) \land \varphi_{e_2}(j,y) \land z=x \times y
 & \text{if } s_1 \neq 1 \neq s_2 \text{ and } s_3 = 1
\cr
\varphi_e(i,z):=\varphi_{e_1}(i,x) \land \varphi_{e_2}(y) \land z=x \times y
 & \text{if } s_1 \neq 1 = s_2 \text{ and } s_3 = 1
\cr
\varphi_e(i,z):=\varphi_{e_1}(x) \land \varphi_{e_2}(i,y)\land z=x \times y
 & \text{if } s_1 = 1 \neq s_2 \text{ and } s_3 = 1
\cr
\varphi_e(z):=\varphi_{e_1}(x) \land \varphi_{e_2}(y) \land z=x \times y
 & \text{if } s_1 = 1 = s_2 \text{ and } s_3 = 1
\cr
\end{cases},
\]
where $e_1$ is of type $s_1 \times s_3$ and $e_2$ is of type $s_3 \times s_2$.

\item If $e = \Apply[f](e_1,\ldots,e_n)$, then 
\begin{align*}
\varphi_e(i,j,x)&:=\exists x_1,\ldots,x_n\, (\varphi_{e_1}(i,j,x_1)\land \cdots \land \varphi_{e_n}(i,j,x_n)\land x=f(x_1,\ldots,x_n)),\\
\varphi_e(i,x)&:=\exists x_1,\ldots,x_n\, (\varphi_{e_1}(i,x_1)\land \cdots \land \varphi_{e_n}(i,x_n)\land x=f(x_1,\ldots,x_n)), \text{ and}\\
\varphi_e(x)&:=\exists x_1,\ldots,x_n\, (\varphi_{e_1}(x_1)\land \cdots \land \varphi_{e_n}(x_n)\land x=f(x_1,\ldots,x_n))
\end{align*}
depending on whether $\tau$ is of general, vector or scalar type, respectively.
\end{itemize}
Notice that the only functions in $\varphi_e$ aside from those used in $\Apply$ in $e$ are complex conjugation ($\bar z$), multiplication of two numbers ($\times$), and the constant functions $0$ and $1$. Also notice that $\varphi_e$ uses neither negation, nor equality conditions on numerical variables, nor equality conditions on variables involving a constant.

By induction on the structure of $e$ one straightforwardly observes that $\varphi_e$ satisfies the conditions (1) and (2) in the statement of the theorem.
Furthermore, it is clear for all operations except for matrix multiplication that when $\varphi_{e'}$ uses at most $3$ base variables than so does $\varphi_e$.
When it comes to matrix multiplication, assume that $\varphi_{e_1}(i,k,x)$ uses base variables $i,j',k$ and 
 $\varphi_{e_2}(k,j,y)$ use base variables $i',k,j$. Then since only $i$, $j$ and $k$ are free variables, we can simply reuse variable $j'$ for $i'$ (or vice versa).
 Hence, $\varphi_e(i,j,z):=z = \Sum k,x,y . (\varphi_{e_1}(i,k,x) \land \varphi_{e_2}(k,j,y), x \times y)$ uses at most $3$ base variables as well.\qed


\subsection{MATLANG + sum iteration}
Let's denote this extension of $\ML$ by $\sum\ML$.
We show that we can extend the previous translation to incorporate expressions of the form $\sum_v e(v)$. More precisely, let $e$ be a $\sum\ML$ expression. Such an expression $e$ is defined on matrix variables which hold to input matrices. To formally define $\sum_v e(v)$ we reserve an infinite set of vector variables $V_1,V_2,\ldots,$ which are to hold the canonical vectors over which the sum ranges. So, in general, we consider expressions of the form 
$$
\sum_v e(V_1/v,V_2,\ldots,V_k),
$$
where $V_1/v$ means that vector variable is instantiated by vector $v$, the other vector variables $V_2,\ldots,V_k$ are ``free'' (a notion we need to define for $\sum\ML$ expressions) and these will be instantiated later by further summation (i.e., in the end a valid $\sum\ML$ expression does not have free vector variables).

It is easy to see that the inductive proof given above still works in the presence of such vector variables. The difference is that $\varphi_e$ will have occurrences of $\Rel_1(V_j)(i,x)$ which, when considering an instance $I$ of the matrix variables the vector variables will be instantiated by means of the sum iteration. As a consequence, for every such occurrence of  $\Rel_1(V_j)(i,x)$ we have an extra pair of free variables (one of $\Tb$ type for the index $i$ in the vector, and one of $\Tn$ type for the value $x$ of the entries in the vector. In summary, $\varphi_e$ is of the form 
$\varphi_e(i,j,x; i_1,x_1,\ldots,i_k,x_k)$ where the last $2k$ variables correspond to the vector variables.

Consider now $$
e:=\sum_v e'(V_1/v,V_2,\ldots,V_k),
$$ 
we assume that $e'$ is of matrix type (if $e'$ is of vector type or scalar we may need to encode these differently, shouldn't be a problem).

Then to translate  $e$  into the calculus we 
\begin{itemize}
\item first consider $\varphi_{\mathsf{Id}}(i,j,x)$ as the expression corresponding to $\diag(\one(M))$ for a matrix variable $M$, this is to generate the identity matrix.
\item
Then, we replace the vector relation  $\Rel_1(V_1)(i_1,x_1)$ in $\varphi_e'(i,j,x; i_1,x_1,\ldots,i_k,x_k)$ by columns returned by $\varphi_{\mathsf{Id}}(i_1,j',x)$ for a given $j'$:
$$
\psi_e(i,j,x;j',i_2,x_2,\ldots,i_k,x_k):=\exists i_1,x_1\, \varphi_{\mathsf{Id}}(i_1,j',x_1)\land      \varphi_{e'}(i,j,x; i_1,x_1,\ldots,i_k,x_k)[V_1(i_1,x_1)/\varphi_{\mathsf{Id}}(i_1,j',x_1)],
$$
where $\varphi_{e'}(i,j,x; i_1,x_1,\ldots,i_k,x_k)[V_1(i_1,x_1)/\varphi_{\mathsf{Id}}(i_1,j',x_1)]$ means that we subtitute $V_1(i_1,x_1)$ by 
$\varphi_{\mathsf{Id}}(i_1,j',x_1)$ everywhere in expression $\varphi_{e'}$. We further existentially quantify $i_1$ and $x_1$ so that we
when $\psi_e$ is evaluated on an instance $I$ (for matrix variables) and index $j$ (and indexes $i_2,\ldots,i_k$ and values $x_2,\ldots,x_k$,
$\psi_e$ evaluates $\varphi_{e'}$ were $V_1$ holds the $j$th column of the identity matrix.
\item It remains to sum up over all these columns in the identity matrix:
$$
\varphi_e(i,j,z;i_2,x_2,\ldots,i_k,x_k):= z= \Sum x,j' (\exists i',x',\, \varphi_{\mathsf{Id}}(i',j',x')\land \psi_e(i,j,x;j',i_2,x_2,\ldots,i_k,x_k),x)
$$
which results in an expression in which the pair of variables for $V_1$ is eliminated.
\end{itemize}

\subsection{Conjecture}
Under certain syntactical restriction on function applications and equality conditions (no numerical equal to base variable) it is possible to
translate any calculus with sum expression which starts from relations encoding matrices and has an output type that correspond to a scalar,
vector or matrix, into $\sum\ML$? Perhaps also no  negation in the expression? The challenge will be to simulate conjunction (disjunction), in other words. For examples, we can have sub-formulas $M(i,j,x)\land M(i',j',x')$ which will be existentially quantified in the end because we can only have 
three variables (indexes $i$ and $j$ and numerical variable) in the final expression. 










