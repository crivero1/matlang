\section{Connection with logic}

We show the expressive power of MATLANG. Let RRA be the class of algebras of binary relations with the operations all, identity, set difference, converse and relational composition. Also, let $\text{FO}^3_b$ denote first-order logic with three variables, equality and infinitely many binary relation symbols. This is, $\text{FO}^3_b$ are $\text{FO}^3$ graph queries.

It is known that the logic captured by RRA is $\text{FO}^3_b$. Now, we show that RRA can be interpreted into MATLANG, thus the expressive power of MATLANG is at least the same as $\text{FO}^3_b$.

Let $U$ be a nonempty finite set with $n$ elements (and thus has an enumeration $u_1,\ldots, u_n$). Let $\mathcal{A}^U\in\text{RRA}$, this is $$\mathcal{A}^U=\langle A, \cup, -, \circ, ^{-1}, I\rangle,$$ where 

\begin{itemize}
	\item $A\subseteq\mathcal{P}(U\times U)$, this is, $\forall R\in A. R\subseteq U\times U$. So $A$ is a set of binary relations over $U$.
	\item $\cup, -, \circ, ^{-1}$ denote the operations union, set difference, relational composition and converse, respectively. All of them defined over binary relations.
	\item $I$ is the constant relation symbol that denotes the set $\lbrace (u,u):u\in U\rbrace$.
\end{itemize}

Let $R\in A$. Define $M^R$ a matrix such that

\[ M^R_{ij}=
    \begin{cases} 
      1 & \text{ if }(u_i, u_j)\in R \\
      0 & \text{ if }(u_i, u_j)\notin R 
   \end{cases}
\]

Note that the MATLANG instance of $\mathcal{A}$, has $|A|$ matrices with dimentions $|U|\times |U|$. Thus binary relations are represented as adyacency matrices.

We know show how to express the RRA operations in MATLANG.

\begin{itemize}
	\item \textbf{All:} let $R\in A$ be any relation. We express $U\times U$ as $M^{U\times U}=\mathbf{1}(M^R)\cdot\mathbf{1}(M^R)^*.$
	\item \textbf{Identity:} let $R\in A$ be any relation. Then we express the constant relation $I$ as $M^I=\text{diag}(\mathbf{1}(M^R)).$
	\item \textbf{Union:} let $R,S\in A$. Then $M^{R\cup S}=\text{apply}[x\vee y](M^R,M^S).$
	\item \textbf{Set difference:} let $R,S\in A$. Then $M^{R-S}=\text{apply}[x\vee\neg y](M^R, M^S).$
	\item \textbf{Converse:} let $R\in A$. Then we express $R^{-1}=\lbrace (u,v)\in U\times U: (v,u)\in R\rbrace$ as $M^{R^{-1}}=\mathbf{t}(M^R).$
	\item \textbf{Relational composition:} let $R,S\in A$, then $M^{R\circ S}=\text{apply}[x>0](M^R\cdot M^S).$ Where 
	\[ x>0=
    \begin{cases} 
      1 & \text{ if } x>0 \\
      0 & \text{ if } x\leq 0 
    \end{cases}
	\]
\end{itemize}

Thus RRA can be interpreted into MATLANG,  and hence so does $\text{FO}^3_b$, this is, $\text{FO}^3$ graph queries.

