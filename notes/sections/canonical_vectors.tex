\section{Adding canonical vectors to MATLANG}
One thing that we cannot do in MATLANG is to obtain a specific entry of a matrix. This entry is expected to be a $1\times 1$ matrix. We can do this by adding the standard unit vectors $e_j$ where 
\[
e_i=
\begin{bmatrix}
    0 \\
    \vdots \\
    1 \\
    \vdots \\
    0
\end{bmatrix}
\rightarrow i\text{-th position}
\]

We know show some examples of what can we express with this new feature. For ilustrative reasons, we asume that all the dimentions are well suited for the corresponding operation.

\begin{itemize}
	\item Get $A_{ij}$ with $e_i^*\cdot A\cdot e_j$.
	\item The expression $e_i\cdot e_j^*$ is the matrix that has a $1$ in the position $i,j$ and zero everywhere else.
	\item Given a vector $v$, the expression $v\cdot e_i^*$ is the matrix 
	\[
\begin{bmatrix}
    0 & \cdots & v_{1} & \cdots & 0 \\
    0 & \cdots & v_{2} & \cdots & 0 \\
    \hdotsfor{5} \\
    0 & \cdots & v_{n} & \cdots & 0 
\end{bmatrix}
\]
	\item Replace column $j$ of $A$ with zeros: $A(I-e_j\cdot e_j^*)$.
	\item Replace column $j$ of $A$ with a vector $v$: $A(I-e_j\cdot e_j^*) + v\cdot e_j^*$.
\end{itemize}

Note that $I=\text{diag}(\mathbf{1}(A))$ and the sum of matrices can be implemented as $\text{apply}\left[ + \right](A, B)$.

\subsection{Ordering}

Let's turn our attention to the order of the canonical vectors $\lbrace v_i\rbrace_i$. If we have the canonical vectors, using a simple matrix product we can compute a function that discriminates if two canonical vectors are the same or not, in particular
\[
  			f(v_i,v_j)=1-v_i^*v_j=\begin{cases}
               0 \text{ if } i=j \\
               1 \text{ if } i\neq j
            \end{cases}
		\]
does exactly that. So, without adding any object, we have equality for canonical vectors. Note that implicitly we are using de identity, the matrix equivalent to the equality relation.


Now, consider the following matrix
\[
Z = \begin{bmatrix}
    0 & 1 & \cdots &  1 \\
    0 & \ddots & \ddots & \vdots \\
    \hdotsfor{3} & 1 \\
    0 & \cdots & \cdots & 0 
\end{bmatrix}.
\]
In some way, this matrix gives us an ordering of the canonical vectors. We can see this in the functions of the form 

\[
  			f(v_i, v_j)=v_i^*Zv_j=\begin{cases}
               1 \text{ if } i < j \\
               0 \text{ if } i \geq j
            \end{cases}
		\]
		
This kind of functions discriminate between canonical vectors, in terms of order. Note that this gives us an answer for any pair of canonical vector, so we have a total order in some sense.

Can we introduce a \textit{local} order of the canonical vectors? For instance, we can ask ourselves if we can define the \textit{successor} or \textit{predecessor} relation of the canonical vectors from a matrix. Let us introduce a new matrix:

\[
S := \begin{bmatrix}
    0 & 1         & 0         & \cdots &  0 \\
    0 & \ddots & 1         & \cdots & 0 \\
    \vdots & \vdots & \ddots & \ddots & \vdots \\
    \hdotsfor{3} & \ddots & 1 \\
    0 & \cdots & \cdots & \cdots & 0 
\end{bmatrix}.
\]

Let's see what this matrix does. Note that

\[
  			Sv_i=\begin{cases}
               v_{i-1} \text{ if } i > 1 \\
              \mathbf{0} \text{ if } i = 1
            \end{cases}
		\]

So it's the matrix equivalent of the \textit{predecessor} relation. For example, now we can compute 

\[
  			g(v_i, v_j)=v_i^*Sv_j=\begin{cases}
               1 \text{ if } i = j - 1 \\
               0 \text{ if not}
            \end{cases}
		\]
		
With this matrix we can compute a function that tells us if the argument is the first canonical vector. Note that we don't have exactly $$\mathbf{1}\cdot Su=0\leftrightarrow u=cv_1, \hspace{1ex} c\in\mathbb{R},$$ it depends on the entries of $u$, but  $$\mathbf{1}\cdot f_{\neq 0}(Su)=0\leftrightarrow u=cv_1, \hspace{1ex} c\in\mathbb{R},$$ and we have that
\[
  			min(u)=(1-\mathbf{1}\cdot f_{\neq 0}(Su))=\begin{cases}
               1 \text{ if } u = cv_1 \\
               0 \text{ if not}
            \end{cases}
		\]

Since $S$ and $Z$ give us an order, it is natural to compute one from the other, so

\begin{align*}
Z&=f_{>0}\left(\prod v. (S+I)\right) \\
S&=Z-f_{>0}(Z^2)
\end{align*}

\label{sec:canonical}