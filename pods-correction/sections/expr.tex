% !TeX spellcheck = en_US
%!TEX root = ../main.tex
In this section we explore the expressive power of $\langfor.$ Given that arithmetic circuits \cite{allender} capture most standard linear algebra algorithms \cite{Raz02,ShpilkaY10}, they seem as a natural candidate for comparison. Intuitively, an arithmetic circuit is similar to a boolean circuit \cite{aroraB2009}, except that it has gates computing the sum and the product function, and processes elements of $\RR$ instead of boolean values. To connect \langfor to arithmetic circuits we need a notion of uniformity of such circuits. After all, a \langfor expression can take matrices of arbitrary dimensions as input and we want to avoid having  different circuits for each dimension. To handle inputs of different sizes, we thus consider a notion of uniform families of arithmetic circuits, defined via a Turing machine generating a description of the circuit for each input size $n$.

What we show in the remainder of this section is that any function $f$ which operates on matrices, and is computed by a uniform family of arithmetic circuits of bounded degree, can also be computed by a \langfor expression, and vice versa. In order to keep the notation light, we will focus on 
 \langfor schemas over ``square matrices'' where each variable has type $(\alpha,\alpha),(\alpha,1),(1,\alpha)$, or $(1,1)$, although all of our results hold without these restrictions as well. In what follows, we will write $\langfor$ to denote $\langforf{\emptyset}$, i.e. the fragment of our language with no additional pointwise functions. We begin by defining circuits and then show how circuit families can be simulated by $\langfor.$

\subsection{From arithmetic circuits to \langfor}
Let us first recall the definition of arithmetic circuits. 
An \textit{arithmetic circuit} $\Phi$ over a set $X=\{x_1,\ldots,x_n\}$ of input variables is a directed
acyclic labeled graph. The vertices of $\Phi$ are called \textit{gates} and denoted by $g_1,\ldots,g_m$;
the edges in $\Phi$ are called \textit{wires}. The children of a gate $g$ correspond to all gates
$g'$ such that $(g,g')$ is an edge. The parents of $g$ correspond to all gates $g'$ 
such that $(g',g)$ is an edge. The \textit{in-degree}, or a \textit{fan-in}, of a gate $g$ refers to its number of children, and 
the \textit{out-degree} to its number of parents. We will not assume any restriction on the in-degree of a gate, and will thus consider circuits with unbounded fan-in. Gates with in-degree $0$ are called \textit{input gates}
and are labeled by either a variable in $X$ or a constant $0$ or $1$. All other gates
are labeled by either $+$ or $\times$, and are referred to as \textit{sum gates} or \textit{product gates}, respectively.
Gates with out-degree $0$ are called \textit{output gates}. When talking about arithmetic circuits, one usually focuses on circuits with $n$ input gates and a single output gate.\looseness=-1

The \textit{size} of $\Phi$, denoted by $|\Phi|$, is its number of gates and wires. The \textit{depth} of $\Phi$, denoted
by $\mathsf{depth}(\Phi)$, is the length of the longest directed path from any of its output gates to any of the input gates. The \textit{degree} of a gate is defined inductively: an input gate has degree~1, a sum gate has a degree equal to the maximum of degrees of its children, and a product gate has a degree equal to the sum of the degrees of its children. When $\Phi$ has a single output gate, the \textit{degree} of $\Phi$, denoted by $\mathsf{degree}(\Phi)$, is defined as the degree of its output gate. If $\Phi$ has a single output gate and its input gates take values from $\RR$, then $\Phi$ corresponds to a polynomial in $\RR[X]$ in a natural way. In this case, the {degree} of $\Phi$ equals the degree of the polynomial corresponding to $\Phi$.
If $a_1,\ldots ,a_n$ are values in $\RR$, then 
the result of the circuit on this input is the value computed by the corresponding polynomial, denoted by $\Phi(a_1,\ldots ,a_k)$.

In order to handle inputs of different sizes, we use the notion of uniform circuit families. An \textit{arithmetic circuit family} is a set of arithmetic circuits $\{\Phi_n\mid n=1,2,\ldots\}$ where $\Phi_n$ has $n$ input variables and a single output gate. An arithmetic circuit family is \textit{uniform} if there exists a \logspace-Turing machine,
which on input $1^n$, returns an encoding of the arithmetic circuit $\Phi_n$ for each $n$.
We observe that uniform arithmetic circuit families are necessarily of polynomial size. 
Another important parameter is the circuit depth. A circuit family is of logarithmic depth, whenever $\mathsf{depth}(\Phi_n)\in \mathcal{O}(\log\, n)$. We  now show that \langfor subsumes uniform arithmetic circuit families that are of logarithmic depth. 

\begin{theorem}
\label{th-circuits-ml}
For any uniform arithmetic circuit family $\{\Phi_n\mid n=1,2,\ldots\}$ of logarithmic depth there is a \langfor schema $\Sch$ and an expression $e_\Phi$ using a matrix variable $v$, with $\ttype(v)=(\alpha,1)$ and $\ttype(e) = (1,1)$, such that for any input values $a_1,\ldots ,a_n$: 
\begin{itemize}
\item If $\I = (\dom,\conc)$ is a \lang\ instance such that $\dom(\alpha) = n$ and $\conc(v) = [a_1 \ldots a_n]^T$.
\item Then $\sem{e_\Phi}{\I} = \Phi_n(a_1,\ldots ,a_n)$.
\end{itemize}
\end{theorem}
It is important to note that the expression $e_\Phi$ does not change depending on the input size, meaning that it is uniform in the same sense as the circuit family being generated by a single Turing machine. The different input sizes for a \langfor instance are handled by the typing mechanism of the language. 

\textit{Proof sketch.} The proof of this Theorem, which is the deepest technical result of the paper, depends crucially on two facts: (i) that any polynomial time Turing machine working within linear space and producing linear size output, can be simulated via a \langfor\ expression; and (ii) that evaluating an arithmetic circuit $\Phi_n$ can be done using two stacks of  depth $n$.

Evaluating  $\Phi_n$ on input $(a_1,\ldots ,a_n)$ can be done in a depth-first manner by maintaining  two stacks: the gates-stack that tracks the current gate being evaluated, and the values-stack that stores the value that is being computed for this gate. The idea behind having two stacks is that whenever the number of items on the gates-stack is higher by one than the number of items on the values-stack, we know that we are processing a fresh gate, and we have to initialize its current value (to 0 if it is a sum gate, and to 1 if it is a product gate), and push it to the values-stack. We then proceed by processing the children of the head of the gates-stack one by one, and aggregate the results using sum if we are working with a sum gate, and by using product otherwise. 

In order  to access the information about the gate we are processing (such as whether it is a sum or a product gate, the list of its children, etc.) we use the uniformity of our circuit family. Namely, we know that we can generate the circuit $\Phi_n$ with a \logspace-Turing machine $M_\Phi$ by running it on the input $1^n$. Using this machine, we can in fact compute all the information needed to run the two-stack algorithms described above. For instance, we can construct a \logspace\ machine that checks, given two gates $g_1$ and $g_2$, whether $g_2$ is a child of $g_1$. Similarly, we can construct a machine that, given $g_1$ and $g_2$ tells us whether $g_2$ is the final child of $g_1$, or the one that produces the following child of $g_1$ (according to the ordering given by the machine $M_\Phi$). Defining these machines based on $M_\Phi$ is similar to the algorithm for the composition of two \logspace\ transducers, and is commonly used to evaluate arithmetic circuits \citep{allender}. 

To simulate the circuit evaluation algorithm that uses two stacks, in \langfor we can use a binary matrix of size $n\times n$, where $n$ is the number of inputs. The idea here is that  the gates-stack corresponds to the first $n-3$ columns of the matrix, with each gate being encoded as a binary number in positions $1,\ldots,n-3$ of a row. The remaining three columns are reserved for the values-stack, the number of elements on the gates stack, and the number of elements on the values stack, respectively. The number of elements is encoded as a canonical vector of size $n$. Here we crucially depend on the fact that the circuit is of logarithmic depth, and therefore the size of the two stacks is bounded by $n$ (apart from the portion before the asymptotic bound kicks-in, which can be hard-coded into the expression $e_\Phi$). Similarly, given that the circuits are of polynomial size, we can assume that gate ids can be encoded into $n-3$ bits.

This matrix is then updated in the same way as the two-stack algorithm. It processes gates one by one, and using the successor relation for canonical vectors determines whether we have more elements on the gates stack. In this case, a new value is added to the values stack ($0$ if the gate is a sum gate, and $1$ otherwise), and the process continues. Information about the next child, last child, or input value, are obtained using the expression which simulates the Turing machine generating this data about the circuit (the machines used never produce an output longer than their input). Given that the size of the circuit is polynomial, say $n^k$, we can initialize the matrix with the output gate only, and run the simulation of the two-stack algorithm for $n^k$ steps (by iterating $k$ times over size $n$ canonical vectors). After this, the value in position  $(1,n-2)$ (the top of the values stack) holds the final results. \qed

\smallskip
While Theorem \ref{th-circuits-ml} gives us an idea on how to simulate arithmetic circuits, it does not tell us which classes of functions over real numbers can be computed by \langfor expressions. In order to answer this question, we note that arithmetic circuits can be used to compute functions over real numbers. Formally, a circuit family $\{\Phi_n\mid n=1,2,\ldots\}$ computes a function $f:\bigcup_{n\geq 1} \mathbb{R}^n\mapsto\mathbb{R}$, if for any $a_1,\ldots a_n\in \mathbb{R}$ it holds that $\Phi_n(a_1,\ldots ,a_n) = f(a_1,\ldots ,a_n)$. To make the connection with \langfor\!, we need to look at circuit families of bounded degree. 

A circuit family $\{\Phi_n\mid n=1,2,\ldots\}$ is said to be of \textit{polynomial degree} if $\mathsf{degree}(\Phi_n)\in O(p(n))$, for some polynomial $p(n)$. Note that polynomial size circuit families are not necessarily of polynomial degree. An easy corollary of Theorem \ref{th-circuits-ml} tells us that all functions computed by uniform family of circuits of polynomial degree and logarithmic depth can be simulated using \langfor expressions. However, we can actually drop the restriction on circuit depth due to the result of Valiant et. al.~\cite{valiant1981fast} and Allender et. al. \cite{AllenderJMV98} which says that any function computed by a uniform circuit family of polynomial degree (and polynomial depth), can also be computed by a uniform circuit family of logarithmic depth. Using this fact, we can conclude the following:


\begin{corollary}
\label{cor-circ-ml}
For any function $f$ computed by a uniform family of arithmetic circuits of polynomial degree, there is an equivalent \langfor formula $e_f$.
\end{corollary}

Note that there is nothing special about circuits that have a single output, and both Theorem \ref{th-circuits-ml} and Corollary \ref{cor-circ-ml} also hold for functions  $f:\bigcup_{n\geq 1} \mathbb{R}^n\mapsto\mathbb{R}^{s(n)}$, where $s$ is a polynomial. Namely, in this case, we can assume that circuits for $f$ have multiple output gates, and that the depth reduction procedure of \cite{AllenderJMV98} is carried out for each output gate separately. Similarly, the construction underlying the proof of Theorem \ref{th-circuits-ml} can be performed for each output gate independently, and later composed into a single output vector.

\subsection{From \langfor to circuits}

Now that we know that arithmetic circuits can be simulated using \langfor expressions, it is natural to ask whether the same holds in the other direction. That is, we are asking whether for each \langfor expression $e$ over some schema $\Sch$ there is a uniform family of arithmetic circuits computing precisely the same result depending on the input size. 

In order to handle the fact that \langfor\ expressions can produce any matrix, and not just a single value, as their output, we need to consider circuits which have multiple output gates. Similarly, we need to encode matrix inputs of a \langfor\ expression in our circuits. We will write $\Phi(A_1,\ldots ,A_k)$, where $\Phi$ is an arithmetic circuit with multiple output gates, and each $A_i$ is a matrix of dimensions $\alpha_i\times \beta_i$, with $\alpha_i,\beta_i \in \{n,1\}$ to denote the input matrices for a circuit $\Phi$. We will also write $\texttt{type}(\Phi)=(\alpha,\beta)$, with $\alpha,\beta\in \{n,1\}$, to denote the size of the output matrix for $\Phi$. We call such circuits \textit{arithmetic circuits over matrices}. When $\{\Phi_n\mid n=1,2,\ldots\}$ is a uniform family of arithmetic circuits over matrices, we will assume that the Turing machine for generating $\Phi_n$ also gives us the information about how to access a position of each input matrix, and how to access the positions of the output matrix, as is usually done when handling matrices with arithmetic circuits \cite{Raz02}. The notion of degree is extended to be the sum of the degrees of all the output gates. With this  at hand, we can now show the following result.

\begin{theorem}
\label{th-ml-to-circuits}
Let $e$ be a \langfor expression over a schema $\Sch$, and let $V_1,\ldots ,V_k$ be the variables of $e$ such that $\ttype(V_i)\in \{(\alpha,\alpha), (\alpha,1), (1,\alpha), (1,1)\}$. Then there exists a uniform arithmetic circuit family over matrices $\Phi_n(A_1,\ldots ,A_k)$ such that:
\begin{itemize}
\item For any instance $\I = (\dom,\conc)$ such that $\dom(\alpha) = n$ and $\conc(V_i) = A_i$ it holds that:
\item $\sem{e}{\I} = \Phi_n(A_1,\ldots ,A_k)$.
\end{itemize}
\end{theorem}

It is not difficult to see that the proof of Theorem \ref{th-circuits-ml} can also be extended to support arithmetic circuits over matrices. In order to identify the class of functions computed by \langfor expressions, we need to impose one final restriction: than on the degree of an expression. 
In particular, we will be interested in \langfor expressions of polynomial degree. Formally, an expression $e$ is of polynomial degree, whenever there is an equivalent circuit family for $e$ of a polynomial degree.  
For example, all \langfor expressions seen so far have polynomial degree.
With this definition, we can now identify the class of functions for which arithmetic circuits and \langfor formulas are equivalent. This is the main technical contribution of the paper. 

\begin{corollary}
\label{th-equivalence}
Let $f$ be a function with input matrices $A_1,\ldots ,A_k$ of dimensions $\alpha\times \beta$, with $\alpha,\beta \in \{n,1\}$. Then, $f$ is computed by a uniform circuit family over matrices of polynomial degree if and only if there is a \langfor expression of polynomial degree for $f$. 
\end{corollary}

Note that this result crucially depends on the fact that expressions in \langfor are of polynomial degree. Some \langfor expressions are easily seen to produce results which are not polynomial. An example of such an expression is, for instance, $e_{\texttt{exp}} = \ffor{v}{X=A}{X\cdot X}$, over a schema $\Sch$ with $\ttype(v)= (\gamma,1)$, and $\ttype(X)=(1,1)$. Over an instance which assigns $n$ to $\gamma$ this expression computes the function $a^{2^n}$, for $A=[a]$. Therefore, a natural question to ask then is whether we can determine the degree of a \langfor expression. Unfortunately, as we show in the following proposition this question is in fact undecidable.

\begin{proposition}
\label{prop-undec}
Given a \langfor expression $e$ over a schema $\Sch$, it is undecidable to check whether $e$ is of polynomial degree.
\end{proposition}

Of course, one might wonder whether it is possible to define a syntactic subclass of \langfor expressions that are of polynomial degree and can still express many important linear algebra algorithms. We identify one such class in Section \ref{ss:sumML}, called \langsum, and in fact show that this class is powerful enough to capture relational algebra on (binary) $K$-relations. 

\subsection{Supporting additional operators}
The equivalence of \langfor and arithmetic circuits we proved above assumes that circuits can only use the sum and product gates (note that even without the sum and the product function, \langfor\ can simulate these operations via matrix sum/product). However, both arithmetic circuits and expressions in $\langfor$ can be allowed to use a multitude of functions over $\RR$. The most natural addition to the set of functions is the division operator, which is crucially needed in many linear algebra algorithms, such as, for instance, Gaussian elimination, or $LU$ decomposition (recall Proposition \ref{prop:gauss}).
Interestingly, the equivalence in this case still holds, mainly due to a surprising result which shows that (almost all) divisions can in fact be removed for arithmetic circuits which allow sum, product, and division gates \cite{allender}.

More precisely, in \cite{strassen1973vermeidung,borodin1982fast,kaltofen1988greatest} it was shown that for any function of the form $f = g/h$, where $g$ and $h$ are relatively prime polynomials of degree $d$, if $f$ is computed by an arithmetic circuit of size $s$, then both $g$ and $h$ can be computed by a circuit whose size is polynomial in $s + d$. Given that we can postpone the division without affecting the final result, this, in essence, tells us that division can be eliminated (pushed to the top of the circuit), and we can work with sum-product circuits instead. The degree of a circuit for $f$, can then be defined as the maximum of degrees of circuits for $g$ and $h$. Given this fact, we can again use the depth reduction procedure of \cite{AllenderJMV98}, and extend Corollary~\ref{th-equivalence} to circuits with division.
\begin{corollary}
\label{cor-division}
Let $f$ be a function taking as its input matrices $A_1,\ldots ,A_k$ of dimensions $\alpha\times \beta$, with $\alpha,\beta \in \{n,1\}$. Then, $f$ is computed by a uniform circuit family over matrices of polynomial degree that allows divisions, if and only if there is a $\langforf{f_/}$ expression of polynomial degree for $f$.
\end{corollary}

An interesting line of future work here is to see which additional functions can be added to arithmetic circuits and \langfor formulas, in order to preserve their equivalence. Note that this will crucially depend on the fact that these functions have to allow the depth reduction of \cite{AllenderJMV98} in order to be supported.