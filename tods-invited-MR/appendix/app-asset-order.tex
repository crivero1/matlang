% !TeX spellcheck = en_US
%!TEX root = ../main.tex
% \floris{This section should be placed in the appropriate place% , i.e., this should be part of proofs for the last section. Let's see how the paper evolves? What is the precise result shown here?}
% \thomas{Updated, I think now it's more clear what is shown here. It should still go in the appendix of the last section? To me it seems order related.}

% Assume we can only perform the restricted versions of $\ffor{v}{X}{e}$, this is, we can only do 
% $\ssum$ and $\sprod$. Note that these operators also iterate over canonical vectors
% in certain order. Since they are operations that \textit{aggregate} information, it doesn't seem
% possible to access this order explicitly and compare canonical vectors, like in the full version of $\langfor$.
% Let's see what we can do if this order is \textit{supplied}. 
We conclude by verifying that the fragment defined in Section \ref{subsec:langlinear}, i.e, 
\langmprod extended with order and $f_{>0}$, 
can perform matrix inversion and compute the determinant. To this aim, we verify that all order
predicates in Section \ref{app:order} can be derived using $\ssum$, $\sprod$, $f_{>0}$ and 
$e_{S_{<}}$. Given this, it suffices to observe that Csanky's algorithm, as shown in Section~\ref{app:inverse}, only relies on expressions using $\ssum$ and $\sprod$ and order information on canonical vectors and $f_/$.
As consequence, our fragment can perform matrix inversion and compute the determinant.


It remains to show that if we have $e_{S_{<}}$, using $\ssum$ and $\sprod$ and $f_{>0}$ we can
can define all order predicates from Section~\ref{app:order}. We note that due to the restricted for-loops
in $\ssum$ and $\sprod$, we do not have access to the intermediate
result in the iterations and as such, it is unclear whether order information can be computed. This is why
we assume access to $e_{S_<}$.

We first remark that if we have $e_{S_{<}}$, we can also obtain
 $e_{S_{\leq}}$ by adding $e_{\mathsf{Id}}$. Hence,
%
%  such that the following holds:
%
% $$
% b_i^T\cdot S_{<} \cdot b_j=\begin{cases}
%                1, \text{ if } i < j.\\
%               \mathbf{0}, \text{ if not.}
%             \end{cases}
% $$
% As a consequence we
we can compute $\EDIT{\islessorequal}$ and $\EDIT{\isless}$ as well. Furthermore, 
\begin{align*}
  e_{\mathsf{min}}&\coloneqq \ssum v. \left[ \sprod w. \EDIT{\islessorequal}(w,v)\right] \times v. \\
  e_{\mathsf{max}}&\coloneqq \ssum v. \left[ \sprod w. \left( 1-\EDIT{\islessorequal}(w,v) \right) \right] \times v.
\end{align*}
Both expressions are only using $\ssum$ and $\sprod$ and $\EDIT{\islessorequal}$, so are in our fragment.
% Consider $S_<$ and the matrix $J$ that consist of all ones (assume they are $n\times n$). From linear algebra we know that
% \[
% S_<^2 \circ \left( S_{<}^2-2\times J\right) \circ \left( S_{<}^2-3\times J\right) = \begin{bmatrix}
%     0 & n-1 & 0 & \cdots &  0 \\
%     0 & \ddots & \ddots & \vdots & \vdots \\
%     0 & \ddots & \ddots & \vdots & 0\\
%     \hdotsfor{4} & n-1 \\
%     0 & \cdots & \cdots & \cdots & 0
% \end{bmatrix}.
% \]
% Recall that $\circ$ is the Hadamard product. We can compute a matrix that has $n-1$ in
% all of its entries as $\ssum v.(1-\mathsf{max}(v))\times e_{\ones}(v)\cdot e_{\ones}(v)^T$.
% So, if we have $f_{/}$ we can define
% $$
% e_{\mathsf{Pred}}\coloneqq  f_{/}\left( f_{\odot}\left(e_{S_{<}}^2, e_{S_{<}}^2-2\times (e_{\ones}(e_{S_{<}})\cdot e_{\ones}(e_{S_{<}})^T), e_{S_{<}}^2-3\times (e_{\ones}(e_{S_{<}})\cdot e_{\ones}(e_{S_{<}})^T) \right),\ssum v.(1-\mathsf{max}(v))\times e_{\ones}(v)\cdot e_{\ones}(v)^T\right)
% $$
% Thus we have $e_{\mathsf{Next}}\coloneqq e_{\mathsf{Pred}}^T$.
% We can now define $\mathsf{prev}(v)$ and $\mathsf{next}(v)$ as in the previous section.
% The same goes with $e_{\mathsf{getPrevMatrix}}(V)$,
% $e_{\mathsf{getNextMatrix}}(V)$, $e_{\mathsf{min}+i}$ and $e_{\mathsf{max}+i}$.
Furthermore, if we have $f_{>0}$ then we can define
$$
e_{\mathsf{Pred}}\coloneqq  e_{S_{<}}- f_{>0}(e_{S_{<}}^2)
$$
Also, recall that  $e_{\mathsf{Next}}\coloneqq e_{\mathsf{Pred}}^T$. As a consequence, 
we can now define $\mathsf{prev}(v)$ and $\mathsf{next}(v)$ as in \ref{app:order}. Similarly,
it is readily verified that also $e_{\mathsf{getPrevMatrix}}(V)$,
$e_{\mathsf{getNextMatrix}}(V)$, $e_{\mathsf{min}+i}$ and $e_{\mathsf{max}+i}$ can be expressed
in our fragment.



% On the other hand, supose we have $e_{\mathsf{Prev}}$
% (and thus $e_{\mathsf{Next}}$) such that the following holds:

% $$
% b_i^T\cdot \mathsf{Prev} \cdot b_j=\begin{cases}
%                1, \text{ if } i = j-1.\\
%               \mathbf{0}, \text{ if not.}
%             \end{cases},
% \hspace{2em}b_i^T\cdot \mathsf{Next} \cdot b_j=\begin{cases}
%                1, \text{ if } i=j+1.\\
%               \mathbf{0}, \text{ if not.}
%             \end{cases}.
% $$
% We can also define $\mathsf{prev}(v)$ and $\mathsf{next}(v)$.
% Note that we can compute
% $$
% e_{S_{\leq}}\coloneqq \ssum v. \left( v\cdot v^T + \ssum w. v\cdot\mathsf{next}(v)^T \right)
% $$
% Here, $(e_{S_{\leq}})_{ij}=1$ if and only if $b_i$ comes 
% before according to $\mathsf{Pred}$ and $\mathsf{Next}$, or is equal to $b_j$. As a consequence we
% can compute $\EDIT{\islessorequal}$ and $\EDIT{\isless}$.

% We can now compute everything as we have $S_{\leq}$ and thus $S_{<}$.

% $$
% b_i^T\cdot \mathsf{Prev} \cdot b_j=\begin{cases}
%                1, \text{ if } b_i \text{ comes immediately before } b_j.\\
%               \mathbf{0}, \text{ if not.}
%             \end{cases},
% \hspace{2em}b_i^T\cdot \mathsf{Next} \cdot b_j=\begin{cases}
%                1, \text{ if } b_i \text{ comes immediately after } b_j.\\
%               \mathbf{0}, \text{ if not.}
%             \end{cases}.
% $$


% The expression in this section are intended to be used over canonical vectors.
% To have access to order, we need a matrix $S_{\leq}$ such that the following holds for 
% canonical vectors $b_i$ and $b_j$:
% $$
% b_i^T\cdot S_{\leq} \cdot b_j=\begin{cases}
%                1, \text{ if } b_i \text{ comes before } b_j\\
%               \mathbf{0}, \text{ if not.}
%             \end{cases}
% $$
% Note that $S_{\leq}$ encodes \textit{total} order. 

% Using this matrix 


% We would also like to have access to 
% \textit{local} information, this is, know if a canonical vector comes immediately before another. 
% Let's call this matrix $\mathsf{Prev}$. 
% The following must hold:
% $$
% b_i^T\cdot \mathsf{Prev} \cdot b_j=\begin{cases}
%                1, \text{ if } b_i \text{ comes immediately before } b_j\\
%               \mathbf{0}, \text{ if not.}
%             \end{cases}
% $$

% Using this matrix, we have that for a canonical vector $b_i$:
% \[
% \mathsf{Prev}\cdot b_i=\begin{cases}
%                b_{i-1}, \text{ if } i > 1 \\
%               \mathbf{0}, \text{ if } i = 1
%             \end{cases}
% \]

% Note that $\mathsf{Prev}$ can be defined using the following \langfor expression:
% $$
% e_{\mathsf{Prev}}\coloneqq  \texttt{for }v,X.\quad X + \left[ (1 - \mathsf{max}(v))\times ve_{\mathsf{max}}^T - (Xe_{\mathsf{max}})\cdot e_{\mathsf{max}}^T + (Xe_{\mathsf{max}})\cdot v^T\right].
% $$

% Here, $X$ starts as 0 and thus in turn $X\cdot e_{\mathsf{max}}$, so we initiate storing $b_1$ in the last column. Next, we add a matrix that has the stored vector $Xe_{\mathsf{max}}$ (the previous canonical vector) in the column indicated by $v$ (the current canonical vector) and $v-Xe_{\mathsf{max}}$ in the last column, to replace the vector stored.
% The last iteration does nothing (sums the zero matrix).

% To get the \textit{next} relation we simply do $e_{\mathsf{Next}} = e_{\mathsf{Prev}}^T$. We have that for a canonical vector $b_i$:
% \[
% \mathsf{Next}\cdot b_i=\begin{cases}
%                b_{i+1}, \text{ if } i < n \\
%               \mathbf{0}, \text{ if } i = n
%             \end{cases}
% \]
