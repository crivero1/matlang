%!TEX root = ../main.tex

One of our main motivations to introduce for-loops is to be able to express classical linear algebra algorithms by a small set of core operators. We have seen that \langfor is
quite expressive as it can check for cliques, compute the transitive closure, and can even
leverage a successor relation on canonical vectors. The big question is how expressive \langfor
actually is. We will answer this question more formally in Section \ref{sec:circuits}, where we compare \langfor with 
arithmetic circuits of polynomial degree.
Before doing so, in this section we will illustrate that \langfor is capable of expressing many classical linear algebra algorithms. More precisely, we start by showing how  \langfor is able to
compute LU decompositions of matrices. These decompositions form the basis of many other algorithms, such as solving linear systems of equations. We further show that \langfor is expressive enough to compute matrix inversion and the determinant. We recall that matrix inversion and determinant need to be explicitly added as ad-hoc operators in \lang~\cite{matlang-journal} and that the LARA language is unable to invert matrices under usual complexity-theoretic assumptions~\cite{BarceloH0S20}.


\subsection{LU decomposition}

\begin{algorithm}[t]
    \caption{\EDIT{LU pseudocode}}
    \label{alg:lu}
    \begin{algorithmic}[1]
        % \Function{LU}{$A$}
        %     \State{$A_0\coloneqq A$}
        %     \For{$i=1,2,\ldots,n$}
        %         \State{$A_i\coloneqq$ GETREDUCEMATRIX$(A_{i-1},i)\cdot A_{i-1}$}
        %     \EndFor
        %     \Return{$A_n$}
        % \EndFunction
        % \State{ }
        \Function{LU}{$A$}
            \State{$A_0\coloneqq A$}
            \State{$T_0\coloneqq I$}
            \For{$i=1,2,\ldots,n$}
                \State{$A_i\coloneqq T_{i-1}\cdot A_{i-1}$}
                \State{$L^{-1}_i\coloneqq$ GETREDUCEMATRIX$(A_i,i)\cdot T_{i-1}$}\Comment{Note that $L_n^{-1}=T_1\cdot\cdots\cdot T_n$}
            \EndFor
            \State{}
            \State{$L\coloneqq (-1)\times L^{-1}_n + 2\times I$ }\Comment{This is equal to $T_1^{-1}\cdot\cdots\cdot T_n^{-1}$}
            \State{$U\coloneqq A_n$ }
            \State{}
            \Return{$L,U$}
        \EndFunction
        \State{ }
        \Function{GETREDUCEMATRIX}{$A,i$}\Comment{This computes $T_i$}
            \State{$c\coloneqq [0,\ldots,0]^T$}
            \For{$j=i+1,\ldots,n$}
                \State{ $c_{j,1} \coloneqq -\dfrac{A_{ij}}{A_{ii}}$}\Comment{Assuming that $A_{ii}\neq 0$}
                % \State{$T^{(i)}_{ij}\coloneqq A_{ij}+c_{j}A_{ii}$}
            \EndFor
            \State{}
            \State{$b_i\coloneqq [0,\ldots,1,\ldots,0]^T$}\Comment{$i$-th canonical vector}
            \State{$T\coloneqq I + c\cdot b^T$}
            \State{}
            \Return{$T$}
        \EndFunction
    \end{algorithmic}
\end{algorithm}

A lower-upper (LU) decomposition factors a matrix $A$ as the product of a lower triangular matrix $L$ and upper triangular matrix $U$.  
This decomposition, and more generally LU decomposition with row pivoting (PLU),  underlies many linear algebra algorithms and 
we next show that \langfor can compute these decompositions.

LU decomposition \EDIT{(described in Algorithm~\ref{alg:lu})} can be seen as a matrix form of Gaussian elimination in which the columns of $A$
are reduced, one by one, to obtain the matrix $U$. The reduction of columns of $A$ is achieved
as follows. Consider the first column $[A_{11},\ldots,A_{n1}]^T$ of $A$ and  define 
$c_1 \coloneqq   [0, \alpha_{21},\ldots, \alpha_{n1}]^T$ 
with $\alpha_{j1} \coloneqq   -\frac{A_{j1}}{A_{11}}$. Let $T_1\coloneqq  I+ c_1\cdot b_1^T$ and consider
$T_1\cdot A$. That is, the $j$th row of $T_1\cdot A$ is obtained by multiplying the first row of $A$ by $\alpha_{j1}$ and adding it to the $j$th row of $A$. As a result, the first column of $T_1\cdot A$ is equal to $[A_{11},0,\ldots,0]^T$, i.e., 
all of its entries below the diagonal are zero.  One then iteratively performs a similar computation, using the matrix $T_i\coloneqq  I+c_i\cdot b_i^T$, where $c_i$ now depends on the $i$th column in $T_{i-1}\cdots T_1\cdot A$. As a consequence, $T_i\cdot T_{i-1}\cdots T_1\cdot A$ is upper triangular
in its first $i$ columns. At the end of this process, $T_n\cdots T_1\cdot A=U$ where $U$ is the desired upper triangular matrix.
Furthermore, it is easily verified that each $T_i$ is invertible and by defining $L\coloneqq  T_1^{-1}\cdot\cdots\cdot T_n^{-1}$ one obtains a lower triangular matrix satisfying $A=L\cdot U$. The above procedure is only successful when the denominators used in the definition of the vectors $c_i$ are non-zero. When this is the case we call a matrix $A$ \textit{LU-factorizable} \cite{num}. 

In case when such a denominator is zero in one of the reduction steps, one can remedy this situation by \textit{row pivoting}. That is, when the $i$th entry of the
$i$th row in $T_{i-1}\cdots T_1\cdot A$ is zero, one replaces the $i$th row by  $j$th row in this matrix, with $j>i$, provided that the $i$th entry of the $j$th row is non-zero. If no such row exists, this implies that all elements below the diagonal are zero already in column $i$ and one can proceed with the next column. One can formulate this in matrix terms by stating that there exists a permutation matrix $P$, which pivots rows, such that $P\cdot A=L\cdot U$. Any matrix $A$ is LU-factorizable \textit{with pivoting} \cite{num}.

To implement the described procedure in \langfor, we first assume that the input matrices are LU-factorizable, and  deal with general matrices later on. The following result tells us how to do this in $\langforf{f_/}$; that is, in \langfor extended with the \textit{division} function  $f_/:\RR^2\to\RR$, which is defined by $f_/(x,y)\coloneqq x/y$, for $y\neq 0$, and $f_/(x,y)\coloneqq   0$, otherwise. %, given by $f_/(x,y) \coloneqq   x/y$, where we define division by zero to be equal zero.

\begin{proposition}\label{prop:gauss}
There exist $\langforf{f_/}$ expressions $e_L(V)$ and $e_U(V)$ such that
$\sem{e_L}{\I}=L$ and $\sem{e_U}{\I}=U$ form an LU-decomposition of $A$,
\EDIT{where $\I$ assigns $V$ to a matrix $A$ that is} LU-factorizable.
\end{proposition}
\begin{proof}
Assume $A$ to be an LU-factorizable matrix. To compute the LU decomposition of a matrix, as described above, we need to compute the vector $c_i$ for each column $i$. We do this in two steps. First, we extract from our input matrix its $i$th column and set all its upper diagonal entries to zero
by means of 
 the 
 expression:
$$\ccol{V}{y} \coloneqq   \Sigma v. \  \EDIT{\isless}(y,v)\cdot(v^T\cdot V \cdot y)\times v.$$
Indeed, when $V$ is assigned to a matrix $A$ and $y$ to $b_i$, we initially assign the zero (column) vector to $A_0$, i.e., 
$A_0\coloneqq  \mathbf{0}$, and then in consecutive iterations,  $A_j\coloneqq  A_{j-1}+ (b_j^T\cdot A\cdot b_i)\times b_i$ if $j>i$ (because $\EDIT{\isless}(b_i,b_j)=1$ if $j>i$) and $A_j\coloneqq A_{j-1}$ otherwise (because $\EDIT{\isless}(b_i,b_j)=0$ for $j\leq i$). The result of this evaluation is the desired column vector.
Using $\ccol{V}{y}$, we can now compute $T_i$ by the following expression (which corresponds to function GETREDUCEMATRIX in Algorithm~\ref{alg:lu}):
$$\red{V}{y} \coloneqq   e_{\mathsf{Id}}+ f_/\bigl(\ccol{V}{y},-(y^T\cdot V\cdot y)\times e_{\ones}(y)\bigr)\cdot y^T.$$
When $V$ is assigned to $A$ and $y$ to $b_i$, $f_/(\ccol{A}{b_i},-(b_i^T\cdot A\cdot b_i)\times \ones)$ is equal to the vector $c_i$ used in the definition of $T_i$. To perform the reduction steps for all columns, we consider
the expression:
$$
e_{U}(V) \coloneqq    \left( \initf{e_{\mathsf{Id}}}{y}{X}{\red{X\cdot V}{y}\cdot X} \right) \cdot V.
$$
That is, when $V$ is assigned $A$, $X$ will be initially $A_0\coloneqq  e_{\mathsf{Id}}$, and then
$A_i\coloneqq  \red{A_{i-1}\cdot A}{b_i}\cdot A=T_i\cdot T_{i-1}\cdots T_1\cdot A$, as desired.

In order to construct $e_L(V)$, we will use the fact that $e_U(A)=T_n\cdot\cdots\cdot T_1\cdot A$ with $L^{-1}=T_n\cdot\cdots\cdot T_1$, and that each $T_i$ can easily be inverted. For this, let
    $$
    e_{L^{-1}}(V) \coloneqq    \initf{e_{\mathsf{Id}}}{y}{X}{\red{X\cdot V}{y}\cdot X}.
    $$
    such that	$e_{\mathsf{U}}(V)=  e_{L^{-1}}(V) \cdot V$. It now suffices to observe that, since $T_n=I$,
    \begin{align*}
    L^{-1}&=(I-c_1\cdot b_1^T)\cdots (I-c_{n-1}\cdot  b_{n-1}^T)
    =I-c_1\cdot b_1^T-\cdots - c_{n-1}\cdot b_{n-1}^T,
    \end{align*}
and therefore,    
    \begin{align*}
    L&=(I+c_1\cdot b_1^T)\cdots (I+c_{n-1}\cdot b_{n-1}^T) =I+c_1\cdot b_1^T+\cdots + c_{n-1}\cdot b_{n-1}^T.
    \end{align*}
    As a consequence, to obtain $L$ from $L^{-1}$ we just need to multiply every entry below the diagonal by $-1$. Since both  $L$ and $L^{-1}$ are lower triangular, this can done 
    by computing $L=-1\times L^{-1} + 2\times I$. Translated into \langfor, this means that we can define
    $$
    e_{L}(V) \coloneqq    -1\times e_{L^{-1}}(V) + 2\times e_{\mathsf{Id}},
    $$
    which concludes the proof of the proposition.    
\end{proof}

When row pivoting is needed, we can also obtain a permutation matrix
$P$ such that $P\cdot A=L\cdot U$ holds by means of an expression in \langfor, provided
that we additionally allow the function $f_{>0}$, 
where $f_{>0}:\RR\to\RR$ is such that $f_{>0}(x)\coloneqq  1$ if $x>0$ and $f_{>0}(x)\coloneqq  0$ otherwise.
Intuitively, $f_{>0}$ introduces a limited form of \texttt{if-then-else} to \langfor, allowing to continue reducing columns only when the right pivot has been found.

\begin{proposition}\label{prop:palu}
There exists expressions $e_{L^{-1}P}(M)$ and $e_U(M)$ in $\langforf{f_/,f_{>0}}$ such that
$L^{-1}\cdot P=\sem{e_{L^{-1}P}}{\I}$ and $U=\sem{e_U}{\I}$ satisfy $L^{-1}\cdot P\cdot A=U$,
\EDIT{when $\I$ assigns $M$ to a matrix $A$ that is PALU factorizable.}
\end{proposition}

\begin{proof}

    We assume that $f_{/}$ and $f_{>0}$ are in $\mathcal{F}$. Let $A$ be an arbitrary matrix.
    By contrast to when $A$ is LU-factorizable, during the LU-decomposition process we may need row interchange (pivoting) in each step of the iteration. Let us assume that row interchange is needed immediately before 
    step $k$, $1\leq k\leq n$. In other words, we now aim to reduce the $k$-th column of $A_k=T_{k-1}\cdots T_1\cdot A$, 
    or $A_k=A$ if $k=1$, but now $A_k$ has a zero pivot, i.e., $(A_{k})_{kk}=0$. 
    Let $P$ be the matrix that denotes the necessary row interchange. If we know
    $P$, then 
    to compute $T_k$ we need to perform $\red{P\cdot X\cdot A}{v}$ in this iteration,
    where $\red{\cdot}$ is the expression in \langfor reducing a column, as defined in the proof of Proposition \ref{prop:gauss}.
    Furthermore, we need to apply the permutation $P$ to the current result, leading to the 
    expression $\initf{e_{\mathsf{Id}}}{v}{X}{\red{P\cdot X\cdot A}{v}\cdot P\cdot X}$. We now remark that
    $P$ is a permutation matrix of the  form $P = I - u\cdot u^T$ and it denotes an interchange (if multiplied on the left) of rows $i$ and $j$ if $u=(b_{i}-b_{j})$. We also note that we are performing a row interchange for column $k$ and thus $i=k$ and $j>k-1$. If no interchange is needed, $i=j=k$ and $P=I$.
    In addition, when $k=n$ no interchange takes place. Finally, if no suitable $b_j$ can
    be found, this implies that no interchange is required as well and we can move on to next column.

    To find the vector $u$ needed for $P$, we can, for example, find the first entry $j\geq k$ in column $k$ of $A_k$ that holds a non-zero value. In $\langforf{f_{>0}}$  we can, more generally, find the first entry in a vector $a$ that holds a non-zero value by using the function $f_{>0}$. Indeed, consider the following expression:

    \begin{align*}\allowdisplaybreaks
       \nneq{a}{u} := &\texttt{for}\, v,X \texttt{.}\, \left( 1-e_{\ones}(v)^T\cdot X \right) \times f_{>0}\left( ( v^T\cdot a )^2 \right)\times v \\
        &\hspace{4em}+ \mathsf{max}(v)\times\left( 1-e_{\ones}(v)^T\cdot X \right)\times \left( 1 - f_{>0}\left( ( v^T\cdot a )^2 \right) \right) \times u \\
    \end{align*}
where $e_{\ones}(v)$ is the expression for the $\ones(v)$ operator (see Example~\ref{ex:onevec}).
Here, $\nneq{a}{u}$ receives two $n$ dimensional vectors $a$ and $u$ and outputs a 
    canonical vector $b_j$ such that $a_j$ is the first non-zero entry of $a$, or $u$ if such non-zero value does not exist in $a$. We check for $f_{>0}((v^T\cdot a)^2)$ 
    in case a negative number is tested. The above expression simply checks in each iteration
    whether $X$ already holds a canonical vector. If so, then $X$ is not updated. Otherwise,
    $X$ is replaced by the current canonical vector $b_j$ if and only if $b_j^T\cdot a$ is non-zero. Furthermore, when the final canonical vector is considered and $X$ does not hold
    a canonical vector yet and $b_n^T\cdot a$ is zero, the vector $u$ is returned.

    We use $\nneq{a}{u}$ to find a pivot for a specific column. Let us assume again that we
    want to find a pivot in column $k$ of $A_k$. We can then first make all entries in that column, with indexes smaller or equal to $k$, zero, just as we did by means of $\ccol{\cdot}{\cdot}$ in the
    definition of $\red{\cdot}{\cdot}$. Except, now we also need to make the $k$th entry zero as well.
    Let us denote by $\ccoleq{\cdot}{\cdot}$ the operation $\ccol{\cdot}{\cdot}$, as defined in the proof of Proposition \ref{prop:gauss}, but using $\mathsf{succ}$ instead of $\mathsf{succ}^+$ (to include the $k$ entry). Given this, we can construct $P=I-u\cdot u^T$ as follows:
    $$
    e_{P_u}(A,u) := e_{\mathsf{Id}} - \bigl(u - \nneq{ \ccoleq{A}{u} }{u} \bigr)\cdot \bigl(u - \nneq{ \ccoleq{A}{u} }{u} \bigr)^T.
    $$ 
    From the explanations given above, it should be clear that $e_{P_u}(A,u)$ computes the necessary permutation matrix of $A_k$ for the column indicated by $u$, or $I$
    if no permutation is needed, or if such permutation does not exist (so we skip the current column). Also, we have to modify the $\red{V}{y}$ operator, resulting in $\redd{V}{y}$, as follows:

    \begin{align*}
        &\redd{V}{y}:= e_{\mathsf{Id}}+ f_{>0}\left( ( y^T\cdot V\cdot y)^2 \right)\times f_/\biggl(\ccoleq{V}{y}, \\
        &\hspace{4em}\Bigl(-(y^T\cdot V\cdot y)\times e_{\ones}(y) + \Bigl( 1 - f_{>0}\bigl(( y^T\cdot V\cdot y)^2 \bigr) \Bigr)\times e_{\ones}(y) \Bigr)\biggr)\cdot y^T. \\
    \end{align*}
    
    So, when $V$ is interpreted by a matrix $B$ and $y=b_i$, it returns $I+c_ib_i^T$ if $B_{ii}$ is not zero. 
    If $B_{ii}=0$ then we divide $\ccoleq{B}{b_i}$ by $e_{\ones}(b_i)$ (so we don't get \textit{undefined}), 
    but we don't add $c_ib_i^T$ precisely because $B_{ii}=0$, and return the identity so nothing happens. We check 
    for $f_{>0}((\cdot)^2)$ in case a negative number is tested.
    Finally, we define
    $$
    e_{L^{-1}P}(V):=\initf{e_{\mathsf{Id}}}{v}{X}{\redd{e_{P_v}(X\cdot V,v)\cdot X\cdot V}{v}\cdot e_{P_v}(X\cdot V,v)\cdot X}
    $$
    and $e_{\mathsf{U}}(V):=e_{L^{-1}P}(V)\cdot V$ as the desired expressions.

    As a final observation, in the definition of $e_{L^{-1}P}(V)$ 
    we interlaced permutation matrices with the $T_i$'s. More specifically, 
    $A_k=T_k\cdot P\cdot T_{k-1}\cdots T_1\cdot A$. We observe, however, that for $\ell\leq k-1$ and
    $T_{\ell}=I-c_\ell\cdot b_\ell^T$, we have that  $b_\ell^T\cdot P=b_\ell^T$ because $b_\ell$ has zeroes in positions in the rows involved in the row exchange $P$. Also, note that  $P^2=I$ and thus 
    $$P\cdot T_\ell\cdot P=P^2-P\cdot c_\ell\cdot b_\ell^T\cdot P=I-\widehat{c}_\ell\cdot b_\ell^T,$$
where $\widehat{c}_\ell:=P\cdot c_\ell$. We define $\widehat{T}_\ell:=I-\widehat{c}_\ell\cdot b_\ell^T$.
    As a consequence,
    $$
    T_k\cdot P\cdot T_{k-1}\cdots T_1=T_k\cdot P\cdot T_{k-1}\cdot P^2\cdot T_{k-2}\cdot P^2\cdots P^2 \cdot T_1\cdot P^2=T_k\cdot (P\cdot T_{k-1}\cdot P)\cdots (P\cdot T_1\cdot P)\cdot P=\widehat{T}_{k-1}\cdots \widehat{T}_1\cdot P,
    $$
    and thus we may assume that $P$ occurs at the end. Hence, we obtain $L^{-1}\cdot P\cdot A=U$.
\end{proof}



\subsection{Determinant and inverse}\label{sec:queries:inverse}
Other key linear algebra operations include the computation of the determinant and
the inverse of a matrix (if the matrix is invertible). As a consequence of the expressibility
in $\langforf{f_/,f_{>0}}$ of LU-decompositions with pivoting, it can be shown that the determinant
and inverse can be expressed as well. 

However, the results
in the next section (connecting \langfor with arithmetic circuits) imply that the determinant
and inverse of a matrix can already be defined in $\langforf{f_/}$. So instead of using LU decomposition with pivoting for matrix inversion and computing the determinant, we provide an alternative solution that will be based on Csanky's algorithm~\cite{Csanky76}. The main objective of this subsection is to show the following result:

\begin{proposition}\label{prop:inverse}
    There are $\langforf{f_/}$ expressions $e_{\mathsf{det}}(V)$ and $e_{\mathsf{inv}}(V)$ such that
    $\sem{e_{\mathsf{det}}}{\I}=\mathsf{det}(A)$ and  
    $\sem{e_{\mathsf{inv}}}{\I}=A^{-1}$ when $\I$ assigns $V$
    to $A$ and $A$ is invertible.
\end{proposition}
% !TeX spellcheck = en_US
%!TEX root = ../../main.tex

We begin by showing that the result holds in a special case when considering non-singular lower or upper triangular matrices.
\begin{lemma}\label{prop:upperlowerinverse}
There are $\langforf{f_/}$ expressions $e_{\mathsf{upperDiagInv}}(V)$ and $e_{\mathsf{lowerDiagInv}}(V)$
such that $\sem{e_{\mathsf{upperDiagInv}}}{\I}=A^{-1}$ when $\I$ assigns $V$
to an invertible upper triangular matrix $A$ and, similarly, $\sem{e_{\mathsf{lowerDiagInv}}}{\I}=A^{-1}$ when $\I$ assigns $V$
to an invertible lower triangular matrix $A$.
\end{lemma}

\begin{proof} We start by considering the expression:
    $$
    e_{\mathsf{ps}}(V)\coloneqq  e_{\mathsf{Id}} + \ssum v.\sprod w. \left[ \EDIT{\islessorequal}(w,v)\times V + (1 - \EDIT{\islessorequal}(w,v))\times e_{\mathsf{Id}} \right].
    $$
    Here, $e_{\mathsf{ps}}(A)$ results in $I+A+A^2+\cdots + A^n$ for any matrix $A$. In the expression, the outer loop defines which power we compute. 
    That is, when $v$ is the $i$th canonical vector, we compute $A^i$.
    Computing $A^i$ is achieved via the inner product loop, which uses $\EDIT{\islessorequal}(w,v)$ to 
    determine whether $w$ comes before or is $v$ in the ordering of canonical vectors.
    When this is the case, we multiply the current result by $A$, and when $w$ is greater 
    than $v$, we use the identity as not to affect the already computed result. We add the identity at the end.

    Now, let $A$ be an $n\times n$ matrix that is upper triangular and let $D_A$ be the diagonal matrix consisting of the diagonal elements of $A$.
	% , i.e.,
%     \[
%     D_A = \begin{bmatrix}
%         a_{11} & \cdots & \cdots &  0 \\
%         0 & a_{22} & \cdots &  0 \\
%         0 & \ddots & \vdots & \vdots \\
%         \vdots & \cdots& \cdots & a_{nn}
%     \end{bmatrix}.
%     \]
    We can compute $D_A$ by the expression:
    $$
    e_{\mathsf{getDiag}}(V) \coloneqq  \ssum v. (v^T\cdot V\cdot v) \times v\cdot v^T.
    $$
    Let $T=A-D_A$. We can write
    $$
    A^{-1}=\left[ D_A+T \right]^{-1}= \left[ D_A\left( I+D_A^{-1}T\right) \right]^{-1} = \left( I+D_A^{-1}T\right)^{-1}D_A^{-1}.
    $$
    We now observe that $D_{A}^{-1}$ simply consists of the inverses of the elements on the diagonal. This can be expressed, as follows:
    $$
    e_{\mathsf{diagInverse}}(V)\coloneqq \ssum v. f_{/}(1,v^T\cdot V\cdot v)\times v\cdot v^T
    $$
    where $f_{/}$ is the division function. In the last equality we take advantage of the fact that the diagonals of $A$ and $D_A$ are the same.

    We now focus on the computation of $\left( I+D_A^{-1}\cdot T\right)^{-1}$. First, by construction, $D_A^{-1}\cdot T$ is strictly upper triangular and thus nilpotent, i.e.,   $\left( D_A^{-1}\cdot T\right)^n=0$  where $n$ is the dimension of $A$.
    Recall the following algebraic identity 
    $$(1+x)\left( \sum_{i=0}^{m}(-x)^i \right)=1-(-x)^{m+1}.$$
    By choosing $m=n-1$ and applying it to $x=D_A^{-1}\cdot T$, we have
    $$
    \left(I+D_A^{-1}\cdot T \right)\left( \sum_{i=0}^{n-1}(-D_A^{-1}\cdot T)^i \right)=I- \left( -D_A^{-1}\cdot T\right)^n =I.
    $$
    Hence,
    $$
    \left(I+D_A^{-1}\cdot T \right)^{-1}=\sum_{i=0}^{n-1}(-D_A^{-1}\cdot T)^i=\sum_{i=0}^{n}(-D_A^{-1}\cdot T)^i.
    $$
    We now observe that
    $$
    e_{\mathsf{ps}}(-1\times D_A^{-1}\cdot T)=\sum_{i=0}^{n}(-D_A^{-1}\cdot T)^i=\left(I+D_A^{-1}\cdot T \right)^{-1},
    $$
    and thus 
    $$
    A^{-1}= e_{\mathsf{ps}}\left(-1\times \left[e_{\mathsf{diagInverse}}(A)\cdot(A-e_{\mathsf{getDiag}}(A))\right] \right)\cdot e_{\mathsf{diagInverse}}(A).
    $$
    Seeing this as a \langfor\ expression:
    $$
    e_{\mathsf{upperDiagInv}}(V)\coloneqq  e_{\mathsf{ps}}\left(-1\times \left[e_{\mathsf{diagInverse}}(V)\cdot(V-e_{\mathsf{getDiag}}(V))\right] \right)\cdot e_{\mathsf{diagInverse}}(V),
    $$
    we see that  when interpreting $V$ as an  upper triangular invertible matrix, 
    $e_{\mathsf{upperDiagInv}}(A)$ evaluates to $A^{-1}$.
   To deal with invertible lower triangular matrices $A$, we observe that  $\left(A^{-1}\right)^T=\left(A^T\right)^{-1}$ and $A^T$ is upper triangular.
    Hence, it suffices to define
    $$
    e_{\mathsf{lowerDiagInv}}(V)\coloneqq  e_{\mathsf{upperDiagInv}}(V^T)^T.
    $$
    This concludes the proof of the lemma.
\end{proof}

\medskip 
\noindent{\textsc{Proof of Proposition \ref{prop:inverse}.}}
With this result at hand, we are now ready to prove Proposition \ref{prop:inverse}. More specifically, we rely on Csanky's algorithm for computing the inverse of a matrix~\cite{Csanky76}.
This algorithm uses the characteristic polynomial $p_A(x)=\mathsf{det}(xI-A)$ of a matrix. When expanded as a polynomial $p_A(x)=\sum_{i=0}^{n} c_i x^i$, it is known that
$A^{-1}=\frac{1}{c_n}\sum_{i=0}^{n-1}c_i A^{n-1-i},$ if $c_n\neq 0$. It also holds that $c_n=(-1)^n\mathsf{det}(A)$, and   $c_0 = 1$. We can thus write $p_A(x)=1 + \sum_{i=1}^n c_ix^i$. Furthermore, the coefficients $c_1,\ldots,c_n$ are known to satisfy the system of equations $S\cdot \bar c=\bar s$ given by:
    $$
    \underbrace{\left(\begin{matrix}
    1 & 0 & 0 & \cdots & 0 & 0\\
    S_1 & 1 & 0 & \cdots  &0 & 0\\
    S_2 & S_1 & 1 & \cdots  &0 & 0\\
    \vdots & \vdots & \vdots & \vdots & \vdots & 0\\
    S_{n-1} & S_{n-2} & S_{n-3} & \cdots & S_1 & 1\\
    \end{matrix}\right)}_{S}\cdot
    \underbrace{\left(\begin{matrix}
    c_1\\
    c_2\\
    c_3\\
    \vdots\\
    c_n\\
    \end{matrix}\right)}_{\bar c}=\underbrace{\left(\begin{matrix}
    S_1\\
    S_2\\
    S_3\\
    \vdots\\
    S_n\\
    \end{matrix}\right)}_{\bar s},
    $$
with $S_i\coloneqq \frac{1}{i+1}\mathsf{tr}(A^i)$, where $\mathsf{tr}(\cdot)$ is the trace operator that sums up the diagonal elements of a matrix. 

    We can now compute the $S_i's$ in $\langfor$ as follows. First, for
    $i=1,\ldots,n$ we can consider
    $$
    e_{\mathsf{powTr}}(V,v)\coloneqq \ssum w. w^T\cdot\left(e_{\mathsf{pow}}(V,v)\cdot V\right)\cdot w
    $$
    with 
    $$
    e_{\mathsf{pow}}(V,v)\coloneqq  \sprod w. (\EDIT{\islessorequal}(w,v)\times V+(1-\EDIT{\islessorequal}(w,v))\times e_{\mathsf{Id}}).
    $$
    We have that $e_{\mathsf{pow}}(A,b_j)=A^{j}$ and thus $e_{\mathsf{powTr}}(A,b_j)=\mathsf{tr}(A^{j})$. Define:
    $$
    e_{S}(V,v)\coloneqq f_{/}(1, 1 + \ssum w. \EDIT{\islessorequal}(w,v))\times e_{\mathsf{powTr}}(V,v).
    $$
    Here $e_{S}(A,b_i)=S_i$. Note that $i+1$ is computed summing up to the dimension indicated by $v$, and adding 1.
    We can now easily construct the vector $\bar s$ used in the system of equations by means of the expression:
    $$
    e_{\bar s}(V)\coloneqq \ssum w.e_{S}(V,w)\times w.
    $$
    We next construct the matrix $S$. We need to be able to \textit{shift} a vector $a$ in $k$ positions, i.e.,
    such that $(a_1,\ldots,a_n)\mapsto (0,\ldots,a_1,\ldots,a_{n-k})$. We use $e_{\mathsf{getNextMatrix}}$ 
    defined in Section~\ref{sec:formatlang:design}, i.e., we define:
    $$
    e_{\mathsf{shift}}(a,v)\coloneqq \ssum w.(w^T\cdot a)\times(e_{\mathsf{getNextMatrix}}(v)\cdot w)
    $$
    performs the desired shift when $u$ is assigned a vector $a$ and $v$ is $b_k$. 
    The matrix $S$ is now obtained as follows:
    $$
    S(V)\coloneqq  e_{\mathsf{Id}} + \ssum v. e_{\mathsf{shift}}(e_{\bar s}(V), v)\cdot v^T
    $$
    We now observe that $S$ is lower triangular with nonzero diagonal entries. So,
    Lemma~\ref{prop:upperlowerinverse} tells us that we can invert it, i.e.,
    $e_{\mathsf{lowerDiagInv}}(S)=S^{-1}$. As a consequence,
    $$
    e_{\bar c}(V)\coloneqq e_{\mathsf{lowerDiagInv}}(S(V))\cdot e_{\bar s}(V).
    $$
    outputs $\bar c$ when $V$ is interpreted as matrix $A$. Observe that we only use the division operator. We now have all coefficients of the characteristic polynomial of $A$.


    We can now define
    $$
    e_{\mathsf{det}}(V)\coloneqq \left( \left(\left( \sprod w. (-1)\times e_{\ones}(V)\right)^T\cdot e_{\mathsf{max}}\right) \times e_{\bar c}(V) \right)^T\cdot e_{\mathsf{max}},
    $$
    an expression that, when $V$ is interpreted as any matrix $A$, outputs $\mathsf{det}(A)$.
    Here, $(\sprod w. (-1)\times e_{\ones}(V))$ is the $n$ dimensional vector with $(-1)^n$ in all of its entries.
    Since $c_n=(-1)^n\mathsf{det}(A)$, we extract $(-1)^n(-1)^n\mathsf{det}(A)=\mathsf{det}(A)$ with $e_{\mathsf{max}}$.

    For the inverse, we have that
    $$
    A^{-1}=\frac{1}{c_n}\sum_{i=0}^{n-1}c_i A^{n-1-i} = \frac{1}{c_n}A^{n-1} + \sum_{i=1}^{n-1}\frac{c_i}{c_n}A^{n-1-i}.
    $$
    We compute $\frac{1}{c_n}A^{n-1}$ as
    $$
    f_{/}(1, e_{\bar c}(A)^T\cdot e_{\mathsf{max}})\times e_{\mathsf{pow}}(A, e_{\mathsf{max}})
    $$
    and $\sum_{i=1}^{n-1}\frac{c_i}{c_n}A^{n-1-i}$ as
    $$
    \ssum v. f_{/}\left( e_{\bar c}(A)^T\cdot v, e_{\bar c}(A)^T\cdot e_{\mathsf{max}} \right)\times e_{\mathsf{invPow}}(A, v),
    $$
    where
    $$
    e_{\mathsf{invPow}}(V, v)\coloneqq  \sprod w. (1-\mathsf{max}(w)) \times \left[ (1 - \EDIT{\islessorequal}(w,v))\times V + \EDIT{\islessorequal}(w,v)\times e_{\mathsf{Id}} \right] + \mathsf{max}(w)\times e_{\mathsf{Id}}.
    $$
    Here, $e_{\mathsf{invPow}}(A, b_i)=A^{n-1-i}$ and $e_{\mathsf{invPow}}(A, b_n)=I$.
    Note that we always multiply by $e_{\mathsf{Id}}$ in the last step.
    To conclude, we define:
    $$
    e_{\mathsf{inv}}(V)\coloneqq  f_{/}(1, e_{\bar c}(V)^T\cdot e_{\mathsf{max}})\times e_{\mathsf{pow}}(V, e_{\mathsf{max}}) + \left[ \ssum v. f_{/}\left( e_{\bar c}(V)^T\cdot v, e_{\bar c}(V)^T\cdot e_{\mathsf{max}} \right)\times e_{\mathsf{invPow}}(V, v) \right],
    $$
    an expression that, when $V$ is interpreted as any invertible matrix $A$, computes $A^{-1}$. \qed

Note that we use a slightly different, but equivalent, system of equations to the originally one proposed in \cite{Csanky76}. Indeed,
for each equation $i=1, \ldots, n$, we divide the equality by $i$ and push the negative sign to the solution $c_1, \ldots, c_n$.

We conclude by observing that we here only use operators $\ssum$ and $\sprod$ defined in section \ref{sec:restrict} and
also assume access to order information.

