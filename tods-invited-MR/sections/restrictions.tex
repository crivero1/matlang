%!TEX root = ../main.tex
% !TeX spellcheck = en_US
We conclude the paper by zooming in on some special fragments of \langfor and in which matrices can take values from an arbitrary (commutative) semiring $K$. In particular, in Section~\ref{ss:sumML} we first consider \langsum, in which iterations can only perform
additive updates, and show that it is equivalent in expressive power to the (positive)
relational algebra on $K$-relations. We then extend \langsum such that also updates involving pointwise-multiplication (called Hadamard product) are allowed in Section~\ref{subsec:langprod}. The resulting fragment, \langprod, is shown to be equivalent in expressive power to weighted logics. Finally, in Section~\ref{subsec:langlinear}, we consider the fragment \langmprod in which updates involving sum and matrix multiplication, and possibly order information, is allowed. From the results in Section~\ref{sec:queries}, we infer that the latter fragment suffices to compute matrix inversion. An overview of the fragments and their relationships are depicted in Figure~\ref{thefigure}.

\subsection{Summation \lang and relational algebra}
\label{ss:sumML}
%When defining $4$-cliques and in several other expressions we have seen so far, we 
% only update $X$ by adding some matrix to it. This restricted form of for-loop proved useful throughout the paper, and we therefore introduce it as a special operator. That is, we define:
%$$\Sigma v. e := \ffor{v}{X}{X + e}.$$
%Here, $e=e(v)$. We define the subfragment of \langfor, called \langsum, to consist of the $\Sigma$ operator plus the ``core'' operators in \lang, namely, transposition, matrix multiplication and addition, scalar multiplication, and pointwise function applications.

Recall from Section \ref{sec:formatlang} that we introduced a special version of the for-loop that only allows to update the computed matrix by adding to it the result of some expression which depends on the current canonical vector. 
  That is, we defined the operator:
$$\Sigma v. e := \ffor{v}{X}{X + e},$$
where $e = e(v)$. We call \langsum the fragment of \langfor consisting of 
 the $\Sigma$ operator plus the ``core'' operators in \lang, namely, transposition, matrix multiplication and addition, scalar multiplication, and pointwise function applications.

One property of \langsum is that it only allows expressions of polynomial degree. Indeed, one can easily show that \langsum can only create matrix entries that are polynomial in the dimension $n$. More precisely, we can show the following (proof omitted):
\begin{proposition}\label{prop:poly}
Every expression in \langsum is of polynomial degree.
\end{proposition}

Interestingly enough, this restricted version of for-loops already allows us to capture the \lang\ operators that are not present in the syntax of \langsum. More precisely, we see from Examples~\ref{ex:onevec} and~\ref{ex:diag} that the one-vector and $\diag$ operator are expressible in \langsum. Combined with the observation that the $4$-clique
expression of Example~\ref{ex:fourcliques} is in \langsum, the following result is immediate.

\begin{corollary}
\lang\ is strictly subsumed by \langsum.
\end{corollary}

What operations over matrices can be defined with \langsum that is beyond \lang? In~\cite{brijder2019matrices}, it was shown that \lang\ is strictly included in the (positive) relational algebra on $K$-relations, denoted by $\mathsf{RA}_{K}^+$~\cite{GreenKT07}.\footnote{The algebra used in~\cite{brijder2019matrices} differs slightly from the one given in~\cite{GreenKT07}. In this paper we work with the original algebra $\mathsf{RA}_{K}^+$ as defined in~\cite{GreenKT07}.} 
It thus seems natural to compare the expressive power of \langsum with $\mathsf{RA}_{K}^+$. 
The main result in this section is that \langsum\ and $\mathsf{RA}_{K}^+$ 
are equally expressive over binary schemas. 
To make this equivalence precise, we next give the 
definition of $\mathsf{RA}_{K}^+$~\cite{GreenKT07} and then show how to connect both formalisms.

A \textit{semiring} $(K, \ksum, \kprod, \kzero, \kone)$ is an algebraic structure where $K$ is a non-empty set, $\ksum$ and $\kprod$ are binary operations over $K$, and $\kzero, \kone \in K$. Furthermore,  $\ksum$ and $\kprod$ are associative operations, $\kzero$ and $\kone$ are the identities of $\ksum$ and $\kprod$, respectively, $\ksum$ is a commutative operation, $\kprod$ distributes over $\ksum$, and $\kzero$ annihilates $K$ (i.e. $\kzero \kprod k = k \kprod \kzero = \kzero$). As usual, we assume that all semirings in this paper are commutative, namely, $\kprod$ is also commutative. We use $\bigksum_L$ or $\bigkprod_L$ for the $\ksum$- or $\kprod$-operation over all elements in $L\subseteq K$, respectively. Typical examples of semirings are the reals $(\RR, +, \times, 0,1)$, the natural numbers $(\NN, +, \times, 0,1)$, and the boolean semiring $(\{0,1\}, \vee, \wedge, 0, 1)$. 

Let $\ddom$ be a data domain and $\att$ a set of \textit{attributes}. A \textit{relational signature} is a finite subset of $\att$. A \textit{relational schema} is a function $\cR$ on finite set of symbols $\fdom(\cR)$ such that $\cR(R)$ is a relation signature for each $R \in \fdom(\cR)$. To simplify the notation, from now on we write $R$ to denote both the symbol $R$ and the relational signature $\cR(R)$.
Furthermore, we write $R \in \cR$ to say that $R$ is a symbol of $\cR$. 
For $R \in \cR$, an \textit{$R$-tuple} is a function $t: R \rightarrow \ddom$. We denote by $\tuples(R)$ the set of all $R$-tuples. Given $X \subseteq R$, we denote by $t[X]$ the restriction of $t$ to the set $X$.

We fix a semiring $(K, \ksum, \kprod, \kzero, \kone)$ and a relational schema $\cR$. A \textit{$K$-relation of} $R \in \cR$ is a function $r: \tuples(R) \rightarrow K$ such that the \textit{support}  $\supp(r):= \{t \in \tuples(R) \mid r(t) \neq \kzero\}$ is finite. 
A \textit{$K$-instance} $\cJ$ of $\cR$ is a function that assigns relational signatures of $\cR$ to $K$-relations. Given $R \in \cR$, we denote by $R^\cJ$ the $K$-relation associated to $R$ by $\cJ$. Recall that $R^\cJ$ is a function and hence  $R^\cJ(t)$ is the value in $K$ assigned to $t$. 
Given a $K$-relation $r$ we denote by $\adom(r)$ the \textit{active domain} of $r$ defined as $\adom(r):= \{t(a) \mid t \in \supp(r) \wedge a \in R\}$. More generally, the active domain of an $K$-instance $\cJ$ of $\cR$ is defined as $\adom(\cJ):= \bigcup_{R \in \cR} \adom(R^\cJ)$. 

An $\mathsf{RA}_{K}^+$  \textit{expression} $\arae$ over $\cR$ is given by the following syntax:
$$
\begin{array}{rcl}
\arae & := & R \ \mid \ \arae \cup \arae \ \mid \  \pi_X(\arae) \ \mid \  \sigma_X(\arae) \ \mid \ \rho_f(\arae) \ \mid \ \arae \bowtie \arae,
\end{array}
$$
where $R \in \cR$, $X \subseteq \att$ is finite, and $f: X \rightarrow Y$ is a one-to-one mapping with $Y \subseteq \att$. One can extend the schema $\cR$ to any expression over $\cR$ recursively as follows: $\cR(R):= R$, $\cR(\arae \cup \arae'):= \cR(\arae)$, $\cR(\pi_X(\arae)):= X$, $\cR(\sigma_X(\arae)):= \cR(\arae)$, $\cR(\rho_f(\arae)):= X$ where $f:X \rightarrow Y$, and $\cR(\arae \bowtie \arae'):= \cR(\arae) \cup \cR(\arae')$ for every expressions $\arae$ and $\arae'$.
We further assume that any expression $\arae$ satisfies the following syntactic restrictions: $\cR(\arae') = \cR(\arae'')$ whenever $\arae = \arae' \cup \arae''$, $X \subseteq \cR(\arae')$ whenever $\arae = \pi_X(\arae')$ or $\arae = \sigma_X(\arae')$, and $Y = \cR(\arae')$ whenever $\arae = \rho_f(\arae')$ with $f: X \rightarrow Y$.

Given an $\mathsf{RA}_{K}^+$ expression $\arae$ and a $K$-instance $\cJ$ of $\cR$, we define the \textit{semantics} $\ssem{\arae}{\cJ}$ as a $K$-relation of $\cR(\arae)$ as follows. For $X \subseteq \att$, let $\operatorname{Eq}_X(t):= \kone$ when $t(a) = t(b)$ for every $a, b \in X$, and $\operatorname{Eq}_X(t):= \kzero$ otherwise. For every tuple $t \in \cR(\arae)$:
$$
\begin{array}{ll}
\!\!\!\text{if $\arae = R$, then} & \ssem{\arae}{\cJ}(t):= R^\cJ(t) \\
\!\!\!\text{if $\arae = \arae_1 \cup \arae_2$, then} & \ssem{\arae}{\cJ}(t):= \ssem{\arae_1}{\cJ}(t) \ksum \ssem{\arae_2}{\cJ}(t)  \\
\!\!\!\text{if $\arae = \pi_X(\arae')$, then} & \ssem{\arae}{\cJ}(t):= \bigksum_{t': t'[X] = t} \ssem{\arae'}{\cJ}(t') \\
\!\!\!\text{if $\arae = \sigma_X(\arae')$, then} & \ssem{\arae}{\cJ}(t):= 
\ssem{\arae'}{\cJ}(t) \kprod \operatorname{Eq}_X(t)  \\
\!\!\!\text{if $\arae = \rho_f(\arae')$, then} & \ssem{\arae}{\cJ}(t):= 
\ssem{\arae'}{\cJ}(t \circ f)  
\\
\!\!\!\text{if $\arae = \arae_1 \bowtie \arae_2$, then} & \ssem{\arae}{\cJ}(t):=  \ssem{\arae_1}{\cJ}(t[Y]) \kprod  \ssem{\arae_2}{\cJ}(t[Z])
\end{array}
$$
where $Y = \cR(\arae_1)$ and $Z = \cR(\arae_2)$. We note that the $\bigksum$-operation in the semantics of $\pi_X(\arae')$ is well-defined given that the support of $\ssem{\arae'}{\cJ}$ is always finite. 

We are now ready for comparing \langsum with $\mathsf{RA}_{K}^+$. For a fair comparison, we need to extend \langsum from $\RR$ to any semiring. Let $\mtr{K}$ denote the set of all $K$-matrices. 
Similarly as for \lang\ over $\RR$, given a \lang\ schema $\Sch$, a $K$-instance $\I$ over $\Sch$ is a pair $\I = (\dom,\conc)$, where $\dom : \DD \mapsto \NN$ assigns a value to each size symbol, and $\conc : \Mnam \mapsto \mtr{K}$ assigns a concrete $K$-matrix to each matrix variable. Then it is straightforward to extend the semantics of \lang, \langfor, and \langsum from $(\RR, +, \times, 0, 1)$ to $(K, \ksum, \kprod, \kzero, \kone)$ by switching $+$ with $\ksum$ and $\times$ with $\kprod$. 

The next step for comparing \langsum with $\mathsf{RA}_{K}^+$  is to encode $K$-matrices as $K$-relations.
Let $\Sch=(\Mnam,\size)$ be a \lang\ schema. On the relational side
we have for each size symbol $\alpha\in\DD\setminus\{1\}$, attributes $\alpha$, $\row_\alpha$, and $\col_\alpha$ in $\att$. Furthermore, for each $V\in\Mnam$ and $\alpha \in \DD$ we denote
by $R_V$ and $R_\alpha$ its corresponding relation name, respectively. Then, given $\Sch$ we define the relational schema $\text{Rel}(\Sch)$ such that $\fdom(\text{Rel}(\Sch)):=  \{R_\alpha \mid \alpha\in\DD\} \cup \{R_V \mid V \in \Mnam\}$ where $\text{Rel}(\Sch)(R_\alpha):= \{\alpha\}$ and:
\[
\text{Rel}(\Sch)(R_V):= \begin{cases}
\lbrace\row_\alpha,\col_\beta \rbrace & \text{ if $ \size(V)=(\alpha,\beta)$} \\
\lbrace\row_\alpha \rbrace & \text{ if $ \size(V)=(\alpha,1)$} \\
\lbrace\col_\beta \rbrace  &
\text{ if $ \size(V)=(1,\beta)$} \\
\lbrace\rbrace & \text{ if $\size(V)=(1,1)$}.
\end{cases}
\]
Consider now a matrix instance $\I = (\dom,\conc)$ over $\Sch$.
Let $V\in\Mnam$ with $\size(V)=(\alpha,\beta)$ and let $\conc(V)$ be its corresponding $K$-matrix of dimension $\dom(\alpha)\times \dom(\beta)$.
To encode $\I$ as a $K$-instance in $\mathsf{RA}_{K}^+$, we use as data domain $\ddom = \mathbb{N} \setminus \{0\}$. Then we construct the $K$-instance $\text{Rel}(\I)$ such that for each $V\in\Mnam$ we define 
$R_V^{\text{Rel}(\I)}(t):=\conc(V)_{ij}$ whenever $t(\row_\alpha) = i \leq \dom(\alpha)$ and $t(\col_\beta) = j \leq \dom(\beta)$, and $\kzero$ otherwise. Furthermore, for each $\alpha \in \DD$ we define $R_\alpha^{\text{Rel}(\I)}(t):=\kone$ whenever $t(\alpha) \leq \dom(\alpha)$, and $\kzero$ otherwise. In other words, $R_\alpha$ and $R_\beta$ encodes the active domain of a matrix variable $V$ with $\size(V)=(\alpha,\beta)$. Given that the $\mathsf{RA}_{K}^+$ framework of \cite{GreenKT07} represents the ``absence'' of a tuple in the relation with $\kzero$, we need to separately encode the indexes in a matrix.
This is where $R_\alpha^{\text{Rel}(\I)}$ and $R_\beta^{\text{Rel}(\I)}$ are used for.
We are now ready to state the first connection between \langsum and $\mathsf{RA}_{K}^+$  by using the previous encoding. 
\begin{proposition}\label{prop:sum_to_ara} 
	For each \langsum expression $e$ over schema $\Sch$ such that $\Sch(e)=(\alpha,\beta)$ with $\alpha\neq 1\neq\beta$, there exists an $\mathsf{RA}_{K}^+$  expression $\Phi(e)$ over relational schema $\text{Rel}(\Sch)$ such that $\text{Rel}(\Sch)(\Phi(e))=\{\row_\alpha,\row_\beta\}$ and 
	such that, for any instance $\I$ over~$\Sch$,
	$$
	\sem{e}{\I}_{i,j}=\ssem{\Phi(e)}{\text{Rel}(\I)}(t)
	$$
	for tuple $t(\mathrm{row}_\alpha)=i$ and $t(\mathrm{col}_\beta)=j$. Similarly for $e$ when $e$ has schema $\Sch(e)=(\alpha,1)$, $\Sch(e)=(1,\beta)$ or $\Sch(e)=(1,1)$, then $\Phi(e)$ has schema $\text{Rel}(\Sch)(\Phi(e))=\{\mathrm{row}_\alpha\}$,
	$\text{Rel}(\Sch)(\Phi(e))=\{\mathrm{col}_\alpha\}$, or
	$\text{Rel}(\Sch)(\Phi(e))=\{\}$, respectively.
\end{proposition}

% !TeX spellcheck = en_US
%!TEX root = ../main.tex

\begin{proof}
We start from a matrix schema $\Sch=(\Mnam,\size)$, where $\Mnam\subset \Mvar$ is a finite set of matrix variables, 
and $\size: \Mvar \mapsto \DD\times \DD$ is a function that maps each matrix variable to a pair of size symbols. 
On the relational side we have for each size symbol $\alpha\in\DD\setminus\{1\}$, attributes $\alpha$, $\row_\alpha$, 
and $\col_\alpha$ in $\att$. We also reserve some special attributes $\gamma_1,\gamma_2,\ldots$ whose role will become clear shortly.
For each $V\in\Mnam$ and $\alpha \in \DD$ we denote
by $R_V$ and $R_\alpha$ its corresponding relation name, respectively. 

Then, given $\Sch$ we define the relational 
schema $\text{Rel}(\Sch)$ such that $\fdom(\text{Rel}(\Sch)) =  \{R_\alpha \mid \alpha\in\DD\} \cup \{R_V \mid V \in \Mnam\}$

where $\text{Rel}(\Sch)(R_\alpha) = \{\alpha\}$ and for all $V\in\Mnam$:
\[
\text{Rel}(\Sch)(R_V) = \begin{cases}
\lbrace\row_\alpha,\col_\beta \rbrace & \text{ if $ \size(V)=(\alpha,\beta)$} \\
\lbrace\row_\alpha \rbrace & \text{ if $ \size(V)=(\alpha,1)$} \\
\lbrace\col_\beta \rbrace  &
\text{ if $ \size(V)=(1,\beta)$} \\
\lbrace\rbrace & \text{ if $\size(V)=(1,1)$}.
\end{cases}
\]

Next, for a matrix instance $\I = (\dom,\conc)$ over $\Sch$,
let $V\in\Mnam$ with $\size(V)=(\alpha,\beta)$ and let $\conc(V)$ be its corresponding $K$-matrix of dimension $\dom(\alpha)\times \dom(\beta)$.
The $K$-instance in $\mathsf{RA}_{K}^+$ according to $\I$ is $\text{Rel}(\I)$ with data domain $\ddom = \mathbb{N} \setminus \{0\}$. For each $V\in\Mnam$ we define 
$R_V^{\text{Rel}(\I)}(t):=\conc(V)_{ij}$ whenever $t(\row_\alpha) = i \leq \dom(\alpha)$ and $t(\col_\beta) = j \leq \dom(\beta)$, and $\kzero$ otherwise. 
Also, for each $\alpha \in \DD$ we define $R_\alpha^{\text{Rel}(\I)}(t):=\kone$ whenever $t(\alpha) \leq \dom(\alpha)$, and $\kzero$ otherwise.
If $\size(V)=(\alpha,1)$ then $R_V^{\text{Rel}(\I)}(t):=\conc(V)_{i1}$ whenever $t(\row_\alpha) = i \leq \dom(\alpha)$ and $\kzero$ otherwise.
Similarly, if $\size(V)=(1,\beta)$ then $R_V^{\text{Rel}(\I)}(t):=\conc(V)_{1j}$ whenever $t(\col_\beta) = j \leq \dom(\beta)$ and $\kzero$ otherwise.
If $\size(V)=(1,1)$ then $R_V^{\text{Rel}(\I)}(()):=\conc(V)_{11}$.


Let $e$ be a \langsum expression. In the following we need to distinguish between matrix variables $v$
that occur in $e$ as part of a sub-expression $\ssum v. (\cdot)$, i.e., those variables that will be used to iterate over by means of canonical vectors, and those that are not. To make this distinction clear, we use $v_1,v_2,\ldots$ for those ``iterator'' variables, and capital $V$ for the other variables occurring in $e$. For simplicity, we assume that each occurrence of $\ssum$ has its own iterator variable associated with it. 

We define free (iterator) variables, as follows.
$\mathsf{free}(V):=\emptyset$, $\mathsf{free}(v):=\{v\}$, $\mathsf{free}(e^T):=\mathsf{free}(e)$, $\mathsf{free}(e_1+e_2):=\mathsf{free}(e_1)\cup \mathsf{free}(e_2)$, $\mathsf{free}(e_1\cdot e_2):=\mathsf{free}(e_1)\cup \mathsf{free}(e_2)$,
 $\mathsf{free}(f_\odot(e_1,\ldots,e_k)):=\mathsf{free}(e_1)\cup\cdots \cup \mathsf{free}(e_k)$, and $\mathsf{free}(e=\ssum V. e_1)=\mathsf{free}(e_1)\setminus\{v\}$. We will explicitly denote the free variables in an expression $e$ by writing $e(v_1,\ldots,v_k)$.

We now use the following induction hypotheses:
\begin{itemize}
	\item If $e(v_1,\ldots,v_k)$ is of type $(\alpha,\beta)$ then there exists a
	\rak expression $\arae$ such that $\text{Rel}(\Sch)(\arae(e))=\{\row_\alpha,\col_\beta,\gamma_1,\ldots,\gamma_k\}$
	and such that 
	$$
	\ssem{\arae(e)}{\text{Rel}(\I)}(t)=\sem{e}{\I[v_1\gets b_{i_1},\ldots,v_k\gets b_{i_k}]}_{i,j}
	$$
	for tuple $t(\mathrm{row}_\alpha)=i$, $t(\mathrm{col}_\beta)=j$ and $t(\gamma_s)=i_s$ for $s=1,\ldots, k$.
	\item If $e(v_1,\ldots,v_k)$ is of type $(\alpha,1)$ then there exists a
	\rak expression $\arae$ such that $\text{Rel}(\Sch)(\arae(e))=\{\row_\alpha,\gamma_1,\ldots,\gamma_k\}$
	and such that 
	$$
	\ssem{\arae(e)}{\text{Rel}(\I)}(t)=\sem{e}{\I[v_1\gets b_{i_1},\ldots,v_k\gets b_{i_k}]}_{i,1}
	$$
	for tuple $t(\mathrm{row}_\alpha)=i$,  and $t(\gamma_s)=i_s$ for $s=1,\ldots, k$.
	And similarly for when $e$ is type $(1,\beta)$.
	\item If $e(v_1,\ldots,v_k)$ is of type $(1,1)$ then there exists a
	\rak expression $\arae$ such that $\text{Rel}(\Sch)(\arae(e))=\{\gamma_1,\ldots,\gamma_k\}$
	and such that 
	$$
	\ssem{\arae(e)}{\text{Rel}(\I)}(t)=\sem{e}{\I[v_1\gets b_{i_1},\ldots,v_k\gets b_{i_k}]}_{1,1}
	$$
	for tuple $t(\gamma_s)=i_s$ for $s=1,\ldots, k$.
\end{itemize}
Clearly, this suffices to show the proposition since we there consider expressions $e$ for which $\mathsf{free}(e)=\emptyset$, in which case the above statements reduce to the one given in the proposition.


The proof is by induction on the structure of \langsum expressions. In line with the simplifications in Section~\ref{sec:queries:simp}, it suffices to consider pointwise function application with $f_\odot$ instead of scalar multiplication. (We also note that we can express the one-vector operator in \langsum, so scalar multiplication can be expressed using $f_\odot$ in \langsum).

Let $e$ be a \langsum expression.
\begin{itemize}
  \item If $e=V$ then $\arae (e):=R_V$.
  \item If $e=v_p$ then $\arae (e):=\sigma_{\lbrace \row_\alpha,\gamma_p\rbrace}\bigl(\rho_{\row_\alpha\to \alpha}(R_\alpha)\bowtie \rho_{\gamma_p\to \alpha}(R_\alpha)\bigr)$ when  $v_p$ is of type $(\alpha,1)$. It is here that we introduce the attribute $\gamma_p$ associated with iterator variable $v_p$.
 We note that 
$$ \ssem{\arae(v_p)}{\text{Rel}(\I)}(t)=\sem{v_p}{\I[v_p\gets b_{j}]}_{i,1}=(b_j)_{i,1}
$$
for $t(\mathrm{row}_\alpha)=i$ and $t[\gamma_p]=j$. Indeed, $(b_j)_{i,1}=\kone$ if $j=i$
and this holds when $t(\mathrm{row}_\alpha)=t[\gamma_p]=j$, and $(b_j)_{i,1}=\kzero$ if $j\neq i$
and this also holds when $t(\mathrm{row}_\alpha)\neq t[\gamma_p]=j$.

  \item If $e(v_1,\ldots,v_k)=(e_1(v_1,\ldots,v_k))^T$ with $\Sch (e_1)=(\alpha, \beta)$ then \[
\arae(e) :=
\begin{cases}
\rho_{\mathrm{row}_\alpha \to \mathrm{col}_\alpha,\mathrm{col}_\beta \to \mathrm{row}_\beta}\bigl(\arae(e_1)\bigr) & \text{if } \alpha \neq 1 \neq \beta; \cr
\rho_{\mathrm{row}_\alpha \to \mathrm{col}_\alpha}\bigl(\arae(e_1)\bigr) & \text{if } \alpha \neq 1 = \beta; \cr
\rho_{\mathrm{col}_\beta \to \mathrm{row}_\beta}\bigl(\arae(e_1)\bigr) & \text{if } \alpha = 1 \neq \beta; \cr
\arae(e_1) & \text{if } \alpha = 1 = \beta.
\end{cases}
\]
\floris{There is an issue here since $e_1$ and $e_2$ can have different free iterator variables. I think we can ensure that both have the same by introducing them somehow, or alternative, since all operators are linear, by pushing $+$ all the way down? Not sure.}
\item If $e=e_1(v_1,\ldots,v_k)+e_2(v_1,\ldots,v_k)$ with $\Sch (e_1)=\Sch (e_2)=(\alpha, \beta)$ then $\arae (e):=\arae (e_1)\cup \arae (e_2)$. We assume here that $e_1$ and $e_2$ have the same free variables. This is without loss of generality. Indeed, as an example, suppose that we have $e_1(v_1,v_2)$
and $e_2(v_2,v_3)$. Then, we can replace $e_1$ by  $e_1(v_1,v_2,v_3)=(v_3^T\cdot v_3)\times e_1(v_1,v_2)$
and similarly, $e_2$ by $e_2(v_1,v_2,v_3)=(v_1^T\cdot v_1)\times e_2(v_2,v_3)$, where in addition we replace scalar multiplication with its simulation using $f_{\odot}$ and the ones vector, as mentioned earlier. 

  \item If $e=f_\odot(e_1,\ldots, e_k)$ with $\Sch(e_i)=\Sch(e_j)$ for all $i,j\in[1,k]$, then $\arae(e):=\arae(e_1)\Join \cdots \Join\arae(e_k)$.

  \item If $e=e_1\cdot e_2$ with $\Sch (e_1)=(\alpha, \gamma)$ and $\Sch (e_2)=(\gamma, \beta)$, we have two cases. If $\gamma = 1$ then $\arae (e):=\arae (e_1)\Join \arae (e_2)$.
If $\gamma\neq 1$ then
$$
\arae (e) := \pi_{\lbrace \row_{\alpha},\col_{\beta}, \gamma_1,\ldots,\gamma_k \rbrace}\left(\rho_{C\to \col_\gamma}(\arae (e_1))\Join \rho_{C\to \row_\gamma}(\arae (e_2)) \right),
$$
when $\text{Rel}(\Sch)(\arae(e_1))=\{\row_\alpha,\col_\gamma,\gamma_1',\ldots,\gamma_{\ell}'\}$,
$\text{Rel}(\Sch)(\arae(e_2))=\{\row_\gamma,\col_\beta,\gamma_1'',\ldots,\gamma_{\ell}''\}$ and $\{\gamma_1,\ldots,\gamma_k\}=\{\gamma_1',\ldots,\gamma_k',\gamma_1'',\ldots,\gamma_m''\}$.

  \item If $e(v_1,\ldots,v_{p-1},v_{p+1},\ldots,v_k)=\ssum v_p. e_1(v_1,\ldots,v_k)$ where $\Sch(e_1)=(\alpha,\beta)$ and $\Sch(V)=(\gamma,1)$. Then we do 
  $$
  \arae (e):=\pi_{\text{Rel}(\Sch)(\arae(e_1))\setminus\{\gamma_p\}} \arae (e_1).
  $$
 Indeed, by induction we know that 
 $$
\ssem{\arae(e_1)}{\text{Rel}(\I)}(t)=\sem{e}{\I[v_1\gets b_{i_1},\ldots,v_k\gets b_{i_k}]}_{i,j}
$$
for tuple $t(\mathrm{row}_\alpha)=i$, $t(\mathrm{col}_\beta)=j$ and $t(\gamma_s)=i_s$ for $s=1,\ldots, k$.
Hence, for $t(\mathrm{row}_\alpha)=i$, $t(\mathrm{col}_\beta)=j$ and $t(\gamma_s)=i_s$ for $s=1,\ldots, k$ and $s\neq p$,
$$
\ssem{\arae(e_1)}{\text{Rel}(\I)}(t):=\bigksum_{i_p=1,\ldots,\dom(\gamma)} \sem{e_1}{\I[v_1\gets b_{i_1},\ldots,v_k\gets b_{i_k}]}_{i,j},$$
which precisely corresponds to 
$$
\sem{\ssum v_p. e_1(v_1,\ldots,v_k)}{\I[v_1\gets b_{i_1},\ldots,v_{p-1}\gets b_{p-1},v_{p+1}\gets b_{p+1},\ldots,v_k\gets b_k]}_{i,j}.
$$

\end{itemize}
All other cases, when expressions have type $(\alpha,1)$, $(1,\beta)$ or $(1,1)$ can be dealt with in a similar way.
\end{proof}


Now we move to the other direction.
To translate $\mathsf{RA}_{K}^+$  into \langsum, we must restrict our comparison to $\mathsf{RA}_{K}^+$  over $K$-relations with at most two attributes. Given that linear algebra works over vectors and matrices, it is reasonable to restrict to unary or binary relations as input. Note that this is only a restriction on the input relations and not on intermediate relations, namely, expressions can create relation signatures of arbitrary size from the binary input relations. 
Thus, from now on we say that a relational schema $\cR$ is \textit{binary} if $|R| \leq 2$ for every $R \in \cR$. We also make the assumption that there is an arbitrary order, denoted by $<$, on the attributes in $\att$. 
This is to identify which attributes correspond to rows and columns when moving to matrices. 
Then, given that relations will be  either unary or binary and there is an order on the attributes, we write $t = (d_1)$ or $t = (d_1,d_2)$ to denote a tuple over a unary or binary relation $R$, respectively, where $d_1$ and $d_2$ is the value of the first and second attribute with respect to $<$.

Consider a binary relational schema $\cR$. With each $R\in \cR$ we associate a matrix variable $V_R$ such that, if $R$ is a binary relational signature, then $V_R$ represents a (square) matrix, and, if not (i.e. $R$ is unary), then $V_R$ represents a vector. Formally, we fix a symbol $\alpha \in \DD \setminus \{1\}$. Let $\text{Mat}(\cR)$ denote the \lang \ schema
$(\Mnam_\cR,\size_\cR)$ such that $\Mnam_\cR = \{ V_R \mid R \in \cR\}$ and $\size_\cR(V_R) = (\alpha, \alpha)$ whenever $|R| = 2$, and $\size_\cR(V_R) = (\alpha, 1)$ whenever $|R|=1$. 
Take now a $K$-instance $\cJ$ of $\cR$ and suppose that $\adom(\cJ) = \{d_1, \ldots, d_n\}$ is the active domain of $\cJ$ (the order $d_1, \ldots, d_n$ over $\adom(\cJ)$ is arbitrary). Then we define the matrix instance $\text{Mat}(\cJ) = (\dom_\cJ,\conc_\cJ)$ such that $\dom_\cJ(\alpha) = n$, $\conc_\cJ(V_R)_{i,j} = R^{\cJ}((d_i, d_j))$ whenever $|R|=2$,  $\conc_\cJ(V_R)_{i} = R^{\cJ}((d_i))$ whenever $|R|=1$, and $\conc_\cJ(V_R)_{1,1} = R^{\cJ}$ whenever $|R|=0$. 
Note that, although each $K$-relation can have a different active domain, we encode them as square matrices by considering the active domain of the $K$-instance. %By again using an inductive proof on the structure of $\mathsf{RA}_{K}^+$ expressions, we obtain the following result.
\begin{proposition}\label{prop:ara_to_sum} 
	Let $\cR$ be a binary relational schema. For each $\mathsf{RA}_{K}^+$  expression $\arae$ over $\cR$  such that $|\cR(\arae)| = 2$, there exists a \langsum  expression $\Psi(\arae)$ over \lang \ schema $\text{Mat}(\cR)$ such that for any $K$-instance $\cJ$ with $\adom(\cJ) = \{d_1, \ldots, d_n\}$ over $\cR$,
	$$
	\ssem{\arae}{\cJ}((d_i, d_j))=\sem{\Psi(\arae)}{\text{Mat}(\cJ)}_{i,j}.
	$$
	Similarly for when $|\cR(\arae)| = 1$, or $|\cR(\arae)| = 0$ respectively.
\end{proposition}

% !TeX spellcheck = en_US
%!TEX root = ../main.tex

Let $\cR$ be binary relational schema. For each $R\in \cR$ we associate a matrix variable 
$V_R$ such that, if $R$ is a binary relational signature, then $V_R$ represents a (square) matrix, 
if $R$ is unary, then $V_R$ represents a vector and if $|R|=0$ then $V_R$ represents a constant. Formally, 
fix a symbol $\alpha \in \DD \setminus \{1\}$. Let $\text{Mat}(\cR)$ denote the \lang \ schema
$(\Mnam_\cR,\size_\cR)$ such that $\Mnam_\cR = \{ V_R \mid R \in \cR\}$ and $\size_\cR(V_R) = (\alpha, \alpha)$ 
whenever $|R| = 2$, $\size_\cR(V_R) = (\alpha, 1)$ whenever $|R|=1$ and $\size_\cR(V_R) = (1, 1)$ whenever $|R|=0$. 
Let $\cJ$ be the $K$-instance of $\cR$ and suppose that $\adom(\cJ) = \{d_1, \ldots, d_n\}$ is 
the active domain (with arbitrary order) of $\cJ$. 
Define the matrix instance $\text{Mat}(\cJ) = (\dom_\cJ,\conc_\cJ)$ such 
that $\dom_\cJ(\alpha) = n$, $\conc_\cJ(V_R)_{i,j} = R^{\cJ}((d_i, d_j))$ whenever $|R|=2$, $\conc_\cJ(V_R)_{i} = R^{\cJ}((d_i))$ 
whenever $|R|=1$, 
\floris{This case relates to nullary relations. What does $R^{\cJ}$ mean?}
and $\conc_\cJ(V_R)_{1,1} = R^{\cJ}$ whenever $|R|=0$. 
Note that we consider the active domain of the whole $K$-instance.

\newcommand{\earae}{e_{\arae}}

We next translate \rak expressions in to \langsum expressions over an extended schema. More specifically, for each attribute $A \in \att$ we define a vector variable $v_A$ of type $(\alpha,1)$. Then for each \rak expression $\arae$ with attributes $A_1, \ldots, A_k$ we define a \langsum expression $\earae(v_{A_1}, \ldots, v_{A_k})$ of type $(1,1)$ such that the following inductive hypothesis holds:
$$
\sem{\earae}{\text{Mat}(\cJ)[v_{A_1} \gets b_{i_1},\ldots, v_{A_k} \gets b_{i_k}]} = 
\ssem{\arae}{\cJ}(t) \ \ \ \ \ \ \  (*)
$$
where $t(A_s)=i_s$ for $s=1,\ldots, k$. The proof of this claim follows by induction on the structure of expressions:
\begin{itemize} \itemsep3mm
	\item If $\arae=R$, then $\earae:=v_{A_1}^T \cdot V_R \cdot v_{A_2}$ if $\mathcal{R}(R)=\{A_1,A_2\}$ with $A_1<A_2$; 
	$\earae:=V_R^T \cdot v_A$ if $\mathcal{R}(R)=\{A\}$; and 
	$\earae:=V_R$ if $\mathcal{R}(R)=\{\}$.
	\item If $\arae=\arae_1\cup \arae_2$ then
	$\earae:=e_{\arae_1} + e_{\arae_2}$.
	\item If $\arae=\pi_{Y}(\arae_1)$ for $Y\subseteq \mathcal{R}(\arae_1)$ and $\{B_1, \ldots, B_l\} = \mathcal{R}(\arae_1) \setminus Y$ then
	$$
	\earae:= \Sigma v_{B_1}. \ \Sigma v_{B_2}. \ \ldots \Sigma v_{B_l}. \ e_{\arae_1}
	$$
	\item If $\arae=\sigma_{Y}(\arae_1)$ with $Y\subseteq\mathcal{R}(\arae_1)$ then
	$$
	\earae:=e_{\arae_1}\cdot \prod_{A,B\in Y} (v_{A}^T \cdot v_{B}).
	$$
	Here $\Pi$ is the matrix multiplication of expressions of type $(1,1)$.
	\item If $\arae=\rho_{X\mapsto Y}(\arae_1)$ then
	$$\earae:=e_{\arae_1}[v_B\gets v_A\mid A\in X, B\in Y, A\mapsto B].$$
	In other words, we rename variable $v_B$ with variable $v_B$ in all the expression $e_{\arae_1}$. 
	\item If $\arae=\arae_1\bowtie \arae_2$ then
	$\earae:=e_{\arae_1} \cdot e_{\arae_1}$ where the product is over expression of type $(1,1)$.
\end{itemize}
One can check, by induction over the construction, that the inductive hypothesis $(*)$ holds in each case.
Now we can prove proposition \ref{prop:ara_to_sum}.

\begin{proof}
As a consequence of the previous discussion above, when $\arae$ is a \rak expression 
such that $\mathcal{R}(\arae)=\{A_1,A_2\}$ with $A_1<A_2$ then we define
$$
\Psi(\arae) \ = \ \Sigma v_{A_1}. \ \Sigma v_{A_2}. \ \earae \cdot (v_{A_1} \cdot v_{A_2}^T). 
$$
Instead, when $\mathcal{R}(\arae)=\{A\}$ we have
$$
\Psi(\arae) \ = \ \Sigma v_{A}. \  (v_{A} \cdot \earae). 
$$
And when $\mathcal{R}(\arae)=\{\}$ we have
$$
\Psi(\arae) \ = \ \earae.
$$
By using the inductive hypothesis $(*)$ one can check that $\Psi(\arae)$ works in each case as expected. 
\end{proof}
 

It is important to remark that the expression $\arae$ of the previous result can have intermediate expressions that are not necessarily binary, given that the proposition only restricts that the input relation and the schema of $\arae$ must have arity at most two. We recall from~\cite{brijder2019matrices} that \lang\ corresponds to $\mathsf{RA}_{K}^+$ where intermediate expressions are at most ternary, and this underlies, e.g. the inability of \lang\ to check for $4$-cliques. In \langsum, we can deal with intermediate relations of arbitrary arity. In fact, each new attribute can be seen to correspond to an application of the $\Sigma$ operator. For example, in the $4$-clique expression, four $\Sigma$ operators are needed, in analogy to how
$4$-clique is expressed in $\mathsf{RA}_{K}^+$.

Given the previous two propositions, we derive the following conclusion which is the first characterization of relational algebra with a (sub)-fragment of linear algebra.
\begin{corollary}
	\langsum and $\mathsf{RA}_{K}^+$  over binary relational schemas are equally expressive. 
\end{corollary}

As a direct consequence, we have that \langsum cannot compute matrix inversion. Indeed, using similar arguments as
in~\cite{matlang-journal}, e.g. by embedding $\mathsf{RA}_{K}^+$  in (infinitary) first-order logic with counting and by leveraging its locality, one can show that \langsum cannot compute the transitive closure of an adjacency matrix. By contrast, the transitive closure can be expressed by means of matrix inversion~\cite{matlang-journal}. We also note that the evaluation of the $\Sigma$ operator is
independent of the order in which the canonical vectors are considered. This is because $\oplus$ is commutative.
Hence, \langsum cannot express the order predicates mentioned in Section~\ref{sec:formatlang}.

We end by noting how \langsum can be compared to Functional Aggregate Queries \cite{FAQAI,FAQ}. Indeed, \langsum can be easily seen to correspond to a restricted version of $\mathsf{FAQ}\text{s}$, where only a single semiring and matrix inputs are allowed. Moreover, the summation in \langsum corresponds to the elimination of a variable in $\mathsf{FAQ}\text{s}$. One can thus argue that  variable elimination techniques of $\mathsf{FAQ}\text{s}$ \cite{FAQ} can be used to efficiently evaluate \langsum expressions. We leave a detailed analysis of this connection for future work.




\subsection{Hadamard product and weighted logics}\label{subsec:langprod}
Similarly to using sum, we can use other operations to update $X$ in the for-loop. The next natural choice is to consider products of matrices. In contrast to matrix sum, we have two options: either we can choose to use matrix product or to use the pointwise matrix product, also called the Hadamard product. We treat  matrix product in the next subsection and first explain here the connection of sum and Hadamard product operators to weighted logics~\cite{DrosteG05}.

For the rest of this section, we fix a semiring $(K, \ksum, \kprod, \kzero, \kone)$. The \textit{Hadamard product} over $K$-matrices is defined as the pointwise application of $\kprod$ between two matrices of the same size. Formally, we define the expression $e \hadprod e'$ where $e, e'$ are expressions  over
some schema $\Sch=(\Mnam,\size)$ satisfying  $\ttype(e) = \ttype(e')$. Then the semantics of $e \hadprod e'$ is the pointwise application of $\kprod$, namely, $\sem{e \hadprod e'}{\I}_{ij} = \sem{e}{\I}_{ij} \kprod \sem{e'}{\I}_{ij}$ for any instance $\I$ of $\cS$. This enables us to define, similar as for the operator $\Sigma v$, the  pointwise-product quantifier $\qhadprod v$ as follows:
$$
\qhadprod v. \  e := \ffor{v}{X\!=\!e_{\kone}}{X \circ e}.
$$
where $e_\kone$ is the \langfor expression for the matrix with the same type as $X$ and all entries equal to the $\kone$-element of $K$ (i.e., we need to initialize $X$ accordingly with the $\kprod$-operator).
We call \langprod  (for ``sum-product \lang''`) the subfragment of \langfor that consists of \langsum \ extended with $\qhadprod v$.

\begin{example}
	Similar to the trace of a matrix, a useful function in linear algebra is to compute the product of the values on the diagonal. 
	Using the $\qhadprod v$ operator, this can be easily expressed:
\begin{equation*}
	 e_{\mathsf{dp}}(V) := \qhadprod v. \ v^T\cdot V \cdot v. \tag*{$\qed$}
\end{equation*}
\end{example}

Clearly, the inclusion of this new operator extends the expressive power to \langsum. For example,  $\sem{e_{\mathsf{dp}}}{\I}$ can be an exponentially large number in the dimension $n$ of the input.
By contrast, one can easily show that all expressions in \langsum can only return numbers polynomial in  $n$. That is, \langprod is more expressive than \langsum and $\mathsf{RA}_{K}^+$. 

To measure the expressive power of \langprod, we use weighted logics~\cite{DrosteG05} (WL) as a yardstick. Weighted logics extend monadic second-order logic from the boolean semiring to any semiring~$K$. Furthermore, it has been used extensively to characterize the expressive power of weighted automata in terms of logic~\cite{droste2009handbook}. We use here the first-order subfragment of weighted logics to suit our purpose and, moreover, we extend its semantics over weighted structures (similar as in~\cite{GradelV17}).

A \textit{relational vocabulary} $\Gamma$ is a finite collection of relation symbols such that each $R \in \Gamma$ has an associated \textit{arity}, denoted by $\arity(R)$.
A \textit{$K$-weighted structure over $\Gamma$} (or just structure) is a pair $\cA = (A, \{R^\cA\}_{R \in \Gamma})$ such that $A$ is a non-empty finite set (i.e. the domain) and, for each $R \in \Gamma$, $R^\cA: A^{\arity(R)} \rightarrow K$ is a function that associates to each \textit{tuple} in $A^{\arity(R)}$ a \textit{weight} in $K$.

Let $X$ be a set of first-order variables. A \textit{$K$-weighted logic (WL) formula} $\varphi$ over $\Gamma$ is defined by the following syntax:
$$
\begin{array}{rcl}
\varphi & := & x = y \ \mid \ R(\bar{x}) \ \mid \ \varphi \ksum \varphi \ \mid \ \varphi \kprod \varphi \ \mid \ \Sigma x. \varphi \ \mid \ \Pi x. \varphi
\end{array}
$$ 
where $x, y \in X$, $R \in \Gamma$, and $\bar{x} = x_1, \ldots, x_k$ is a sequence of variables in $X$ such that $k=\arity(R)$. As usual, we say that $x$ is a free variable of $\varphi$, if $x$ is not below $\Sigma x$ or $\Pi x$ quantifiers (e.g. $x$ is free in $\Sigma y. R(x,y)$ but $y$ is not). 
Given that $K$ is fixed, from now on we talk about structures and formulas without mentioning $K$ explicitly.  

An \textit{assignment} $\sigma$ over a structure $\cA = (A, \{R^\cA\}_{R \in \Gamma})$ is a function $\sigma: X \rightarrow A$. Given $x \in X$ and $a \in A$, we denote by $\sigma[x \mapsto a]$ a new assignment such that $\sigma[x \mapsto a](y) = a$ whenever $x = y$ and $\sigma[x \mapsto a](y) = \sigma(y)$ otherwise. For $\bar{x} = x_1, \ldots, x_k$,  we write $\sigma(\bar{x})$ to say $\sigma(x_1),\ldots, \sigma(x_k)$. Given a structure $\cA = (A, \{R^\cA\}_{R \in \Gamma})$ and an assignment $\sigma$, we define the \textit{semantics} $\ssem{\varphi}{\cA}(\sigma)$ of $\varphi$:
$$
\begin{array}{ll}
\text{if $\varphi= x = y$, then} & \ssem{\varphi}{\cA}(\sigma):= 
\left\{
\begin{array}{ll}
\kone & \text{if $\sigma(x) = \sigma(y)$} \\
\kzero & \text{otherwise}
\end{array}
\right. \\
\text{if $\varphi= R(\bar{x})$, then} & \ssem{\varphi}{\cA}(\sigma):= R^\cA(\sigma(\bar{x})) \\
\text{if $\varphi= \varphi_1 \ksum \varphi_2$, then} & \ssem{\varphi}{\cA}(\sigma):= \ssem{\varphi_1}{\cA}(\sigma) \ksum \ssem{\varphi_2}{\cA}(\sigma)  \\
\text{if $\varphi= \varphi_1 \kprod \varphi_2$, then} & \ssem{\varphi}{\cA}(\sigma):= \ssem{\varphi_1}{\cA}(\sigma) \kprod \ssem{\varphi_2}{\cA}(\sigma)  \\
\text{if $\varphi= \Sigma x. \, \varphi'$, then} & \ssem{\varphi}{\cA}(\sigma):=  \bigksum_{a \in A} \ssem{\varphi'}{\cA}(\sigma[x \mapsto a]) \\
\text{if $\varphi= \Pi x. \, \varphi'$, then} & \ssem{\varphi}{\cA}(\sigma):=  \bigkprod_{a \in A} \ssem{\varphi'}{\cA}(\sigma[x \mapsto a])
\end{array}
$$
When $\varphi$ contains no free variables, we omit $\sigma$ and write $\ssem{\varphi}{\cA}$ instead of $\ssem{\varphi}{\cA}(\sigma)$.

For comparing the expressive power of \langprod with WL, we have to show how to encode \lang\ instances into structures and vice versa. For this, we make two assumptions to align both formalisms: (1) we restrict structures to relation symbols of arity at most two and (2) we restrict instances to square matrices. The first assumption is for the same reasons as when comparing \langsum with $\mathsf{RA}_K^+$, and the second assumption is to have a crisp translation between both languages. Indeed, understanding the relation of \langprod with WL for non-square matrices is slightly more complicated and we leave this for future work. 

Let $\Sch=(\Mnam,\size)$ be a schema of square matrices, that is, there exists an $\alpha$ such that $\size(V) \in \{1, \alpha\} \times \{1,\alpha\}$ for every $V \in \Mnam$.
Define the relational vocabulary $\text{WL}(\Sch):= \{R_V \mid V \in \Mnam\}$ such that $\arity(R_V) = 2$ if $\size(V) = (\alpha, \alpha)$, $\arity(R_V) = 1$ if $\size(V) \in \{(\alpha,1), (1,\alpha)\}$, and $\arity(R_V) = 0$ otherwise.
Then given a matrix instance $\I = (\dom,\conc)$ over $\Sch$ define the structure $\text{WL}(\I):= (\{1, \ldots, n\}, \{R_V^{\I}\}_{V \in \Mnam} )$ such that $\dom(\alpha) = n$ and $R_V^{\I}(i, j):= \conc(V)_{i,j}$ if $\size(V) = (\alpha, \alpha)$, $R_V^{\I}(i):= \conc(V)_{i}$ if $\size(V) \in \{(\alpha,1), (1,\alpha)\}$, and $R_V^{\I}:= \conc(V)$ if $\size(V) = (1,1)$.

To encode weighted structures into matrices and vectors, the story is similar as for $\mathsf{RA}_K^+$. Let $\Gamma$ be a relational vocabulary where $\arity(R) \leq 2$. 
Define $\text{Mat}(\Gamma):= (\Mnam_\Gamma,\size_\Gamma)$ such that $\Mnam_\Gamma:= \{ V_{R} \mid R \in \Gamma\}$ and $\size_\Gamma(V_{R})$ is equal to $(\alpha, \alpha), (\alpha, 1)$, or $(1,1)$ if $\arity(R)=2$, $\arity(R)=1$, or $\arity(R)=0$, respectively, for some $\alpha \in \DD$. Similarly, let $\cA= (A, \{R^{\cA}\}_{R \in \Gamma})$ be a structure with $A = \{a_1, \ldots, a_n\}$, ordered arbitrarily.
Then we define the matrix instance $\text{Mat}(\cA) = (\dom,\conc)$ such that $\dom(\alpha) = n$, $\conc(V_{R})_{i,j} = R^{\cA}(a_i, a_j)$ if $\arity(R)=2$, $\conc(V_{R})_{i} = R^{\cA}(a_i)$ if $\arity(R)=1$, and $\conc(V_{R}) = R^{\cA}$ otherwise.

Let $\Sch$ be a \lang\ schema of square matrices and $\Gamma$ a relational vocabulary of relational symbols of arity at most $2$. We can show the equivalence of \langprod and WL as follows. 
\begin{proposition} \label{prop:wl}
Weighted logics over $\Gamma$ and \langprod over $\Sch$ have the same expressive power. More specifically,
\begin{itemize}
	\item for each \langprod expression $e$ over $\Sch$ such that $\Sch(e)=(1,1)$, there exists a WL-formula $\Phi(e)$ over $\text{WL}(\Sch)$ such that for every instance $\I$ of~$\Sch$:
	$$
	\sem{e}{\I} \ = \ \ssem{\Phi(e)}{\text{WL}(\I)}.
	$$
	\item for each WL-formula $\varphi$ over $\Gamma$ without free variables, there exists an \langprod expression $\Psi(\varphi)$ such that for any structure $\cA$ over~$\text{Mat}(\Gamma)$:
	$$
	\ssem{\varphi}{\cA} \ = \ \sem{\Psi(\varphi)}{\text{Mat}(\cA)}.
	$$
\end{itemize}	
\end{proposition}

% !TeX spellcheck = en_US
%!TEX root = ../../main.tex
\begin{proof}
Both directions are proved by induction on the structure of expressions.

\smallskip

%First, let $\Sch=(\Mnam,\size)$ be a schema of square matrices, that is, there exists an $\alpha$ such 
%that $\size(V) \in \{1, \alpha\} \times \{1,\alpha\}$ for every $V \in \Mnam$.
%Define the relational vocabulary $\text{WL}(\Sch) = \{R_V \mid V \in \Mnam\}$ such that $\arity(R_V) = 2$ 
%if $\size(V) = (\alpha, \alpha)$, $\arity(R_V) = 1$ if $\size(V) \in \{(\alpha,1), (1,\alpha)\}$, and 
%$\arity(R_V) = 0$ otherwise.
%Then given a matrix instance $\I = (\dom,\conc)$ over $\Sch$ with  $\dom(\alpha) = n$ define the structure 
%$\text{WL}(\I) = (\{1, \ldots, n\}, \{R_V^{\I}\} )$ such that 
%$R_V^{\I}(i, j) = \conc(V)_{i,j}$ if $\size(V) = (\alpha, \alpha)$, $R_V^{\I}(i) = \conc(V)_{i}$ 
%if $\size(V) \in \{(\alpha,1), (1,\alpha)\}$, and $R_V^{\I} = \conc(V)$ if $\size(V) = (1,1)$.

\noindent \emph{From \langprod to Weighted Logics.} 
Similar to the proof of Proposition~\ref{prop:sum_to_ara}, for each expression $e(v_1, \ldots, v_k)$ of type $(\alpha, \alpha)$ we must encode in WL the $\alpha$ and the vector variables $v_1, \ldots, v_k$. For this, we use variables $x_{\alpha}^\row$, $x_{\alpha}^\col$, and $x_{v_i}$ for each variable $v_1, \ldots, v_k$. Then we use the following inductive hypothesis (similar to Proposition~\ref{prop:sum_to_ara}):

\newcommand{\varphie}{\varphi_e}
\newcommand{\xr}{x_{\alpha}^\row}
\newcommand{\xc}{x_{\alpha}^\col}

\begin{itemize}
	\item If $e(v_1,\ldots,v_k)$ is of type $(\alpha,\alpha)$ then there exists a formula $\varphie(x_{\alpha}^\row,x_{\alpha}^\col, x_{v_1}, \ldots, x_{v_k})$ such that
	$$
	\ssem{\varphie}{\text{WL}(\I)}(\sigma) \ = \ \sem{e}{\I[v_1\gets b_{i_1},\ldots,v_k\gets b_{i_k}]}_{i,j}
	$$
	for assignment $\sigma$ with $\sigma(\xr)=i$, $\sigma(\xc)=j$ and $\sigma(x_{v_s})=i_s$ for $s=1,\ldots, k$.
	
	\item If $e(v_1,\ldots,v_k)$ is of type $(\alpha,1)$ then there exists a WL formula $\varphie(x_{\alpha}^\row, x_{v_1}, \ldots, x_{v_k})$ such that
	$$
	\ssem{\varphie}{\text{WL}(\I)}(\sigma) \ = \ \sem{e}{\I[v_1\gets b_{i_1},\ldots,v_k\gets b_{i_k}]}_{i}
	$$
	for assignment $\sigma$ with $\sigma(\xr)=i$ and $\sigma(x_{v_s})=i_s$ for $s=1,\ldots, k$.
	And similarly for when $e$ is of type $(1,\alpha)$.
	
	\item If $e(v_1,\ldots,v_k)$ is of type $(1,1)$ then there exists a WL formula $\varphie( x_{v_1}, \ldots, x_{v_k})$ such that
	$$
	\ssem{\varphie}{\text{WL}(\I)}(\sigma) \ = \ \sem{e}{\I[v_1\gets b_{i_1},\ldots,v_k\gets b_{i_k}]}
	$$
	for assignment $\sigma$ with $\sigma(x_{v_s})=i_s$ for $s=1,\ldots, k$.
\end{itemize}
If we prove the previous statement we are done, because the last bullet is what we want to show when $e$ has no free vector variables. 
Then rest of the proof is to go by induction on the structure of \langprod expressions.
For a WL-formula $\varphi$ and variables $x,y$, we will write  $\varphi[x \mapsto y]$ for the formula $\varphi$ when $x$ is replaced with $y$ all over the formula (syntactically).
Let $e$ be a \langprod expression.
\begin{itemize} \itemsep3mm
  \item If $e:=V$ and $\Sch(e)= (\alpha, \alpha)$ then $\varphie:=R_V(\xr, \xc)$. Similarly, if $\Sch(e)$ is of type $(\alpha,1)$, $(1, \alpha)$, or $(1,1)$, then $\varphie:=R_V(\xr)$, $\varphie:=R_V(\xc)$, and $\varphie:=R_V$, respectively.
  
  \item If $e:=v$, for $v\in \{v_1,\ldots ,v_k\}$, and $\Sch(v)= (\alpha,1)$ then $\varphie := \xr = x_v$. Similarly, if $\Sch(v)= (1,\alpha)$ then $\varphie := \xc = x_v$.
  
  \item if $e:= e_1^T$ and $\Sch(e)=(\alpha,\alpha)$ then
  $$
  \varphie:= \varphi_{e_1}[\xr \mapsto \xc, \xc \mapsto \xr].
  $$
  Similarly, if $\Sch(e)$ is equal to $(\alpha,1)$ or $(1,\alpha)$ then $\varphie:=\varphi_{e_1}[\xr \mapsto \xc]$ and $\varphie:=\varphi_{e_1}[\xc \mapsto \xr]$, respectively.   


	\item If $e=e_1+e_2$ with $\Sch (e_1)=\Sch (e_2)$, then $\varphie:= \varphi_{e_1} \ksum \varphi_{e_2}$.
	
	\item If $e=f_\odot(e_1,\ldots, e_k)$ with $\Sch(e_i)=\Sch(e_j)$ for all $i,j\in[1,k]$, then $\varphie:= \varphi_{e_1} \kprod \varphi_{e_2} \cdots \kprod \varphi_{e_k}$.
	
	\item If $e=e_1\cdot e_2$ with $\Sch (e_1)=\Sch (e_2)=(\alpha, \alpha)$,  then $\varphie:= \Sigma y. \  \varphi_{e_1}[\xc \mapsto y] \kprod \varphi_{e_2}[\xr \mapsto y]$ where $y$ is a fresh variable not mentioned in $\varphi_{e_1}$ or $\varphi_{e_2}$. Instead, if $\Sch (e_1)= (\alpha', 1)$ and $\Sch (e_2)=(1, \alpha'')$ with $\alpha', \alpha'' \in \{\alpha, 1\}$, then $\varphie := \varphi_{e_1} \kprod \varphi_{e_2}$.
	
	\item If $e=\ssum v. e_1(v)$, then we define $\varphie := \Sigma x_{v}. \  \varphi_{e_1}(x_v)$.

  \item If $e=\qhadprod v. e_1(v)$, then $\varphie := \sprod x_{v}.\  \varphi_{e_1}(x_v)$.
\end{itemize}
From the construction it is now straightforward to check that the inductive hypothesis holds for all cases. To conclude this direction, define $\Phi(e) := \varphie$ for every expression $e$, and we are done.

\medskip
%We now encode weighted structures into matrices and vectors. Let $\Gamma$ be a relational vocabulary 
%where $\arity(R) \leq 2$. 
%Define $\text{Mat}(\Gamma) = (\Mnam_\Gamma,\size_\Gamma)$ such 
%that $\Mnam_\Gamma = \{ V_{R} \mid R \in \Gamma\}$ and $\size_\Gamma(V_{R})$ is equal to 
%$(\alpha, \alpha), (\alpha, 1)$, or $(1,1)$ if $\arity(R)=2$, $\arity(R)=1$, or $\arity(R)=0$, 
%respectively, for some $\alpha \in \DD$. Similarly, let $\cA = (A, \{R^{\cA}\}_{R \in \Gamma})$ 
%be a structure with $A = \{a_1, \ldots, a_n\}$, ordered arbitrarily.
%Then we define the matrix instance $\text{Mat}(\cA) = (\dom,\conc)$ such that $\dom(\alpha) = n$, 
%$\conc(V_{R})_{i,j} = R^{\cA}(a_i, a_j)$ if $\arity(R)=2$, $\conc(V_{R})_{i,1} = R^{\cA}(a_i)$ if $\arity(R)=1$, 
%and $\conc(V_{R})_{1,1} = R^{\cA}$ otherwise.

\newcommand{\evarphi}{e_\varphi}

\noindent \emph{From Weighted Logics to \langprod.} 
Similar to the previous direction, we have to encode the \langprod variables of a formula $\varphi$ with vector variables in the equivalent \langprod expression $\evarphi$. For this, for each \langprod variable $x$ we define a vector variable $v_x$ of type $(\alpha, 1)$. 
Take a structure $\cA = (A, \{R^{\cA}\}_{R \in \Gamma})$ with $A = \{a_1, \ldots, a_n\}$, ordered arbitrarily.
Then for each formula $\varphi(x_1, \ldots, x_k)$ we define an expression $\evarphi(v_{x_1}, \ldots, v_{x_k})$ of type $(1,1)$ such that for every assignment $\sigma$ of $x_1, \ldots, x_k$ we have:
$$
\sem{\evarphi}{\text{Mat}(\cA)[v_{x_1} \gets b_{i_1},\ldots,v_{x_1}\gets b_{i_k}]} \ = \ \ssem{\varphi}{\cA}(\sigma) 
$$
such that $\sigma(x_{s}) = a_{i_s}$ for every $s \leq k$. Note that when the formula has no free variables, the proof of the proposition is shown. Finally, we proceed by induction over the formula $\varphi$ over $\Gamma$.
\begin{itemize} \itemsep3mm
  \item If $\varphi:=x=y$, then $\evarphi:= v_x^T \cdot v_{y}$.
  \item If $\varphi:=R(x,y)$, then $\evarphi:=v_x^T \cdot V_R \cdot v_{y}$. Similarly, if $\varphi:=R(x)$ or $\varphi:=R$, then $\evarphi:= V_R^T \cdot v_{x}$  and $\evarphi:= V_R$, respectively. 
  \item If $\varphi = \varphi_1 \ksum \varphi_2$, then $\evarphi:= e_{\varphi_1} + e_{\varphi_2}$.
  \item If $\varphi = \varphi_1 \kprod \varphi_2$, then $\evarphi:= f_\odot(e_{\varphi_1},e_{\varphi_2})$.
  \item If $\varphi = \ssum x.\  \varphi_1$, then $\evarphi :=\ssum v_x.\ e_{\varphi_1}$.
  \item If $\varphi = \qhadprod x. \varphi_1$, then $\evarphi := \sprod v_x.\ e_{\varphi_1}$.
\end{itemize}
The inductive hypothesis can be proved following the above construction. To finish the proof, we define $\Psi(\varphi) := \evarphi$ and the proposition is shown.
\end{proof}


\subsection{Matrix multiplication as a quantifier}\label{subsec:langlinear}
In a similar way, we can consider a fragment in which sum and the usual product of matrices can be used
in for-loops, as defined in Section \ref{sec:formatlang}. Recall that, for an expression $e$ we define the operator:
$$
\sprod v.\,  e=\ffor {v}{X = e_{\mathsf{Id}}}{X\cdot e},
$$
where $e_{\mathsf{Id}}$ is the identity matrix and $e=e(v)$. We call \langmprod the subfragment of \langfor that consists of \langsum extended with $\sprod v$. It is readily verified that  $\qhadprod v$ can be expressed in terms of $\sprod v$.
Furthermore, by contrast to the Hadamard product, matrix multiplication is a non-commutative operator. As a consequence, one can formulate expressions that are not invariant under the order in which the canonical vectors
are processed. 

\begin{proposition}
	Every expression in \langprod can be defined in \langmprod. Moreover, there exists an expression that uses the $\sprod v$ quantifier that cannot be defined in \langprod. 
\end{proposition}

Furthermore, when allowing for the $f_{>0}$ function,  \langsum extended with $\sprod v$ suffices to compute the transitive closure.
Indeed, one can use the expression $e_{\mathsf{TC}}(V):=f_{>0}\bigl(\sprod v.\, (e_{\mathsf{Id}}+V)\bigr)$ for this purpose because
 $\sem{e_{\mathsf{TC}}}{\I}=f_{>0}\bigl((I+A)^n\bigr)$ when $\I$ assigns an $n\times n$ adjacency matrix $A$ to $V$, and non-zero entries in $(I+A)^n$ coincide with non-zero entries in  the transitive closure of $A$.
Furthermore, if we additionally allow access to the matrix $S_{<}$, defining the (strict) order on canonical vectors, then Csanky's matrix inversion algorithm becomes expressible (if $f_/$ is allowed). We denote this fragment by \langmprod+$S_{<}$. 

% !TeX spellcheck = en_US
%!TEX root = ../main.tex

We now verify that \langmprod, extended with order and $f_{>0}$, 
can perform matrix inversion and compute the determinant. To this aim, we verify that all order
predicates in Section \ref{sec:formatlang:design} can be derived using $\ssum$, $\sprod$, $f_{>0}$ and 
$e_{S_{<}}$. Given this, it suffices to observe that Csanky's algorithm, as shown in Section~\ref{sec:queries:inverse}, only relies on expressions using $\ssum$ and $\sprod$ and order information on canonical vectors and $f_/$.
As consequence, our fragment can perform matrix inversion and compute the determinant.


It remains to show that if we have $e_{S_{<}}$, using $\ssum$ and $\sprod$ and $f_{>0}$ we can
can define all order predicates from Section~\ref{sec:formatlang:design}. We note that due to the restricted for-loops
in $\ssum$ and $\sprod$, we do not have access to the intermediate
result in the iterations and as such, it is unclear whether order information can be computed. This is why
we assume access to $e_{S_<}$.

We first remark that if we have $e_{S_{<}}$, we can also obtain
 $e_{S_{\leq}}$ by adding $e_{\mathsf{Id}}$. Hence,
we can compute $\mathsf{succ}$ and $\mathsf{succ}^+$ as well. Furthermore, 
\begin{align*}
  e_{\mathsf{min}}&:=\ssum v. \left[ \sprod w. \mathsf{succ}(w,v)\right] \times v. \\
  e_{\mathsf{max}}&:=\ssum v. \left[ \sprod w. \left( 1-\mathsf{succ}(w,v) \right) \right] \times v.
\end{align*}
Both expressions are only using $\ssum$ and $\sprod$ and $\mathsf{succ}$, so are in our fragment.
Furthermore, if we have $f_{>0}$ then we can define
$$
e_{\mathsf{Pred}}:= e_{S_{<}}- f_{>0}(e_{S_{<}}^2)
$$
Also, recall that  $e_{\mathsf{Next}}:=e_{\mathsf{Pred}}^T$. As a consequence, 
we can now define $\mathsf{prev}(v)$ and $\mathsf{next}(v)$ as in \ref{sec:formatlang:design}. Similarly,
it is readily verified that also $e_{\mathsf{getPrevMatrix}}(V)$,
$e_{\mathsf{getNextMatrix}}(V)$, $e_{\mathsf{min}+i}$ and $e_{\mathsf{max}+i}$ can be expressed
in our fragment.


% What is interesting is that
%  provided that we allow for the $f_{>0}$ function.
 We do not know if the inversion and determinant algorithms are expressible in \langmprod, without order information, or even  in  \langprod, with or without order information. We leave the study of \langmprod and, in particular, the relationship to full \langfor, for future work.
 
Finally, in Figure~\ref{thefigure} we show a diagram of all the fragments of \langfor over square matrices introduced in this section and their corresponding equivalent formalisms.

\begin{figure}
	
	\begin{tikzpicture}[->,>=stealth, semithick, auto, initial text= {}, initial distance= {3mm}, accepting distance= {4mm}, node distance=0.5cm, semithick]
	
	\node [rectangle, draw=black, fill=white, minimum height=4mm, minimum width=2cm, rounded corners] (ML) at (0, 0) {$\texttt{ML}$};
	
	\node [inner sep=0mm] (SML) at ($(ML) + (0,0.65)$) {$\texttt{sum}\text{-}\texttt{ML}\equiv \texttt{RA}^+_K$};
	
	\node [circle, radius=4mm,draw=black, fill=black, inner sep=0mm] (pCLIQUE) at ($(ML) + (1.4,0.1)$) {};
	\node [right of=pCLIQUE,inner sep=0mm, node distance=0.65cm] (CLIQUE) {$\textsc{4Clique}$};
	
	\begin{pgfonlayer}{background}
	\node (SMLc)[draw=black, inner sep=1.5mm, rounded corners,fit=(SML)(ML)(CLIQUE)] {};
	\end{pgfonlayer}
	
	
	\node [inner sep=0mm] (sp-ML) at ($(SML) + (0,0.7)$) {$\texttt{sp}\text{-}\texttt{ML} \equiv \texttt{WL}$};
	
	\node [circle, radius=4mm,draw=black, fill=black, inner sep=0mm] (pDP) at ($(pCLIQUE) + (1.7,0.2)$) {};
	\node [right of=pDP,inner sep=0.5mm, node distance=0.3cm] (DP) {$\textsc{DP}$};
	
	
	\begin{pgfonlayer}{background}
	\node (sp-MLc) [draw=black, inner sep=1.5mm, rounded corners,fit=(SMLc)(ML)(sp-ML)(DP)] {};
	\end{pgfonlayer}
	
	
	\node [inner sep=0mm] (PML)  at ($(sp-ML) + (0,0.7)$) {$\texttt{prod}\text{-}\texttt{ML} + S_{<}$};
	
	\node [circle, radius=4mm,draw=black, fill=black, inner sep=0mm] (pINV) at ($(pDP) + (1,1)$) {};
	\node [right of=pINV,inner sep=0mm, node distance=0.35cm] (INV) {$\textsc{Inv}$};
	
	\node [circle, radius=4mm,draw=black, fill=black, inner sep=0mm] (pDET) at ($(pDP) + (1,0.2)$) {};
	\node [right of=pDET,inner sep=0mm, node distance=0.35cm] (DET) {$\textsc{Det}$};
	
	
	\begin{pgfonlayer}{background}
	\node (PMLc) [draw=black, inner sep=1.5mm, rounded corners,fit=(SMLc)(ML)(sp-MLc)(PML)(INV)(DET)] {};
	\end{pgfonlayer}
	
	
	\node [inner sep=0mm] (forML) at ($(PML) + (1,0.7)$) {$\texttt{for}\text{-}\texttt{ML} \equiv \text{Arithmetic Circuits}$};
	
	\node [circle, radius=4mm,draw=black, fill=black, inner sep=0mm] (pPALU) at ($(pINV) + (1, 0.5)$) {};
	\node [right of=pPALU,inner sep=0mm, node distance=0.5cm] (PALU) {$\textsc{PLU}$};
	
	\begin{pgfonlayer}{background}
	\node [draw=black, inner sep=1.5mm, rounded corners,fit=(SMLc)(ML)(sp-MLc)(PMLc)(forML)(PALU)] {};
	\end{pgfonlayer}
	
	\end{tikzpicture}
	
	\caption{Fragments of \langfor over square matrices and their equivalences. The functions \textsc{4Clique}, \textsc{DP} (diagonal product), \textsc{Inv}, \textsc{Det}, and \textsc{PLU} decomposition are placed in their fragments.} \label{thefigure}
\end{figure}
