% !TeX spellcheck = en_US
%!TEX root = ../../main.tex

\newcommand{\earae}{e_{\arae}}

\begin{proof}	
%Similar than for Proposition~\ref{prop:sum_to_ara}, the proof is by induction on the structure of expression $Q$.
%Let $\cR$ be binary relational schema. For each $R\in \cR$ we associate a matrix variable 
%$V_R$ such that, if $R$ is a binary relational signature, then $V_R$ represents a (square) matrix, 
%if $R$ is unary, then $V_R$ represents a vector and if $|R|=0$ then $V_R$ represents a constant. Formally, 
%fix a symbol $\alpha \in \DD \setminus \{1\}$. Let $\text{Mat}(\cR)$ denote the \lang \ schema
%$(\Mnam_\cR,\size_\cR)$ such that $\Mnam_\cR = \{ V_R \mid R \in \cR\}$ and $\size_\cR(V_R) = (\alpha, \alpha)$ 
%whenever $|R| = 2$, $\size_\cR(V_R) = (\alpha, 1)$ whenever $|R|=1$ and $\size_\cR(V_R) = (1, 1)$ whenever $|R|=0$. 
%Let $\cJ$ be the $K$-instance of $\cR$ and suppose that $\adom(\cJ) = \{d_1, \ldots, d_n\}$ is 
%the active domain (with arbitrary order) of $\cJ$. 
%Define the matrix instance $\text{Mat}(\cJ) = (\dom_\cJ,\conc_\cJ)$ such 
%that $\dom_\cJ(\alpha) = n$, $\conc_\cJ(V_R)_{i,j} = R^{\cJ}((d_i, d_j))$ whenever $|R|=2$, $\conc_\cJ(V_R)_{i} = R^{\cJ}((d_i))$ 
%whenever $|R|=1$, 
%\floris{This case relates to nullary relations. What does $R^{\cJ}$ mean?}
%and $\conc_\cJ(V_R)_{1,1} = R^{\cJ}$ whenever $|R|=0$. 
%Note that we consider the active domain of the whole $K$-instance.
By induction we translate \rak expressions into \langsum expressions over an extended schema. More specifically, for each attribute $A \in \att$ we define a vector variable $v_A$ of type $(\alpha,1)$. Then for each \rak expression $\arae$ with attributes $A_1, \ldots, A_k$ we define a \langsum expression $\earae(v_{A_1}, \ldots, v_{A_k})$ of type $(1,1)$ such that the following inductive hypothesis holds:
$$
\sem{\earae}{\text{Mat}(\cJ)[v_{A_1} \gets b_{i_1},\ldots, v_{A_k} \gets b_{i_k}]} = 
\ssem{\arae}{\cJ}(t) \ \ \ \ \ \ \  (*)
$$
where $t(A_s)=d_{i_s}$ for $s=1,\ldots, k$. The proof of this claim follows by induction on the structure of expressions:
\begin{itemize} \itemsep3mm
	\item If $\arae=R$, then $\earae\coloneqq v_{A_1}^T \cdot V_R \cdot v_{A_2}$ if $\mathcal{R}(R)=\{A_1,A_2\}$ with $A_1<A_2$; 
	$\earae\coloneqq V_R^T \cdot v_A$ if $\mathcal{R}(R)=\{A\}$; and 
	$\earae\coloneqq V_R$ if $\mathcal{R}(R)=\{\}$.
	\item If $\arae=\arae_1\cup \arae_2$ then
	$\earae\coloneqq e_{\arae_1} + e_{\arae_2}$.
	\item If $\arae=\pi_{Y}(\arae_1)$ for $Y\subseteq \mathcal{R}(\arae_1)$ and $\{B_1, \ldots, B_l\} = \mathcal{R}(\arae_1) \setminus Y$ then
	$
	\earae\coloneqq  \Sigma v_{B_1}. \ \Sigma v_{B_2}. \ \ldots \Sigma v_{B_l}. \ e_{\arae_1}
	$.
	\item If $\arae=\sigma_{Y}(\arae_1)$ with $Y\subseteq\mathcal{R}(\arae_1)$ then
	$
	\earae\coloneqq e_{\arae_1} \kprod \bigkprod_{A,B\in Y} (v_{A}^T \cdot v_{B})
	$.
	Here $\bigkprod$ is the matrix multiplication of expressions of type $(1,1)$.
	\item If $\arae=\rho_{f}(\arae_1)$ with $f: X \rightarrow Y$ then
	$\earae\coloneqq e_{\arae_1}[v_{\EDIT{A}} \gets v_{\EDIT{f(A)}}\mid A\in X]$.
	In other words, we rename variable $v_{\EDIT{A}}$ with variable $v_{\EDIT{f(A)}}$ in the expression $e_{\arae_1}$. 
	\item If $\arae=\arae_1\bowtie \arae_2$ then
	$\earae\coloneqq e_{\arae_1} \cdot \EDIT{e_{\arae_2}}$ where the product is over expression of type $(1,1)$.
\end{itemize}
By induction, one can check that the inductive hypothesis $(*)$ holds in each case. 

Finally, we can prove Proposition~\ref{prop:ara_to_sum}.
As a consequence of the previous discussion, when $\arae$ is a \rak expression 
such that $\mathcal{R}(\arae)=\{A_1,A_2\}$ with $A_1<A_2$ then we define
$$
\Psi(\arae) \ = \ \Sigma v_{A_1}. \ \Sigma v_{A_2}. \ \earae \cdot (v_{A_1} \cdot v_{A_2}^T). 
$$
Instead, when $\mathcal{R}(\arae)=\{A\}$ we have $
\Psi(\arae) \ = \ \Sigma v_{A}. \  (v_{A} \cdot \earae)$.
And when $\mathcal{R}(\arae)=\{\}$ we have $
\Psi(\arae) \ = \ \earae
$.
By using the inductive hypothesis $(*)$ one can check that $\Psi(\arae)$ works in each case as expected. 
\end{proof}
 