\section{Reviewer 3}
\label{sec:reviewer-3}
\bigskip

\begin{comment}
	(\ldots) However, the
	presentation can still be improved by a lot as indicated below.
	Also there is some problem in the way how the expressiveness of
	MATLANG is compared with arithmetic circuits. 
	
	(\ldots) While the small errors are easily fixable, the presentation and the comparison
	to circuit families of polynomial degree needs a major revision.
\end{comment}

\answer \TODO
\bigskip

\begin{comment}
	The authors define a MATLANG expression of polynomial degree as any MATLANG
	expression that has an equivalent circuit family of polynomial degree.
	Afterwards there is the mind blowing result that this class exactly corresponds
	to the class of circuits of polynomial degree. Of course this result does not
	provide any scientific value, as it just repeats the definition. It is not
	clear at all how this class of MATLANG expressions looks like. Actually it is
	undecidable if a given MATLANG expression has a polynomial degree.
	
	Instead there should be an (ideally syntactic) definition that is intrinsic to
	MATLANG. Especially it should not be necessary to refer to circuit families in
	order to provide a definition of polynomial degree for MATLANG expressions. If
	no syntactic definition is possible than a sensible semantic definition will
	also work.
\end{comment}

\answer \TODO
\bigskip

\begin{comment}
	I start with the comparison to arithmetic circuits. In Section 5.2 you
	construct MATLANG expressions that uses an input vector of the same arity as
	the circuit and outputs a single value, the same that the circuit will
	produce. Theorem 5.1 does not talk about the sizes of other matrices used.
	
	In your construction you use square matrices and vectors of the same size as
	the input vector. As a result of this design decision, you must limit the
	construction to circuits of logarithmic depth. For the other direction
	however, you produce circuits of polynomial depth.
	
	Here you introduce MATLANG expressions of polynomial degree in order to have
	a MATLANG class and a circuit class of equal expressiveness. Instead, I
	propose to change your construction of MATLANG expressions in a way that
	allows to handle all polynomial arithmetic circuits.
	
	Replace Algorithm 1 with an algorithm that computes (and stores) the output
	values of all gates in topological order. This algorithm is way simpler and
	does not need a stack.
	
	Then allow your MATLANG construction to use intermediate values of polynomial
	size. The main data structure is a vector that has as many entries as you have
	gates in the circuit. Now you can iterate over all gates (w.l.o.g. the gates
	are sorted in topological order) and compute all values. This is a single for
	loop.
	
	Of course you still need the construction from the appendix to compute the
	nextgate() function, i.e. to simulate the TM that constructs the circuit.
	Probably it would be simpler if this TM would directly construct a vector that
	contains all input gates. Then you only need to call this function once for
	each gate and just use an additional for-loop for the aggregation.
	
	Now you have a natural class of MATLANG expressions that exactly corresponds
	to arithmetic circuits. I do not see why MATLANG expressions should not be
	allowed to use intermediate results that have bigger arity than the input.
	This is a restriction that you never formulated and that is also not imposed
	on the circuits.
	
	Of course, you can still discuss restricted settings, but please use sensible
	definitions. E.g. if you restrict the  size of intermediate results in MATLANG
	a corresponding restriction would be on the width of the arithmetic circuit.
\end{comment}

\answer \TODO
\bigskip

\begin{comment}
	You should introduce and describe a consistent notation that allows to easily
	distinguish whether some MATLANG expressions (1) construct some vector or matrix; or
	(2) is a Boolean test (i.e., evaluates to a scalar value 0 or 1).
\end{comment}

\answer \TODO
\bigskip

\begin{comment}
	Also, you should adopt the notation that iterator variables are easily
	distinguishable from other variables throughout the article. Especially if you
	give expressions like for $col(V,y)$ in line 572, it would be really helpful to
	immediately see that $y$ is meant to be a variable that can only take canonical
	vectors as values.
\end{comment}

\answer Done.
\bigskip

\begin{comment}
	Actually, I would even prefer a notation where iterator variables (like $i,j$)
	range over indices, i.e. take values from $1$ to $n$. Then you can still write $b_i$
	if you need the canonical vector, but you could also just write $V_{ij}$ instead
	of $v^t * V * y$, which is way easier to parse for a human. Obviously, such a
	notation would be just syntactic sugar.
\end{comment}

\answer \TODO.

\thomas{The former is a good idea, but besides replacing using $V_{uv}$ instead
of $u^t * V * v$, there are no major advantages, while it has major impact in rewriting expressions. Will do if time allows it.}
\bigskip

\begin{comment}
	I also suggest to consider introducing some $\texttt{if } e_1 \texttt{ then } e_2 \texttt{ else } e_3$
	construction, where $e_1$ is some expression that evaluates to a scalar $0$ or $1$.
	This is used a lot in the article.
\end{comment}

\answer \TODO
\bigskip

\begin{comment}
	You have to rename the $succ$ and $succ+$ expressions, as they in fact do not test
	for successors. What you call successor is the less or equal relation and what
	you call $succ+$ is the less than relation. So $succ$ could be renamed to
	$islessorequal$ and $succ+$ could be named $isless$, which also directly reminds the
	reader that this expression is a Boolean test.
\end{comment}

\answer Done.
\bigskip

\begin{comment}
	You seem to mix $\to$ and $\mapsto$ in function specifications in a random way.
	Function signatures use $\to$ ($f \colon A \to B$, where $A$ and $B$ are domain and
	image of $f$), while $\mapsto$ is used for concrete mappings (e.g., $f \colon n
	\mapsto n^2$). And you should use \verb+\colon+ instead of $:$, because $:$ is treated as
	a division operator by Latex and thus the spacing is not correct.
\end{comment}

\answer \TODO

\bigskip

\begin{comment}
	And you should use \verb+\colon+ instead of $:$, because $:$ is treated as
	a division operator by Latex and thus the spacing is not correct.
\end{comment}

\answer Done.

\thomas{Most problems are with $:=$, so it was replaced by 'coloneqq' command}
\bigskip

\begin{comment}
	LU-Decomposition: You should definitely provide some pseudocode for the LU Decomposition
	algorithm in order to allow a simpler comparison with your MATLANG
	expressions. Right now the algorithm is given as prose. Furthermore it is not
	even complete as the definition of $c_i$ with $i \neq 1$ is missing.
\end{comment}

\answer \TODO
\bigskip

\begin{comment}
	Algorithm 1: the aggregate function is working in a completely different way than your
	MATLANG construction. The MATLANG constructions is a sum over 5 expressions,
	which especially implies that the order of evaluation is irrelevant. However
	the algorithm is written in a way that the order of the statements is very
	important. Especially it is not the case that the five expressions correspond
	to five different cases of the algorithm as you claim.
	(\ldots)
	some constructions are way more complex than needed.
\end{comment}

\answer \TODO
\bigskip

\begin{comment}
	Algorithm 1: simplify $e_{iterate}$ by using the loop init $X_1 = e_{start}$. Then you can
	remove the outer ``if-then-else''.
\end{comment}

\answer \TODO
\bigskip

\begin{comment}
	Algorithm 1: $e_{pop}$ can just pop both stacks simultaneously by just removing one row
	from the matrix. The complicated $IdUpTo$ expression is not needed. Just
	compute
	$$
	e_{pop} := - G_{top} * G_{top}^T * X_k + e_{Prev} * V_{top} * e_{V_{top}}^T
	$$
	to pop both stacks. The first summand computes the delta to remove a line
	from the matrix and the second summand reads the pointer for the value
	stack. The pointer for the gate stack will be added in $e_{agg\_(not)\_last}$.
	Note that this is just a delta. Thus $e_{aggregate}$ will need an additional
	$X_{k}$` summand.
\end{comment}

\answer \TODO
\bigskip

\begin{comment}
	Algorithm 1: $e_{agg\_prod}$ also looks way to complicated. Why do you need $IdVal$? Just
	manipulate the single matrix entry directly:
	$$
	e_{agg\_prod} := V^T * V_{top} * (V^T * (e_{Prev} * V_{top}) - 1)
	$$
	The $-1$ is to accomodate for the value that is already on the stack.
	No for-loop needed (hidden in the $IdVal$ expression).
\end{comment}

\answer \TODO
\bigskip

\begin{comment}
	Algorithm 1: I find it confusing that in Figure 3 you list the combined effect of two
	of the expressions. The figure should describe each of the expressions on
	its own. Especially as these combinations that you describe are not possible
	in the algorithm. The algorithm combines always three of the expressions.
	Also the Figure is wrong. In the upper two cases, there should be no pointer
	for the gate stack. And in the lower cases there should be an empty value
	stack according to your construction.
\end{comment}

\answer \TODO
\bigskip

\begin{comment}
	Algorithm 1: Also, please note that you overload $n$ with many different meanings:
	(1) arity of the circuit, (2) size of the matrices, and (3) length of a bit vector describing a gate id.
\end{comment}

\answer \TODO
\bigskip


\begin{comment}
	Algorithm 1: Why are gate ids of linear length in the input? This would correspond to an
	circuit of exponential size.
\end{comment}

\answer \TODO
\bigskip


\begin{comment}
	Proposition 5.2: First of all, add a runtime bound to the Turing machine. I do not believe that
	your construction can simulate every linear space machine without restrictions
	on the runtime.
\end{comment}

\answer \TODO
\bigskip


\begin{comment}
	Proposition 5.2: And now: Why do you write up the proof using the most complicated type of TMs
	possible? For your construction it would be perfectly sufficient to have a
	single tape machine. Your computations take at most two IDs of size $O(log(n))$
	as input and produce one such ID as output. This even fits on a single tape if
	you restrict to length $n$. You are assuming $n > n_0$ for some sufficiently large $n_0$
	anyway. Using only a single tape makes the construction much more readable, as you can
	get rid of many indices.
\end{comment}

\answer \TODO
\bigskip

\begin{comment}
	Proposition 5.2: If you change Algorithm 1 as laid out above, it could make sense to
	actually allow for an output ``tape'' that is formed like a square matrix. The
	head of the output tape could move in four directions (simulated by two
	vectors wit a single 1 entry). It is obvious (to all that know TMs) that this
	does not add additional power and one can easily translate the TM producing
	the circuit to such a TM with a square output. The advantage would be that
	you could directly produce the adjacency matrix of the circuit which you
	could use immediately in the expression that evaluates your circuit.
\end{comment}

\answer \TODO
\bigskip

\begin{comment}
	Proposition 5.2: Also, you do not need to consider the case of small $n$ at all. This case is already considered in the proof of Theorem 1. You can restrict Proposition 5.2 to all
	$n$ greater than some $n_0$ (and maybe say that it also holds without this
	restriction). But no need to waste the space for the proof.
\end{comment}

\answer \TODO
\bigskip

\begin{comment}
	Proposition 5.2: And please try to simplify the construction of the TM. Why do you encode the
	position of the head in a special way if it is at the edge of the tape? Just
	adjust the size of the tape such that the end markers are included in the
	length. Then you do not need to special code this.
\end{comment}

\answer \TODO
\bigskip

\begin{comment}
	From MATLANG to circuits: Why do you restrict the result to MATLANG expressions, where all types only
	use the size symbol alpha? The construction should work in exactly the same
	way in the general case. OK, for uniformness you need that all sizes can be
	computed from the input size by a logspace TM, which results from the
	definition of uniform circuits where the TM just gets one input parameter.
	Probably you should discuss this (as it is the usual definition), but in your
	setting a slightly more general setting of uniform circuits would make sense.
	
	In any case, the proof needs to be reformulated in order to avoid all these
	pointless case distinctions on the types of the subexpressions. Do all
	induction cases with $\texttt{type}_S(V)=(\alpha,\beta)$ and provide one induction step for
	every operation. The cases where one or both of alpha beta are 1 are special
	cases of the general case and subsumed by the general case. No need to do any
	case distinctions.
\end{comment}

\answer \TODO
\bigskip

\begin{comment}
	Comparison with $K$-Relations: Your definition of the renaming operator is nonstandard. Usually this operator
	takes a function $f$ that renames the variables of a given relation, i.e., the
	domain of $f$ are the attributes of the relation/expression inside the operator
	and the image are the new (renamed) attributes. You define it the other way
	round. When you use the operator, you mix standard and your non-standard
	definition. Please stay with the established definition.
\end{comment}

\answer \TODO
\bigskip

\begin{comment}
	Comparison with $K$-Relations: The construction of algebra expression from MATLANG expressions is much more
	complicated than necessary. You should not do case distinctions on the types of
	matrices, as the general construction works independently of whether some
	dimension is 1 or not.
\end{comment}

\answer \TODO
\bigskip

\begin{comment}
	Comparison with $K$-Relations: Just use $Rel(S)(R_V) := \{row, col\}$ for every matrix $V$. row and col encode the domain of the indices of the matrix, as in your construction. The only
	difference is, that this domain could be the singleton $\{1\}$. And you should omit
	the subscripts alpha and beta of row and col. They are not needed, as the domain
	is encoded by the relation. If you omit the subscripts you will not need to
	talk about types at all in most parts of the proof. The soundness of the
	MATLANG expression ensures that the domains of row and col are correct.
	
	To always provide a col attribute you need a new relation $R_1$ with attribute
	col and a single number 1 inside the relation. You can then change $Q(v_p)$ to be
	$$
	\sigma_{row, \gamma_p} (\rho_{\alpha -> \gamma_p}(R_\alpha) \texttt{join} \rho_{\alpha ->
			row}(R_\alpha) \texttt{join} R_1)
	$$
	This construction simplifies the definition for transposition to just
	rename row $\rightarrow$ col and col $\rightarrow$ row. Also for the other operators you only need
	to talk about one case. And types are working flawlessly. E.g. matrix product
	becomes rename col $\rightarrow$ C for the first expression and row $\rightarrow$ C for the second
	expression before doing the join. However you have to explain what the join
	does, i.e., that it just computes the same sum as the matrix product.
\end{comment}

\answer \TODO
\bigskip

\begin{comment}
line 183: this is ugly to read. Please align at := and at if.
\end{comment}

\answer \TODO
\bigskip

\begin{comment}
	line 216: this also should be aligned.
\end{comment}

\answer \TODO
\bigskip

\begin{comment}
	line 1386: Please rephrase. One could get the impression that the halting problem for linear time TMs is undecidable (which it certainly is not).
\end{comment}

\answer \TODO
\bigskip

\begin{comment}
	line 1435: please provide a pageref for figure 4 or mention that it is at the very end of the article.
\end{comment}

\answer \TODO
\bigskip

\begin{comment}
	line 1443: You have to restrict the expression e, such that it cannot use X. Otherwise Proposition 6.1 is definitely not true.
\end{comment}

\answer \TODO
\bigskip

\begin{comment}
	line 1449: Using your ill-defined definition of expressions of polynomial
	degree, the proof of Proposition 6.1 is nontrivial and cannot be omitted. You
	have to show that every expression of sum-MATLANG can be converted to a
	circuit family of polynomial degree, meeting the syntactic definition of the circuits.
	Showing that you cannot produce superpolynomial matrix entries is not enough.
\end{comment}

\answer \TODO
\bigskip

\begin{comment}
	line 1681: change $Q_1$ to $Q_2$.
\end{comment}

\answer \TODO
\bigskip

\begin{comment}
	line 1731: again, X should not be used inside e.
\end{comment}

\answer \TODO
\bigskip

\begin{comment}
	line 1912 (figure 4): If you adapt the constructions as indicated above, the
	figure is ok. Otherwise you need to specify which subclass of MATLANG is
	equivalent to which subclass of arithmetic circuits. E.g., right now, you
	only can convert MATLANG expressions that use a single size symbol alpha, as
	you needlessly restrict your construction.
\end{comment}

\answer \TODO
\bigskip




%%% Local Variables:
%%% mode: latex
%%% TeX-master: "response.tex"
%%% End:
