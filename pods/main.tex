%%
%% This is file `sample-sigconf.tex',
%% generated with the docstrip utility.
%%
%% The original source files were:
%%
%% samples.dtx  (with options: `sigconf')
%% 
%% IMPORTANT NOTICE:
%% 
%% For the copyright see the source file.
%% 
%% Any modified versions of this file must be renamed
%% with new filenames distinct from sample-sigconf.tex.
%% 
%% For distribution of the original source see the terms
%% for copying and modification in the file samples.dtx.
%% 
%% This generated file may be distributed as long as the
%% original source files, as listed above, are part of the
%% same distribution. (The sources need not necessarily be
%% in the same archive or directory.)
%%
%% The first command in your LaTeX source must be the \documentclass command.
\documentclass[sigconf]{acmart}

\settopmatter{printacmref=false} % Removes citation information below abstract
\renewcommand\footnotetextcopyrightpermission[1]{} % removes footnote with conference information in first column
\pagestyle{plain} % removes running headers

%%
%% \BibTeX command to typeset BibTeX logo in the docs
\AtBeginDocument{%
  \providecommand\BibTeX{{%
    \normalfont B\kern-0.5em{\scshape i\kern-0.25em b}\kern-0.8em\TeX}}}

%% Rights management information.  This information is sent to you
%% when you complete the rights form.  These commands have SAMPLE
%% values in them; it is your responsibility as an author to replace
%% the commands and values with those provided to you when you
%% complete the rights form.
%\setcopyright{acmcopyright}
%\copyrightyear{2018}
%\acmYear{2018}
%\acmDOI{10.1145/1122445.1122456}
%
%%% These commands are for a PROCEEDINGS abstract or paper.
%\acmConference[Woodstock '18]{Woodstock '18: ACM Symposium on Neural
%  Gaze Detection}{June 03--05, 2018}{Woodstock, NY}
%\acmBooktitle{Woodstock '18: ACM Symposium on Neural Gaze Detection,
%  June 03--05, 2018, Woodstock, NY}
%\acmPrice{15.00}
%\acmISBN{978-1-4503-XXXX-X/18/06}
\usepackage{enumitem}
\usepackage{xspace}
\newcommand{\paths}{\text{PATHS}}

%%%%%%%%%%%%%%%%%%%%%%%%%%%%%%%%
%			COMMENTS
%%%%%%%%%%%%%%%%%%%%%%%%%%%%%%%%
\usepackage[textwidth=2cm,textsize=small]{todonotes}

\newcommand{\domagoj}[1]{\todo[inline, color=blue!15]{{\bf Domagoj:} #1}}
\newcommand{\cristian}[1]{\todo[inline, color=orange!30]{{\bf Cristian:} #1}}
\newcommand{\thomas}[1]{\todo[inline, color=green!30]{{\bf Thomas:} #1}}
\newcommand{\floris}[1]{\todo[inline, color=red!30]{{\bf Floris:} #1}}

%%% UNCOMMENT BELOW TO REMOVE COMMENTS
%\renewcommand{\domagoj}[1]{}
%\renewcommand{\cristian}[1]{}
%\renewcommand{\thomas}[1]{}
%\renewcommand{\floris}[1]{}

% Pseudocode packages
\usepackage{algorithm} 
\usepackage[noend]{algpseudocode}

% Tikz
\usepackage{tikz}
\usetikzlibrary{trees}
\usetikzlibrary{tikzmark}
\usetikzlibrary{matrix, positioning}

% Other packages
\usepackage[overload]{empheq}
\usepackage{mathtools}
\DeclarePairedDelimiter{\ceil}{\lceil}{\rceil}

%% Macros comments
\newcommand{\tover}[1]{\textcolor{red}{#1}}
\newcommand{\td}[1]{\textcolor{blue}{[TODO: #1]}}

%% Macros logics
\newcommand{\NN}{\mathbb{N}}
\newcommand{\ZZ}{\mathbb{Z}}
\newcommand{\MM}{\mathbb{M}}
\newcommand{\SE}{\mathbb{S}}
\newcommand{\BB}{\mathbb{B}}
\newcommand{\RR}{\mathbb{R}}
\newcommand{\cF}{\mathcal{F}}
\newcommand{\cI}{\mathcal{I}}
\newcommand{\cM}{\mathcal{M}}
\newcommand{\cS}{\mathcal{S}}
\newcommand{\cX}{\mathcal{X}}
\newcommand{\cV}{\mathcal{V}}
\newcommand{\QFBILIA}{\textsf{QFBILIA}}
\newcommand{\nnf}{\textsf{f}}
\newcommand{\Cat}{\operatorname{Cat}}
\newcommand{\ssum}{\textstyle\sum}
\newcommand{\sprod}{\textstyle\prod}
%\newcommand{\for}{\textbf{for }}

\newcommand{\ite}{\textsf{ite}}
\newcommand{\limp}{\Rightarrow}
\newcommand{\flc}{\rightarrow}


\newcommand{\bsingle}{\textsf{bag}}
\newcommand{\bplus}{\oplus}
\newcommand{\bminus}{\ominus}

%% Macros tools
\newcommand{\spen}{\textsc{spen}}
\newcommand{\zzz}{\textsc{Z3}}

%% Environments
\newtheorem{mythm}{Theorem}[section]
\newtheorem{mydef}[mythm]{Definition}
\newtheorem{myprop}[mythm]{Proposition}
\newtheorem{mylem}[mythm]{Lemma}
\newtheorem{myex}[mythm]{Example}
\newtheorem{mycor}[mythm]{Corollary}

\newtheorem*{myrem}{Remark} %% based on amsthm
\newtheorem*{mynota}{Notation}
\newtheorem*{mylem*}{Lemma}
\newtheorem{myclaim}{Claim}
\newtheorem*{myclaim*}{Claim}
\newtheorem*{myprop*}{Proposition}
\newtheorem*{comp}{Efficiency Study}


\newenvironment{point}[1]
{\subsection*{#1}}%
{}

\newcommand{\flist}{\text{\sc FList}}
\newcommand{\set}{\text{\sc Set}}
\newcommand{\fset}{\text{\sc FSet}}
\newcommand{\B}{\text{\bf B}}
%\newcommand{\G}{\mathcal{G}}
\newcommand{\K}{\mathcal{K}}
\newcommand{\LOG}{\text{\sc Log}}
\newcommand{\length}{\text{\rm length}}
\newcommand{\BK}{\text{\sc BK}}
%\newcommand{\cP}{\mathcal{P}}
%\newcommand{\cV}{\mathcal{V}}
%\newcommand{\cC}{\mathcal{C}}
%\newcommand{\cS}{\mathcal{S}}
%\newcommand{\cA}{\mathcal{A}}
%\newcommand{\cR}{\mathcal{R}}
%\newcommand{\cQ}{\mathcal{Q}}
\newcommand{\cG}{\mathcal{G}}
\newcommand{\cT}{\mathcal{T}}
\newcommand{\pr}{\mathbf{Pr}}
\newcommand{\Dyck}{\mathcal{D}}
\newcommand{\expected}{\mathbf{E}}
\newcommand{\bv}{\mathbf{v}}
\newcommand{\bV}{\mathbf{V}}
\newcommand{\bs}{\mathbf{s}}
\newcommand{\bsigma}{\mathbf{\sigma}}
\newcommand{\bw}{\mathbf{w}}
\newcommand{\ba}{\mathbf{a}}
\newcommand{\bq}{\mathbf{q}}
\newcommand{\bx}{\mathbf{x}}
\newcommand{\by}{\mathbf{y}}

\newcommand{\bstring}{\{0,1\}^\ast}

\newcommand{\quot}[1]{#1/\!\equiv}

\newcommand{\then}{\,|\,}
\newcommand{\body}{q}
\newcommand{\bchain}{\text{bc}}

\newcommand{\owner}{\text{\rm owner}}
\newcommand{\pred}{\text{\rm pred}}
\newcommand{\mine}{\text{\rm mine}}
\newcommand{\suc}{\text{\rm succ}}


\newcommand{\bP}{\mathbf{P}}
\newcommand{\bB}{\mathbf{B}}
\newcommand{\bA}{\mathbf{A}}
\newcommand{\bR}{\mathbf{R}}
\newcommand{\bS}{\mathbf{S}}
\newcommand{\bH}{\mathbf{H}}
\newcommand{\bQ}{\mathbf{Q}}

\newcommand{\forkm}[1]{F^{#1}}
\newcommand{\mfork}{F_m}
\newcommand{\mgfork}{F_{m,g}}
\newcommand{\last}{\text{\rm last}}
\newcommand{\best}{\text{\rm best}}
\newcommand{\cho}{\text{\rm choose}}

\newcommand{\ie}{i.e.$\!$ }

\newcommand{\longest}{{\text{\rm longest}}}

\newcommand{\subbody}{{\text{\rm sub-state}}}

\newcommand{\df}{\text{\rm DF}}
\newcommand{\fg}{\text{\rm FG}}
\newcommand{\fr}{\text{\rm FR}}
\newcommand{\bdf}{\text{\rm {\bf DF}}}
\newcommand{\bfg}{\text{\rm {\bf FG}}}
\newcommand{\bfr}{\text{\rm {\bf FR}}}

\newcommand{\meet}{\text{\rm meet}}

\DeclareMathOperator*{\argmax}{argmax}


%\newcommand{\rpa}{r_p^\alpha}
\newcommand{\rpa}{r_p}

\newcommand{\ameet}{\text{\rm all-meet}}
\newcommand{\base}{\text{\rm base}}

\newcommand{\ucl}{\text{\rm up-cl}}
\newcommand{\dcl}{\text{\rm down-cl}}


\newcommand{\sem}[2]{\llbracket #1 \rrbracket(#2)}
\newcommand{\ssem}[1]{\llbracket #1 \rrbracket}


\newcommand{\Mnam}{\mathcal{M}}
\newcommand{\Mvar}{\mathcal{V}}
\newcommand{\Fun}{\mathcal{F}}
\newcommand{\Mlang}{\text{MATLANG}}
\newcommand{\llet}{\texttt{let } V = e_1 \texttt{ in } e_2}
\newcommand{\ones}{\mathbf{1}}
\newcommand{\diag}{\texttt{diag}}
\newcommand{\apply}[1]{\texttt{apply}[#1]}

\newcommand{\I}{\mathcal{I}}
\newcommand{\Voc}{\mathcal{S}}
\newcommand{\Sch}{\mathcal{S}}

\newcommand{\mtr}[1]{\textsf{Mat}[#1]}

\newcommand{\dom}{\mathcal{D}}
\newcommand{\conc}{\texttt{mat}}

\newcommand{\DD}{\texttt{Symb}}
\newcommand{\size}{\texttt{size}}

\newcommand{\ddim}{\texttt{dim}}
\newcommand{\ttype}{\texttt{type}_{\Sch}}
\newcommand{\ttypeo}{\texttt{type}_{\Sch_1}}

\newcommand{\type}{\texttt{type}}

\newcommand{\lang}{\texttt{MATLANG}}
\newcommand{\langfor}{\texttt{for}-\texttt{MATLANG} }
\newcommand{\langsum}{\texttt{sum}-\texttt{MATLANG} }

\newcommand{\ffor}[3]{\texttt{for}\, #1,#2 \texttt{.}\, #3}

%\newcommand{\initf}[4]{\texttt{for}[#1]\, #2,#3 \texttt{.}\, #4}
\newcommand{\initf}[4]{\texttt{for}\, #2,#3\!=\! #1 \texttt{.}\, #4}

\newcommand{\lleq}[2]{\texttt{leq}(#1,#2)}
\newcommand{\mmin}[1]{\texttt{min}(#1)}

\newcommand{\ccol}[2]{\texttt{col(}#1,#2\texttt{)}}
\newcommand{\red}[2]{\texttt{reduce(}#1,#2\texttt{)}}
\newcommand{\nneq}[2]{\texttt{neq(}#1,#2\texttt{)}}
\newcommand{\ccoleq}[2]{\texttt{col}_{\texttt{eq}}\texttt{(}#1,#2\texttt{)}}
\newcommand{\getdiag}[1]{\texttt{get{\_}diag}(#1)}
\newcommand{\diaginverse}[1]{\texttt{diag{\_}inverse}(#1)}

\newcommand{\Dist}{\texttt{Dist}}

\newcommand{\push}[2]{#1\texttt{.push}(#2)}
\newcommand{\pop}[1]{#1\texttt{.pop}}
\newcommand{\getsize}[1]{#1\texttt{.size}}
\newcommand{\gettop}[1]{#1\texttt{.top}}

\newcommand{\isplus}[1]{\texttt{isplus}\left( #1 \right)}
\newcommand{\isprod}[1]{\texttt{isprod}\left(#1 \right)}
\newcommand{\isone}[1]{\texttt{isone}\left(#1 \right)}
\newcommand{\isinput}[1]{\texttt{isinput}\left(#1 \right)}
\newcommand{\getfirst}[1]{\texttt{getfirst}\left(#1 \right)}
\newcommand{\getinput}[1]{\texttt{getinput}\left(#1 \right)}
\newcommand{\getroot}{\texttt{getroot}()}
\newcommand{\isnotlast}[2]{\texttt{not{\_}last}\left(#1,#2 \right)}
\newcommand{\nextgate}[2]{\texttt{next{\_}gate}\left(#1, #2 \right)}

\newcommand{\Iden}[1]{\texttt{Iden}(#1)}
\newcommand{\pondIden}[2]{\texttt{E}\left[ #1, #2 \right]}

%Complexity theory
\newcommand{\np}{{\sc NP}}
\newcommand{\nl}{{\sc NLOGSPACE}}
\newcommand{\logspace}{{\sc LOGSPACE}}


% Semirings

%\usepackage{bbold}
\DeclareMathAlphabet{\mymathbb}{U}{BOONDOX-ds}{m}{n}





%%
%% Submission ID.
%% Use this when submitting an article to a sponsored event. You'll
%% receive a unique submission ID from the organizers
%% of the event, and this ID should be used as the parameter to this command.
%%\acmSubmissionID{123-A56-BU3}

%%
%% The majority of ACM publications use numbered citations and
%% references.  The command \citestyle{authoryear} switches to the
%% "author year" style.
%%
%% If you are preparing content for an event
%% sponsored by ACM SIGGRAPH, you must use the "author year" style of
%% citations and references.
%% Uncommenting
%% the next command will enable that style.
%%\citestyle{acmauthoryear}

%%
%% end of the preamble, start of the body of the document source.
\begin{document}

%%
%% The "title" command has an optional parameter,
%% allowing the author to define a "short title" to be used in page headers.
%\title{LOLA: A query language for Linear Algebra with LOops}
\title{Expressive power of linear algebra query languages}
%%
%% The "author" command and its associated commands are used to define
%% the authors and their affiliations.
%% Of note is the shared affiliation of the first two authors, and the
%% "authornote" and "authornotemark" commands
%% used to denote shared contribution to the research.
\author{Floris Geerts}
\affiliation{%
  \institution{University of Antwerp}
%  \streetaddress{P.O. Box 1212}
%  \city{Dublin}
%  \state{Ohio}
%  \postcode{43017-6221}
}
\email{floris.geerts@uantwerp.be}


\author{Thomas Mu\~noz}
\affiliation{%
  \institution{PUC Chile and IMFD Chile}
%  \streetaddress{1 Th{\o}rv{\"a}ld Circle}
%  \city{Hekla}
%  \country{Iceland}
}
\email{tfmunoz@uc.cl}

\author{Cristian Riveros}
\affiliation{%
  \institution{PUC Chile and IMFD Chile}
%  \streetaddress{1 Th{\o}rv{\"a}ld Circle}
%  \city{Hekla}
%  \country{Iceland}
}
\email{cristian.riveros@uc.cl}

\author{Domagoj Vrgo\v{c}}
\affiliation{%
  \institution{PUC Chile and IMFD Chile}
%  \streetaddress{1 Th{\o}rv{\"a}ld Circle}
%  \city{Hekla}
%  \country{Iceland}
}
\email{dvrgoc@ing.puc.cl}

%%
%% By default, the full list of authors will be used in the page
%% headers. Often, this list is too long, and will overlap
%% other information printed in the page headers. This command allows
%% the author to define a more concise list
%% of authors' names for this purpose.
%\renewcommand{\shortauthors}{Trovato and Tobin, et al.}

%%
%% The abstract is a short summary of the work to be presented in the
%% article.
\begin{abstract}
Linear algebra algorithms often require some sort of iteration or recursion as is illustrated by standard algorithms for Gaussian elimination, matrix inversion, and transitive closure. A key characteristic shared by these 
algorithms is that they allow looping for a number of steps that is bounded by the matrix dimension. 
In this paper we extend the matrix query language \lang with this type of recursion, and show that this suffices to express  classical linear algebra algorithms. We study the expressive power of this language and show that it naturally corresponds to arithmetic circuit families, which are often said to capture linear algebra. Furthermore, we analyze several sub-fragments of our language, and show that their expressive power is closely tied to logical formalisms on semiring-annotated relations.
% how they can capture the expressive power of annotated relations and weighted logics.
\end{abstract}

%%
%% The code below is generated by the tool at http://dl.acm.org/ccs.cfm.
%% Please copy and paste the code instead of the example below.
%%%
%\begin{CCSXML}
%<ccs2012>
% <concept>
%  <concept_id>10010520.10010553.10010562</concept_id>
%  <concept_desc>Computer systems organization~Embedded systems</concept_desc>
%  <concept_significance>500</concept_significance>
% </concept>
% <concept>
%  <concept_id>10010520.10010575.10010755</concept_id>
%  <concept_desc>Computer systems organization~Redundancy</concept_desc>
%  <concept_significance>300</concept_significance>
% </concept>
% <concept>
%  <concept_id>10010520.10010553.10010554</concept_id>
%  <concept_desc>Computer systems organization~Robotics</concept_desc>
%  <concept_significance>100</concept_significance>
% </concept>
% <concept>
%  <concept_id>10003033.10003083.10003095</concept_id>
%  <concept_desc>Networks~Network reliability</concept_desc>
%  <concept_significance>100</concept_significance>
% </concept>
%</ccs2012>
%\end{CCSXML}
%
%\ccsdesc[500]{Computer systems organization~Embedded systems}
%\ccsdesc[300]{Computer systems organization~Redundancy}
%\ccsdesc{Computer systems organization~Robotics}
%\ccsdesc[100]{Networks~Network reliability}
%
%%%
%%% Keywords. The author(s) should pick words that accurately describe
%%% the work being presented. Separate the keywords with commas.
%\keywords{datasets, neural networks, gaze detection, text tagging}
%%%
%% This command processes the author and affiliation and title
%% information and builds the first part of the formatted document.
\maketitle

\section{Introduction}
%!TEX root = /Users/fgeerts/Documents/MLforloops/pods/main.tex
%With the raise of Machine Learning applications we have witnessed in recent years, there is an increasing need for languages that allow us to reason about matrix operations. 

%Matrix operations are at the very core of most Machine Learning algorithms deployed these days. Due to this, there is an increasing need to define languages allowing us to reason about matrix manipulation and more generally, about linear algebra. Several such proposals have been put forward by the database community, mostly focusing on how matrices are manipulated statically. On the other hand, many linear algebra operations require some sort of iteration or recursion. Examples of these are computing the Gaussian elimination, the inverse of a matrix, or its transitive closure.
%
%By observing that the recursion required in these algorithms amounts to repeating 
Linear algebra-based algorithms have become a key component in data analytical workflows. As such, there is growing interest in the database community to integrate linear algebra functionality in relational database management systems \cite{}. In particular, from a query language perspective, several proposals have recently been put forward to unify relational algebra and linear algebra. One such a proposal is LARA~\cite{}, a minimalistic language in which a number of atomic operations on so-called associative tables are proposed. Since such tables generalize relational tables, matrices and tensors, the aimed unification of linear and relational algebra is obtained. A key operator in LARA is the extension operator, which is parametrized by (user-defined) functions. Hence, the capabilities of LARA are inherently tied to the allowed functions. Fragments of LARA are known to be expressive complete to first-order logic with aggregation~\cite{}. Furthermore, under complexity-theoretic assumptions, LARA can not compute the inverse of a matrix.\cite{}. This is primarily due to the absence of a recursion mechanism in LARA. 


Another proposal, on which we will built on, is \lang, a pure matrix query language \cite{}. It consists of a number of atomic linear algebra operators, such as matrix transposition, multiplication and addition, and multiplication of a matrix by a scalar, just to name a few. Also here there are close connections to first-order logic with aggregates. In fact, \lang\ is subsumed in the three-variable fragment of that logic, resulting in the inability of \lang\ to check for four-cliques in adjacency matrices and to compute the inverse of a matrix. Furthermore, \lang\ was recently shown to be equivalent to a restricted version of the annotated relational algebra (ARA) \cite{}. The latter is alike the relational algebra on $K$-relations \cite{}, with $K$ a semiring, and  where input and output relations are restricted to be binary (to make a crisp connection to matrices) and at most three distinct attributes can be used. 

When faced with complex linear algebra procedures, LARA requires the introduction of new user-defined functions to be used in the extension function, and similarly, \lang\ requires the addition of new atomic linear algebra operations. This begs the question which and how many such functions or operators need to be added before these languages can accommodate for 
linear algebra procedures commonly used in practice. In other words, what is the yardstick with which to compare the expressive power of such languages? 

We take as answer to this question inspiration from complexity theory, where it is often said that ``efficient'' arithmetic circuits are sufficient to capture most of linear algebra \cite{}. An arithmetic circuit is like a boolean circuit but instead of representing a boolean function, it represents a polynomial in real variables. As such, an efficient arithmetic circuit can be regarded as one that corresponds to a polynomial of bounded degree, i.e., a degree bounded by a polynomial in the number of variables. Such circuits thus seem well-suited as a yardstick for comparing  linear-algebra based query languages. It leads to the question whether a linear algebra query language can be defined that matches (polynomial degree) arithmetic circuits in expressive power.

The main contribution of this paper is, yes we can. We propose an extension of \lang, referred to by as \langfor, for which can identify a sub-fragment that is equivalent to arithmetic circuits of polynomial degree. Here, intuitively, the sub-fragment consists of \langfor-expression that can be compiled into a polynomial degree circuit. As a consequence, \langfor\ inherits all expressiveness properties of circuits, and thus can compute the determinant of matrix, perform matrix inversion, can compute the characteristic polynomial of a matrix, and is able to solve linear systems of equations.

The key difference in the design of \langfor, compared to the two previous approaches, is that we introduce a limited form of recursion. As our choice of recursion mechanism, we take inspiration from textbooks on linear algebra algorithms \cite{}, i.e., we introduce \texttt{for} loops to \lang. Indeed, most algorithm are based on \texttt{for} loops that iterate over the indices of matrices involved. We illustrate how \langfor\ works by an example.
%
%
% \floris{This is example is getting too complicated! Any suggestions?}
% % %
% \begin{example}
% Consider a linear system of equations $L\cdot x=a$ with $L$ a matrix of dimensions $n\times n$, $a$ a vector of dimension $n\times 1$, and $x$ a vector of variables of dimensions $n\times 1$. Furthermore, assume that $L$ is a non-singular lower triangular matrix, i.e., all entries above the diagonal are zero and all entries on the diagonal are non-zero. To solve the system for $x$, it suffices to apply forward substitution, i.e.,
% $$	x_1:= a_1, \quad
% x_i:= \frac{1}{L_{ii}}\left(a_i -\sum_{j=1}^{i-1}L_{ij}a_j\right) \quad i\in[2,n]$$
% To view this procedure as a query in \langfor we proceed as follows.
% We use a matrix variable $M$ to store $L$ and a vector variable $V$ to store $a$.
% Furthermore, we reserve two special vector variables $v$ and $w$ which will range over
% the canonical vectors of the same dimension as $L$ and $a$. More specifically, they range
% over $b_1=(1,0,\ldots,0)$, $b_2=(0,1,0,\ldots,0)$,\ldots, $b_n=(0,\ldots,0,1)$ \textit{in this order}. We further have a special variable $X$ to hold intermediate results. In this example,
% $X$ will hold the solution $x$ for the system of equations at the end of the evaluation. We also
% require a second variable $Y$ which will hold $\sum_{j=1}^{i-1}L_{ij}a_j$ for a given $i$. Finally,
% we also allow the use the division function $f_/(x,y)=x/y$.
%
% We need to more operators in this example:
% the operator $\mathsf{min}(v)$ such that $\mathsf{min}(v):=1$ if $v=b_1$ and $\mathsf{min}(v):=0$ otherwise, and the $\mathsf{pred}^+(v,w)$ such that $\mathsf{pred}^+(b_j,b_i)=1$
%  if $j<i$ and $\mathsf{pred}^+(b_j,b_i)=0$ otherwise. These identify the first canonical vector and a strict predecessor relation on canonical vectors, respectively, We show later in the paper that these can be expressed in \langfor. Given these, we consider the following \langfor\ expressions:
% \begin{tabbing}
% $e_1(M,V,v)$\=:=\texttt{for}$\,w,Y\,.\, Y + \mathsf{pred}^+(w,v) (v^T\cdot M\cdot w)(w^T\cdot V)\times v$\\
% $e(M,V)$\=:=\texttt{for}$\,v,X\,.\, X + (\mathsf{min}(v)(v^T\cdot V)\times v) +(1-\mathsf{min}(v))e_1(M,V,v)$
% \end{tabbing}
% \end{example}
% % %


\medskip
\noindent
\textbf{Related work.} 

\begin{itemize}
\item MATLANG, LARA, SIMQL.

\item K-Relations, ARA, FAQ.

\item Weighted Logics, ...

\item Logics for linear algebra: Grohe (bounded lfp), Dawar (), Pakusa, Holm.

\item Algoritmic side: for loop optimization AGM style.
\end{itemize}


\subsection{old stuff}
% In another line of work~\cite{}, matrices are encoded as relational tables and an extensions of SQL is proposed to carry out matrix manipulations. In particular, \cite{} extends SQL with a limited form of recursion -- alike dynamic programming - such that linear algebra-based procedures for learning feed-forward neural networks can be declaratively specified. To our knowledge, the precise expressive power of the resulting language has not been characterized. For example, it is unclear whether matrix inversion can be expressed.
%
% Based on these works, a natural question arises: how to add a natural form of recursion to linear algebra-based query languages? Inspecting any linear algebra textbook, one sees that most linear algebra procedures heavily rely on the use of for-loops in which iterations happen over the dimensions of the matrices involved. We thus propose to extend \lang\ with limited recursion in the form of for-loops, resulting the language \langfor. To define this recursion in a natural way, we simulate a loop of the form \texttt{for i=1,...n do} by leveraging canonical vectors. In other words, we use the canonical vectors $b_1=(1,0,\ldots)$, $b_2=(0,1,\ldots)$, \ldots, to access specific rows and columns, and iterate over these vectors. As an almost direct consequence, the expressive power of \langfor goes well beyond \lang. It can check for cliques of any given size, compute the transitive closure of a graph, and as we will show, compute important linear algebra operators such as LU-decomposition, determinant, matrix inverse, among other things.
%
% More generally, we show that \langfor\ is closely related to arithmetic circuits and we show that anything computable by an arithmetic circuit of polynomial degree can be computed in \langfor, and vice versa, provided that \langfor expressions can be compiled, in a uniform way into arithmetic circuits. Since these circuits are often said to ``capture'' linear algebra, we see our results as as a justification for our language.
%
% Furthermore, the introduction of recursion to \lang has some interesting consequences. First of all, when consecutive iterations can only perform updates in an additive way, we show that \langfor and the annotated relation algebra are equivalent. Secondly, when iterations update in a multiplicative manner,
% \langfor is equivalent to weighted logics.
%


\begin{itemize}
\item Explain why this is important.
\item Say what sort of things we would like to express.
\item Give a quick tour of our minimal language (examples).
\item Stress our concrete contributions.
\item Relate to previous work \cite{matlang,BrijderGBW19,Geerts19,HutchisonHS17}.
\end{itemize}

\domagoj{A good example for the intro is all shortest paths via Floyd-Warshall.}

\noindent
$\ffor{e_k}{\Dist}{ }$
\\
\hspace*{0.5cm} $\ffor{e_i}{\Dist}{ }$
\\
\hspace*{1cm} $\ffor{e_j}{\Dist}{ }$
\\
\hspace*{1.5cm} 
$\texttt{curr} := e_i^*\cdot \Dist \cdot e_j$\\
\hspace*{1.5cm} 
$\texttt{new} := e_i^*\cdot \Dist \cdot e_k + e_k^*\cdot \Dist\cdot e_j$\\
\hspace*{1.5cm}
$\Dist + \texttt{update}(\texttt{curr},\texttt{new})\times (e_i\cdot e_j^*)$

where

\[
  			\texttt{update}(x,y)=\begin{cases}
               0, \text{ if } x<=y \\
               -x + y, \text{ if } x > y
            \end{cases}.
		\]




\section{MATLANG}\label{sec:matlang}
Recall the basics of MATLANG \cite{matlang}, and linear algebra. Perhaps stress where MATLANG falls short with respect to natural linear algebra questions.

\section{Extending MATLANG with for loops}\label{sec:formatlang}
%!TEX root = ../main.tex
% !TeX spellcheck = en_US

To extend \lang\ with recursion, we take inspiration from classical linear algebra algorithms, such as those described in \cite{num}. Many of these algorithms are based on \textit{for-loops} in which the termination condition for each loop is determined by the matrix dimensions. We have seen how the transitive closure of a matrix can be computed using for-loops in the Introduction. Here we add this ability to \lang, and show that the resulting language, called $\langfor,$ can compute properties outside of the scope of \lang. We see more advanced examples, such as Gaussian elimination and matrix inversion, later in the paper. 

\subsection{Syntax and semantics of \langfor} The syntax of \langfor is defined just as for \lang but with an extra rule in the grammar:
\medskip

\begin{tabular}{lcll}
 $\ffor{v}{X}{e}$ & (canonical for loop, with $v, X \in \Mvar$). 
\end{tabular}

\medskip
\noindent Intuitively, $X$ is a matrix variable which is iteratively updated according to the expression $e$. We simulate iterations of the form ``\texttt{for} $i\in [1..n]$'' by letting $v$ loop over the \textit{canonical vectors} $b_1^n,\ldots,b_n^n$ of dimension $n$. Here,
$b_1^n = [1\ 0 \cdots 0]^T$, $b_2^n = [0\ 1\ 0 \cdots 0]^T$, etc. When $n$ is clear from the context we simply write $b_1,b_2,\ldots$. In addition, the expression $e$ in the rule above may depend on $v$. 

We next make the semantics precise and start by
declaring the type of loop expressions.
Given a schema $\Sch=(\Mnam,\size)$, the type of a \langfor expression $e$, denoted $\ttype(e)$, is defined inductively as in \lang but with following extra rule:
\begin{itemize}
\item $\ttype(\ffor{v}{X}{e}) := (\alpha,\beta)$, if \\
$\ttype(e)=\ttype(X) =(\alpha,\beta)$ and $\ttype(v) = (\gamma,1)$.
\end{itemize}
We note that $\Sch$ now necessarily includes $v$ and $X$ as variables and assigns size symbols to them.
We also remark that in the definition of the type of $\ffor{v}{X}{e}$, we require that $\ttype(X) = \ttype(e)$ as this expression updates the content of the variable $X$ in each iteration using the result of $e$. We further restrict the type of 
$v$ to be a vector, i.e., $\ttype(v)=(\gamma,1)$, since $v$ will be instantiated with canonical vectors.
A \langfor\ expression $e$ is well-typed over a schema $\Sch=(\Mnam,\size)$ if its type is defined. 

For well-typed expressions we next define their semantics. This is done in an inductive way, just as for \lang. To define the semantics of $\ffor{v}{X}{e}$ over an instance $\I$, we need the following notation. Let $\I$ be an instance and $V\in \Mnam$. Then $\I[V_1 := A_1, \ldots, V_l:=A_l]$ denotes an instance that coincides with $\I$, except that the value of the matrix variable $V_i$ is given by the matrix $A$, for $i\in\lbrace 1, \ldots, l\rbrace$. Assume that
$\ttype(v)= (\gamma,1)$, and $\ttype(e) = (\alpha,\beta)$ and $n := \dom(\gamma)$. Then, $\sem{\ffor{v}{X}{e}}{\I}$ is defined iteratively, as follows:
\begin{itemize}
\item Let $A_0 := \mathbf{0}$ be the zero matrix of size $\dom(\alpha)\times \dom(\beta)$.
\item For $i=1,\ldots n$, compute $A_i:= \sem{e}{\I[v := b^{n}_i, X:= A_{i-1}]}$.
\item Finally, set $\sem{\ffor{v}{X}{e}}{\I}:= A_{n}$.
\end{itemize}

For better understanding how \langfor  works, we next provide some  examples.
We start by showing that the one-vector and $\diag$ operators are redundant
in \langfor.
\begin{example}\label{ex:onevec}
We first show how the one-vector operator $\ones(e)$ can be expressed using \texttt{for} loops.
It suffices to consider the expression
$$e_{\ones}:=\ffor{v}{X}{X+v},$$
with $\ttype(v)=(\alpha,1)=\ttype(X)$ if $\ttype(e)=(\alpha,\beta)$. This expression is well-typed
and is of type $(\alpha,1)$. When evaluated over some instance $\I$ with $n=\dom(\alpha)$, $\sem{e_{\ones}}{\I}$ is defined as follows.
Initially, $A_0:=\mathbf{0}$. Then $A_i:=A_{i-1}+b_i^n$, i.e., the $i$th canonical vector is added to $A_{i-1}$.
Finally, $\sem{e_{\ones}}{\I}:=A_n$ and this now clearly coincides with $\sem{\ones(e)}{\I}$.\qed
\end{example}

\begin{example}\label{ex:diag}
We next show that the $\diag$ operator is redundant in \langfor.
Indeed, it suffices to consider the expression
$$e_{\mathsf{diag}}:=
\ffor{v}{X}{X + (v^T\cdot e) \times v\cdot v^T},$$ where $e$ is a \langfor\  expression of type $(\alpha,1)$. For this expression to be well-typed, $v$ has to be a vector variable of type $\alpha\times 1$ and $X$ a matrix variable of type $(\alpha,\alpha)$. Then, $\sem{e_{\mathsf{diag}}}{\I}$ is defined as follows.
Initially, $A_0$ is the zero matrix of dimension $n\times n$, where $n=\dom(\alpha)$. Then, in each iteration
$i\in[1..n]$, $A_{i}:=A_{i-1}+  ((b_i^n)^T\cdot\sem{e}{\I})\times (b_i^n\cdot (b_i^n)^T)$. In other words, $A_i$ is obtained by adding the matrix with value $(\sem{e}{\I})_i$ on position $(i,i)$ to $A_{i-1}$. Hence, $\sem{e_{\mathsf{diag}}}{\I}:=A_n=\sem{\diag(e)}{\I}$.\qed
 \end{example}

These examples illustrate that we can limit \langfor to consist of the following ``core'' operators: transposition, matrix multiplication and addition, scalar multiplication, pointwise function application, and for-loops. More specific, \langfor is defined by the following simplified syntax:
$$
e ::= V \ \mid \ e^T \!\ \mid \ e_1 \cdot e_2 \ \mid \ e_1 + e_2 \ \mid \ e_1\times e_2  \ \mid \  f(e_1,\ldots ,e_k) \ \mid \ \ffor{v}{X}{e}
$$
Similarly as for \lang, we write $\langforf{\Fun}$ for some set $\Fun$ of functions when these are required for the task at hand.

As a final example, we show that we can compute whether a graph contains a 4-$\textsf{clique}$ using \langfor.
\begin{example}\label{ex:fourcliques}
To test for $4$-cliques it suffices to consider the following expression with for-loops nested four times:
\begin{tabbing}
\texttt{for\,}\=$u,\,X_1.\ \ X_1 \ + $\\
\> \texttt{for\,}\=$v,\,X_2.\ \ X_2 \ +$ \\
\>\>\texttt{for\,}\=$w,\,X_3.\ \ X_3 \ +$ \\
\>\>\>\texttt{for\,}\=$x,\,X_4.\ \ X_4 \ +$ \\
\>\>\>\>$u^T\cdot V\cdot v \cdot u^T\cdot V\cdot w\cdot u^T\cdot V\cdot x \cdot $\\
\>\>\>\>$v^T\cdot V\cdot w \cdot v^T\cdot V\cdot x\cdot w^T\cdot V\cdot x \cdot g(u,v,w,x)$
\end{tabbing}
with $g(u,v,w,x)=f(u,v)\cdot f(u,w)\cdot f(u,x)\cdot f(v,w)\cdot f(v,x)\cdot f(w,x)$ and
$f(u,v)=1-u^T\cdot v$. Note that $f(b_i^n,b_j^n)=1$ if $i\neq j$ and $f(b_i^n,b_j^n)=0$ otherwise.
Hence, $g(b_i^n,b_j^n,b_k^n,b_\ell^n)=1$ if and only if all $i,j,k,l$ are pairwise different.
When evaluating the expression on an instance $\I$ such that $V$ is assigned to the adjacency 
matrix of a graph, the expression above evaluates to a non-zero value if and only if the graph
contains a four-clique.\qed
\end{example}

Given that \lang can not check for 4-cliques \cite{matlang-journal}, we easily obtain the following.

\begin{proposition}
\label{cor-ml-fml}
For any collection of functions $\Fun$, 
$\langf{\Fun}$ is properly subsumed by $\langforf{\Fun}$.
\end{proposition} 

\subsection{Design decisions behind \langfor}\label{sec:formatlang:design}
\noindent\textbf{Loop Initialization.} As the reader may have observed, in the semantics of for-loops we 
always initialize $A_0$ to the zero matrix~$\mathbf{0}$ (of appropriate dimensions). It is often convenient
to start the iteration given some concrete matrix  originating from the result of evaluation a \langfor\ expression $e_0$. To make this explicit, we write $\initf{e_0}{v}{X}{e}$ and its semantics is defined as above
with the difference that $A_0:=\sem{e_0}{\I}$. We observe, however, that $\initf{e_0}{v}{X}{e}$ can already
be expressed in \langfor. In other words, we do not loose generality by assuming an initialization of $A_0$ by $\mathbf{0}$.
The key insight is that in \langfor\ we can check during evaluation whether or not
the current canonical vector $b_i^n$ is equal to $b_1^n$. This 
is due to the fact that for-loops iterate over the canonical vectors in a fixed order. We discuss this more in the next paragraph.
In particular, we can define a \langfor expression $\mmin$, which when evaluated on an instance, returns $1$ if its input vector is $b_1^n$, and returns $0$ otherwise. Given $\mmin$, consider now the
\langfor\ expression
 $$\ffor{v}{X}{\mmin{v}\cdot e(v,X/e_0) + (1-\mmin{v})\cdot e(v,X)},$$
 where we explicitly list $v$ and $X$ as matrix variables on which $e$ potentially depends on, and where
 $e(v,X/e_0)$ denotes the expression obtained by replacing every occurrence of $X$ in $e$ with $e_0$.
When evaluating this expression on an instance $\I$, $A_0$ is initial set to the zero matrix, in the first iteration (when  $v=b_1^n$ and thus $\mmin{v}=1$)
we have $A_1=\sem{e}{\I[v:=b_1^n,X:=\sem{e_0}{\I}]}$, and for consecutive iterations (when only the part related to $1-\mmin{v}$ applies) $A_i$ is updated as before. Clearly, the result of this evaluation is equal to
$\sem{\initf{e_0}{v}{X}{e}}{\I}$. 


As an illustration, we consider the Floyd-Warshall algorithm given in the Introduction. 

\begin{example}\label{ex:floyd}
Consider the following expression:
\begin{tabbing}
$e_{FW} := $ \texttt{for\,}\=$v_k,\, X_1\!=\!A.\ \ X_1 \ + $\\
\> \texttt{for\,}\=$v_i, \, X_2.\ \ X_2 \ +$ \\
\>\>\texttt{for\,}\=$v_j,\, X_3.\ \ X_3 \ +$ \\
\>\>\>$(v_i^T\cdot X_1\cdot v_k \cdot v_k^T\cdot X_1\cdot v_j)\times v_i\cdot v_j^T$
\end{tabbing}
The expression $e_{FW}$ simulates the Floyd-Warshall algorithm by updating the matrix $A$, which is stored in the variable $X_1$. The inner sub-expression here constructs an $n\times n$ matrix that contains one in the position $(i,j)$ if and only if one can reach the vertex $j$ from $i$ by going through $k$, and zero elsewhere. If an instance $\I$ assigns to $A$ the adjacency matrix of a graph, then $\sem{e_{FW}}{\I}$ will be equal to the matrix produced by the algorithm given in the Introduction.
\qed
\end{example}


\noindent\textbf{Order.} By introducing for-loops we not only extend \lang\ with bounded recursion, we also introduce order information. Indeed, the semantics of the \texttt{for} operator assumes that the canonical vectors $b_1,b_2,\ldots$
are accessed in this order. It implies, among other things, that \langfor\ expressions are not permutation-invariant.
We can, for example, return the bottom right-most entry in a matrix. Indeed, consider the expression $e_{\mathsf{max}} := \ffor{v}{X}{v}$ which, for it to be well-typed, requires both $v$ and $X$ to be of type $(\alpha,1)$. Then, $\sem{e_{\mathsf{max}}}{\I}=b_n^n$, for $n=\dom(\alpha)$, simply because initially, $X=\mathbf{0}$, but $X$ will be overwritten by $b_1^n,b_2^n,\ldots,b_n^n$, in this order. Hence, at the end of the evaluation $b_n^n$ is returned.
To extract the bottom right-most entry from a matrix, we now simply use $e_{\mathsf{max}}^T\cdot V\cdot e_{\mathsf{max}}$.

Although the order is implicit in \langfor, we can explicitly use this order in \langfor expressions. More precisely, the order on canonical vectors is made accessible by
using the matrix:
\[
S_{\leq} = \begin{bmatrix}
1 & 1 & \cdots &  1 \\
0 & 1 & \cdots & 1\\
\vdots & \vdots & \ddots & 1 \\
0 & 0 & \cdots & 1 
\end{bmatrix}.
\] 
We observe that $S_{\leq}$ has the property that $b_i^T\cdot S_{\leq} \cdot b_j=1$, for two canonical vectors $b_i$ and $b_j$ of the same dimension, if and only if $i\leq j$. Otherwise, $b_i^T\cdot S_{\leq} \cdot b_j=0$. 
Interestingly, we can build the matrix $S_{\leq}$ with the following \langfor expression:
$$
e_{\leq}=\ffor{v}{X}{X + \left((X\cdot e_{\mathsf{max}}) + v \right)\cdot v^T + v\cdot e^T_{\mathsf{max}}},
$$
where $e_{\mathsf{max}}$ is as defined above. The intuition behind this expression is that by using the last canonical vector $b_n$, as returned by $e_{\mathsf{max}}$, we have access to the last column of $X$ (via the product $X\cdot e_{\mathsf{max}}$). We use this column such that after the $i$-th iteration, this column contains the $i$-th column of $S_{\leq}$. This is done by incrementing $X$ with $v\cdot e_{\mathsf{max}}^T$.
To construct $S_{\leq}$, in the $i$-th iteration we further increment $X$ with 
(i)~the current last column in $X$ (via $X\cdot e_{\mathsf{max}}\cdot v^T$) which holds
the $(i-1)$-th column of $S_{\leq}$; and (ii)~the current canonical vector (via $v\cdot v^T$). Hence, after iteration $i$, $X$ contains the first $i$ columns of $S_{\leq}$ and holds the $i$th column of $S_{\leq}$ in its last column. It is now readily verified that $X=S_{\leq}$ after the $n$th iteration.

It should be clear that if we can compute $S_{\leq}$ using $e_{\leq}$, then we can easily define the following predicates and vectors related with the order of canonical vectors:
\begin{itemize}
	\item $\mathsf{succ}(u,v)$ such that $\mathsf{succ}(b_i^n,b_j^n)=1$ if $i\leq j$ and $0$ otherwise. Similarly, we can define
	$\mathsf{succ}^+(u,v)$ such that  $\mathsf{succ}^+(b_i^n,b_j^n)=1$ if $i < j$ and $0$ otherwise;
	\item $\mathsf{min}(u)$ such that  $\mathsf{min}(b_i^n)=1$ if $i=1$ and $\mathsf{min}(b_i^n)=0$ otherwise; 
	\item $\mathsf{max}(u)$ such that  $\mathsf{max}(b_i^n)=1$ if $i=n$ and $\mathsf{min}(b_i^n)=0$ otherwise; and
	\item $e_{\mathsf{min}}$ and $e_{\mathsf{max}}$ such that $\sem{e_{\mathsf{min}}}{\I}=b_1^n$ and 
	$\sem{e_{\mathsf{max}}}{\I}=b_n^n$, respectively.
\end{itemize}

\input{./sections/definitions/order-predicates.tex}

Having order information available results in \langfor to be quite expressive.
We heavily rely on order information in the next sections to compute the inverse of matrices and more generally to simulate low complexity Turing machines and arithmetic circuits.


%\section{Order}\label{sec:order}
%%
%
% \noindent{\bf Order.}
With the introduction of \texttt{for} loops we not only extend \lang\ with bounded recursion, we also introduce order information. Indeed, the semantics of the \texttt{for} operator assumes that the canonical vectors $b_1,b_2,\ldots$
are accessed in this order. It implies, among other things, that \langfor\ expressions are not permutation-invariant.
We can, for example, return the bottom right-most entry in a matrix. Indeed, consider the expression $v_{max} := \ffor{v}{X}{v}$ which, for it to well-typed requires both $v$ and $X$ to be of type $(\alpha,1)$. Then, $\sem{v_{max}}{\I}=b_n^n$, for $n=\dom(\alpha)$, simply because the $X=\mathbf{0}$ initially, but $X$ will be overwritten by $b_1^n,b_2^n,\ldots,b_n^n$, in this order. Hence, at the end of the evaluation $b_n^n$ is returned.
To extract the bottom right-most entry we now simply use $v_{max}^T\cdot M\cdot v_{max}$.


%
% assumes that canonical vectors come in some particular order.

In fact, although the order is implicit in \langfor\, we can explicitly use this order in \langfor\ expressions. More precisely, one can the define the predicates in \langfor:
\begin{itemize}
	\item $\mathsf{succ}(u,v)$ such that when evaluated on $\I$, $\mathsf{succ}(b_i^n,b_j^n)=1$ if $i\leq j$ and $\mathsf{succ}(b_i^n,b_j^n)=0$ otherwise;
	$\mathsf{ssucc}(u,v)$ such that when evaluated on $\I$, $\mathsf{succ}(b_i^n,b_j^n)=1$ if $i < j$ and $\mathsf{succ}(b_i^n,b_j^n)=0$ otherwise;
	\item $\mathsf{min}(u)$ such that when evaluated on $\I$, $\mathsf{min}(b_i^n)=1$ if $i=1$ and $\mathsf{min}(b_i^n)=0$ otherwise; and
	\item $\mathsf{max}(u)$ such that when evaluated on $\I$, $\mathsf{max}(b_i^n)=1$ if $i=n$ and $\mathsf{min}(b_i^n)=0$ otherwise.	
	\end{itemize}
The definitions of these expressions is not entirely trivial and are detailed in the appendix.
We here only highlight that successor information on canonical vectors is made accessible by
using the following matrix:
%
% To define an order relation for canonical vectors, notice that the following matrix:
\[
Z_{eq} = \begin{bmatrix}
    1 & 1 & \cdots &  1 \\
    0 & \ddots & \ddots & \vdots \\
    \hdotsfor{3} & 1 \\
    0 & \cdots & \cdots & 1 
\end{bmatrix}.
\] 
We observe that $Z_{eq}$ has the property that $b_i^T\cdot Z_{eq} \cdot v_j$, for two canonical vectors $b_i,b_j$ of the same dimension, is equal to one if and only $i\leq j$, and is zero otherwise. Now, $Z_{eq}$ can be defined in \langfor as follows:
$$Z_{\text{eq}}=\ffor{v}{X}{X + \left((X\cdot v_{max}) + v \right)\cdot v^T + v\cdot v^T_{max}},$$
where $v_{max}$ is as defined above. The intuition behind this expression is that using the last canonical vector $v_{max}$, we have access to the final column of $X$ (via the product $X\cdot v_{max}$), to which we add the current canonical vector $v$, thus constructing $Z_{eq}$ by filling it column by column.

\floris{I commented out the details about the order predicates. Let's put this in the appendix.
I wonder what else we can say here. Clearly, the order is crucial for all our results later on.
We could comment on order-invariant \langfor expressions (probably undecidable)? Any suggestions?
}
%
%  expressions
%
%  however, it does not give us an immediate access to this order. An interesting question is whether, using the \texttt{for} loops, we can define a \langfor expression which allow us to define the order of canonical vectors. Next we show that this is indeed possible.
%
%  \floris{I am here. Not sure how insightful details about the encodings are. Should we just state that we can do it?}
%  % To begin with, we can easily obtain the last canonical vector using the expression $$v_{max} := \ffor{v}{X}{v}$$
% %
% % The fundamental property of iteration we use here is that the result variable is always initiated with the null matrix. Therefore the loop above will simply keep on storing the current canonical vector before returning the final one. The ability to do this sort of manipulation was one of the reasons why we initiate $A_0$ in the semantics of \texttt{for} to null matrix.
%
% To define an order relation for canonical vectors, notice that the following matrix:
% \[
% Z_{eq} = \begin{bmatrix}
%     1 & 1 & \cdots &  1 \\
%     0 & \ddots & \ddots & \vdots \\
%     \hdotsfor{3} & 1 \\
%     0 & \cdots & \cdots & 1
% \end{bmatrix},
% \]
% has the property that $e_i^*\cdot Z_{eq} \cdot e_j$, for two canonical vectors $e_i,e_j$ of the same dimension, is equal to one if and only $i\leq j$, and is zero otherwise. $Z_{eq}$ can easily de defined in \langfor as follows:
% $$Z_{\text{eq}}=\ffor{v}{X}{X + \left[ (Xv_{max}) + v \right]v^* + vv^*_{max}},$$
% where $v_{max}$ is as defined above. The intuition behind this expression is that using the last canonical vector $v_{max}$, we have access to the final column of $X$ (via the product $X\cdot v_{max}$), to which we add the current canonical vector $v$, thus constructing $Z_{eq}$ by filling it column by column. By defining $$\lleq{u}{v} := u^*\cdot Z_{eq} \cdot v,$$
% we obtain an order relation that allows us to discern whether one canonical vector comes before the other in the order given by \texttt{for}. If we want a strict order, we can just use $Z_< := Z_{eq} - I$.
%
% Interestingly, we can also define the predecessor relation between canonical vectors. For this, we require the following matrix:
% %\[
% %S := \begin{bmatrix}
% %    0 & 1         & 0         & \cdots &  0 \\
% %    0 & \ddots & 1         & \cdots & 0 \\
% %    \vdots & \vdots & \ddots & \ddots & \vdots \\
% %    \hdotsfor{3} & \ddots & 1 \\
% %    0 & \cdots & \cdots & \cdots & 0
% %\end{bmatrix}.
% %\]
% \[
% S = \begin{bmatrix}
%     0 & 1 & \cdots &  0 \\
%     0 & \ddots & \ddots & \vdots \\
%     \hdotsfor{3} & 1 \\
%     0 & \cdots & \cdots & 0
% \end{bmatrix},
% \]
% Using this matrix, we have that for a canonical vector $e_i$:
% \[
%   			S\cdot e_i=\begin{cases}
%                e_{i-1}, \text{ if } i > 1 \\
%               \mathbf{0}, \text{ if } i = 1
%             \end{cases}
% 		\]
% where $\mathbf{0}$ is a vector of zeros of the same type as $e_i$. Notice also that $\ones(v)^*\cdot S \cdot v$ is equal to zero, for a canonical vector $v$, if and only if $v = e_1$ is the first canonical vector, and zero otherwise. To define the first canonical vector in the order given by \texttt{for}, we can then write:
% $$v_{min} := \ffor{v}{X}{\mmin{v}\cdot v},$$
% where the expression $\mmin{v}$ is defined as $\mmin{v} := 1 - \ones(v)^*\cdot S \cdot v$, and, when evaluated over canonical vectors, will result in $1$ if and only if $v=e_1$ is the first canonical vector. Finally, we denote that $S$ can be defined using the following \langfor expression:
% \begin{multline*}
% S:= \texttt{for }v,X.\quad X + \\
% \left[ (1 - v^*v_{max})vv_{max}^* - (Xv_{max}) v_{max}^* + (Xv_{max})v^*\right].
% \end{multline*}
%
%
%


\section{Algorithms in Linear Algebra}\label{sec:queries}
\floris{I did not do a pass over this yet...}
The real power of \langfor comes from the fact that \texttt{for} loops allow us to express many classical linear algebra algorithms. Here we illustrate this by showing how to compute the $LU$ factorization of a matrix using Gaussian eliminations, which is one of the most commonly used matrix algorithms. Recall that $A$ is said to be LU factorizable if there exists matrices $T_1,\ldots, T_{n}$ where $T_i=E_{n}^{(i)}\cdots E_{i+1}^{(i)}$ for some elementary matrices $E_{j}^{(i)}=I+\alpha_{ij}\cdot e_{i}e_{j}^{*}$ such that $T_{n}\cdots T_1A=U$ holds, where $U$ is an upper triangular matrix.

Now, assume that $A$ is a square matrix which allows $LU$ factorization without row pivoting (we deal with this case later on). This means that after reducing the column $i$ of $A$ (i.e. we make all of the entries below the diagonal zero in the column $i$), we will end up with a matrix which has the element in position $i+1,i+1$ different from zero, for each $i$ strictly less than the dimension of $A$. 

To reduce the first column of $A$ it suffices to multiply $A$ from the left with the matrix $T_1 := I + c_1\cdot e_1^*$, where the vector $I$ is the identity matrix, $e_1$ is the first canonical vector, and $c_1$ is the vector 
\[
c_1 :=
\begin{bmatrix}
    0 \\
    \alpha_{21} \\
    \vdots \\
    \alpha_{n1}
\end{bmatrix},
\]
with $\alpha_{j1} := -\frac{A_{j1}}{A_{11}}$ is the number with which we need to multiply the first row of $A$ in order to reduce the row $j$ of $A$. More generally, if $A$ is reduced up to column $i-1$, reducing the $i$th column is achieved by computing $T_i \cdot A$, where $T_i := I + c_i\cdot e_i^*$, with $c_i$ being the vector with zeros up to position $i$, and with the value $-\frac{A_{ji}}{A_{ii}}$ in position $j > i$. To express the matrices $T_i$ in \langfor we will use the following expression:
$$\ccol{A}{y} := \ffor{v}{X}{(y^*\cdot Z_{<} \cdot v)(v^*\cdot A \cdot y)v + X},$$
which, when $y$ is interpreted by the $j$th canonical vector, computes the vector which has zeroes in positions $1\ldots j$, and values $A_{ij}$ in positions $i>j$. Intuitively, the product $y^*\cdot Z_{<}\cdot v$ makes all positions up to $j$ equal zero, and $v^*\cdot A\cdot y$ extracts the element $A_{ij}$ of $A$ when $v$ is interpreted by the $i$th canonical vector, and $y$ by the $j$th one.



Using this expression, we can now compute $T_i$s by writing:
$$\red{A}{y} := I + f_/(\ccol{A}{y},-(y^*\cdot A\cdot y)\cdot \ones(y))y^*,$$
where $f_/$ is the division function. Notice that $f_/(\ccol{A}{y},-(y^*\cdot A\cdot y)\cdot \ones(y))$ in fact computes the vectors $c_i$ defined above. The Gaussian elimination can now be performed by the following expression:
$$
U(A) :=  \left( \initf{I}{v}{X}{\red{X\cdot A}{v}\cdot X} \right) \cdot A,
$$
which computes the $U$ from the $LU$ factorization of $A$, whenever $A$ can be $LU$ factorized without row interchange. Notice that the latter assumption is crucial, since it will ensure that the element of $A$ in position $ii$ is different from zero after each \texttt{reduce} step. As explained previously, the inner \texttt{for} loop in fact computes the product of matrices $T_{n}\cdots T_1$, which in fact amounts to the matrix $L^{-1}$ from the $LU$ factorization of $A$. Given that each $T_i$ is easily invertible, we can also recover $L$.

As stated previously, the construction above works under the assumption that no row pivoting is needed when performing the Gaussian eliminations, giving us:
\begin{proposition}\label{prop:gauss}
There is a \langfor expression $e(X)$ such that by interpreting $X$ with a matrix $A$ will result in the matrix $U$ from the $LU$ factorization of $A$ whenever $A$ is $LU$ factorizable.
\end{proposition}

Now, since $A^{-1}=U^{-1}L^{-1}$ and using that $U$ is upper triangular, we can obtain the determinant and inverse of $A$ and thus


\thomas{I don't know how much of the proof goes here or if it's ok as it is.}


 \begin{proposition}\label{prop:determinant}
There is a \langfor expression $e(X)$ such that by interpreting $X$ with a matrix $A$ will result in the $1\times 1$ matrix with $\texttt{det}(A)$ in its only entry whenever $A$ is $LU$ factorizable.
\end{proposition}

\begin{proposition}\label{prop:inverse}
There is a \langfor expression $e(X)$ such that by interpreting $X$ with a matrix $A$ will result in the matrix $A^{-1}$ whenever $A$ is $LU$ factorizable.
\end{proposition}

If $A$ needs row interchange to be $LU$ factorizable (this is, $PA$ is $LU$ factorizable) we say that $A$ is $PALU$ factorizable, where the factorization is $PA=LU$. To accomplish this, we need something extra.

 \begin{proposition}\label{prop:palu}
There is a \langfor expression $e(X)$ such that by interpreting $X$ with a matrix $A$ will result in the matrix $U$ from the $PALU$ factorization of $A$ whenever $A$ is $PALU$ factorizable if and only if there exists a \langfor expression $e'(X)$ such that by interpreting $X$ with a vector $A$ outputs a canonical vector $e_k$ where $A_k$ is the first non zero entry of $A$.
\end{proposition}


\section{Expressiveness of for loops}\label{sec:circuits}
% !TeX spellcheck = en_US
%!TEX root = ../main.tex

%To avoid nasty technical stuff, we will try to connect with circuits:
%\begin{itemize}
%\item Upper bound (AC?)
%\item Lower bound (transitive closure?)
%\end{itemize}

% OTHER IDEAS ABOUT PRESENTATION
%OPTION 1:
% LINK WITH TMS FIRST (NO CIRCUITS)!
% NOW INTRODUCE CIRCUITS
% USING THIS LINK, ASSUME THAT FOR EVALUATING A CIRCUIT INFO ABOUT A GATE IS AVAILABLE THROUGH AN EFFICIENT TM (NO UNIFORMITY)
% STATE THE RESULT ABOUT SIMULATION OF A SINGLE CIRCUIT IN ML
% NOW ESTABLISH A LINK WITH UNIFORM CIRCUITS -- JUST SAY THAT WE ALREADY KNOW HOW TO SIMULATE THE MACHINE, SO ALL GOOD
% NOW GO IN THE OTHER DIRECTION WITH A MORE EXPRESSIVE CLASS OF CIRCUITS

%OPTION 2:
% DEFINE ARITHMETIC CIRCUITS AND EVALUATE A SINGLE CIRCUIT
% LINK WITH TMS WHEN ASSUMING THAT TMS ARE GIVEN FOR CIRCUIT DATA
% ASSUME THAT FOR EVALUATING A CIRCUIT INFO ABOUT A GATE IS AVAILABLE THROUGH AN EFFICIENT TM (NO UNIFORMITY)
% NOW GO IN THE OTHER DIRECTION WITH A MORE EXPRESSIVE CLASS OF CIRCUITS
% NOW ESTABLISH A LINK WITH UNIFORM CIRCUITS


%In this section we explore the expressive power of $\langfor$. Given that \langfor expressions compute functions over matrices whose entries are not only boolean, but can in fact be arbitrary elements from $\RR$, a natural candidate for comparison is the class of arithmetic circuits \cite{allender}. As we show in the remainder of this section, \langfor captures the expressive power of arithmetic circuits. 
%To show this result, we first recall the definition of arithmetic circuits.
%
%In order to derive this result, we first look at the connection of \langfor and Turing machines. When observing a \langfor expression over some schema, perhaps the most crucial characteristic is the fact that depending on the instance, the number of iterations that a for loop does changes, as does the size of the matrix computed by our expression. This is analogous to how the number of steps a Turing machine takes changes depending on the input size, and gives some intuition on why \langfor expressions might be able to simulate Turing machines. Indeed, what we show is that \langfor expressions can actually  simulate Turing machines that use linear space and run in polynomial time. Formally, we prove the following.
%
%\begin{theorem}
%\label{th-tm-ml}
%Let $T$ be a Turing machine with $\ell$ read only input tapes, a work tape, and an output tape. 
%\end{theorem}
%
%Next we move to comparison with arithmetic circuits. 

In this section we explore the expressive power of $\langfor.$ Given that arithmetic circuits \cite{allender} capture most standard linear algebra algorithms \cite{Raz02,ShpilkaY10}, they seem as a natural candidate for comparison. Intuitively, an arithmetic circuit is similar to a boolean circuit \cite{aroraB2009}, except that it has gates computing the sum and the product function, and processes elements of $\RR$ instead of boolean values. To connect \langfor to arithmetic circuits we need a notion of uniformity of such circuits. After all, a \langfor expression can take matrices of arbitrary dimensions as input and we want to avoid having  different circuits for each dimension. To handle inputs of different sizes, we thus consider a notion of uniform families of arithmetic circuits, defined via a Turing machine generating a description of the circuit for each input size $n$.

What we show in the remainder of this section is that any function $f$ which operates on matrices, and is computed by a uniform family of arithmetic circuits of bounded degree, can also be computed by a \langfor expression, and vice versa. In order to keep the notation light, we will focus on 
%circuits that take $n$ inputs and have a single output, and
 \langfor schemas over ``square matrices'' where each variable has type $(\alpha,\alpha),(\alpha,1),(1,\alpha)$, or $(1,1)$, although all of our results hold without these restrictions as well. In what follows, we will write $\langfor$ to denote $\langforf{\emptyset}$, i.e. the fragment of our language with no additional pointwise functions. We begin by defining circuits and then show how circuit families can be simulated by $\langfor.$

\subsection{From arithmetic circuits to \langfor}
Let us first recall the definition of arithmetic circuits. 
An \textit{arithmetic circuit} $\Phi$ over a set $X=\{x_1,\ldots,x_n\}$ of input variables is a directed
acyclic labeled graph. The vertices of $\Phi$ are called \textit{gates} and denoted by $g_1,\ldots,g_m$;
the edges in $\Phi$ are called \textit{wires}. The children of a gate $g$ correspond to all gates
$g'$ such that $(g,g')$ is an edge. The parents of $g$ correspond to all gates $g'$ 
such that $(g',g)$ is an edge. The \textit{in-degree}, or a \textit{fan-in}, of a gate $g$ refers to its number of children, and 
the \textit{out-degree} to its number of parents. We will not assume any restriction on the in-degree of a gate, and will thus consider circuits with unbounded fan-in. Gates with in-degree $0$ are called \textit{input gates}
and are labeled by either a variable in $X$ or a constant $0$ or $1$. All other gates
are labeled by either $+$ or $\times$, and are referred to as \textit{sum gates} or \textit{product gates}, respectively.
Gates with out-degree $0$ are called \textit{output gates}. When talking about arithmetic circuits, one usually focuses on circuits with $n$ input gates and a single output gate.\looseness=-1

The \textit{size} of $\Phi$, denoted by $|\Phi|$, is its number of gates and wires. The \textit{depth} of $\Phi$, denoted
by $\mathsf{depth}(\Phi)$, is the length of the longest directed path from any of its output gates to any of the input gates. The \textit{degree} of a gate is defined inductively: an input gate has degree~1, a sum gate has a degree equal to the maximum of degrees of its children, and a product gate has a degree equal to the sum of the degrees of its children. When $\Phi$ has a single output gate, the \textit{degree} of $\Phi$, denoted by $\mathsf{degree}(\Phi)$, is defined as the degree of its output gate. If $\Phi$ has a single output gate and its input gates take values from $\RR$, then $\Phi$ corresponds to a polynomial in $\RR[X]$ in a natural way. In this case, the {degree} of $\Phi$ equals the degree of the polynomial corresponding to $\Phi$.
%
%\floris{We may consider changes $v_1,\ldots,v_n$ to $a_1,\ldots,a_n$ is the use of $v$ is a bit overloaded. }
If $a_1,\ldots ,a_n$ are values in $\RR$, then 
%we define 
the result of the circuit on this input is the value computed by the corresponding polynomial, denoted by $\Phi(a_1,\ldots ,a_k)$.

%\domagoj{Should we already present the result without talking about uniformity?}

% MAYBE OVERCOMPLICATING
%In order to talk about efficient evaluation of arithmetic circuits in terms of complexity classes defined by Turing machines, we need a notion of uniformity of circuit families. An \textit{ arithmetic circuit family} is a set of arithmetic circuits $\{\Phi_n\mid n=1,2,\ldots\}$ where $\Phi_n$ has $n$ input variables. An arithmetic circuit family is \textit{ uniform} if there exist \logspace-Turing machines\footnote{There are several versions of this definition that give a different amount of resources to the Turing machine. See \cite{allender} for an in-depth discussion on the subject.} $M_1$ and $M_2$, such that:
%\begin{itemize}
%\item On input $1^n$, the machine $M_1$ returns an encoding of the arithmetic circuit $\Phi_n$ for each $n$;
%\item On input $1^n$, and an encoding of a gate $g$, the machine $M_2$ outputs the relevant information about $g$ (e.g. whether it is a sum or a product gate, the list of its children, whether it is an output gate, etc.).
%\end{itemize}
%
%The idea here is that the machine $M_1$ gives an overview of the circuit itself, while the machine $M_2$ allows us to efficiently evaluate the circuits by traversing the circuit graph in a depth-first fashion, starting from the output node. We observe that uniform arithmetic circuit families are necessarily of polynomial size.
%
%SHOULD EXPLAIN BRIEFLY HOW TO EVALUATE A CIRCUIT!

%In order to talk about efficient evaluation of arithmetic circuits in terms of complexity classes defined by Turing machines, we need a notion of uniformity of circuit families. 

%\floris{The following paragraph is not clear to me. I would define circuit families straightaway and drop the para below?}
%\cristian{I agree with Floris. Actually it is confusing because constructing the circuit is not about efficient evaluation.}
%In order to talk about efficient evaluation of arithmetic circuits, we need a way to compute their output algorithmically. Namely, we need a way to access their components such as the children of some gate, the value of an input gate, the operation  carried out by the gate, etc. The most general way to do this is via Turing machines. Additionally, this will allow us to handle inputs of arbitrary size, similarly as when working with Turing machines. This idea is captured by the notion of uniform circuit families. 
In order to handle inputs of different sizes, we use the notion of uniform circuit families. An \textit{arithmetic circuit family} is a set of arithmetic circuits $\{\Phi_n\mid n=1,2,\ldots\}$ where $\Phi_n$ has $n$ input variables and a single output gate. An arithmetic circuit family is \textit{uniform} if there exists a \logspace-Turing machine,
%\footnote{There are several alternatives to this definition that give a different amount of resources to the Turing machine. See \cite{allender} for an in-depth discussion on the subject.} 
which on input $1^n$, returns an encoding of the arithmetic circuit $\Phi_n$ for each $n$.
% Additionally, we assume that there is a \logspace-machine which, when started on input $1^n$, and an encoding of a gate $g$, outputs the relevant information about $g$ in $\Phi_n$ (e.g. whether it is a sum or a product gate, the list of its children, whether it is an output gate, etc.).
We observe that uniform arithmetic circuit families are necessarily of polynomial size. %, however, their degree can grow exponentially. A circuit family $\{\Phi_n\mid n=1,2,\ldots\}$ is said to be of polynomial degree if $\mathsf{degree}(\Phi_n)\in O(p(n))$, for some polynomial $p(n)$. Similarly, a circuit family is of logarithmic depth, whenever $\mathsf{depth}(\Phi_n)\in O(logn)$. We can now show that \langfor subsumes uniform arithmetic circuit families that are of polynomial degree and logarithmic depth. 
Another important parameter is the circuit depth. A circuit family is of logarithmic depth, whenever $\mathsf{depth}(\Phi_n)\in \mathcal{O}(log\, n)$. We  now show that \langfor subsumes uniform arithmetic circuit families that are of logarithmic depth. 

%\domagoj{The thing about the second machine is crucial, but it might be a bit confusing here, as we only need it to explain how one actually computes the output of a circuit. Because of this I'm thinking of maybe presenting the connection without uniformity.}

%\floris{why the $^*$ in the definition}
\begin{theorem}
\label{th-circuits-ml}
For any uniform arithmetic circuit family $\{\Phi_n\mid n=1,2,\ldots\}$ of logarithmic depth there is a \langfor schema $\Sch$ and an expression $e_\Phi$ using a matrix variable $v$, with $\ttype(v)=(\alpha,1)$ and $\ttype(e) = (1,1)$, such that for any input values $a_1,\ldots ,a_n$: 
% to the circuit $\Phi_n$:
\begin{itemize}
\item If $\I = (\dom,\conc)$ is a \lang\ instance such that $\dom(\alpha) = n$ and $\conc(v) = [a_1 \ldots a_n]^T$.
\item Then $\sem{e_\Phi}{\I} = \Phi_n(a_1,\ldots ,a_n)$.
\end{itemize}
\end{theorem}
It is important to note that the expression $e_\Phi$ does not change depending on the input size, meaning that it is uniform in the same sense as the circuit family being generated by a single Turing machine. The different input sizes for a \langfor instance are handled by the typing mechanism of the language. 

% OLD VERSION:
%\textit{Proof sketch.}
%The expression $e_\Phi$ on input $[a_1,\ldots ,a_n]^T$ basically simulates a depth-first evaluation of the arithmetic circuit $\Phi_n$. 
%In order to do this, we consider an algorithm for evaluating $\Phi_n$ on input $(a_1,\ldots ,a_n)$ that maintains  two stacks: the gates-stack that tracks the current gate being evaluated, and the values-stack that stores the value that is being computed for this gate. The idea behind having two stacks is that whenever the number of items on the gates-stack is higher by one than the number of items on the values-stack, we know that we are processing a fresh gate, and we have to initialize its current value (to 0 if it is a sum gate, and to 1 if it is a product gate), and push it to the values-stack. We then proceed by processing the children of the head of the gates-stack one by one, and aggregate the results using sum if we are working with a sum gate, and by using product otherwise. 
%
%The main challenge is how  to access the information about each gate we are processing, such as whether it is a sum or a product gate, the list of its children, or whether it is an input gate? This is where the uniformity condition comes in handy. Namely, we know that we can generate the circuit $\Phi_n$ with a \logspace-Turing machine $M_\Phi$ by running it on the input $1^n$. Using this machine, we can in fact compute all the information needed to run the two-stack algorithms described above. 
%
%For instance, we can construct a \logspace\ machine that checks, given two gates $g_1$ and $g_2$ whether $g_2$ is a child of $g_1$. Similarly, we can construct a machine that, given $g_1$ and $g_2$ tells us whether $g_2$ is the final child of $g_1$, or the one that produces the following child of $g_1$ (according to the ordering given by the machine $M_\Phi$). Defining these machines based of $M_\Phi$ is similar to the algorithm for the composition of two \logspace\ transducers, and is based on "hard-coding" the values such as $g_1$ and $g_2$ into the machine $M_\Phi$. An important detail here is also that these computations can be carried out using only linear amount of space on the output tape. We remark that this, more operational, definition of arithmetic circuits is quite commonly used when evaluating them \cite{allender}.
%
%In order to access this information in \langfor, we show that any polynomial-time Turing machine that on input $w_1,\ldots ,w_\ell\in \{0,1\}^n$ runs in linear space, and produces an output $w\in \{0,1\}^n$ of size $n$, can in fact be simulated using a \langfor expression that, when interpreted on an instance assigning $w_1,\ldots ,w_n$ to its variables, produces a vector of size $n\times 1$ corresponding to $w$.
%
%To simulate the circuit evaluation algorithm that uses two stacks, in \langfor we can use a binary matrix of size $n\times n$, where $n$ is the number of inputs. The idea here is that each gate, encoded as a binary number, is represented by the positions $1,\ldots,n-3$ of each row. The remaining three columns are reserved for the values-stack, the number of elements on the gates stack, and the number of elements on the values stack, respectively. The number of elements is encoded as a canonical vector of size $n$. Here we crucially depend on the fact that the circuit is of logarithmic depth, and therefore the size of the two stacks is bounded by $n$ (apart from the portion before the asymptotic bound kicks-in, which can be hard coded into the formula). Similarly, given that the circuits are of polynomial size, we can assume that gate ids can be encoded into $n-3$ bits.
%
%%\floris{The ``order'' information referred to below is the one of $M_{\Phi}$, right? We may want to clarify this a bit more in order not to confuse with the ordering on canonical vectors..}
%This matrix is then updated in the same way as the two-stack algorithm. It processes gates one by one, and using the successor relation for canonical vectors determines whether we have more elements on the gates stack. In this case, a new value is added to the values stack ($0$ if the gate is a sum gate, and $1$ otherwise), and the process continues. Information about the next child, last child, or input value, are obtained using the expression which simulates the Turing machine generating this data about the circuit. Given that the size of the circuit is polynomial, say $n^k$, we can initialize the matrix with the output gate only, and run the simulation of the two-stack algorithm for $n^k$ steps (by iterating $k$ times over size $n$ canonical vectors). After this, the value in position  $(1,n-2)$ (the top of the values stack) holds the final results. \qed

\domagoj{New proof sketch.}
\textit{Proof sketch.} The proof of this Theorem, which is the deepest technical result of the paper, depends crucially on two facts: (i) that any polynomial time Turing machine working within linear space and producing linear size output, can be simulated via a \langfor\ expression; and (ii) that evaluating an arithmetic circuit $\Phi_n$ can be done using two stacks of  depth $n$.

Evaluating  $\Phi_n$ on input $(a_1,\ldots ,a_n)$ can be done in a depth-first manner by maintaining  two stacks: the gates-stack that tracks the current gate being evaluated, and the values-stack that stores the value that is being computed for this gate. The idea behind having two stacks is that whenever the number of items on the gates-stack is higher by one than the number of items on the values-stack, we know that we are processing a fresh gate, and we have to initialize its current value (to 0 if it is a sum gate, and to 1 if it is a product gate), and push it to the values-stack. We then proceed by processing the children of the head of the gates-stack one by one, and aggregate the results using sum if we are working with a sum gate, and by using product otherwise. 

In order  to access the information about the gate we are processing (such as whether it is a sum or a product gate, the list of its children, etc.) we use the uniformity of our circuit family. Namely, we know that we can generate the circuit $\Phi_n$ with a \logspace-Turing machine $M_\Phi$ by running it on the input $1^n$. Using this machine, we can in fact compute all the information needed to run the two-stack algorithms described above. For instance, we can construct a \logspace\ machine that checks, given two gates $g_1$ and $g_2$, whether $g_2$ is a child of $g_1$. Similarly, we can construct a machine that, given $g_1$ and $g_2$ tells us whether $g_2$ is the final child of $g_1$, or the one that produces the following child of $g_1$ (according to the ordering given by the machine $M_\Phi$). Defining these machines based of $M_\Phi$ is similar to the algorithm for the composition of two \logspace\ transducers, and is commonly used to evaluate arithmetic circuits \citep{allender}. %Note that the output of these machine is either $1\slash 0$, for machines checking a condition, or a gate encoding, which fits into $n$ bits.

%In order to access this information in \langfor, we show that any polynomial-time Turing machine that on input $w_1,\ldots ,w_\ell\in \{0,1\}^n$ runs in linear space, and produces an output $w\in \{0,1\}^n$ of size $n$, can in fact be simulated using a \langfor expression that, when interpreted on an instance assigning $w_1,\ldots ,w_n$ to its variables, produces a vector of size $n\times 1$ corresponding to $w$.

To simulate the circuit evaluation algorithm that uses two stacks, in \langfor we can use a binary matrix of size $n\times n$, where $n$ is the number of inputs. The idea here is that  the gates-stack corresponds to the first $n-3$ columns of the matrix, with each gate being encoded as a binary number in positions $1,\ldots,n-3$ of a row. The remaining three columns are reserved for the values-stack, the number of elements on the gates stack, and the number of elements on the values stack, respectively. The number of elements is encoded as a canonical vector of size $n$. Here we crucially depend on the fact that the circuit is of logarithmic depth, and therefore the size of the two stacks is bounded by $n$ (apart from the portion before the asymptotic bound kicks-in, which can be hard-coded into the expression $e_\Phi$). Similarly, given that the circuits are of polynomial size, we can assume that gate ids can be encoded into $n-3$ bits.

%\floris{The ``order'' information referred to below is the one of $M_{\Phi}$, right? We may want to clarify this a bit more in order not to confuse with the ordering on canonical vectors..}
This matrix is then updated in the same way as the two-stack algorithm. It processes gates one by one, and using the successor relation for canonical vectors determines whether we have more elements on the gates stack. In this case, a new value is added to the values stack ($0$ if the gate is a sum gate, and $1$ otherwise), and the process continues. Information about the next child, last child, or input value, are obtained using the expression which simulates the Turing machine generating this data about the circuit (the machines used never produce an output longer than their input). Given that the size of the circuit is polynomial, say $n^k$, we can initialize the matrix with the output gate only, and run the simulation of the two-stack algorithm for $n^k$ steps (by iterating $k$ times over size $n$ canonical vectors). After this, the value in position  $(1,n-2)$ (the top of the values stack) holds the final results. \qed

\smallskip
While Theorem \ref{th-circuits-ml} gives us an idea on how to simulate arithmetic circuits, it does not tell us which classes of functions over real numbers can be computed by \langfor expressions. In order to answer this question, we note that arithmetic circuits can be used to compute functions over real numbers. Formally, a circuit family $\{\Phi_n\mid n=1,2,\ldots\}$ computes a function $f:\bigcup_{n\geq 1} \mathbb{R}^n\mapsto\mathbb{R}$, if for any $a_1,\ldots a_n\in \mathbb{R}$ it holds that $\Phi_n(a_1,\ldots ,a_n) = f(a_1,\ldots ,a_n)$. To make the connection with \langfor\!, we need to look at circuit families of bounded degree. 

A circuit family $\{\Phi_n\mid n=1,2,\ldots\}$ is said to be of \textit{polynomial degree} if $\mathsf{degree}(\Phi_n)\in O(p(n))$, for some polynomial $p(n)$. Note that polynomial size circuit families are not necessarily of polynomial degree. An easy corollary of Theorem \ref{th-circuits-ml} tells us that all functions computed by uniform family of circuits of polynomial degree and logarithmic depth can be simulated using \langfor expressions. However, we can actually drop the restriction on circuit depth due to the result of Valiant et. al.~\cite{valiant1981fast} and Allender et. al. \cite{AllenderJMV98} which says that any function computed by a uniform circuit family of polynomial degree (and polynomial depth), can also be computed by a uniform circuit family of logarithmic depth. Using this fact, we can conclude the following:

%An important observation here is that, due to the depth reduction result for arithmetic circuits \cite{AllenderJMV98}, we can in fact show a version of Theorem \ref{th-circuits-ml} with no condition about the circuit depth. To formalize this, we need to define the class of functions computed by arithmetic circuits. Formally, a circuit family $(\Phi_n)_n$ computes a function $f:\bigcup \mathbb{R}^n\mapsto\mathbb{R}$, if for any $v_1,\ldots v_n\in \mathbb{R}$ it holds that $\Phi_n(v_1,\ldots ,v_n) = f(v_1,\ldots ,v_n)$. Given that any function computed by 

\begin{corollary}
\label{cor-circ-ml}
For any function $f$ computed by a uniform family of arithmetic circuits of polynomial degree, there is an equivalent \langfor formula $e_f$.
\end{corollary}

%\domagoj{If needed the corollary can be merged into Theorem 6.1. I prefer it here to separate the degree vs. size discussion.}

Note that there is nothing special about circuits that have a single output, and both Theorem \ref{th-circuits-ml} and Corollary \ref{cor-circ-ml} also hold for functions  $f:\bigcup_{n\geq 1} \mathbb{R}^n\mapsto\mathbb{R}^{s(n)}$, where $s$ is a polynomial. Namely, in this case, we can assume that circuits for $f$ have multiple output gates, and that the depth reduction procedure of \cite{AllenderJMV98} is carried out for each output gate separately. Similarly, the construction underlying the proof of Theorem \ref{th-circuits-ml} can be performed for each output gate independently, and later composed into a single output vector.

\subsection{From \langfor to circuits}

Now that we know that arithmetic circuits can be simulated using \langfor expressions, it is natural to ask whether the same holds in the other direction. That is, we are asking whether for each \langfor expression $e$ over some schema $\Sch$ there is a uniform family of arithmetic circuits computing precisely the same result depending on the input size. 

%BEFORE:
%\floris{The challenge mentioned below is not entirely clear, especially since later an inductive proof is given. Perhaps this can be clarified?}

%One challenge here is that when working with \langfor, even if the final result of our expression is a single number (i.e. a matrix of size $1\times 1$), applying various expressions can lead to intermediate results which are of arbitrary size, which would make an inductive translation impossible. 

%In order to circumvent this issue, we need to consider arithmetic circuits which can take several matrices as input, and can also compute a matrix as their output. This can be easily handled by circuits which have multiple output gates. We will write $\Phi(A_1,\ldots ,A_k)$, where $\Phi$ is an arithmetic circuit with multiple output gates, and each $A_i$ is a matrix of dimensions $\alpha_i\times \beta_i$, with $\alpha_i,\beta_i \in \{n,1\}$ to denote the input matrices for a circuit $\Phi$. We will also write $\ttype(\Phi)=(\alpha,\beta)$, with $\alpha,\beta\in \{n,1\}$, to denote the size of the output matrix for $\Phi$. We call such circuits \textit{arithmetic circuits over matrices}. When $\{\Phi_n\mid n=1,2,\ldots\}$ is a uniform family of arithmetic circuits over matrices, we will assume that the Turing machine for generating $\Phi_n$ also gives us the information about how to access a position of each input matrix, and how to access the positions of the output matrix, as is usually done when handling matrices with arithmetic circuits \cite{Raz02}. The notion of degree is extended to be the sum of the degrees of all the output gates. With this  at hand, we can now show the following result.

%NOW:
In order to handle the fact that \langfor\ expressions can produce any matrix, and not just a single value, as their output, we need to consider circuits which have multiple output gates. Similarly, we need to encode matrix inputs of a \langfor\ expression in our circuits. We will write $\Phi(A_1,\ldots ,A_k)$, where $\Phi$ is an arithmetic circuit with multiple output gates, and each $A_i$ is a matrix of dimensions $\alpha_i\times \beta_i$, with $\alpha_i,\beta_i \in \{n,1\}$ to denote the input matrices for a circuit $\Phi$. We will also write $\texttt{type}(\Phi)=(\alpha,\beta)$, with $\alpha,\beta\in \{n,1\}$, to denote the size of the output matrix for $\Phi$. We call such circuits \textit{arithmetic circuits over matrices}. When $\{\Phi_n\mid n=1,2,\ldots\}$ is a uniform family of arithmetic circuits over matrices, we will assume that the Turing machine for generating $\Phi_n$ also gives us the information about how to access a position of each input matrix, and how to access the positions of the output matrix, as is usually done when handling matrices with arithmetic circuits \cite{Raz02}. The notion of degree is extended to be the sum of the degrees of all the output gates. With this  at hand, we can now show the following result.

%One way to do this is to allow multiple output gates, and have some sort of enoding of input positions and output positions. This is, for example, how one treats the classical claim that arithmetic circuits can compute matrix product \cite{?}. While this is indeed a viable option, we find it far more elegant to extend arithmetic circuits such that they can receive matrices as inputs, and can compute matrix result as their output.

%Formally, an \textit{ arithmetic circuit over matrices} is an expression $\Phi(A_1,\ldots ,A_k)$, where $\Phi$ is an arithmetic circuit with multiple output gates, and $A_1,\ldots ,A_k$ are matrices of dimension $\alpha\times \beta$, with $\alpha,\beta \in \{n,1\}$. Each matrix entry of $A_i$ corresponds to an input gate of $\Phi$. That is, if some $A_i$ is of dimension $n\times n$, then this will correspond to $n^2$ input gates of $\Phi$, and similarly for other supported dimensions. The matrices $A_1,\ldots ,A_k$ do not need to be all of the same dimension. Furthermore, we will assume that $\Phi$ has $\alpha\times \beta$ output gates, with $\alpha,\beta \in \{n,1\}$, and denote this by $\mathtt{type}(\Phi) = \alpha \times \beta$. The idea here is that $\Phi$ on input $A_1,\ldots ,A_k$ computes a matrix of dimension $\alpha\times \beta$. Here we will abuse the notation and write $\Phi(A_1,\ldots ,A_k)$ both when entries of $A_i$ denote uninstantiated inputs (i.e. variables), and when they interpret concrete values. The notion of uniform circuits families, degree, etc., are defined same as before. Note that this adds no expressive power to (ordinary) arithmetic circuits with multiple output gates. With this notation at hand, we can now show that \langfor is subsumed by arithmetic circuit families over matrices.

%\domagoj{Should probably make the definition more precise. I.e. variables vs instantiations!}

\begin{theorem}
\label{th-ml-to-circuits}
Let $e$ be a \langfor expression over a schema $\Sch$, and let $V_1,\ldots ,V_k$ be the variables of $e$ such that $\ttype(V_i)\in \{(\alpha,\alpha), (\alpha,1), (1,\alpha), (1,1)\}$. Then there exists a uniform arithmetic circuit family over matrices $\Phi_n(A_1,\ldots ,A_k)$ such that:
\begin{itemize}
\item For any instance $\I = (\dom,\conc)$ such that $\dom(\alpha) = n$ and $\conc(V_i) = A_i$ it holds that:
\item $\sem{e}{\I} = \Phi_n(A_1,\ldots ,A_k)$.
\end{itemize}
\end{theorem}

%\textit{Proof sketch.} The approach we take here is showing how to compose various circuits, while still being able to generate them using \logspace\ machines. This is an inductive construction, which handles the complexity restriction in a similar way as when composing two \logspace\ transducers. \qed

It is not difficult to see that the proof of Theorem \ref{th-circuits-ml} can also be extended to support arithmetic circuits over matrices. In order to identify the class of functions computed by \langfor expressions, we need to impose one final restriction: than on the degree of an expression. Formally, the \textit{degree of \langfor expression $e$} over a schema $\Sch$, is the minimum of the degrees of any circuit family  $\{\Phi_n\mid n=1,2,\ldots\}$ that is equivalent to $e$. That is, the expression $e$ is of polynomial degree, whenever there is an equivalent circuit family for $e$ of a polynomial degree.  
For example, all \langfor expressions seen so far have polynomial degree.
With this definition, we can now identify the class of functions for which arithmetic circuits and \langfor formulas are equivalent. This is the main technical contribution of the paper. 

\begin{corollary}
\label{th-equivalence}
Let $f$ be a function with input matrices $A_1,\ldots ,A_k$ of dimensions $\alpha\times \beta$, with $\alpha,\beta \in \{n,1\}$. Then, $f$ is computed by a uniform circuit family over matrices of polynomial degree if and only if there is a \langfor expression of polynomial degree for $f$. 
\end{corollary}

Note that this result crucially depends on the fact that expressions in \langfor are of polynomial degree. Some \langfor expression are easily seen to produce results which are not polynomial. An example of such an expression is, for instance, $e_{\texttt{exp}} = \ffor{v}{X=A}{X\cdot X}$, over a schema $\Sch$ with $\ttype(v)= (\gamma,1)$, and $\ttype(X)=(1,1)$. Over an instance which assigns $n$ to $\gamma$ this expression computes the function $a^{2^n}$, for $A=[a]$. Therefore, a natural question to ask then is whether we can determine the degree of a \langfor expression. Unfortunately, as we show in the following proposition this question is in fact undecidable.

\begin{proposition}
\label{prop-undec}
Given a \langfor expression $e$ over a schema $\Sch$, it is undecidable to check whether $e$ is of polynomial degree.
\end{proposition}

%In general, determining a syntactic subclass of \langfor expressions which are of polynomial degree seems similar to the analysis of terminating conditions of programs with for loops \cite{florisKnows?}. We leave the study of \langfor fragments which can be simulated by circuits of polynomial degree for future work.

Of course, one might wonder whether it is possible to define a syntactic subclass of \langfor expressions that are of polynomial degree and can still express many important linear algebra algorithms. We identify one such class in Section \ref{ss:sumML}, called \langsum, and in fact show that this class is powerful enough to capture relational algebra on (binary) $K$-relations. 


\floris{What I am missing is what it is precisely that prevents us from doing any circuit in \langfor and what do these results actually imply. What cannot be done?}

\subsection{Supporting additional operators}

%MISSING:
%
%1. UNDECIDABILITY
%
%2. DIVISION
%
%3. COMMENT ON GENERAL DEPTH REDUCTION DEPENDING ON THE FUNCTIONS

The equivalence of \langfor and arithmetic circuits we prove above assumes that circuits can only use the sum and product gates (note that even without the sum and the product function, \langfor\ can simulate these operations via matrix sum/product). However, both arithmetic circuits and expressions in $\langfor$ can be allowed to use a multitude of functions over $\RR$. The most natural addition to the set of functions is the division operator, which is crucially needed in many linear algebra algorithms, such as, for instance, Gaussian elimination, or $LU$ decomposition (recall Proposition \ref{prop:gauss}).
%\floris{We need to link back to the previous section perhaps.}
Interestingly, the equivalence in this case still holds, mainly due to a surprising result which shows that (almost all) divisions can in fact be removed for arithmetic circuits which allow sum, product, and division gates \cite{allender}.

More precisely, in \cite{strassen1973vermeidung,borodin1982fast,kaltofen1988greatest} it was shown that for any function of the form $f = g/h$, where $g$ and $h$ are relatively prime polynomials of degree $d$, if $f$ is computed by an arithmetic circuit of size $s$, then both $g$ and $h$ can be computed by a circuit whose size is polynomial in $s + d$. Given that we can postpone the division without affecting the final result, this, in essence, tells us that division can be eliminated (pushed to the top of the circuit), and we can work with sum-product circuits instead. The degree of a circuit for $f$, can then be defined as the maximum of degrees of circuits for $g$ and $h$. Given this fact, we can again use the depth reduction procedure of \cite{AllenderJMV98}, and extend Corollary 
% \ref{cor-circ-ml} to circuits with division.
\ref{th-equivalence} to circuits with division.
\begin{corollary}
\label{cor-division}
Let $f$ be a function taking as its input matrices $A_1,\ldots ,A_k$ of dimensions $\alpha\times \beta$, with $\alpha,\beta \in \{n,1\}$. Then, $f$ is computed by a uniform circuit family over matrices of polynomial degree that allows divisions, if and only if there is a $\langforf{f_/}$ expression of polynomial degree for $f$.
\end{corollary}

\floris{Again, should we link back to the previous section?}

An interesting line of future work here is to see which additional functions can be added to arithmetic circuits and \langfor formulas, in order to preserve their equivalence. Note that this will crucially depend on the fact that these functions have to allow the depth reduction of \cite{AllenderJMV98} in order to be supported.

\floris{The depth reduction thing is thus crucial for simulating by \langfor. This was mentioned in the proof, but as mentioned before, we may want to make more explicit what goes wrong otherwise, as to better provide an understanding of what can be done with matrices?}

\cristian{I don't follow this comment from Floris. However, we should say here that \langfor can always simulate arithmetic circuits with new operators if the depth of the family is poly-logarithmic. }


\section{Restricting the power of for loops}\label{sec:restrict}
% !TeX spellcheck = en_US

We give here a nice overview of the section.

\newcommand{\hprod}{\circ}

\subsection{Sumation matlang and relational algebra}

When defining the identity matrix and several other expressions, we actually only update $X$ by adding some matrix to it. This restricted form of the $\texttt{for}$ loop proved to be useful throughout the paper, and we will therefore introduce it a special operator. That is, we define:
$$\Sigma v. e := \ffor{v}{X}{X + e}.$$
We define the subfragment of \langfor, called \langsum, to consist of the $\Sigma$ operator plus the ``core'' operators in \lang, namely, transposition, matrix multiplication and addition, scalar multiplication, and pointwise function applications.

Apart from defining the identity matrix with \langsum, the sum quantifier also allows for computing the trace of a matrix $A$ using the expression $tr(A) := \Sigma v. v^*\cdot A \cdot v$. Interestingly enough, this restricted version of \texttt{for} already allows us to capture the \lang\ operators that are not present in the syntax of \langsum. More precisely, we have:
\begin{itemize}
\item {\em Function application.} Notice that in \lang, a function is applied pointwise to matrices of arbitrary size, while \langsum only allows functions that process matrices of size $1\times 1$. Using the summation operator we can lift this condition, and allow applying a function $f:\mathbb{\RR}^n \mapsto \mathbb{\RR}$ on expressions $e_1,\ldots ,e_n$ of arbitrary (but equal) type by writing 
$$\Sigma x_i \Sigma x_j. f(x_i^T\cdot e_1\cdot x_j, \ldots ,x_i^T\cdot e_n\cdot x_j) \cdot x_i\cdot x_j^T,$$
which simply reconstructs the matrix obtained by applying $f$ to every position of $e_1$ through $e_n$, by using the fact that for two canonical vectors $b_i^m$ and $b_j^n$, the product $b_i^m \cdot (b_j^n)^*$ defines a $m\times n$ matrix whose only non-zero entry is in the position $ij$.
\item {\em One vector.} We can define $$\ones(e) := \Sigma v.\, v.$$ where $\ttype(v) = (\alpha, 1)$ and $\ttype(e) = (\alpha, \beta)$ for some $\beta$. 
\item {\em Diagonal of a vector.} The operator $\diag(e)$ can be defined as:
$$\diag(e) := \Sigma v. (v^T\cdot e) \cdot vv^T.$$
\end{itemize}
Furthermore, one can easily check that the 4-clique expression of Example~\ref{ex:fourcliques} can be defined in \langsum. Therefore, we conclude the following result. 
\begin{corollary}
\lang\ is strictly subsumed by \langsum.
\end{corollary}

%To show that the inclusion here is strict, we illustrate how one can detect whether a undirected graph has a four clique, which it is not definable in \lang~\cite{BrijderGBW19}. For this, we define an expression $f(u,v) := 1 - u^*\cdot v$. Notice that when $u$ and $v$ are interpreted by two canonical vector of the same dimension, we have that:
%\[
%  			f(u,v)=1-u^*v=\begin{cases}
%               0 \text{ if } u=v \\
%               1 \text{ if } u\neq v
%            \end{cases}
%		\]
%
%To distinguish four-cliques, we will need to determine whether we are dealing with four different nodes. For this, we will utilize the function $$g(u,v,w,r)=f(u,v)\cdot f(u,w)\cdot f(u,r)\cdot f(v,w)\cdot f(v,r)\cdot f(w,r),$$
%which, when evaluated over four canonical vectors of the same dimension, will give us 1 if and only if the four vectors are distinct. With this at hand, we can now define:		
%\begin{multline*}
%\texttt{4-clique}(A) := \ssum v_1.\ssum v_2. \ssum v_3. \ssum v_4.\\ (v_1^*Av_2)(v_1^*Av_3)(v_1^*Av_4)(v_2^*Av_3)(v_2^*Av_4)(v_3^*Av_4) \cdot
%\\g(v_1,v_2,v_3,v_4).
%\end{multline*}
%
%When $A$ is an adjacency matrix of an undirected graph $G$, then we have that $\texttt{4-clique}(A)$ is different from zero if and only if $G$ has a four-clique. Using this, and the fact that \lang\ is subsumed by First Order Logic with aggregates that uses only three variables \cite{matlang}, we immediately obtain the following:
%
%\begin{corollary}
%There is a  \langfor expression that is not expressible in \lang.
%\end{corollary}

What operations over matrices can be defined with \langsum that is beyond \lang? In~\cite{brijder2019matrices}, it was shown that \lang\ was strictly included in the relational algebra of $K$-relations~\cite{GreenKT07}, called here Annotated Relational Algebra (ARA).
Then a natural idea is to compare the expressive power of \langsum with ARA. For making this comparison clear, in the following we give the formal definition of ARA~\cite{GreenKT07} to then see how to connect both formalism.

\newcommand{\ddom}{\mathbb{D}}
\newcommand{\fdom}{\operatorname{dom}}
\newcommand{\att}{\mathbb{A}}
\newcommand{\tuples}{\mathbf{tuples}}
\newcommand{\supp}{\operatorname{supp}}
\newcommand{\cJ}{\mathcal{J}}
\newcommand{\cR}{\mathcal{R}}
\newcommand{\adom}{\mathbf{adom}}

\newcommand{\ksum}{\oplus}
\newcommand{\kprod}{\odot}
\newcommand{\bigksum}{\bigoplus}
\newcommand{\bigkprod}{\bigodot}
\newcommand{\kzero}{\mymathbb{0}}
\newcommand{\kone}{\mymathbb{1}}

\newcommand{\row}{\mathsf{row}}
\newcommand{\rows}{\mathsf{rows}}
\newcommand{\col}{\mathsf{col}}
\newcommand{\cols}{\mathsf{cols}}


Let $\ddom$ be a data domain and $\att$ a set of attributes. A relational signature is a finite subset of $\att$. A relational schema is a function $\cR$ on finite set of symbols $\fdom(\cR)$ such that $\cR(R)$ is a relation signature for each $R \in \fdom(\cR)$. To simplify the notation, from now on we write $R$ to denote both the symbol $R$ and the relational signature $\cR(R)$.
Furthermore, we write $R \in \cR$ to say that $R$ is a symbol of $\cR$. 
For $R \in \cR$, an $R$-tuple is a function $t: R \rightarrow \ddom$. We denote by $\tuples(R)$ the set of all $R$-tuples. Given $X \subseteq R$, we denote by $t[X]$ the restriction of $t$ to the set $X$.

A semiring $(K, \ksum, \kprod, \kzero, \kone)$ is an algebraic structure where $K$ is a non-empty set, $\ksum$ and $\kprod$ are binary operations over $K$, and $\kzero, \kone \in K$. Furthermore,  $\ksum$ and $\kprod$ are associative operations, $\kzero$ and $\kone$ are the identities of $\ksum$ and $\kprod$ respectively, $\ksum$ is a commutative operation, $\kprod$ distributes over $\ksum$, and $\kzero$ annihilates $K$ (i.e. $\kzero \kprod k = k \kprod \kzero = \kzero$). As usual, we assume that all semirings in this paper are commutative, namely, $\kprod$ is also commutative. We use $\bigksum_X$ or $\bigkprod_X$ for the $\ksum$- or $\kprod$-operation over all elements in $X$, respectively. Typical examples of semirings are the reals $(\RR, +, \times, 0,1)$, the natural numbers $(\RR, +, \times, 0,1)$, and the boolean semiring $(\{0,1\}, \vee, \wedge, 0, 1)$. 

Fix a semiring $(K, \ksum, \kprod, \kzero, \kone)$ and a relational schema $\cR$. A $K$-relation of $R \in \cR$ is a function $r: \tuples(R) \rightarrow K$ such that the support  $\supp(r) = \{t \in \tuples(R) \mid r(t) \neq \kzero\}$ is finite. 
A $K$-instance $\cJ$ of $\cR$ is a function that assigns relational signatures of $\cR$ to $K$-relations. Given $R \in \cR$, we denote by $R^\cJ$ the $K$-relation associated to $R$. Recall that $R^\cJ$ is a function and then $R^\cJ(t)$ is the value in $K$ assign to $t$. 
Given a $K$-relation $r$ we denote by $\adom(r)$ the active domain of $r$ defined as $\adom(r) = \{t(a) \mid t \in \supp(r) \wedge a \in R\}$.
Then the active domain of an $K$-instance $\cJ$ of $\cR$ is defined as $\adom(\cJ) = \bigcup_{R \in \cR} \adom(R^\cJ)$. 

An ARA expression $\alpha$ over $\cR$ is given by the following syntax:
$$
\begin{array}{rcl}
\alpha & := & R \ \mid \ \alpha \cup \alpha \ \mid \  \pi_X(\alpha) \ \mid \  \sigma_X(\alpha) \ \mid \ \rho_f(\alpha) \ \mid \ \alpha \bowtie \alpha
\end{array}
$$
where $R \in \cR$, $X \subseteq \att$ is finite, and $f: X \rightarrow Y$ is a one to one mapping with $Y \subseteq \att$. One can extend the relational schema $\cR$ to any ARA expressions over $\cR$ recursively as follows: $\cR(R) = R$, $\cR(\alpha \cup \alpha') = \cR(\alpha)$, $\cR(\pi_X(\alpha)) = X$, $\cR(\sigma_X(\alpha)) = \cR(\alpha)$, $\cR(\rho_f(\alpha)) = X$ where $f:X \rightarrow Y$, and $\cR(\alpha \bowtie \alpha') = \cR(\alpha) \cup \cR(\alpha')$ for every ARA expressions $\alpha$ and $\alpha'$.
We further assume that any ARA expression $\alpha$ satisfies the following syntactic restrictions: $\cR(\alpha') = \cR(\alpha'')$ whenever $\alpha = \alpha' \cup \alpha''$, $X \subseteq \cR(\alpha')$ whenever $\alpha = \pi_X(\alpha')$ or $\alpha = \sigma_X(\alpha')$, and $Y = \cR(\alpha')$ whenever $\alpha = \rho_f(\alpha')$ with $f: X \rightarrow Y$.

Given an ARA expression $\alpha$ and a $K$-instance $\cJ$ of $\cR$, we define the semantics $\ssem{\alpha}{\cJ}$ as a $K$-relation of $\cR(\alpha)$ as follows. For $X \subseteq \att$, let $\operatorname{Eq}_X(t) = \kone$ when $t(a) = t(b)$ for every $a, b \in X$, and $\operatorname{Eq}_X(t) = \kzero$ otherwise. For every tuple $t \in \cR(\alpha)$:
$$
\begin{array}{ll}
\text{if $\alpha = R$, then} & \ssem{\alpha}{\cJ}(t) = R^\cJ(t) \\
\text{if $\alpha = \alpha' \cup \alpha''$, then} & \ssem{\alpha}{\cJ}(t) = \ssem{\alpha'}{\cJ}(t) \ksum \ssem{\alpha''}{\cJ}(t)  \\
\text{if $\alpha = \pi_X(\alpha')$, then} & \ssem{\alpha}{\cJ}(t) = \bigodot_{t': t'[X] = t} \ssem{\alpha'}{\cJ}(t') \\
\text{if $\alpha = \sigma_X(\alpha')$, then} & \ssem{\alpha}{\cJ}(t) = 
\ssem{\alpha'}{\cJ}(t) \kprod \operatorname{Eq}_X(t)  \\
\text{if $\alpha = \rho_f(\alpha')$, then} & \ssem{\alpha}{\cJ}(t) = 
\ssem{\alpha'}{\cJ}(t \circ f)  
\\
\text{if $\alpha = \alpha' \bowtie \alpha''$, then} & \ssem{\alpha}{\cJ}(t) =  \ssem{\alpha'}{\cJ}(t[Y]) \kprod  \ssem{\alpha''}{\cJ}(t[Z])
\end{array}
$$
where $Y = \cR(\alpha')$ and $Z = \cR(\alpha'')$. It is important to note that the $\bigksum$-operation on the semantics of $\pi_X(\alpha')$ is well defined given that the support of $\ssem{\alpha'}{\cJ}$ is always finite. 

We are ready for comparing the expressive power of \langsum with ARA. First of all, we need to extend \langsum from $\RR$ to any semiring $(K, \ksum, \kprod, \kzero, \kone)$. Indeed, one can easily verify that the semantics of \lang, \langfor and \langsum can be translated from $\RR$ to $K$ by switching from matrices over $(\RR, +, \times, 0, 1)$ to matrices over $(K, \ksum, \kprod, \kzero, \kone)$.
From now on we denote by  $\mtr{K}$ the set of all $K$-matrices. Similar than for \lang\ over $\RR$, given a \lang\ schema $\Sch$ a $K$-instance $\I$ over $\Sch$ is a pair $\I = (\dom,\conc)$, where $\dom : \DD \mapsto \NN$ assigns a value to each size symbol, and $\conc : \Mnam \mapsto \mtr{K}$ assigns a concrete $K$-matrix to each matrix variable. It is straightforward to extend the semantics of \lang, \langfor, and \langsum from $(\RR, +, \times, 0, 1)$ to $(K, \ksum, \kprod, \kzero, \kone)$ by switching $+$ with $\ksum$ and $\times$ with $\kprod$. 

The next step for comparing \langsum with ARA is to represent $K$-matrices as $K$-relations.
Recall that $\Sch=(\Mnam,\size)$ is a \lang\ schema, where $\Mnam\subset \Mvar$ is a finite set of matrix variables, and $\size: \Mvar \mapsto \DD\times \DD$ is a function that maps each matrix variable to a pair of size symbols. On the relational side
we have for each size symbol $\alpha\in\DD\setminus\{1\}$, attributes $\alpha$, $\row_\alpha$, and $\col_\alpha$ in $\att$. For each $V\in\Mnam$ and $\alpha \in \DD$ we denote
by $R_V$ and $R_\alpha$ its corresponding relation name, respectively. Then, given $\Sch$ we define the relational schema $\text{Rel}(\Sch)$ such that $\fdom(\text{Rel}(\Sch)) =  \{R_\alpha \mid \alpha\in\DD\} \cup \{R_V \mid V \in \Mnam\}$ where $\text{Rel}(\Sch)(R_\alpha) = \{\alpha\}$ and:
\[
\text{Rel}(\Sch)(R_V) = \begin{cases}
\lbrace\row_\alpha,\col_\beta \rbrace & \text{ if $ \size(V)=(\alpha,\beta)$} \\
\lbrace\row_\alpha \rbrace & \text{ if $ \size(V)=(\alpha,1)$} \\
\lbrace\col_\beta \rbrace  &
\text{ if $ \size(V)=(1,\beta)$} \\
\lbrace\rbrace & \text{ if $\size(V)=(1,1)$}.
\end{cases}
\]
Consider now a matrix instance $\I = (\dom,\conc)$ over $\Sch$.
Let $V\in\Mnam$ with $\size(V)=(\alpha,\beta)$ and let $\conc(V)$ be its corresponding $K$-matrix of dimension $\dom(\alpha)\times \dom(\beta)$.
To encode $\I$ as a $K$-instance in ARA, we use as data domain $\ddom = \mathbb{N} \setminus \{0\}$. Then we construct the $K$-instance $\text{Rel}(\I)$ such that for each $V\in\Mnam$ we define 
$R_V^{\text{Rel}(\I)}(t):=\conc(V)_{ij}$ whenever $t(\row_\alpha) = i \leq \dom(\alpha)$ and $t(\col_\beta) = j \leq \dom(\beta)$, and $\kzero$ otherwise. Furthermore, for each $\alpha \in \DD$ we define $R_\alpha^{\text{Rel}(\I)}(t):=\kone$ whenever $t(\alpha) \leq \dom(\alpha)$, and $\kzero$ otherwise. In other words, $R_\alpha$ and $R_\beta$ encodes the active domain of a matrix variable $V$ with $\size(V)=(\alpha,\beta)$. Given that the ARA framework of \cite{GreenKT07} represents the ``absence'' of a tuple in the relation with $\kzero$, we need to find a way to encode all entries of a matrix in ARA. For instance, we need to be able to encode a $\kzero$-matrix of dimension $(\alpha,\beta)$ in~ARA.

We are ready to state the first connection between \langsum and ARA by using the previous encoding.
\begin{proposition}
	For each \langsum expression $e$ over \lang \ schema $\Sch$ such that $\Sch(e)=(\alpha,\beta)$ with $\alpha\neq 1\neq\beta$, there exists an ARA expression $\Phi(e)$ over relational schema $\text{Rel}(\Sch)$ such that $\text{Rel}(\Sch)(\Phi(e))=\{\row_\alpha,\row_\beta\}$ and 
	such that for any instance $\I$ over~$\Sch$,
	$$
	\sem{e}{\I}_{i,j}=\ssem{\Phi(e)}{\text{Rel}(\I)}(t)
	$$
	for tuple $t(\mathrm{row}_\alpha)=i$ and $t(\mathrm{col}_\beta)=j$. Similarly for when $e$ has schema $\Sch(e)=(\alpha,1)$, $\Sch(e)=(1,\beta)$ or $\Sch(e)=(1,1)$, then $\Phi(e)$ has schema $\text{Rel}(\Sch)(\Phi(e))=\{\mathrm{row}_\alpha\}$,
	$\text{Rel}(\Sch)(\Phi(e))=\{\mathrm{col}_\alpha\}$, or
	$\text{Rel}(\Sch)(\Phi(e))=\{\}$, respectively.
\end{proposition}
To translate ARA into \langsum, we must restrict our comparison to ARA over binary $K$-relations. Given that linear algebra works over vector and matrices, it is reasonable to restrict to unary or binary relations as input. Note that this is only a restriction to the input relations and not to intermediate relations, namely, expressions can create relation signatures of arbitrary size from the binary input relations. Thus, from now we say that a relational schema $\cR$ is binary if $|R| \leq 2$ for every $R \in \cR$. We also make the assumption that there is an (arbitrary) order, denoted by $<$, on the attributes in $\att$. 
This is to identify which attributes correspond to rows and columns when moving the matrix setting. 
Then, given that relations will be  either unary or binary and there is an order in the attributes, we write $t = (v)$ or $t = (u,v)$ to denote a tuple over a unary or binary relation $R$, respectively, where $u$ and $v$ is the value of the first and second attribute with respect to $<$.

Consider a binary relational schema $\cR$. With each $R\in \cR$ we associate a matrix variable $V_R$ such that, if $R$ is a binary relational signature, then $V_R$ represents a (square) matrix, and, if not (i.e. $R$ is unary), then $V_R$ represents a vector. Formally, fix a symbol $\alpha \in \DD \setminus \{1\}$. Let $\text{Mat}(\cR)$ denote the \lang \ schema
$(\Mnam,\size)$ such that $\Mnam = \{ V_R \mid R \in \cR\}$ and $\size(V_R) = (\alpha, \alpha)$ whenever $|R| = 2$, and $\size(V_R) = (\alpha, 1)$ whenever $|R|=1$. 
Take now a $K$-instance $\cJ$ of $\cR$ and suppose that $\adom(\cJ) = \{d_1, \ldots, d_n\}$ is the active domain of $\cJ$ (i.e. the order over $\adom(\cJ)$ is arbitrary). Then we define the matrix instance $\text{Mat}(\cJ) = (\dom,\conc)$ such that $\dom(\alpha) = n$, $\conc(V_R)_{i,j} = R^{\cJ}((d_i, d_j))$ whenever $|R|=2$, and $\conc(V_R)_{i} = R^{\cJ}((d_i))$ whenever $|R|=1$. 
Note that, although each binary $K$-relation can have different active domain, we encode them as square matrices by considering the active domain of the whole $K$-instance.

\begin{proposition}
	Let $\cR$ be a binary relational schema. For each ARA expression $\alpha$ over $\cR$  such that $|\cR(\alpha)| = 2$, there exists an ARA expression $\Psi(\alpha)$ over \lang \ schema $\text{Mat}(\cR)$ such that for any $K$-instance $\cJ$ with $\adom(\cJ) = \{d_1, \ldots, d_n\}$ over $\cR$,
	$$
	\ssem{\alpha}{\cJ}((d_i, d_j))=\sem{\Psi(\alpha)}{\text{Mat}(\cJ)}_{i,j}.
	$$
	Similarly for when $|\cR(\alpha)| = 1$, or $|\cR(\alpha)| = 0$ respectively.
\end{proposition} 

It is important to remark that the $\alpha$ of the previous result can have intermediate expressions that are not necessary binary given that the proposition only restricts that the input relation and the schema of $\alpha$ must have arity at most two. 

Given the previous two propositions we derive the following conclusion which is the first characterization of relational algebra with a (sub)-fragment of linear algebra.
\begin{corollary}
	\langsum and ARA over binary relational schemas are equally expressive. 
\end{corollary}


\subsection{Comparison with weighted logics}

% !TeX spellcheck = en_US
% !TEX root = ../main.tex

Similar than for matrix sum and \langsum, we can use other operations to update $X$ in the for-loop. A natural choise is to consider product of matrices instead of sum. In contrast to matrix sum, we can choose to use matrix product or pointwise matrix product, also called Hadamard product in linear algebra. We postpone the discussion of matrix product as a quantifier to the next subsection, in order to explain here the connection of sum and Hadamard product to weighted logics.

\newcommand{\hadprod}{\circ} 
\newcommand{\qhadprod}{\Pi^{\hadprod}} 

Fix a semiring $(K, \ksum, \kprod, \kzero, \kone)$ and a \lang\ schema  $\Sch=(\Mnam,\size)$. The Hadamard product over $K$-matrices can be defined as the pointwise application of $\kprod$ between two matrices of the same size, namely, we define the expression $e \hadprod e' := \kprod (e, e')$ where $e, e'$ are expressions with respect to $\cS$ and $\ttype(e) = \ttype(e')$. Then the semantics of $e \hadprod e'$ is the pointwise application of $\kprod$, namely, $\sem{e \hadprod e'}{\I}_{ij} = \sem{e}{\I} \kprod \sem{e'}{\I}_{ij}$ for every instance $\I$. With the Hadamard product we can easily define the pointwise-product quantifier $\qhadprod v$ similar than for the sum quantifier as follows:
$$
\qhadprod v. \  e := \ffor{v}{X\!=\!\kone}{X \circ e}.
$$
where $\kone$ is a matrix of size $\ttype(X)$ with all entries equal to the $\kone$-element of $K$ (i.e. we need to initialize $X$ accordingly with the $\kprod$-operator).
We define the subfragment of \langfor, called \langprod, to consist of \langsum \ plus the $\qhadprod$ operator.

\cristian{Should we give an example here of how to use this operator?}

The inclusion of this new operator adds expressive power to \langsum. For example,  $\sem{\qhadprod v. 2}{\cI} = 2^n$ where $n$ is the dimension of $n$, however, all expressions defined by \langsum grows at most polynomial in the values of $\cI$ and $n$.  Therefore, by the result of the previous section \langprod defined functions that goes beyond the expressive power of ARA. To measure the expressive power of \langprod, we use weighted logics~\cite{DrosteG05} (WL) as a yardstick. Weighted logics extends monadic second order logic from the boolean semiring to any semiring $K$. Furthermore, it has been used extensively to characterize the expressive power of weighted automata in terms of logic~\cite{droste2009handbook}. We use here the first-order subfragment of weighted logics to suit our purpose and, moreover, we extend its semantics over weighted structure (similar than in~\cite{GradelV17}).

\newcommand{\cA}{\mathcal{A}}
\newcommand{\bbX}{\mathbb{X}}
\newcommand{\arity}{\operatorname{arity}}

A relational vocabulary $\Gamma$ is a finite collection of relation symbols $R_1, \ldots, R_l$ such that each $R_i$ has an associated arity, denoted by $\arity(R_i)$.
A $K$-weighted structure over $\Gamma$ is a pair $\cA = (A, \{R_i^\cA\})$ such that $A$ is a non-empty finite set (i.e. the domain) and, for each $R_i \in \Gamma$ and $k = \arity(R_i)$, $R_i^\cA: A^k \rightarrow K$ is a function that associates to each tuple in $A^k$ a weight in $K$. 

Let $\bbX$ be a set of (first-order) variables. A $K$-weighted logic (WL) expression $\varphi$ over $\Gamma$ is defined by the following syntax:
$$
\begin{array}{rcl}
\varphi & := & x = y \ \mid \ R_i(\bar{x}) \ \mid \ \varphi \ksum \varphi \ \mid \ \varphi \kprod \varphi \ \mid \ \Sigma x. \varphi \ \mid \ \Pi x. \varphi
\end{array}
$$ 
where $x, y \in \bbX$, $R_i \in \Gamma$, and $\bar{x}$ is a sequence of variables in $\bbX$ whose length is $\arity(R_i)$. An assignment $\sigma$ over weighted structure $\cA = (A, \{R_i^\cA\})$ is a function $\sigma: X \rightarrow A$. We write $\sigma(\bar{x})$ to denote the result of applying $\sigma$ to each variable in the sequence $\bar{x}$ of variables in $\bbX$. Given $x \in \bbX$ and $a \in A$, we denote by $\sigma[x \mapsto a]$ a new assignment such that, for every $y \in \bbX$, $\sigma[x \mapsto a](y) = a$ whenever $x = y$ and $\sigma[x \mapsto a](y) = \sigma(y)$, otherwise. Then, given a weighted structure $\cA = (A, \{R_i^\cA\})$ and an assignment $\sigma$, we define the semantics $\sem{\varphi}{\cA, \sigma}$ of WL expression $\varphi$ as follows:
$$
\begin{array}{ll}
\text{if $\varphi := x = y$, then} & \ssem{\varphi}{\cA}(\sigma) = 
\left\{
\begin{array}{ll}
\kone & \text{if $\sigma(x) = \sigma(y)$} \\
\kzero & \text{otherwise}
\end{array}
\right. \\
\text{if $\varphi := R_i(\bar{x})$, then} & \ssem{\varphi}{\cA}(\sigma) = R_i^\cA(\sigma(\bar{x})) \\
\text{if $\varphi := \varphi \ksum \varphi'$, then} & \ssem{\varphi}{\cA}(\sigma) = \ssem{\varphi}{\cA}(\sigma) \ksum \ssem{\varphi'}{\cA}(\sigma)  \\
\text{if $\varphi := \varphi \kprod \varphi'$, then} & \ssem{\varphi}{\cA}(\sigma) = \ssem{\varphi}{\cA}(\sigma) \kprod \ssem{\varphi'}{\cA}(\sigma)  \\
\text{if $\varphi := \Sigma x. \varphi'$, then} & \ssem{\varphi}{\cA}(\sigma) =  \bigksum_{a \in A} \ssem{\varphi}{\cA}(\sigma[x \mapsto a]) \\
\text{if $\varphi := \Pi x. \varphi'$, then} & \ssem{\varphi}{\cA}(\sigma) =  \bigkprod_{a \in A} \ssem{\varphi}{\cA}(\sigma[x \mapsto a])
\end{array}
$$
\cristian{Should we show an example here?}

For comparing the expressive power of \langprod with WL, we have to show how to encode \lang instances with weighted relations and viceversa. For this, we need to take two assumptions to put both formalisms at the same level: (1) we restrict WL to relation symbols of arity at most two and (2) we restrict \langprod to square matrices. The first assumption follows the same reasons than for comparing \langsum with ARA, and the second assumption is for simplyfing the presentation of WL. More specific, we could have consider a variant of WL with several domains

\cristian{Work in progress.}



\subsection{Matrix multiplication as a quantifier}

Analogously to the summation with respect to canonical vectors, we can define the product. More precisely, 
if $e$ is an \langfor expression (that possibly uses the variable $v$), we would like to define the result of evaluating $\sem{e}{\I[v := e_1]} \cdot \sem{e}{\I[v := e_2]}\cdots \cdot \sem{e}{\I[v := e_n]}$, where the product is evaluated from left to right. For this purpose, we define the following operator:
$$\sprod v. e=\ffor {v}{X}{X\cdot e + min(v)\cdot e}.$$
Here we use the factor $min(v)\cdot e$ to handle the case of the first canonical vector which starts with $X$ being equal to the null matrix. 

Using the product operator we can express multiple interesting properties. To begin with, we can compute the product of diagonal elements of a matrix using the expression $$dp(A) := \sprod v. v^*\cdot A \cdot v.$$

Another property of interest is computing the transitive closure of a graph adjacency matrix $A$. It is well known the transitive closure of this matrix, denoted $tc(A)$ equals to the matrix consisting of non-zero entries of $(I + A)^n$, where $n$ is the dimension of $A$. Using the product operator we can define:
$$tc(A) := f_{>0}(\sprod v. (I + A)),$$
where $f_{>0}(x) := 1$ if $x>0$, and $f_{>0}(x) = 0$ otherwise, is used to make the result a zero-one matrix. Notice that the expression for $tc(A)$ ignores the canonical vectors, and simply multiplies the previous result with $(I + A)$, thus computing the desired value.

Using the combination of canonical sum and product, we can also define more general operators over matrices, such as the power sum operator, which, given a square matrix $A$, computes $I + A + A^2 + \cdots + A^n$. This operator, denoted by $ps(A)$ can be defined as follows:
$$ps(A) := \ssum v.\sprod w. \left( (w^*Zv)\cdot (A-I) + I\right),$$
where $Z$ is the (strict) order matrix defined above. The outer loop here defines which power we compute. That is, when $v$ is the $i$th canonical vector, we compute $A^i$. Computing $A^i$ is achieved via the inner product loop, which uses $w^*Zv$ to determine whether $w$ comes before $v$ in the ordering of canonical vectors. When this is the case we multiply the current result by $A$, and when $w$ is greater then or equal to $v$, we use $I$ not to affect the already computed result.


% \section{Connecting Linear Algebra with Relational Algebra}
% We prove the ARA theorem here.

If we are short on pages, the proof sketch can make it here.

Perhaps discuss quantifier connection here and not in the previous section?

%\section{Applications}
%Here we show the real strength of our framework by connecting it to some well established areas:

\begin{itemize}
\item Graph query languages
\item Machine learning
\item Query optimization
\end{itemize}


\section{Conclusions}\label{sec:conclude}
We proposed \langfor, an extension of \lang with limited recursion,
and showed that it is able to capture most of linear algebra due to its
connection to arithmetic circuits. We further revealed interesting connections
to logics on annotated relations. Our focus was on language design and
expressivity. An interesting direction for future work relates to efficient
evaluation of (fragments) of \langfor. A possible starting point is \cite{Christ_2013}
in which a general methodology for communication-optimal algorithms
for for-loop linear algebra programs is proposed.
%Additionally, our results of Section \ref{ss:sumML} show interesting opportunities for exploring how relational joins can be evaluated using matrix operations, and vice versa.



%%
%% The next two lines define the bibliography style to be used, and
%% the bibliography file.

\vspace{-1ex}
\bibliographystyle{ACM-Reference-Format}


%UNCOMMENT IF FULL BIB IS DESIRED
% \bibliography{biblio}
\bibliography{nourlbiblio}

%%
%% If your work has an appendix, this is the place to put it.

\newpage 
\appendix

\onecolumn
\section*{Appendix}
The real work begins here.

\section{ARA}
\newcommand{\MLm}{\mathsf{MATLANG}}
\newcommand{\ML}{$\MLm$\xspace}
\newcommand{\ARAm}{\mathsf{ARA}}
\newcommand{\ARA}{$\ARAm$\xspace}
\newcommand{\ARAC}{$(\ARAm+\zeta_k)(k)$\xspace}
\newcommand{\ARACTWO}{$(\ARAm+\zeta_2)(2)$\xspace}
\newcommand{\Rel}{\mathrm{Rel}}
\newcommand{\Mat}{\mathrm{Mat}}

% \DeclareMathOperator{\sdiff}{\triangle}
% By \emph{function} we will always mean a total function. For a function $f: X \to Y$ and $Z \subseteq X$, the \emph{restriction} of $f$ to $Z$, denoted by $f|_Z$, is the function $Z \to Y$ where $f|_Z(x) = f(x)$ for all $x \in Z$.

%
%
%
% From the outset, we also fix countable infinite sets $\mathbf{rel}$, $\mathbf{att}$, and $\mathbf{dom}$, the elements of which are called \emph{relation names}, \emph{attributes}, and \emph{domain elements}, respectively. We assume an equivalence relation $\sim$ on $\mathbf{att}$ that partitions $\mathbf{att}$ into an infinite number of equivalence classes that are each infinite. When $A \sim B$, we say that $A$ and $B$ are \emph{compatible}. Intuitively, $A \sim B$ will mean that $A$ and $B$ have the same set of domain values. A function $f: X \to Y$ with $X$ and $Y$ sets of attributes is called \emph{compatible} if $f(A) \sim A$ for all $A \in X$.
%
\subsection{Annotated-Relation Algebra (\ARA)} 
For completeness, we start by recalling the definition of the \ARA query language. We  here closely follow the exposition given in~\cite{brijder2019matrices}.

Let $\mathbf{att}$ and $\mathbf{dom}$ denote countable infinite sets of  \emph{attributes} and \emph{domain elements}, respectively. A notion of compatibility between attributes is assumed. More formally,
we assume that an equivalence relation $\sim$ on $\mathbf{att}$ is present which partitions $\mathbf{att}$ into an infinite number of equivalence classes that are each infinite. When $A \sim B$, we say that $A$ and $B$ are \emph{compatible}. Intuitively, $A \sim B$ will mean that $A$ and $B$ have the same set of domain values. A function $f: X \to Y$ with $X$ and $Y$ sets of attributes is called \emph{compatible} if $f(A) \sim A$ for all $A \in X$.

A \emph{relation schema} is a finite subset of $\mathbf{att}$. A \emph{database schema} is a function $\mathcal{R}$ on a finite set $N$ of relation names, assigning a relation schema $\mathcal{R}(R)$ to each $R \in N$.
% The \emph{arity} of a relation name $R$ is the cardinality $|\mathcal{R}(R)|$ of its schema. The \emph{arity} of $\mathcal{R}$ is the largest arity among relation names $R \in N$.

The \emph{(positive) Annotated-Relation Algebra}, abbreviated by \ARA, is defined as follows. With each expression $\varphi$ in \ARA one also assigns a relation schema $\mathcal{R}(\varphi)$, by extending the initial schema
$\mathcal{R}$. An \emph{\ARA expression} $\varphi$ over a database schema $\mathcal{R}$ is equal to 
\begin{itemize}
\item a relation name $R$ of $\mathcal{R}$;
\item $\mathbf{1}(\psi)$, where $\psi$ is an \ARA expression, and $\mathcal{R}(\varphi) \coloneqq  \mathcal{R}(\psi)$;
\item $\psi_1 \cup \psi_2$, where $\psi_1$ and $\psi_2$ are \ARA expressions with $\mathcal{R}(\psi_1) = \mathcal{R}(\psi_2)$, and $\mathcal{R}(\varphi) \coloneqq  \mathcal{R}(\psi_1)$;
\item $\pi_Y(\psi)$, where $\psi$ is an \ARA expression and $Y \subseteq \mathcal{R}(\psi)$, and $\mathcal{R}(\varphi) \coloneqq  Y$;
% \item $\pi_Y^\star(\psi)$, where $\psi$ is an \ARA expression and $Y \subseteq \mathcal{S}(\psi)$, and $\mathcal{S}(\varphi) \coloneqq  Y$;
\item $\sigma_{Y}(\psi)$, where $\psi$ is an \ARA expression, $Y \subseteq \mathcal{R}(\psi)$, the elements of $Y$ are mutually compatible, and $\mathcal{R}(\varphi) \coloneqq  \mathcal{R}(\psi)$;
\item $\rho_{\mathcal{R}(\psi) \mapsto Y}(\psi)$, where $\psi$ is an \ARA expression and $\mathcal{R}(\psi) \mapsto Y$ is a compatible one-to-one correspondence of attributes with $Y \subseteq \mathbf{att}$, and $\mathcal{R}(\varphi) \coloneqq  Y$; or
\item $\psi_1 \Join \psi_2$, where $\psi_1$ and $\psi_2$ are \ARA expressions, and $\mathcal{R}(\varphi) \coloneqq  \mathcal{R}(\psi_1) \cup \mathcal{R}(\psi_2)$.
\end{itemize}

We next define the semantics of \ARA expression.
A \emph{domain assignment} is a function $\mathbb{D}: \mathbf{att} \to
2^{\mathbf{dom}}$ such that $A \sim B$ implies
$\mathbb{D}(A) = \mathbb{D}(B)$. Let $X$ be a relation schema. A \emph{tuple} over
$X$ with respect to $\mathbb{D}$
is a function $t: X \to \mathbf{dom}$ such that
$t(A) \in \mathbb{D}(A)$ for all $A \in X$. We denote by
$\mathcal{T}_{\mathbb{D}}(X)$ the set of tuples over $X$ with respect to $\mathbb{D}$. Note that
$\mathcal{T}_{\mathbb{D}}(X)$ is finite.  A \emph{relation} $r$ over
$X$ with respect to $\mathbb{D}$ is a function $r:
\mathcal{T}_{\mathbb{D}}(X) \to K$ for a \emph{semiring} $(K,+,*,0,1)$. So a relation annotates every tuple
over $X$ with respect to $\mathbb{D}$ with a value from $K$.  If $\mathcal{R}$ is a
database schema, then an \emph{instance $\mathcal{I}$ of
$\mathcal{R}$ with respect to $\mathbb{D}$} is a function that assigns to every
relation name $R$ of $\mathcal{R}$ a relation $\mathcal{I}(R):
\mathcal{T}_{\mathbb{D}}(\mathcal{R}(R)) \to K$.

The semantics of \ARA expressions is defined, as follows.

\begin{description}


\item[One] For every relation schema $X$,
  we define $\mathbf{1}_X: \mathcal{T}_{\mathbb{D}}(X) \to K$ where $\mathbf{1}_X(t) = 1$ for every $t \in \mathcal{T}_{\mathbb{D}}(X)$. 


\item[Union] Let $r_1, r_2: \mathcal{T}_{\mathbb{D}}(X) \to K$. Define $r_1 \cup r_2: \mathcal{T}_{\mathbb{D}}(X) \to K$ as $(r_1 \cup r_2)(t) = r_1(t) + r_2(t)$.


\item[Projection] Let $r: \mathcal{T}_{\mathbb{D}}(X) \to K$ and $Y \subseteq X$. Define $\pi_{Y}(r): \mathcal{T}_{\mathbb{D}}(Y) \to K$ as
\[
(\pi_{Y}(r))(t) = \sum_{\substack{t' \in \mathcal{T}_{\mathbb{D}}(X),\\ t'|_{Y} = t}} \!\! r(t').
\]


\item[Selection] Let $r: \mathcal{T}_{\mathbb{D}}(X) \to K$ and $Y \subseteq X$ where the elements of $Y$ are mutually compatible. Define $\sigma_{Y}(r): \mathcal{T}_{\mathbb{D}}(X) \to K$ such that
\[
(\sigma_{Y}(r))(t) =
\begin{cases}
r(t) & \text{if } t(A)=t(B) \text{ for all } A, B \in Y;\cr
0    & \text{otherwise}.
\end{cases}
\]


\item[Renaming] Let $r: \mathcal{T}_{\mathbb{D}}(X) \to K$ and $\varphi: X \to Y$ a compatible one-to-one correspondence. We define $\rho_\varphi(r): \mathcal{T}_{\mathbb{D}}(Y) \to K$ as $\rho_\varphi(r)(t) = r(t \circ \varphi)$.

\item[Join] Let $r_1: \mathcal{T}_{\mathbb{D}}(X_1) \to K$ and $r_2: \mathcal{T}_{\mathbb{D}}(X_2) \to K$. Define $r_1 \Join r_2: \mathcal{T}_{\mathbb{D}}(X_1 \cup X_2) \to K$ as $(r_1 \Join r_2)(t) = r_1(t|_{X_1})*r_2(t|_{X_2})$.
\end{description}

The above operations provide semantics for \ARA in a natural manner. Formally, let $\mathcal{R}$ be a database schema, let $\varphi$ be an \ARA expression over $\mathcal{R}$, and let $\mathcal{I}$ be an instance of $\mathcal{R}$. The \emph{output} relation $\varphi(\mathcal{I})$ of $\varphi$ under $\mathcal{I}$ is defined as follows. If $\varphi = R$ with $R$ a relation name of $\mathcal{R}$, then $\varphi(\mathcal{I}) \coloneqq  \mathcal{I}(R)$. If $\varphi = \mathbf{1}(\psi)$, then $\varphi(\mathcal{I}) \coloneqq  \mathbf{1}_{\mathcal{S}(\psi)}$. If $\varphi = \psi_1 \cup \psi_2$, then $\varphi(\mathcal{I}) \coloneqq  \psi_1(\mathcal{I}) \cup \psi_2(\mathcal{I})$. If $\varphi = \pi_{X}(\psi)$, then $\varphi(\mathcal{I}) \coloneqq  \pi_{X}(\psi(\mathcal{I}))$. If $\varphi = \sigma_{Y}(\psi)$, then $\varphi(\mathcal{I}) \coloneqq  \sigma_{Y}(\psi(\mathcal{I}))$. If $\varphi = \rho_\varphi(\psi)$, then $\varphi(\mathcal{I}) \coloneqq  \rho_\varphi(\psi(\mathcal{I}))$. Finally, if $\varphi = \psi_1 \Join \psi_2$, then $\varphi(\mathcal{I}) \coloneqq  \psi_1(\mathcal{I}) \Join \psi_2(\mathcal{I})$.


\subsection{An extension of \ARA}
We extend \ARA with the following two operators:
\begin{itemize}
 \item $\pi_Y^\star(\psi)$, where $\psi$ is an \ARA expression and $Y \subseteq \mathcal{R}(\psi)$, and $\mathcal{R}(\varphi) \coloneqq  Y$;
 \item $\textsf{Apply}[f](\psi_1,\ldots,\psi_k)$, where $\psi_1,\ldots,\psi_k$ are \ARA expressions with $\mathcal{R}(\psi_1)=\cdots=\mathcal{R}(\psi_k)$, 
 $f$ is a function $K^k\to K$,
 and 
 $\mathcal{R}(\varphi)=\mathcal{R}(\psi_1)$.
\end{itemize}
The semantics of these operators is given by:
\begin{description}
\item[$\star$-Projection] Let $r: \mathcal{T}_{\mathbb{D}}(X) \to K$ and $Y \subseteq X$. Define $\pi_{Y}^\star(r): \mathcal{T}_{\mathbb{D}}(Y) \to K$ as
\[
(\pi_{Y}(r))(t) = \prod_{\substack{t' \in \mathcal{T}_{\mathbb{D}}(X),\\ t'|_{Y} = t}} \!\! r(t').
\]
\item[Function application] Let $r_{i}: \mathcal{T}_{\mathbb{D}}(X) \to K$ for $i=1,\ldots,k$. Define $\textsf{Apply}[f](r_1,\ldots,r_k): \mathcal{T}_{\mathbb{D}}(Y) \to K$ as
\[
(\textsf{Apply}[f](r_1,\ldots,r_k))(t) = f(r_1(t),\ldots,r_k(t)).
\]
\end{description}
Hence, if $\varphi=\pi^\star_{Y}(\psi)$ then 
$\varphi(\mathcal{I})\coloneqq \pi^\star_{Y}(\psi(\mathcal{I}))$, and
if $\varphi=\textsf{Apply}[f](\psi_1,\allowbreak \ldots,\psi_k)$ then we have
$\varphi(\mathcal{I})\coloneqq \textsf{Apply}[f](\psi_1(\mathcal{I}), \ldots,\psi_k(\mathcal{I}))$. We let $\Omega$ denote a set of pointwise functions that can be used in function applications and write \ARA$_\Omega$ to make this explicit.


\subsection{Upper bound on expressivity}
There is a straightforward translation from \ARA$_{\Omega}$ expressions into the relational algebra with aggregation $\text{ALG}_{\text{aggr}}(\Omega',\Theta)$ as defined in~\cite{LIBKIN2003}. Here, $\Omega'$ consists of the functions in $\Omega$ and complemented with the unary functions $1:K\to K:k\mapsto 1$, to deal with $\mathbf{1}$ operator and $0:K\to K:k\mapsto 0$ to deal with selection and binary functions $f_+:K^2\to K:(k,\ell)\mapsto k+\ell$ and $f_*:K^2\to K:(k,\ell)\mapsto k*\ell$.
Furthermore, $\Theta$ consist of aggregate functions corresponding to the semiring sum and product, lifted to multi-sets. More precisely,
$\Theta$ includes $f_+^1,f_+^2,f_+^3,\ldots$ such that $f_+^n$ maps
$n$-element multi-sets in $K$ to their sum, and 
 $f_*^1,f_*^2,f_*^3,\ldots$ such that $f_*^n$ maps
 $n$-element multi-sets in $K$ to their product.

The language $\text{ALG}_{\text{aggr}}(\Omega,\Theta)$ is defined over a ``pure'' relational schema in which attributes are typed. It is easy to see that with every \ARA schema $\mathcal{R}$ we can associate a relational schema encoding the same information. Intuitively, we have one attribute of type $\mathbf{dom}$ for each $A\in\mathcal{R}$ and a special attribute $\text{Val}$ of type $K$ which is to hold the semiring values.

Given this translation, it is known that every expression in  $\text{ALG}_{\text{aggr}}(\Omega,\Theta)$ corresponds to an expression in the finite rank fragment $\mathcal{L}_{\infty,\omega}^*(\textbf{Cnt})$ of infinitary logic with counting $\mathcal{L}_{\infty,\omega}(\textbf{Cnt})$~\cite{LIBKIN2003,Hella:2001}. 
Since this logic is local, \ARA inherits this locality. As an example, 
transitive closure and connectivity of graphcs cannot be expressed in \ARA$_{\Omega}$.





\end{document}
\endinput
%%
%% End of file `sample-sigconf.tex'.
