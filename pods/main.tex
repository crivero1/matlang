%%
%% This is file `sample-sigconf.tex',
%% generated with the docstrip utility.
%%
%% The original source files were:
%%
%% samples.dtx  (with options: `sigconf')
%% 
%% IMPORTANT NOTICE:
%% 
%% For the copyright see the source file.
%% 
%% Any modified versions of this file must be renamed
%% with new filenames distinct from sample-sigconf.tex.
%% 
%% For distribution of the original source see the terms
%% for copying and modification in the file samples.dtx.
%% 
%% This generated file may be distributed as long as the
%% original source files, as listed above, are part of the
%% same distribution. (The sources need not necessarily be
%% in the same archive or directory.)
%%
%% The first command in your LaTeX source must be the \documentclass command.
\documentclass[sigconf]{acmart}

\settopmatter{printacmref=false} % Removes citation information below abstract
\renewcommand\footnotetextcopyrightpermission[1]{} % removes footnote with conference information in first column
\pagestyle{plain} % removes running headers

%%
%% \BibTeX command to typeset BibTeX logo in the docs
\AtBeginDocument{%
  \providecommand\BibTeX{{%
    \normalfont B\kern-0.5em{\scshape i\kern-0.25em b}\kern-0.8em\TeX}}}

%% Rights management information.  This information is sent to you
%% when you complete the rights form.  These commands have SAMPLE
%% values in them; it is your responsibility as an author to replace
%% the commands and values with those provided to you when you
%% complete the rights form.
%\setcopyright{acmcopyright}
%\copyrightyear{2018}
%\acmYear{2018}
%\acmDOI{10.1145/1122445.1122456}
%
%%% These commands are for a PROCEEDINGS abstract or paper.
%\acmConference[Woodstock '18]{Woodstock '18: ACM Symposium on Neural
%  Gaze Detection}{June 03--05, 2018}{Woodstock, NY}
%\acmBooktitle{Woodstock '18: ACM Symposium on Neural Gaze Detection,
%  June 03--05, 2018, Woodstock, NY}
%\acmPrice{15.00}
%\acmISBN{978-1-4503-XXXX-X/18/06}
\usepackage{enumitem}
\usepackage{xspace}
\newcommand{\paths}{\text{PATHS}}

%%%%%%%%%%%%%%%%%%%%%%%%%%%%%%%%
%			COMMENTS
%%%%%%%%%%%%%%%%%%%%%%%%%%%%%%%%
\usepackage[textwidth=2cm,textsize=small]{todonotes}

\newcommand{\domagoj}[1]{\todo[inline, color=blue!15]{{\bf Domagoj:} #1}}
\newcommand{\cristian}[1]{\todo[inline, color=orange!30]{{\bf Cristian:} #1}}
\newcommand{\thomas}[1]{\todo[inline, color=green!30]{{\bf Thomas:} #1}}
\newcommand{\floris}[1]{\todo[inline, color=red!30]{{\bf Floris:} #1}}

%%% UNCOMMENT BELOW TO REMOVE COMMENTS
%\renewcommand{\domagoj}[1]{}
%\renewcommand{\cristian}[1]{}
%\renewcommand{\thomas}[1]{}
%\renewcommand{\floris}[1]{}

%% Macros comments
\newcommand{\tover}[1]{\textcolor{red}{#1}}
\newcommand{\td}[1]{\textcolor{blue}{[TODO: #1]}}

%% Macros logics
\newcommand{\NN}{\mathbb{N}}
\newcommand{\ZZ}{\mathbb{Z}}
\newcommand{\MM}{\mathbb{M}}
\newcommand{\SE}{\mathbb{S}}
\newcommand{\BB}{\mathbb{B}}
\newcommand{\RR}{\mathbb{R}}
\newcommand{\cF}{\mathcal{F}}
\newcommand{\cI}{\mathcal{I}}
\newcommand{\cM}{\mathcal{M}}
\newcommand{\cS}{\mathcal{S}}
\newcommand{\cX}{\mathcal{X}}
\newcommand{\cV}{\mathcal{V}}
\newcommand{\QFBILIA}{\textsf{QFBILIA}}
\newcommand{\nnf}{\textsf{f}}
\newcommand{\Cat}{\operatorname{Cat}}
\newcommand{\ssum}{\textstyle\sum}
\newcommand{\sprod}{\textstyle\prod}
%\newcommand{\for}{\textbf{for }}

\newcommand{\ite}{\textsf{ite}}
\newcommand{\limp}{\Rightarrow}
\newcommand{\flc}{\rightarrow}


\newcommand{\bsingle}{\textsf{bag}}
\newcommand{\bplus}{\oplus}
\newcommand{\bminus}{\ominus}

%% Macros tools
\newcommand{\spen}{\textsc{spen}}
\newcommand{\zzz}{\textsc{Z3}}

%% Environments
\newtheorem{mythm}{Theorem}[section]
\newtheorem{mydef}[mythm]{Definition}
\newtheorem{myprop}[mythm]{Proposition}
\newtheorem{mylem}[mythm]{Lemma}
\newtheorem{myex}[mythm]{Example}
\newtheorem{mycor}[mythm]{Corollary}

\newtheorem*{myrem}{Remark} %% based on amsthm
\newtheorem*{mynota}{Notation}
\newtheorem*{mylem*}{Lemma}
\newtheorem{myclaim}{Claim}
\newtheorem*{myclaim*}{Claim}
\newtheorem*{myprop*}{Proposition}
\newtheorem*{comp}{Efficiency Study}


\newenvironment{point}[1]
{\subsection*{#1}}%
{}

\newcommand{\flist}{\text{\sc FList}}
\newcommand{\set}{\text{\sc Set}}
\newcommand{\fset}{\text{\sc FSet}}
\newcommand{\B}{\text{\bf B}}
%\newcommand{\G}{\mathcal{G}}
\newcommand{\K}{\mathcal{K}}
\newcommand{\LOG}{\text{\sc Log}}
\newcommand{\length}{\text{\rm length}}
\newcommand{\BK}{\text{\sc BK}}
%\newcommand{\cP}{\mathcal{P}}
%\newcommand{\cV}{\mathcal{V}}
%\newcommand{\cC}{\mathcal{C}}
%\newcommand{\cS}{\mathcal{S}}
%\newcommand{\cA}{\mathcal{A}}
%\newcommand{\cR}{\mathcal{R}}
%\newcommand{\cQ}{\mathcal{Q}}
\newcommand{\cG}{\mathcal{G}}
\newcommand{\cT}{\mathcal{T}}
\newcommand{\pr}{\mathbf{Pr}}
\newcommand{\Dyck}{\mathcal{D}}
\newcommand{\expected}{\mathbf{E}}
\newcommand{\bv}{\mathbf{v}}
\newcommand{\bV}{\mathbf{V}}
\newcommand{\bs}{\mathbf{s}}
\newcommand{\bsigma}{\mathbf{\sigma}}
\newcommand{\bw}{\mathbf{w}}
\newcommand{\ba}{\mathbf{a}}
\newcommand{\bq}{\mathbf{q}}
\newcommand{\bx}{\mathbf{x}}
\newcommand{\by}{\mathbf{y}}

\newcommand{\bstring}{\{0,1\}^\ast}

\newcommand{\quot}[1]{#1/\!\equiv}

\newcommand{\then}{\,|\,}
\newcommand{\body}{q}
\newcommand{\bchain}{\text{bc}}

\newcommand{\owner}{\text{\rm owner}}
\newcommand{\pred}{\text{\rm pred}}
\newcommand{\mine}{\text{\rm mine}}
\newcommand{\suc}{\text{\rm succ}}


\newcommand{\bP}{\mathbf{P}}
\newcommand{\bB}{\mathbf{B}}
\newcommand{\bA}{\mathbf{A}}
\newcommand{\bR}{\mathbf{R}}
\newcommand{\bS}{\mathbf{S}}
\newcommand{\bH}{\mathbf{H}}
\newcommand{\bQ}{\mathbf{Q}}

\newcommand{\forkm}[1]{F^{#1}}
\newcommand{\mfork}{F_m}
\newcommand{\mgfork}{F_{m,g}}
\newcommand{\last}{\text{\rm last}}
\newcommand{\best}{\text{\rm best}}
\newcommand{\cho}{\text{\rm choose}}

\newcommand{\ie}{i.e.$\!$ }

\newcommand{\longest}{{\text{\rm longest}}}

\newcommand{\subbody}{{\text{\rm sub-state}}}

\newcommand{\df}{\text{\rm DF}}
\newcommand{\fg}{\text{\rm FG}}
\newcommand{\fr}{\text{\rm FR}}
\newcommand{\bdf}{\text{\rm {\bf DF}}}
\newcommand{\bfg}{\text{\rm {\bf FG}}}
\newcommand{\bfr}{\text{\rm {\bf FR}}}

\newcommand{\meet}{\text{\rm meet}}

\DeclareMathOperator*{\argmax}{argmax}


%\newcommand{\rpa}{r_p^\alpha}
\newcommand{\rpa}{r_p}

\newcommand{\ameet}{\text{\rm all-meet}}
\newcommand{\base}{\text{\rm base}}

\newcommand{\ucl}{\text{\rm up-cl}}
\newcommand{\dcl}{\text{\rm down-cl}}


\newcommand{\sem}[2]{\llbracket #1 \rrbracket(#2)}


\newcommand{\Mnam}{\mathcal{M}}
\newcommand{\Mvar}{\mathcal{V}}
\newcommand{\Fun}{\mathcal{F}}
\newcommand{\Mlang}{\text{MATLANG}}
\newcommand{\llet}{\texttt{let } M = e_1 \texttt{ in } e_2}
\newcommand{\ones}{\mathbf{1}}
\newcommand{\diag}{\texttt{diag}}
\newcommand{\apply}[1]{\texttt{apply}[#1]}

\newcommand{\I}{\mathcal{I}}
\newcommand{\Voc}{\mathcal{S}}
\newcommand{\Sch}{\mathcal{S}}

\newcommand{\mtr}[1]{\texttt{Matrices}[#1]}

\newcommand{\dom}{\mathcal{D}}
\newcommand{\conc}{\texttt{mat}}

\newcommand{\DD}{\texttt{Symb}}
\newcommand{\size}{\texttt{size}}

\newcommand{\ddim}{\texttt{dim}}
\newcommand{\ttype}{\texttt{type}_{\Sch}}
\newcommand{\type}{\texttt{type}}

\newcommand{\lang}{\texttt{MATLANG}}
\newcommand{\langfor}{\texttt{for}-\texttt{MATLANG} }

\newcommand{\ffor}[3]{\texttt{for}\, #1,#2 \texttt{.}\, #3}

%\newcommand{\initf}[4]{\texttt{for}[#1]\, #2,#3 \texttt{.}\, #4}
\newcommand{\initf}[4]{\texttt{for}\, #2,#3\!=\! #1 \texttt{.}\, #4}

\newcommand{\lleq}[2]{\texttt{leq}(#1,#2)}
\newcommand{\mmin}[1]{\texttt{min}(#1)}

\newcommand{\ccol}[2]{\texttt{col(}#1,#2\texttt{)}}
\newcommand{\red}[2]{\texttt{reduce(}#1,#2\texttt{)}}

%%
%% Submission ID.
%% Use this when submitting an article to a sponsored event. You'll
%% receive a unique submission ID from the organizers
%% of the event, and this ID should be used as the parameter to this command.
%%\acmSubmissionID{123-A56-BU3}

%%
%% The majority of ACM publications use numbered citations and
%% references.  The command \citestyle{authoryear} switches to the
%% "author year" style.
%%
%% If you are preparing content for an event
%% sponsored by ACM SIGGRAPH, you must use the "author year" style of
%% citations and references.
%% Uncommenting
%% the next command will enable that style.
%%\citestyle{acmauthoryear}

%%
%% end of the preamble, start of the body of the document source.
\begin{document}

%%
%% The "title" command has an optional parameter,
%% allowing the author to define a "short title" to be used in page headers.
%\title{LOLA: A query language for Linear Algebra with LOops}
\title{Expressive power of linear algebra query languages}
%%
%% The "author" command and its associated commands are used to define
%% the authors and their affiliations.
%% Of note is the shared affiliation of the first two authors, and the
%% "authornote" and "authornotemark" commands
%% used to denote shared contribution to the research.
\author{Floris Geerts}
\affiliation{%
  \institution{University of Antwerp}
%  \streetaddress{P.O. Box 1212}
%  \city{Dublin}
%  \state{Ohio}
%  \postcode{43017-6221}
}
\email{floris.geerts@uantwerp.be}


\author{Thomas Mu\~noz}
\affiliation{%
  \institution{PUC Chile and IMFD Chile}
%  \streetaddress{1 Th{\o}rv{\"a}ld Circle}
%  \city{Hekla}
%  \country{Iceland}
}
\email{tfmunoz@uc.cl}

\author{Cristian Riveros}
\affiliation{%
  \institution{PUC Chile and IMFD Chile}
%  \streetaddress{1 Th{\o}rv{\"a}ld Circle}
%  \city{Hekla}
%  \country{Iceland}
}
\email{cristian.riveros@uc.cl}

\author{Domagoj Vrgo\v{c}}
\affiliation{%
  \institution{PUC Chile and IMFD Chile}
%  \streetaddress{1 Th{\o}rv{\"a}ld Circle}
%  \city{Hekla}
%  \country{Iceland}
}
\email{dvrgoc@ing.puc.cl}

%%
%% By default, the full list of authors will be used in the page
%% headers. Often, this list is too long, and will overlap
%% other information printed in the page headers. This command allows
%% the author to define a more concise list
%% of authors' names for this purpose.
%\renewcommand{\shortauthors}{Trovato and Tobin, et al.}

%%
%% The abstract is a short summary of the work to be presented in the
%% article.
\begin{abstract}
Linear algebra algorithms often require some sort of iteration or recursion as is illustrated by standard algorithms for Gaussian elimination, matrix inversion, and transitive closure. A key characteristic shared by these 
algorithms is that they allow looping for a number of steps that is bounded by the matrix dimension. 
In this paper we extend the matrix query language \lang with this type of recursion, and show that this suffices to express the majority of classical linear algebra algorithms. We study the expressive power of this language and show that it naturally corresponds to arithmetic circuit families, which are often said to capture linear algebra. Furthermore, we analyze several sub-fragments of our language, and show that their expressive power is closely tied to logical formalisms on semiring-annotated relations.
% how they can capture the expressive power of annotated relations and weighted logics.
\end{abstract}

%%
%% The code below is generated by the tool at http://dl.acm.org/ccs.cfm.
%% Please copy and paste the code instead of the example below.
%%%
%\begin{CCSXML}
%<ccs2012>
% <concept>
%  <concept_id>10010520.10010553.10010562</concept_id>
%  <concept_desc>Computer systems organization~Embedded systems</concept_desc>
%  <concept_significance>500</concept_significance>
% </concept>
% <concept>
%  <concept_id>10010520.10010575.10010755</concept_id>
%  <concept_desc>Computer systems organization~Redundancy</concept_desc>
%  <concept_significance>300</concept_significance>
% </concept>
% <concept>
%  <concept_id>10010520.10010553.10010554</concept_id>
%  <concept_desc>Computer systems organization~Robotics</concept_desc>
%  <concept_significance>100</concept_significance>
% </concept>
% <concept>
%  <concept_id>10003033.10003083.10003095</concept_id>
%  <concept_desc>Networks~Network reliability</concept_desc>
%  <concept_significance>100</concept_significance>
% </concept>
%</ccs2012>
%\end{CCSXML}
%
%\ccsdesc[500]{Computer systems organization~Embedded systems}
%\ccsdesc[300]{Computer systems organization~Redundancy}
%\ccsdesc{Computer systems organization~Robotics}
%\ccsdesc[100]{Networks~Network reliability}
%
%%%
%%% Keywords. The author(s) should pick words that accurately describe
%%% the work being presented. Separate the keywords with commas.
%\keywords{datasets, neural networks, gaze detection, text tagging}
%%%
%% This command processes the author and affiliation and title
%% information and builds the first part of the formatted document.
\maketitle

\section{Introduction}
% !TeX spellcheck = en_US
%!TEX root = ../main.tex

Linear algebra-based algorithms have become a key component in data analytical workflows. As such, there is a growing interest in the database community to integrate linear algebra functionalities into relational database management systems \cite{Jermaine/17/LAonRA,2019Boehm,LARA_Berlin_2016,JankovLYCZJG19,Khamis0NOS18}. In particular, from a query language perspective, several proposals have recently been put forward to unify relational algebra and linear algebra. Two notable examples of this are: \lara~\cite{HutchisonHS17}, a minimalistic language in which a number of atomic operations on 
%so-called 
associative tables are proposed, and \lang, a query language for 
%manipulating 
matrices \cite{matlang-journal}.\looseness=-1

Both \lara and \lang have been studied by the database theory community, showing interesting connections to relational algebra and logic. For example, fragments of \lara are known to capture first-order logic with aggregation~\cite{BarceloH0S20}, and \lang has been recently shown to be equivalent to a restricted version of the (positive) relational algebra on $K$-relations, \rak~\cite{brijder2019matrices}, where $K$ denotes a semiring. On the other hand, 
%there are 
some standard constructions in linear algebra 
%that 
are out of reach for these languages. For instance, it was shown that under standard complexity-theoretic assumptions, \lara can not compute the inverse of a matrix or its determinant~\cite{BarceloH0S20}, and operations such as the transitive closure of a matrix are known to be inexpressible in \lang~\cite{matlang-journal}. Given that these are fundamental constructs in linear algebra, one might wonder how to extend \lara or \lang in order to allow expressing such properties.

One approach would be to add these constructions explicitly to the language. Indeed, this was done for \lang in~\cite{matlang-journal}, and \lara in ~\cite{BarceloH0S20}. In these works, the authors have extended the core language with the trace, the inverse, the determinant, or the eigenvectors operators and study the expressive power of the result. However, one can argue that there is nothing special in these operators, apart they have been used historically in linear algebra textbooks and they extend the expressibility of the core language. The question here is whether these new operators form a sound and natural choice to extend the core language, or are they just some particular queries that we would like to support. 

In this paper we take a more principled approach by studying what are the atomic operations needed to define standard linear algebra algorithms. Inspecting any linear algebra textbook, one sees that most linear algebra procedures heavily rely on the use of for-loops in which iterations happen over the dimensions of the matrices involved. To illustrate this, let us consider the example of computing the transitive closure of a graph. This can be done using a modification of the Floyd-Warshall algorithm~\cite{cormen}, which takes as its input an $n\times n$ adjacency matrix $A$ representing our graph, and operates according to the following pseudo-code:
%\vspace{-1ex}
\begin{tabbing}
\quad\texttt{for}\=\,  $k = 1..n$ \texttt{do}\\
\> \texttt{for}\=\,  $i = 1..n$ \texttt{do}\\
\> \> \texttt{for}\=\,  $j = 1..n$ \texttt{do}\\
\> \> \> $A[i,j] := A[i,j] + A[i,k] \cdot A[k,j]$
\end{tabbing}
%\vspace{-1ex}
After executing the algorithm, all of the non zero entries signify an edge in the (irreflexive) transitive closure graph.
%  (and in fact, the number of paths of length at most $n$ between the two nodes).
%
\cristian{I am not sure about this sentence. Depending on the order how is iterated, maybe you can compute more than $n$. Actually, if you don't add the identity to the original matrix, even the transitive closure will not work.}
%\cristian{This is not exactly true. If we use $\cdot$ as $+$ and $+$ as min, it works, namely if we use the min/plus semiring. Should we say it?}

% FOR THE REAL ALGORITHM WE TREAT ZERO AS INFINITY!!!
%\vspace{-1ex}
%\begin{tabbing}
%\quad\texttt{for}\=\,  $k = 1..n$ \texttt{do}\\
%\> \texttt{for}\=\,  $i = 1..n$ \texttt{do}\\
%\> \> \texttt{for}\=\,  $j = 1..n$ \texttt{do}\\
%\> \> \> \texttt{if} $A[i,j]$\=\, $> A[i,k] + A[k,j]$\\
%\> \> \> \> $A[i,j] := A[i,k]+ A[k,j]$
%\end{tabbing}
%%\vspace{-1ex}
%After executing the algorithm, all of the non zero entries signify an edge in the transitive closure graph (and also the length of the shortest path between the two nodes). 


By
%When 
examining standard linear algebra algorithms such as Gaussian elimination, $LU$-decomposition, computing the inverse of a matrix, or its determinant, we can readily see that this pattern continues. Namely, we observe that there are two main components to such algorithms: (i) the ability to iterate up to the matrix dimension; and (ii) the ability to access a particular position in our matrix. In order to allow this behavior in a query language, we propose to extend \lang with limited recursion in the form of for-loops, resulting in the language \langfor. To  simulate the two components of standard linear algebra algorithms in a natural way, we simulate a loop of the form \texttt{for}\, $i=1..n$ \texttt{do} by leveraging canonical vectors. In other words, we use the canonical vectors $b_1=(1,0,\ldots)$, $b_2=(0,1,\ldots)$, \ldots, to access specific rows and columns, and iterate over these vectors. In this way,
% As a consequence of this,
we obtain a language able to compute important linear algebra operators such as $LU$-decomposition, determinant, matrix inverse, among other things.

Of course, a natural question to ask now is whether this really results in a language suitable for linear algebra? We argue that the correct way to approach this question is to compare our language to arithmetic circuits, which have been shown to capture the vast majority of existing matrix algorithms, from basic ones such as computing the determinant and the inverse, to complex procedures such as discrete Fourier transformation, and Strassen's algorithm (see \cite{ShpilkaY10,allender} for an overview of the area), and can therefore be considered to effectively capture linear algebra. In the main technical result of this paper, we show that \langfor indeed computes the same class of functions over matrices as the ones computed by arithmetic circuit families of bounded degree.  As a consequence, \langfor inherits all expressiveness properties of circuits, and thus can simulate any linear algebra algorithm definable by circuits.

Having established that \langfor indeed provides a good basis for a linear algebra language, we move to a more fine-grained analysis of the expressiveness of its different fragments. For this, we aim to provide a connection with logical formalisms, similarly as was done by linking \lara and \lang to first-order logic with aggregates~\cite{BarceloH0S20,matlang-journal}. As we show, capturing different logics correspond to restricting how matrix variables are updated in each iteration of the for-loops allowed in \langfor. For instance, if we only allow to add some temporary result to a variable in each iteration (instead of rewriting it completely like in any programming language), we obtain a language, called \langsum, which is equivalent to \rak, directly extending an analogous result shown for \lang, mentioned earlier~\cite{brijder2019matrices}. We then study updating matrix variables based on another standard linear algebra operator, the Hadamard product, resulting in a fragment called \langprod, which we show to be equivalent to weighted logics~\cite{DrosteG05}. Finally, in \langmprod 
we 
%consider updating 
update the variables based on the standard matrix product, and link this fragment 
%of our language 
to the ones discussed previously.  

\smallskip
\noindent
\textbf{Contribution and outline.} 
% Our main contributions can be summarized as follows.
\begin{itemize}[leftmargin=0.5cm]
	\item After we recall \lang in Section~\ref{sec:matlang}, we show in Section~\ref{sec:formatlang}
	how for-loops can be added to \lang in a natural way. We also observe that
	\langfor strictly extends \lang. In addition, we discuss some design decisions behind the definition of \langfor, noting that our use of canonical vectors results in the availability of an order relation.
	
	\item In Section~\ref{sec:queries} we show that \langfor can compute important linear algebra algorithms in a natural way. We provide expressions in \langfor for LU decomposition (used to solve linear systems of equations), the determinant and matrix inversion.
	\item More generally, in Section~\ref{sec:circuits} we report our main technical contribution.
	 We show that every  uniform arithmetic circuits of polynomial degree correspond to a \langfor expression, and vice versa, when a \langfor expression has polynomial degree, then there is an equivalent uniform family of arithmetic circuits. As a consequence, \langfor inherits all expressiveness properties of such circuits.
	\item  Finally, in Section~\ref{sec:restrict} we generalize the semantics of \langfor to matrices with values in a semiring $K$, and show that two natural fragment of \langfor, \langsum, and \langprod, are equivalent to the (positive) relational algebra and weighted logics on binary $K$-relations, respectively. We also briefly comment on a minimal fragment of \langfor, based on \langmprod, that is able to compute matrix inversion.
\end{itemize}
Due to space limitations, most proofs are referred to the appendix.
%
% \floris{This is example is getting too complicated! Any suggestions?}
% % %
% \begin{example}
% Consider a linear system of equations $L\cdot x=a$ with $L$ a matrix of dimensions $n\times n$, $a$ a vector of dimension $n\times 1$, and $x$ a vector of variables of dimensions $n\times 1$. Furthermore, assume that $L$ is a non-singular lower triangular matrix, i.e., all entries above the diagonal are zero and all entries on the diagonal are non-zero. To solve the system for $x$, it suffices to apply forward substitution, i.e.,
% $$	x_1:= a_1, \quad
% x_i:= \frac{1}{L_{ii}}\left(a_i -\sum_{j=1}^{i-1}L_{ij}a_j\right) \quad i\in[2,n]$$
% To view this procedure as a query in \langfor we proceed as follows.
% We use a matrix variable $M$ to store $L$ and a vector variable $V$ to store $a$.
% Furthermore, we reserve two special vector variables $v$ and $w$ which will range over
% the canonical vectors of the same dimension as $L$ and $a$. More specifically, they range
% over $b_1=(1,0,\ldots,0)$, $b_2=(0,1,0,\ldots,0)$,\ldots, $b_n=(0,\ldots,0,1)$ \textit{in this order}. We further have a special variable $X$ to hold intermediate results. In this example,
% $X$ will hold the solution $x$ for the system of equations at the end of the evaluation. We also
% require a second variable $Y$ which will hold $\sum_{j=1}^{i-1}L_{ij}a_j$ for a given $i$. Finally,
% we also allow the use the division function $f_/(x,y)=x/y$.
%
% We need to more operators in this example:
% the operator $\mathsf{min}(v)$ such that $\mathsf{min}(v):=1$ if $v=b_1$ and $\mathsf{min}(v):=0$ otherwise, and the $\mathsf{pred}^+(v,w)$ such that $\mathsf{pred}^+(b_j,b_i)=1$
%  if $j<i$ and $\mathsf{pred}^+(b_j,b_i)=0$ otherwise. These identify the first canonical vector and a strict predecessor relation on canonical vectors, respectively, We show later in the paper that these can be expressed in \langfor. Given these, we consider the following \langfor\ expressions:
% \begin{tabbing}
% $e_1(M,V,v)$\=:=\texttt{for}$\,w,Y\,.\, Y + \mathsf{pred}^+(w,v) (v^T\cdot M\cdot w)(w^T\cdot V)\times v$\\
% $e(M,V)$\=:=\texttt{for}$\,v,X\,.\, X + (\mathsf{min}(v)(v^T\cdot V)\times v) +(1-\mathsf{min}(v))e_1(M,V,v)$
% \end{tabbing}
% \end{example}
% % %


\smallskip
\noindent
\textbf{Related work.} 
We already mentioned \lara~\cite{HutchisonHS17} and \lang~\cite{matlang-journal}
whose expressive power was further analyzed in~\cite{BarceloH0S20,brijder2019matrices,Geerts19,Geerts20}.
Extensions of \texttt{SQL} for matrix manipulations are reported in~\cite{Jermaine/17/LAonRA}. Most relevant
is~\cite{JankovLYCZJG19} in which a recursion mechanism is added to \texttt{SQL} which resembles for-loops.
The expressive power of this extension is unknown, however. Classical logics with aggregation~\cite{Hella:2001} and fixed-point logics with counting~\cite{GroheP17} can also be used for linear algebra. More generally, for the descriptive complexity of linear algebra we refer to~\cite{dghl_rank,holm_phd}. Most of these works require to encode real numbers inside relations, whereas we treat real numbers as atomic values. We refer to relevant papers related to arithmetic circuits and logical formalisms on semiring-annotated relations in the corresponding sections later in the paper.


% \subsection{old stuff}
% In another line of work~\cite{}, matrices are encoded as relational tables and an extensions of SQL is proposed to carry out matrix manipulations. In particular, \cite{} extends SQL with a limited form of recursion -- alike dynamic programming - such that linear algebra-based procedures for learning feed-forward neural networks can be declaratively specified. To our knowledge, the precise expressive power of the resulting language has not been characterized. For example, it is unclear whether matrix inversion can be expressed.
%
% Based on these works, a natural question arises: how to add a natural form of recursion to linear algebra-based query languages? Inspecting any linear algebra textbook, one sees that most linear algebra procedures heavily rely on the use of for-loops in which iterations happen over the dimensions of the matrices involved. We thus propose to extend \lang\ with limited recursion in the form of for-loops, resulting the language \langfor. To define this recursion in a natural way, we simulate a loop of the form \texttt{for i=1,...n do} by leveraging canonical vectors. In other words, we use the canonical vectors $b_1=(1,0,\ldots)$, $b_2=(0,1,\ldots)$, \ldots, to access specific rows and columns, and iterate over these vectors. As an almost direct consequence, the expressive power of \langfor goes well beyond \lang. It can check for cliques of any given size, compute the transitive closure of a graph, and as we will show, compute important linear algebra operators such as LU-decomposition, determinant, matrix inverse, among other things.
%
% More generally, we show that \langfor\ is closely related to arithmetic circuits and we show that anything computable by an arithmetic circuit of polynomial degree can be computed in \langfor, and vice versa, provided that \langfor expressions can be compiled, in a uniform way into arithmetic circuits. Since these circuits are often said to ``capture'' linear algebra, we see our results as as a justification for our language.
%
% Furthermore, the introduction of recursion to \lang has some interesting consequences. First of all, when consecutive iterations can only perform updates in an additive way, we show that \langfor and the annotated relation algebra are equivalent. Secondly, when iterations update in a multiplicative manner,
% \langfor is equivalent to weighted logics.
%

%
% \begin{itemize}
% \item Explain why this is important.
% \item Say what sort of things we would like to express.
% \item Give a quick tour of our minimal language (examples).
% \item Stress our concrete contributions.
% \item Relate to previous work \cite{matlang,BrijderGBW19,Geerts19,HutchisonHS17}.
% \end{itemize}
%
% \domagoj{A good example for the intro is all shortest paths via Floyd-Warshall.}
%
% \noindent
% $\ffor{e_k}{\Dist}{ }$
% \\
% \hspace*{0.5cm} $\ffor{e_i}{\Dist}{ }$
% \\
% \hspace*{1cm} $\ffor{e_j}{\Dist}{ }$
% \\
% \hspace*{1.5cm}
% $\texttt{curr} := e_i^*\cdot \Dist \cdot e_j$\\
% \hspace*{1.5cm}
% $\texttt{new} := e_i^*\cdot \Dist \cdot e_k + e_k^*\cdot \Dist\cdot e_j$\\
% \hspace*{1.5cm}
% $\Dist + \texttt{update}(\texttt{curr},\texttt{new})\times (e_i\cdot e_j^*)$
%
% where
%
% \[
%   			\texttt{update}(x,y)=\begin{cases}
%                0, \text{ if } x<=y \\
%                -x + y, \text{ if } x > y
%             \end{cases}.
% 		\]




\section{MATLANG}\label{sec:matlang}
We start by recalling the matrix query language \lang, introduced in \cite{matlang,matlang-journal}, which serves as our starting point.

\smallskip
\noindent
\textbf{Syntax.}\,  Let $\Mvar = \{V_1, V_2, \ldots\}$ be a countably infinite set of \textit{matrix variables} and $\Fun=\bigcup_{k>1}\Fun_k$ with
$\Fun_k$ a set of \textit{functions} of the  form $f:\RR^k \to \RR$, where $\RR$ denotes the set of real numbers. The syntax of \lang\ expressions is defined by the following grammar\footnote{The original syntax also permits the operator $\llet$, which replaces every occurrence of $V$ in $e_2$ with the value of $e_1$. Since this is just syntactic sugar, we omit this operator. We also explicitly include matrix addition and scalar multiplication, although these can be simulated by pointwise function applications. Finally, we use transposition instead of conjugate transposition since we work with matrices over $\RR$.}:


\begin{tabular}{lcll}
$e$ & $::=$ & $V\in \Mvar$ & (matrix variable)\\
 & $|$ & $e^T$ & (transpose)\\ 
 & $|$ & $\ones(e)$ & (one-vector)\\ 
 & $|$ & $\diag(e)$ & (diagonalization of a vector)\\  
 & $|$ & $e_1 \cdot e_2$ & (matrix multiplication)\\   
 & $|$ & $e_1 + e_2$ & (matrix addition)\\   
 & $|$ & $a\times e$ & (scalar multiplication, $a\in\RR$)\\   
  & $|$ & $f(e_1,\ldots ,e_k)$ & (pointwise application of $f\in\Fun_k$).    
\end{tabular}
\vspace{1ex}

$\lang$ is parametrized by a collection of functions $\Fun$ but in the remainder of the paper we only make this dependence explicit, and write
\lang$(\Fun)$, for some set $\Fun$ of functions, when these functions are crucial
for some results to hold. 
When we simply write \lang, we mean that any function can be used (including not using any function at all).
%When we simply write \lang, we mean that the only functions used are binary multiplication and addition.
%\domagoj{Double check the last part with the rest.}



\smallskip
\noindent
\textbf{Schemas and typing.}\,
To define the semantics of \lang\ expressions we need a notion of schema and well-typedness of expressions. A \lang\ \textit{schema} $\Sch$ is a pair $\Sch=(\Mnam,\size)$, where $\Mnam\subset \Mvar$ is a finite set of matrix variables, and $\size: \Mnam \mapsto \DD\times \DD$ is a function that maps each matrix variable in $\Mnam$ to a pair of \textit{size symbols}. The $\size$ function helps us determine whether certain matrix operations, such as matrix multiplication, can be performed for matrices adhering to a schema. 
We denote size symbols by greek letters $\alpha,\beta,\gamma$. We also assume that $1\in \DD$. 
To help us determine whether a \lang\ expression can always be evaluated, we define the \textit{type} of an expression $e$, with respect to a schema $\Sch$, denoted by $\ttype(e)$, inductively as follows:
\begin{itemize}
\item $\ttype(V):= \size(V)$, for a matrix variable $V\in\Mnam$;
\item $\ttype(e^T):= (\beta,\alpha)$ if $\ttype(e)=(\alpha,\beta)$;
% , and undefined if $\ttype(e)$ is undefined;
\item $\ttype(\ones(e)):= (\alpha,1)$ if $\ttype(e)=(\alpha,\beta)$;
 % and undefined if $\ttype(e)$ is undefined;
\item $\ttype(\diag(e)):= (\alpha,\alpha)$, if $\ttype(e)=(\alpha,1)$;
% , and undefined otherwise;
\item $\ttype(e_1 \cdot e_2):= (\alpha,\gamma)$ if  $\ttype(e_1)=(\alpha,\beta)$, and $\ttype(e_2)=(\beta,\gamma)$;
 % and undefined otherwise;
\item $\ttype(e_1 + e_2):=(\alpha,\beta)$ if $\ttype(e_1)=\ttype(e_2)=(\alpha,\beta)$;
 % and undefined otherwise;
\item $\ttype(a\times e):=(\alpha,\beta)$ if $\ttype(e)=(\alpha,\beta)$; and
\item $\ttype(f(e_1,\ldots ,e_k)):= (\alpha,\beta)$, whenever $\ttype(e_1) = \cdots = \ttype(e_k) := (\alpha,\beta)$ and $f\in\Fun_k$.
 % and is undefined otherwise.
\end{itemize}
When $\Sch$ is clear from the context we simply write $\type(e)$. We call an expression \textit{well-typed} according to the schema $\Sch$, if it has a defined type. 
A well-typed expression can be evaluated regardless of the actual sizes of the matrices assigned to matrix variables, as we describe next.

% $\dim(M)$ gives the dimension of the matrix $M$, where $\dim(M)\in \mathbb{N}^2$. We
\smallskip
\noindent
\textbf{Semantics.}\, We use $\mtr{\RR}$ to denote the set of all real matrices and for 
$A\in\mtr{\RR}$, $\dim(A)\in\NN^2$ denotes its dimensions.
 % over some field $\mathbb{F}$.
A (\lang) \textit{instance} $\I$ over a schema $\Sch$ is a pair $\I = (\dom,\conc)$, where $\dom : \DD \mapsto \NN$ assigns a value to each size symbol (and thus in turn  dimensions to each matrix variable), and $\conc : \Mnam \mapsto \mtr{\RR}$ assigns a concrete matrix to each matrix variable $V\in \Mnam$, such that $\dim(\conc(V)) = \dom(\alpha)\times \dom(\beta)$ if $\size(V) = (\alpha,\beta)$. That is, an instance tells us the dimensions of each matrix variable, and also the concrete matrices assigned to the variable names in $\Mnam$. We assume that $\dom(1) = 1$, for every instance $\I$. If $e$ is a well-typed expression according to $\Sch$, then we denote by $\sem{e}{\I}$ the matrix obtained by evaluating $e$ over $\I$, and define it as follows:
\begin{itemize}
\item $\sem{V}{\I} := \conc(V)$, for $V\in \Mnam$;
\item $\sem{e^T}{\I} := \sem{e}{\I}^T$, where $A^T$ is the transpose of a matrix $A$;
\item $\sem{\ones(e)}{\I}$ is a $n\times 1$ vector with $1$ as all of its entries, where $\dim(\sem{e}{\I})=(n,m)$;
\item $\sem{\diag(e)}{\I}$ is a diagonal matrix with the vector $\sem{e}{\I}$ on its main diagonal, and zero in every other position;
\item $\sem{e_1\cdot e_2}{\I} := \sem{e_1}{\I} \cdot \sem{e_2}{\I}$;
\item $\sem{e_1+ e_2}{\I} := \sem{e_1}{\I} + \sem{e_2}{\I}$;
\item $\sem{a\times e}{\I} := a\times \sem{e}{\I}$; and
\item $\sem{f(e_1,\ldots ,e_k)}{\I}$ is a matrix $A$ of the same size as $\sem{e_1}{\I}$, and where $A_{ij}$ has the value $f(\sem{e_1}{\I}_{ij},\ldots ,\sem{e_k}{\I}_{ij})$.
\end{itemize}
This concludes the description and semantics of \lang. We next provide some simple example.

\begin{example}Consider the \lang$(f_\odot)$ expression with $f_\odot:\RR^2\to\RR:(x,y)\mapsto x\cdot y$:
$$\mathsf{cwalk}:= (\ones(V))^T\cdot f_{\odot}\bigl(V\cdot V, \diag(\ones(V)\bigr)\cdot\ones(V).$$
Let $\Sch$  consist of $\Mnam:=\{V\}$ and $\size(V):=(\alpha,\alpha)$ such that
matrices assigned to $V$ by instances $\I$ over $\Sch$ are square matrices.
It is readily verified that $\mathsf{cwalk}$ is well-typed and more specifically, $\ttype(\mathsf{cwalk})=(1,1)$, i.e., it returns an element of $\RR$ on any  instance $\I$. Let $\I$ be such that $\dom(\alpha)=n$ and  $\conc(V)$ is an adjacency matrix $A$ of an undirected graph $G$ consisting of $n$ vertices. Then, it is readily verified that $\sem{\mathsf{cwalk}}{\I}$ returns the number of paths of length two in $G$ which start in and end at the same vertex.\qed
\end{example}
Although \lang\ forms a solid basis for a matrix query language, it is limited in expressive power. Indeed, \lang\ is subsumed by first order logic with aggregates that uses only three variables \cite{matlang}. As consequence, no \lang\ expression exists that can compute the transitive closure of a graph (represented by its adjacency matrix) or can compute the inverse of a matrix. Furthermore, no \lang\ expression exists which detects four-cliques in a graph \cite{matlang}.
Also, $\lang$ is not expressive enough to perform classical linear algebra algorithms such as LU-decomposition (Gaussian elimination).
 Rather than extending \lang\ with specific linear algebra operators, such as matrix inversion, we next introduce a limited form of recursion in \lang.
As we will see shortly, this extension allows us to express many linear algebra algorithms, including matrix inversion and LU-decomposition.
\floris{The paragraph above needs to be modified as it may overlap too much with the yet-to-be-written Introduction.}

%FIRST TRY

%
%
% Recall the basics of \lang\ \cite{matlang}, and linear algebra. Perhaps stress where \lang\ falls short with respect to natural linear algebra questions.
%
% \bigskip

%
%Since the baseline for our study is the \lang\ language introduced in \cite{matlang}, here we briefly recap its syntax and semantics. Let $\Mnam = \{M_1, M_2,\ldots\}$ be a countably infinite set of {\em matrix names}, $\Mvar = \{V_1, V_2, \ldots\}$ a countably infinite set of {\em matrix variables}, and $\Fun$ a set of functions $f:\mathbb{C}^n \mapsto \mathbb{C}$, where $\mathbb{C}$ denotes the set of complex numbers. A {\em vocabulary} $\Voc$ is a triple $\Voc = (\Mnam', \Mvar, \Fun)$, where $\Mnam'\subset \Mnam$ is a finite subset of matrix names. An {\em $\Voc$-instance} $\I$ maps every $M\in \Mnam'$ to a concrete matrix, and assigns a dimension $(m,n)$, with $m,n\in \mathbb{N}$ to every matrix variable. That is, if $M\in \Mnam'$, then $\I(M)$ is a matrix over $\mathbb{C}$ of some dimension, and if $V\in \Mvar$, then $\ddim(\I(V)) = (m,n)$; that is, $V$ is a placeholder for a matrix of a specific dimension\footnote{Note that in \cite{matlang} the authors introduce the notion of abstract typing for vocabulary symbols. To simplify the notation, and stay closer to standard definitions of First order logic, we opt to assign matrix types to variables directly on an instance level.}.
%
%The syntax of \lang\ expressions over the vocabulary $\Voc$ is defined by the following grammar:
%
%\begin{tabular}{lcll}
%$e$ & $::=$ & $M\in \Mnam'$ & (matrix name)\\
% & $|$ & $V\in \Mvar$ & (matrix variable)\\
% & $|$ & $\llet$ & (local binding)\\
% & $|$ & $e^*$ & (conjugate transpose)\\ 
% & $|$ & $\ones(e)$ & (one-vector)\\ 
% & $|$ & $\diag(e)$ & (diagonalization of a vector)\\  
% & $|$ & $e_1 \cdot e_2$ & (matrix multiplication)\\   
% & $|$ & $\apply{f}(e_1,\ldots ,e_n)$ & (pointwise application of $f$).    
%\end{tabular}
%
%To define the semantics of a \lang\ expression $e$ over $\Voc$, we first need to know whether $e$ can be evaluated due to matrix dimension constraints, since, for example,  the product $M_1 \cdot M_2$ of two matrices is not always defined. To overcome this, we define the {\em type} of each expression $e$ with respect to an instance $\I$, denoted by $\ttype(e)^\I$ in Table \ref{tab-types}.
%
%\begin{table}
%\begin{tabular}{rcll}
%$\ttype(M)^\I$ & $=$ & $\dim(\I(M))$, for $M\in \Mnam'$\\
%$\ttype(V)^\I$ & $=$ & $\dim(\I(V))$, for $V\in \Mvar$\\
%%$\ttype(\llet)^\I$ & $=$ & this is a dumb operator\\
%$\ttype(e^*)^\I$ & $=$ & $(m,n)$, if $\ttype(e)^\I = (n,m)$\\%, and undefined otherwise\\
%$\ttype(\ones(e))^\I$ & $=$ & $(n,1)$, if $\ttype(e)^\I = (n,m)$\\%, and undefined otherwise\\
%$\ttype(\diag(e))^\I$ & $=$ & $(n,n)$, if $\ttype(e)^\I = (n,1)$\\
%$\ttype(e_1\cdot e_2)^\I$ & $=$ & $(n,k)$, if $\ttype(e_1)^\I = (n,m)$, and $\ttype(e_2)^\I = (m,k)$\\
%$\ttype(\apply{f}(e_1,\ldots ,e_n))^\I$ & $=$ & $(n,m)$, if $\ttype(e_1)^\I = \ldots = \ttype(e_k)^\I = (m,n)$, and $f:\mathbf{C}^n\mathbf{C}$\\
%\end{tabular}
%\label{tab-types}
%\caption{Type of \lang\ expression $e$ over an instance $\I$.}
%\end{table}










%
% Since the baseline for our study is the \lang\ language introduced in \cite{matlang}, here we briefly recap its syntax and semantics.


\section{Extending MATLANG with for loops}\label{sec:formatlang}
%!TEX root = ../main.tex
% !TeX spellcheck = en_US



To extend \lang\ with recursion, we take inspiration from classical linear algebra algorithms, such as those described in \cite{num}. Many of these algorithms are based on \textit{for loops} in which the termination condition for each loop is determined by the matrix dimension. We have seen how the transitive closure of a matrix can be computed using for loops in the Introduction. Here we add this ability to \lang, and show that the resulting language, called $\langfor,$ can compute properties outside of the scope of \lang. We see more advanced examples, such as Gaussian elimination and computing an inverse of a matrix, later in the paper. 

\subsection{Syntax and semantics of \langfor} The syntax of \langfor is defined just as for \lang\, but with an extra rule in the grammar:
\medskip

\begin{tabular}{lcll}
 $\ffor{v}{X}{e}$ & (canonical for loop, with $v, X \in \Mvar$). 
\end{tabular}

\medskip
\noindent Intuitively, $X$ is a matrix variable which is iteratively updated according to the expression $e$. We simulate iterations of the form ``\texttt{for} $i\in [1..n]$'' by letting $v$ loop over the \textit{canonical vectors} $b_1^n,\ldots,b_n^n$ of dimension $n$. Here,
%
% That is,
% for each $n\in \mathbb{N}$, we denote by $b_1^n,\ldots ,b_n^n$ the  canonical vectors of dimension $n$; that is,
$b_1^n = [1\ 0 \cdots 0]^T$, $b_2^n = [0\ 1\ 0 \cdots 0]^T$, etc. When $n$ is clear from the context we simply write $b_1,b_2,\ldots$. In addition, the expression $e$ in the rule above may depend on $v$. 

We next make the semantics precise and start by
declaring the type of loop expressions.
Given a schema $\Sch$, the type of a \langfor expression $e$, denoted $\ttype(e)$, is defined inductively as in \lang\, but with following extra rule:
\begin{itemize}
\item $\ttype(\ffor{v}{X}{e}) := (\alpha,\beta)$, if \\
$\ttype(e)=\ttype(X) =(\alpha,\beta)$ and $\ttype(v) = (\gamma,1)$.
\end{itemize}
We note that $\Sch$ now necessarily includes $v$ and $X$ as variables and assigns size symbols to them.
%Similarly as when defining \lang\, here we start with a core set of matrix operations (i.e sum, product, and the transpose of a matrix). In addition to this, we allow applying functions, and  the $\ffor{v}{X}{e}$ construct, which allows looping over the canonical vectors of a specified dimension, and updating the context of the variable $X$ in each iteration. The latter operator is inspired by classical Linear Algebra algorithms \cite{num}, which commonly use loops whose termination conditions are determined by the matrix dimension, and will allow us to express many properties of interest.
%A \langfor {\em schema} $\Sch$ is a pair $\Sch=(\Mnam,\size)$, where $\Mnam\subset \Mvar$ is a finite set of matrix variables, and $\size: \Mvar \mapsto \DD\times \DD$ is a function that maps each matrix variable to a pair of {\em size symbols}. Given a schema $\Sch$, the type of a \langfor expression, denoted $\ttype(e)$, is defined inductively as follows:
%\begin{itemize}
%\item $\ttype(V) = \size(V)$, for a matrix variable $V$,
%\item $\ttype(e^*) = (\beta,\alpha)$, if $\ttype(e)=(\alpha,\beta)$; and undefined if $\ttype(e)$ is undefined,
%\item $\ttype(e_1 \cdot e_2) = (\alpha,\gamma)$, whenever $\ttype(e_1)=(\alpha,\beta)$, and $\ttype(e_2)=(\beta,\gamma)$; and is undefined otherwise,
%\item $\ttype(e_1 + e_2) = \ttype(e_1)$, if $\ttype(e_1) = \ttype(e_2)$; and is undefined otherwise,
%\item $\ttype(f(e_1,\ldots ,e_n)) = (1,1)$, whenever $\ttype(e_1)= \cdots =\ttype(e_n)=(1,1)$, and $f:\mathbb{C}^n\mapsto \mathbb{C}$; and is undefined otherwise,
%\item $\ttype(f(e_1,\ldots ,e_n)) = (\alpha,\beta)$, whenever $\ttype(e_1) = \ldots = \ttype(e_k) = (\alpha,\beta)$, and $f:\mathbb{C}^n \mapsto  \mathbb{C}$; and is undefined otherwise, and,
%\item $\ttype(\ffor{v}{X}{e}) = \ttype(e)$, if $\ttype(X) = \ttype(e)$, and $\ttype(v) = (\gamma,1)$; and is undefined otherwise.
%\end{itemize}
%\cristian{Is this the same than in the previous section? Is there any difference?}
% To extend \lang\, we add $\ffor{v}{X}{e}$ construct, which allows looping over the canonical vectors of a specified dimension, and updating the context of the variable $X$ in each iteration. The latter operator is i
%
% A \langfor {\em schema} $\Sch$ is a pair $\Sch=(\Mnam,\size)$, where $\Mnam\subset \Mvar$ is a finite set of matrix variables, and $\size: \Mvar \mapsto \DD\times \DD$ is a function that maps each matrix variable to a pair of {\em size symbols}. 
We also remark that in the definition of the type of $\ffor{v}{X}{e}$, we require that $\ttype(X) = \ttype(e)$ as this expression updates the content of the variable $X$ in each iteration using the result of $e$. We further restrict the type of 
$v$ to be a vector, i.e., $\ttype(v)=(\gamma,1)$, since $v$ will be instantiated with canonical vectors.
% As we will see below, evaluating $e$ will depend on a canonical vector stored in $v$, and the current content of $X$. %Another difference from \lang\ is that we allow function application only on scalars (more precisely, on $1\times 1$ matrices).
%
A \langfor\ expression $e$ is well-typed over a schema $\Sch$ if its type is defined. 

For well-typed expressions we next define their semantics. This is done in an inductive way, just as for \lang. To define the semantics of $\ffor{v}{X}{e}$ over an instance $\I$, we need the following notation. Let $\I$ be an instance and $V\in \Mnam$. Then $\I[V := A]$ denotes an instance that coincides with $\I$, except that the value of the matrix variable $V$ is given by the matrix $A$. Assume that
$\ttype(v)= (\gamma,1)$, and $\ttype(e) = (\alpha,\beta)$ and $n := \dom(\gamma)$. Then, $\sem{\ffor{v}{X}{e}}{\I}$ is defined iteratively, as follows:
\begin{itemize}
\item Let $A_0 := \mathbf{0}$ be the zero matrix of size $\dom(\alpha)\times \dom(\beta)$.
\item For $i=1,\ldots n$, compute $A_i:= \sem{e}{\I[v := b^{n}_i, X:= A_{i-1}]}$.
\item Finally, set $\sem{\ffor{v}{X}{e}}{\I}:= A_{n}$.
\end{itemize}

%%
% A \langfor {\em instance} $\I$ over a schema $\Sch$, is a pair $\I = (\dom,\conc)$, where $\dom : \DD \mapsto \mathbb{N}$ assigns a value to each size symbol, and $\conc : \Mnam \mapsto \mtr{\mathbb{C}}$ assigns a concrete matrix to each matrix variable $M\in \Mnam$, such that $\dim(\conc(M)) = \dom(\alpha)\times \dom(\beta)$, where $\size(M) = (\alpha,\beta)$. As before, we assume that $\dom(1) = 1$, for every instance $\I$.
%  %(meaning that $e_1^n = \begin{bmatrix} 1 \\ 0 \\ \vdots \\ 0 \end{bmatrix}$, etc.).
% If $\I$ is an instance, $V$ a matrix variable such that $\size(V)= (\alpha,\beta)$, and $M$ a matrix of dimension $\dom(\alpha)\times \dom(\beta)$, then $\I[V := M]$ denotes an instance that coincides with $\I$, apart from the fact that the value of the matrix variable $V$ is the matrix $M$.
% %If $e$ is a well-typed expression according to $\Sch$, then we denote by $\sem{e}{\I}$ the matrix obtained by evaluating $e$ over $\I$, and define it as follows:
% %\begin{itemize}
% %\item $\sem{M}{\I} = \conc(M)$, for $M\in \Mnam$;
% %\item $\sem{e^*}{\I} = \sem{e}{\I}^*$, where $M^*$ is the conjugate transpose of a matrix $M$;
% %\item $\sem{e_1\cdot e_2}{\I} = \sem{e_1}{\I} \cdot \sem{e_2}{\I}$;
% %\item $\sem{e_1 + e_2}{\I} = \sem{e_1}{\I} + \sem{e_2}{\I}$;
% %\item $\sem{f(e_1,\ldots ,e_n)}{\I}$ is a $1\times 1$ matrix whose only entry has the value $f(\sem{e_1}{\I},\ldots ,\sem{e_n}{\I})$. Here we abuse the notation and use $\sem{e}{\I}$ to denote both a $1\times 1$ matrix, and a scalar from $\mathbb{C}$.
% %\item $\sem{f(e_1,\ldots ,e_n)}{\I}$ is a matrix $A$ of the same size as $\sem{e_1}{\I}$, and where $A_{ij}$ has the value $f(\sem{e_1}{\I}_{ij},\ldots ,\sem{e_n}{\I}_{ij})$.
% %\end{itemize}
% %\cristian{The same as with the type function, we can reuse the semantics of the previous section.}
%
% If $e$ is a well-typed expression according to $\Sch$, then we denote by $\sem{e}{\I}$ the matrix obtained by evaluating $e$ over $\I$, and define it as in \lang\, adding the new operator.
% \floris{Perhaps it is insightful to already give an easy example here, e.g., clique. Perhaps even better are the very simple expressions for diag and one vector. The text below is quite
% cryptic.}
%
% Notice that evaluating $\ffor{v}{X}{e}$ over an instance that already assigns values to $v$ and $X$, these values get overwritten immediately. Notice that if $e$ does not use the variable $v$, then it will have the same value in each iteration. Sometimes when we want to make it explicit that the expression $e$ uses matrix variables $X_1,\ldots ,X_n$, we write $e(X_1,\ldots ,X_n)$, where $X_1,\ldots ,X_n$ are all the matrix variables mentioned in $e$. Analogously with First Order Logic, we could define the notion of free and bound variables (i.e. the ones in the scope of a \texttt{for} operator), however, since we explicitly rewrite the value of these variables, this distinction will not play an important role.
%
%
% \cristian{I don't understand this comment of free and bound variables. What do you mean? I believe than similar than in first order variables, one can define this concept, but here it is not used.}
%
% %\subsection{Examples of \langfor expressions}




%%needs dimensions:
%Next we illustrate the versatility of the introduced language. To begin, we note that, unlike the original \lang\ proposal, we have at our disposals canonical vectors of arbitrary dimension. The most basic use of canonical vectors is for accessing a position $ij$ of some matrix $M$, a property which lies outside of the scope of \lang. For this, we can simply use the expression $(e_i^{n})^*\cdot M \cdot e_j^m$, where $M$ is a matrix of size $m\times n$, $e_i^n$ is the $i$th canonical vector of dimension $n$, and likewise, $e_j^m$ is the $j$th canonical vector of dimension $m$.

For better understanding how \langfor  works, we next provide some  examples.
We start by showing that the one-vector and $\diag$ operators are redundant
in \langfor.
% next present a series of properties that the language can express and introduce some operators that will be commonly used throughout the paper.
% , and show how the proposed language captures original \lang.
\begin{example}
We first show how the one-vector operator $\ones(e)$ can be expressed using \texttt{for} loops.
It suffices to consider the expression
$$e_{\ones}:=\ffor{v}{X}{X+v},$$
with $\ttype(v)=(\alpha,1)=\ttype(X)$ if $\ttype(e)=(\alpha,\beta)$. This expression is well-typed
and is of type $(\alpha,1)$. When evaluated over some instance $\I$ with $n=\dom(\alpha)$, $\sem{e_{\ones}}{\I}$ is defined as follows.
Initially, $A_0:=\mathbf{0}$. Then $A_i:=A_{i-1}+b_i^n$, i.e., the $i$th canonical vector is added to $A_{i-1}$.
Finally, $\sem{e_{\ones}}{\I}:=A_n$ and this now clearly coincides with $\sem{\ones(e)}{\I}$.\qed
\end{example}

% \begin{example}\label{ex:forml-identity}
% We first show how one can construct the identity matrix of a certain dimension. For this, it suffices to use the expression $$e_{\mathsf{Id}}:=\ffor{v}{X}{X + v\cdot v^T}.$$ For this expression to be well-typed, $v$ has to be a vector variable of type $\alpha\times 1$ and $X$ a matrix variable of type $(\alpha,\alpha)$. When evaluated over some instance $\I$, this loop starts by initializing $X$ as the zero matrix of dimension $n\times n$, where $n=\dom(\alpha)$, and then adds to $X$ the matrix $b_1^n\cdot (b_1^n)^T$ in the first iteration, the matrix $b_2^n\cdot (b_2^n)^T$ in the second iteration, and so on, until finally $b_n^n\cdot (b_n^n)^T$ is added. Hence, $X$ equals the identity matrix of dimension $n\times n$ in the end.\qed
% \end{example}
%
\begin{example}\label{ex:diag}
% Indeed, a minor variation of the expression $e_{\mathsf{Id}}$ suffices:
As another example, we show that the $\diag$ operator is redundant in \langfor.
Indeed, it suffices to consider the expression
$$e_{\mathsf{diag}}:=
\ffor{v}{X}{X + (v^T\cdot e) \times v\cdot v^T},$$ where $e$ is a \langfor\  expression of type $(\alpha,1)$. For this expression to be well-typed, $v$ has to be a vector variable of type $\alpha\times 1$ and $X$ a matrix variable of type $(\alpha,\alpha)$. Then, $\sem{e_{\mathsf{diag}}}{\I}$ is defined as follows.
Initially, $A_0$ is the zero matrix of dimension $n\times n$, where $n=\dom(\alpha)$. Then, in each iteration
$i\in[1..n]$, $A_{i}:=A_{i-1}+  (b_i^T\cdot\sem{e}{\I})\times (b_i^n\cdot (b_i^n)^T)$. In other words, $A_i$ is obtained by adding the matrix with value $(\sem{e}{\I})_i$ on position $(i,i)$ to $A_{i-1}$. Hence, $A_n=\sem{\diag(e)}{\I}$ and
thus $\sem{e_{\mathsf{diag}}}{\I}=\sem{\diag(e)}{\I}$.\qed%
 \end{example}
% That is, $A_i$ is obtained from $A_{i-1}$the $i$th entry of
% $\sem{e}{\I})^T$ is multiplied
%
% The difference with $e_{\mathsf{Id}}$ is that instead of the value $1$, the value
% $b_i^T\cdot\sem{e}{\I}=\sem{e}{\I}_{i}$ is put on position $i$ on the diagonal, when evaluating $e_{\mathsf{diag}}$ on $\I$.
%  One can verify that for the above expression to be well-typed, $v$ needs to be of type $(\alpha,1)$ and $X$ of type $(\alpha,\alpha)$. When evaluated over some instance $\I$, $X$ is initialised with the zero matrix of dimension $n\times n$, where $n=\dom(\alpha)$. Then, in iteration $i$, the $n\times n$-matrix
% $b_i^T\cdot \sem{e}{\I}\times (b_i\cdot b_i^T)$ is added. In other words, on the $i$th element of the diagonal, represented by $b_i\cdot b_i^T$, the value $b_i^T\cdot\sem{e}{\I}=\sem{e}{\I}_{i}$ is added. 
% That is, the expression $e_{\mathsf{diag}}$ evaluates to $\sem{\diag(e)}{\I}$.\qed

% It is also readily verified that the one-vector operator in \lang\ becomes redundant in \langfor.

%\floris{I commented out the part related to function applications, i.e., that function applications on scalars is sufficient. It is interesting but may be too much at this point? }

% Furthermore, general function applications can be simulated in \langfor\ by just assuming function applications $f(e_1,\ldots,e_k)$ for $f\in\Fun_k$ where $\ttype(e_i)=(1,1)$ for all $i\in[k]$, as is illustrated next.
% As before, we write \langfor$(\Fun)$ when \langfor\ expressions require certain function applications in some set $\Fun$.
%
% \begin{example}Consider the \langfor$(f)$ expression
% \begin{multline*}
% e=\texttt{for }v,X.\, X + \\
% \texttt{for }w,Y.\, Y + v\cdot f(v^T\cdot e_1\cdot w, \ldots, v^T\cdot e_k\cdot w)\cdot w^T.
% \end{multline*}
% In the expression above, we basically loop over all positions in $\sem{e_i}{\I}$, one by one, and apply the function $f$ on those positions.  One can see that $\sem{e}{\I}=\sem{f(e_1,\ldots,e_k)}{\I}$ for $e_i$'s of arbitrary type, but that $f$ is only
% applied on expressions of type $(1,1)$, such as $v^T\cdot e_i\cdot w$. \qed
% \end{example}
% %
%
% \smallskip
% \noindent
% \textbf{Core operators}. The previous example show that we can define \langfor\ using a small number of core operators, whilst still extending \lang. In particular, it suffices to define \langfor to consist of expressions of the form
% %
% %
% %
% % Note that the  \texttt{for} operator is much more powerful that it seems. We can simulate some operations of the original \lang\ semantics. Assume the allowed expressions are
%
% \begin{tabular}{lcll}
% $e$ & $:=$ & $V\in \Mvar$ & (matrix variable)\\
%  % & $|$ & $e^T$ & (transpose)\\
%  & $|$ & $e_1 \cdot e_2$ & (matrix multiplication)\\
%  % & $|$ & $e_1 + e_2$ & (matrix addition)\\
%  & $|$ & $\text{apply}[f](e_1,\ldots ,e_k)$ & (application of $f\in \Fun$)\\
%  & $|$ & $\ffor{v}{X}{e}$ & (canonical for loop, with $v, X \in \Mvar$).
% \end{tabular}
%
% Where function application $\text{apply}[f]$ works only on scalars. This is
%
% \begin{itemize}
% \item $\ttype(\text{apply}[f](e_1,\ldots ,e_n)) = (1,1)$, whenever $\ttype(e_1)= \cdots =\ttype(e_n)=(1,1)$, and $f:\mathbb{C}^n\mapsto \mathbb{C}$; and is undefined otherwise.
% \item $\sem{\text{apply}[f](e_1,\ldots ,e_n)}{\I}$ is a $1\times 1$ matrix whose only entry has the value $f(\sem{e_1}{\I},\ldots ,\sem{e_n}{\I})$.
% \end{itemize}

These examples illustrate that we can limit \langfor to consist of the following ``core'' operators: transposition, matrix multiplication and addition, scalar multiplication, pointwise function application, and for-loops. More specific, \langfor is defined by the following simplified syntax:
$$
e ::= V \ \mid \ e^T \ \mid \ e_1 \cdot e_2 \ \mid \ e_1 + e_2 \ \mid \ a\times e  \ \mid \  f(e_1,\ldots ,e_k) \ \mid \ \ffor{v}{X}{e}
$$

Similarly as for \lang, we write $\langforf{\Fun}$ for some set $\Fun$ of functions when these are required for the task at hand.

%  $f(e_1,\ldots,e_k)$ for $f\in\Fun_k$ where each $e_i$
% has type $(1,1)$.

%As a final example, we show how the Floyd-Warshall algorithm for computing the transitive closure of a matrix given in the Introduction can be defined using \langfor.
%\begin{example}\label{ex:floyd}
%The following expression in $\langf{f_>}$ computes the transitive closure of a $n\times n$ matrix $A$.
%\begin{tabbing}
%\texttt{for }\=$e_k,X_1.\, X_1 + $\\
%\> \texttt{for }\=$e_i,X_2.\, X_2 +$ \\
%\>\>\texttt{for }\=$e_j,X_3.\, X_3 +$ \\
%\>\>\>$(e_i^T\cdot A\cdot e_k \cdot e_k^T\cdot A\cdot e_j)\times e_i\cdot e_j^T$
%\end{tabbing}
%\end{example}


As a final example, we show that we can compute whether a graph contains a $4-\textsf{clique}$ using \langfor.
\begin{example}\label{ex:fourcliques}
To test for $4$-cliques it suffices to consider the expression
\begin{tabbing}
\texttt{for }\=$u,X_1.\, X_1 + $\\
\> \texttt{for }\=$v,X_2.\, X_2 +$ \\
\>\>\texttt{for }\=$w,X_3.\, X_3 +$ \\
\>\>\>\texttt{for }\=$x,X_4.\, X_4 +$ \\
\>\>\>\>$u^T\cdot V\cdot v \cdot u^T\cdot V\cdot w\cdot u^T\cdot V\cdot x \cdot $\\
\>\>\>\>$v^T\cdot V\cdot w \cdot v^T\cdot V\cdot x\cdot w^T\cdot V\cdot x \cdot g(u,v,w,x)$\\
\end{tabbing}
with $g(u,v,w,x)=f(u,v)\cdot f(u,w)\cdot f(u,x)\cdot f(v,w)\cdot f(v,x)\cdot f(w,x)$ and
$f(u,v)=1-u^T\cdot v$. Note that $f(b_i^n,b_j^n)=1$ if $i\neq j$ and $f(b_i^n,b_j^n)=0$ otherwise.
Hence, $g(b_i^n,b_j^n,b_k^n,b_\ell^n)=1$ if and only if all $i,j,k,l$ are pairwise different.
When evaluating the expression on an instance $\I$ such that $V$ is assigned to the adjacency 
matrix of a graph, the expression above evaluates to a non-zero value if and only if the graph
contains a four-clique.\qed
\end{example}
%
% % =\begin{cases}
% %                0 \text{ if } u=v \\
% %                1 \text{ if } u\neq v
% %             \end{cases}
% % 		\]
%
% To distinguish four-cliques, we will need to determine whether we are dealing with four different nodes. For this, we will utilize the function $$g(u,v,w,r)=f(u,v)\cdot f(u,w)\cdot f(u,r)\cdot f(v,w)\cdot f(v,r)\cdot f(w,r),$$

As mentioned previously, given \lang\ can not express 4-cliques \cite{BrijderGBW19}, we easily obtain the following.

\begin{proposition}
\label{cor-ml-fml}
For any collection of functions $\Fun$, 
$\langf{\Fun}$ is properly subsumed by $\langforf{\Fun}$.
\end{proposition} 







%\domagoj{This is used to simulate circuits, so I put in into a lemma. Perhaps just move to the appendix before that particular proof.}
%
%Another interesting consequence of allowing for loops in \lang\ is that the pointwise product and sum can be defined without direct access to these operators. Namely, the following result tells us that we can always assume to have access to the product and the sum.
%%This will be of crucial importance when comparing \langfor\ to arithmetic circuits in Section \ref{sec:circuits} and to annotated relations in Section \ref{sec:restrict}. Formally, we have:
%
%\begin{lemma}
%\label{lm-prod-sum}
%Let $f_\odot^k:\RR^k\mapsto \RR$ and $f_\oplus^k:\RR^k\mapsto \RR$ be the product and the sum function with $k$ attributes, respectively. Then it holds that $\langforf{\emptyset} \equiv \langforf{\{f_\odot^k,f_\oplus^k \ | \ k\in \mathbf{N}\}}$.
%\end{lemma}
%\cristian{This result is very ugly, why we want this? The interesting result is that you only need the function between constants to define the point-wise application of functions.}
%
%In what follows, we will sometimes abuse the notation and write $\langfor$ both when we are talking about $\langfor(\emptyset)$, and when referring to the language in general. On the other hand, when stating formal results, we will always make the set of functions explicit, and will use $\langfor$ to denote $\langfor(\emptyset)$.

\subsection{Design decisions behind \langfor}

\noindent\textbf{Loop Initialization.} As the reader may have observed, in the semantics of \texttt{for} loops we 
always initialise $A_0$ to the zero matrix~$\mathbf{0}$ (of appropriate dimensions). It is often convenient
to start the iteration given some concrete matrix  originating from the result of evaluation a \langfor\ expression $e_0$. To make this explicit, we write $\initf{e_0}{v}{X}{e}$ and its semantics is defined as above
with the difference that $A_0:=\sem{e_0}{\I}$. We observe, however, that $\initf{e_0}{v}{X}{e}$ can already
be expressed in \langfor. In other words, we do not loose generality by assuming an initialisation of $A_0$ by $\mathbf{0}$.
The key insight is that in \langfor\ we can check during evaluation whether or not
the current canonical vector $b_i^n$ is equal to the $b_1^n$. This is not entirely trivial to see and is due to fact that \texttt{for} loops iterate over the canonical vectors in a fixed order. We discuss this more in the next subsection, when we talk about the order. In particular, we can define a \langfor expression $\mmin$, which when evaluated on an instance, returns $1$ if its input vector is $b_1^n$, and returns $0$ otherwise. Given $\mmin$, consider now the
\langfor\ expression
 $$\ffor{v}{X}{\mmin{v}\cdot e(v,X/e_0) + (1-\mmin{v})\cdot e(v,X)},$$
 where we explicitly list $v$ and $X$ as matrix variables on which $e$ potentially depends on, and where
 $e(v,X/e_0)$ denotes the expression obtained by replacing every occurrence of $X$ in $e$ with $e_0$.
%
%  As mentioned previously, all of the iterations in \texttt{for} loops start by initializing $A_0$ to the null matrix. We have already seen that this property is useful for defining the extremal points of our ordering, however, sometimes we would like to start the iteration using some concrete matrix $B$. Using the operator $\mmin$, we can actually achieve this as follows. First, let $\ffor{v}{X}{e(v,X)}$ be the \texttt{for} loop that begins iterating from the null matrix. Here we write $e(v,X)$ to denote the variables that $e$ potentially (but not necessarily) uses. To start the iteration from $B$ it now suffices to use the following loop:
% $$\ffor{v}{X}{\mmin{v}\cdot e(v,X/B) + (1-\mmin{v})\cdot e(v,X)} \quad (\dag),$$
% where $e(v,X/B)$ denotes the expression obtained by replacing every occurrence of $X$ in $e$ with $B$.
When evaluating this expression on an instance $\I$, $A_0$ is initial set to the zero matrix, in the first iteration (when  $v=b_1^n$ and thus $\mmin{v}=1$)
we have $A_1=\sem{e}{\I[v:=b_1^n,X:=\sem{e_0}{\I}]}$, and for consecutive iterations (when only the part related to $1-\mmin{v}$ applies) $A_i$ is updated as before. Clearly, the result of this evaluation is equal to
$\sem{\initf{e_0}{v}{X}{e}}{\I}$. 

%As an illustration, we consider the transitive closure program from the beginning of this section. Similarly as for \lang, we write $\langforf{\Fun}$ for some set $\Fun$ of functions when these are required for the task at hand.
%%As an illustration, we consider the transitive closure program from the beginning of this section. Similarly as for \lang, we write \langfor$(\Fun)$ for some set $\Fun$ of functions when these are required for the task at hand. Analogously, we write $\langfor$ to denote $\langfor(\emptyset)$, but we might abuse the notation and write $\langfor$ to refer to the language in general.
%\begin{example}
% In $\langforf{f_>}$ we can express transitive closure with the following expression
%$$
%f_{>0}\left(\initf{e_{\mathsf{Id}}}{v}{X}{X\cdot (e_{\mathsf{Id}}+V)}\right),
%$$
%where $e_{\mathsf{Id}}$ is a \langfor\ expression constructing the identity matrix $I$. The expression $e_{\mathsf{Id}}$
%is just like the expression $e_{\mathsf{diag}}$ in Example~\ref{ex:diag}, but now only puts value $1$ on the diagonal.
% Observe that we start the loop using an initialisation by $I$. Hence, when evaluated on an instance $\I$ such that $V$ is an adjacency matrix $A$ of a graph, the expression above simply computes $(I+A)^n$, followed by the pointwise function application $f_{>0}$. In other words, it returns the adjacency matrix of the transitive closure of graph.\qed
%\end{example}

%As an illustration, we consider the Floyd-Warshall algorithm for computing the transitive closure of a matrix given in the Introduction. Similarly as for \lang, we write $\langforf{\Fun}$ for some set $\Fun$ of functions when these are required for the task at hand.

As an illustration, we consider the Floyd-Warshall algorithm given in the Introduction. 

%\begin{example}\label{ex:floyd}
%Consider the following expression in $\langf{f_>}$, where $f_>:\RR^2\to\RR$, is a function such that $f_>(x,y) = 1$ when $x>y$, and is zero otherwise:
%\begin{tabbing}
%\texttt{for\,}\=$e_k,\, X_1\!=\!A.\ \ X_1 \ + $\\
%\> \texttt{for\,}\=$e_i, \, X_2.\ \ X_2 \ +$ \\
%\>\>\texttt{for\,}\=$e_j,\, X_3.\ \ X_3 \ +$ \\
%\>\>\> $f_>(e_i^T\cdot X_1 \cdot e_j\ ,\ $\=$e_i^T\cdot X_1\cdot e_k + e_k^T\cdot X_1\cdot e_j)\ \cdot$\\
%\>\>\>\>$(e_i^T\cdot X_1\cdot e_k + e_k^T\cdot X_1\cdot e_j) \times e_i\cdot e_j^T$
%\end{tabbing}
%The expression $e_{FW}$ simulates the Floyd-Warshall algorithm by updating the matrix $A$, which is stored in the variable $X_1$. The inner sub-expression here constructs an $n\times n$ matrix that contains the distance of the path between $i$ and $j$ that goes through $k$ if and only if this distance is less than what we had previously here.. ZERO SHOULD BE INFINITY.
%$(i,j)$ if and only if one can reach the vertex $j$ from $i$ by going through $k$, and zero elsewhere. If an instance $\I$ assigns to $A$ the adjacency matrix of a graph, then $\sem{e_{FW}}{\I}$ contains the shortest distance between a vertex $i$ and the vertex $j$.
%% The transitive closure in then obtained using the $\langf{f_>}$ expression $f_>(e_{FW})$, where $f_{>0}:\RR\to\RR$ is such that $f_{>0}(x):=1$ if $x>0$ and $f_{>0}(x):=0$ otherwise.  
%\cristian{The same with the problem of the shortest path mentioned in the intro.}
%\qed
%\end{example}

\begin{example}\label{ex:floyd}
Consider the following expression:
\begin{tabbing}
$e_{FW} := $ \texttt{for\,}\=$v_k,\, X_1\!=\!A.\ \ X_1 \ + $\\
\> \texttt{for\,}\=$v_i, \, X_2.\ \ X_2 \ +$ \\
\>\>\texttt{for\,}\=$v_j,\, X_3.\ \ X_3 \ +$ \\
\>\>\>$(v_i^T\cdot X_1\cdot v_k \cdot v_k^T\cdot X_1\cdot v_j)\times v_i\cdot v_j^T$
\end{tabbing}
The expression $e_{FW}$ simulates the Floyd-Warshall algorithm by updating the matrix $A$, which is stored in the variable $X_1$. The inner sub-expression here constructs an $n\times n$ matrix that contains one in the position $(i,j)$ if and only if one can reach the vertex $j$ from $i$ by going through $k$, and zero elsewhere. If an instance $\I$ assigns to $A$ the adjacency matrix of a graph, then $\sem{e_{FW}}{\I}$ will be equal to the matrix produced by the algorithm given in the Introduction.
%contains a non zero value if and only if $j$ is reachable from $i$. 
%If we use the function $f_{>0}:\RR\to\RR$, with $f_{>0}(x):=1$ if $x>0$ and $f_{>0}(x):=0$ otherwise, then the $\langf{f_{>0}}$ expression $f_{>0}(e_{FW})$ computes precisely the same matrix as the algorithm from the Introduction.
% The transitive closure in then obtained using the $\langf{f_>}$ expression $f_>(e_{FW})$, where $f_{>0}:\RR\to\RR$ is such that $f_{>0}(x):=1$ if $x>0$ and $f_{>0}(x):=0$ otherwise.  
%\cristian{The same with the problem of the shortest path mentioned in the intro.}
\qed
\end{example}


\noindent\textbf{Order.} With the introduction of \texttt{for} loops we not only extend \lang\ with bounded recursion, we also introduce order information. Indeed, the semantics of the \texttt{for} operator assumes that the canonical vectors $b_1,b_2,\ldots$
are accessed in this order. It implies, among other things, that \langfor\ expressions are not permutation-invariant.
We can, for example, return the bottom right-most entry in a matrix. Indeed, consider the expression $e_{\mathsf{max}} := \ffor{v}{X}{v}$ which, for it to be well-typed, requires both $v$ and $X$ to be of type $(\alpha,1)$. Then, $\sem{e_{\mathsf{max}}}{\I}=b_n^n$, for $n=\dom(\alpha)$, simply because initially, $X=\mathbf{0}$, but $X$ will be overwritten by $b_1^n,b_2^n,\ldots,b_n^n$, in this order. Hence, at the end of the evaluation $b_n^n$ is returned.
To extract the bottom right-most entry from a matrix, we now simply use $e_{\mathsf{max}}^T\cdot V\cdot e_{\mathsf{max}}$.

Although the order is implicit in \langfor, we can explicitly use this order in \langfor expressions. More precisely, the order on canonical vectors is made accessible by
using the following matrix:
\[
S_{\leq} = \begin{bmatrix}
1 & 1 & \cdots &  1 \\
0 & 1 & \cdots & 1\\
\vdots & \vdots & \ddots & 1 \\
0 & 0 & \cdots & 1 
\end{bmatrix}.
\] 
We observe that $S_{\leq}$ has the property that $b_i^T\cdot S_{\leq} \cdot b_j=1$, for two canonical vectors $b_i$ and $b_j$ of the same dimension, if and only if $i\leq j$. Otherwise, $b_i^T\cdot S_{\leq} \cdot b_j=0$. 
Interestingly, we can build matrix $S_{\leq}$ with the following \langfor expression:
$$
e_{\leq}=\ffor{v}{X}{X + \left((X\cdot e_{\mathsf{max}}) + v \right)\cdot v^T + v\cdot e^T_{\mathsf{max}}},
$$
where $e_{\mathsf{max}}$ is as defined above. The intuition behind this expression is that by using the last canonical vector $b_n$, as returned by $e_{\mathsf{max}}$, we have access to the last column of $X$ (via the product $X\cdot e_{\mathsf{max}}$). We use this column such that after the $i$-th iteration, this column contains the $i$-th column of $S_{\leq}$. This is done by incrementing $X$ with $v\cdot e_{\mathsf{max}}^T$.
To construct $S_{\leq}$, in the $i$-th iteration we further increment $X$ with 
(i)~the current last column in $X$ (via $X\cdot e_{\mathsf{max}}\cdot v^T$) which holds
the $(i-1)$-th column of $S_{\leq}$; and (ii)~the current canonical vector (via $v\cdot v^T$). Hence, after iteration $i$, $X$ contains the first $i$ columns of $S_{\leq}$ and holds the $i$th column of $S_{\leq}$ in its last column. It is now readily verified that $X=S_{\leq}$ after the $n$th iteration.

It should be clear that if we can compute $S_{\leq}$ using $e_{\leq}$, then we can easily define the following predicates and vectors related with the order of cannonical vectors:
\begin{itemize}
	\item $\mathsf{succ}(u,v)$ such that $\mathsf{succ}(b_i^n,b_j^n)=1$ if $i\leq j$ and $0$ otherwise. Similarly, we can define
	$\mathsf{succ}^+(u,v)$ such that  $\mathsf{succ}^+(b_i^n,b_j^n)=1$ if $i < j$ and $0$ otherwise;
	\item $\mathsf{min}(u)$ such that  $\mathsf{min}(b_i^n)=1$ if $i=1$ and $\mathsf{min}(b_i^n)=0$ otherwise; 
	\item $\mathsf{max}(u)$ such that  $\mathsf{max}(b_i^n)=1$ if $i=n$ and $\mathsf{min}(b_i^n)=0$ otherwise; and
	\item $e_{\mathsf{min}}$ and $e_{\mathsf{max}}$ such that $\sem{e_{\mathsf{min}}}{\I}=b_1^n$ and 
	$\sem{e_{\mathsf{max}}}{\I}=b_n^n$, respectively.
\end{itemize}
The definitions of these expressions is not entirely trivial and are detailed in the appendix.

Having order information available results in \langfor to be quite expressive. We heavily rely on order information in the next sections to compute the inverse of matrices and more generally to simulate low complexity Turing machines and arithmetic circuits.

%%%% HERE PREVIOUS VERSION OF ORDER

%\noindent\textbf{Order.} With the introduction of \texttt{for} loops we not only extend \lang\ with bounded recursion, we also introduce order information. Indeed, the semantics of the \texttt{for} operator assumes that the canonical vectors $b_1,b_2,\ldots$
%are accessed in this order. It implies, among other things, that \langfor\ expressions are not permutation-invariant.
%We can, for example, return the bottom right-most entry in a matrix. Indeed, consider the expression $e_{\mathsf{max}} := \ffor{v}{X}{v}$ which, for it to be well-typed, requires both $v$ and $X$ to be of type $(\alpha,1)$. Then, $\sem{e_{\mathsf{max}}}{\I}=b_n^n$, for $n=\dom(\alpha)$, simply because initially, $X=\mathbf{0}$, but $X$ will be overwritten by $b_1^n,b_2^n,\ldots,b_n^n$, in this order. Hence, at the end of the evaluation $b_n^n$ is returned.
%To extract the bottom right-most entry from a matrix, we now simply use $e_{\mathsf{max}}^T\cdot V\cdot e_{\mathsf{max}}$.
%
%Interestingly, although the order is implicit in \langfor, we can explicitly use this order in \langfor expressions. More precisely, one can the define the following order predicates in \langfor\!:
%\begin{itemize}
%	\item $\mathsf{succ}(u,v)$ such that $\mathsf{succ}(b_i^n,b_j^n)=1$ if $i\leq j$ and $0$ otherwise. Similarly, we can define
%	$\mathsf{succ}^+(u,v)$ such that  $\mathsf{succ}^+(b_i^n,b_j^n)=1$ if $i < j$ and $0$ otherwise;
%	\item $\mathsf{min}(u)$ such that  $\mathsf{min}(b_i^n)=1$ if $i=1$ and $\mathsf{min}(b_i^n)=0$ otherwise; 
%	\item $\mathsf{max}(u)$ such that  $\mathsf{max}(b_i^n)=1$ if $i=n$ and $\mathsf{min}(b_i^n)=0$ otherwise; and
%	\item $e_{\mathsf{min}}$ and $e_{\mathsf{max}}$ such that $\sem{e_{\mathsf{min}}}{\I}=b_1^n$ and 
%	$\sem{e_{\mathsf{max}}}{\I}=b_n^n$, respectively.
%\end{itemize}
%The definitions of these expressions is not entirely trivial and are detailed in the appendix.
%We here only highlight that successor information on canonical vectors is made accessible by
%using the following matrix:
%\[
%S_{\leq} = \begin{bmatrix}
%1 & 1 & \cdots &  1 \\
%0 & 1 & \cdots & 1\\
%\vdots & \vdots & \ddots & 1 \\
%0 & 0 & \cdots & 1 
%\end{bmatrix}.
%\] 
%We observe that $S_{\leq}$ has the property that $b_i^T\cdot S_{\leq} \cdot b_j=1$, for two canonical vectors $b_i$ and $b_j$ of the same dimension, if and only if $i\leq j$. Otherwise, $b_i^T\cdot S_{\leq} \cdot b_j=0$. It should be clear that if we can compute $S_{\leq}$ using an expression $e_{\leq}$ in \langfor, then we can define
%$
%\mathsf{succ}(v,w):=v^T\cdot e_{\leq} \cdot w.
%$
%For $e_{\leq}$ we can take the following \langfor expression:
%$$
%e_{\leq}=\ffor{v}{X}{X + \left((X\cdot e_{\mathsf{max}}) + v \right)\cdot v^T + v\cdot e^T_{\mathsf{max}}},
%$$
%where $e_{\mathsf{max}}$ is as defined above. The intuition behind this expression is that by using the last canonical vector $b_n$, as returned by $e_{\mathsf{max}}$, we have access to the last column of $X$ (via the product $X\cdot e_{\mathsf{max}}$). We use this column such that after the $i$-th iteration, this column contains the $i$-th column of $S_{\leq}$. This is done by incrementing $X$ with $v\cdot e_{\mathsf{max}}^T$.
%To construct $S_{\leq}$, in the $i$-th iteration we further increment $X$ with 
%(i)~the current last column in $X$ (via $X\cdot e_{\mathsf{max}}\cdot v^T$) which holds
%the $(i-1)$-th column of $S_{\leq}$; and (ii)~the current canonical vector (via $v\cdot v^T$). Hence, after iteration $i$, $X$ contains the first $i$ columns of $S_{\leq}$ and holds the $i$th column of $S_{\leq}$ in its last column. It is now readily verified that $X=S_{\leq}$ after the $n$th iteration.
%
%Having order information available results in \langfor to be quite expressive. We heavily rely on order information in the next sections to compute the inverse of matrices and more generally to simulate low complexity Turing machines and arithmetic circuits.


%\subsection{Loop Initialization} As the reader may have observed, in the semantics of \texttt{for} loops we 
%always initialise $A_0$ to the zero matrix $\mathbf{0}$ (of appropriate dimensions). It is often convenient
%to start the iteration given some concrete matrix  originating from the result of evaluation a \langfor\ expression $e_0$. To make this explicit, we write $\initf{e_0}{v}{X}{e}$ and its semantics is defined as above
%with the difference that $A_0:=\sem{e_0}{\I}$. We observe, however, that $\initf{e_0}{v}{X}{e}$ can already
%be expressed in \langfor. In other words, we do not loose generality by assuming an initialisation of $A_0$ by $\mathbf{0}$.
%The key insight is that in \langfor\ we can check during evaluation whether or not
%the current canonical vector $b_i^n$ is equal to the $b_1^n$. This is not entirely trivial to see and is due to fact that \texttt{for} loops iterate over the canonical vectors in a fixed order. We discuss this more in the next section. In particular, we can define a \langfor expression $\mmin$, which when evaluated on an instance, returns $1$ if its input vector is $b_1^n$, and returns $0$ otherwise. Given $\mmin$, consider now the
%\langfor\ expression
% $$\ffor{v}{X}{\mmin{v}\cdot e(v,X/e_0) + (1-\mmin{v})\cdot e(v,X)},$$
% where we explicitly list $v$ and $X$ as matrix variables on which $e$ potentially depends on, and where
% $e(v,X/e_0)$ denotes the expression obtained by replacing every occurrence of $X$ in $e$ with $e_0$.
%%
%%  As mentioned previously, all of the iterations in \texttt{for} loops start by initializing $A_0$ to the null matrix. We have already seen that this property is useful for defining the extremal points of our ordering, however, sometimes we would like to start the iteration using some concrete matrix $B$. Using the operator $\mmin$, we can actually achieve this as follows. First, let $\ffor{v}{X}{e(v,X)}$ be the \texttt{for} loop that begins iterating from the null matrix. Here we write $e(v,X)$ to denote the variables that $e$ potentially (but not necessarily) uses. To start the iteration from $B$ it now suffices to use the following loop:
%% $$\ffor{v}{X}{\mmin{v}\cdot e(v,X/B) + (1-\mmin{v})\cdot e(v,X)} \quad (\dag),$$
%% where $e(v,X/B)$ denotes the expression obtained by replacing every occurrence of $X$ in $e$ with $B$.
%When evaluating this expression on an instance $\I$, $A_0$ is initial set to the zero matrix, in the first iteration (when  $v=b_1^n$ and thus $\mmin{v}=1$)
%we have $A_1=\sem{e}{\I[v:=b_1^n,X:=\sem{e_0}{\I}]}$, and for consecutive iterations (when only the part related to $1-\mmin{v}$ applies) $A_i$ is updated as before. Clearly, the result of this evaluation is equal to
%$\sem{\initf{e_0}{v}{X}{e}}{\I}$. 
%
%As an illustration, we consider the transitive closure program from the beginning of this section. Similarly as for \lang, we write \langfor$(\Fun)$ for some set $\Fun$ of functions when these are required for the task at hand.
%%As an illustration, we consider the transitive closure program from the beginning of this section. Similarly as for \lang, we write \langfor$(\Fun)$ for some set $\Fun$ of functions when these are required for the task at hand. Analogously, we write $\langfor$ to denote $\langfor(\emptyset)$, but we might abuse the notation and write $\langfor$ to refer to the language in general.
%\begin{example}
% We can express transitive closure in \langfor$(f_>)$. Indeed, consider the expression 
%$$
%f_{>0}\left(\initf{e_{\mathsf{Id}}}{v}{X}{X\cdot (e_{\mathsf{Id}}+V)}\right),
%$$
%where $e_{\mathsf{Id}}$ is a \langfor\ expression constructing the identity matrix $I$. The expression $e_{\mathsf{Id}}$
%is just like the expression $e_{\mathsf{diag}}$ in Example~\ref{ex:diag}, but now only puts value $1$ on the diagonal.
% Observe that we start the loop using an initialisation by $I$. Hence, when evaluated on an instance $\I$ such that $V$ is an adjacency matrix $A$ of a graph, the expression above simply computes $(I+A)^n$, followed by the pointwise function application $f_{>0}$. In other words, it returns the adjacency matrix of the transitive closure of graph.\qed
%\end{example}
% \smallskip
% \noindent
% \textbf{More examples}. We conclude by showing that the two examples at the beginning of this section can be
% expressed in \langfor. Indeed, for $4\mathsf{clique}$ it suffices to consider
%
% The modified \texttt{for} expression will now use $B$ as the assignment of $X$ in the very first iteration (as determined by $\mmin{v}$), and will proceed as before whenever $v$ is not equal to the first canonical vector, thus defining the desired result. Since starting iteration with a non null matrix is often useful, we will denote  by $\initf{B}{v}{X}{e}$ the expression $(\dag)$.

% \noindent{\bf For loops.}
% The biggest novelty of $\langfor$ is the for operator. To illustrate how this operator can be used, we first show how one can construct the identity matrix of needed dimension in \langfor.

%%NOT DOABLE IF NO ACCESS TO DIMENSION
%\noindent{\bf Elementary matrix operations.} Here we show the true value of adding canonical vectors to the language, by illustrating how they allow us to express elementary matrix operations \cite{linalg}:
%\begin{enumerate}
%\item {\bf Row switching.} To switch the row $i$ and row $j$ of an $m\times n$ matrix $M$, we can simply use the expression $T^{ij} \cdot M$, where:
%$$T^{ij} = I^m - e_i^m\cdot (e_i^m)^* - e_j^m\cdot (e_j^m)^* + e_i^m\cdot (e_j^m)^* + e_j^m\cdot (e_i^m)^*.$$
%\item {\bf Row multiplication.} Multiplying a row by a scalar $c$ is performed by 
%\item {\bf Row addition.}
%\end{enumerate}
%Analogously, we can perform these transformations of columns, by using the transposed matrix $M^*$, and the canonical vectors of appropriate dimension.

% To begin with, consider the expression $$v_{max} := \ffor{v}{X}{v}.$$
% we can easily obtain the last canonical vector using the expression
% \medskip
%
% \noindent{\bf Core operators}
% Note that the  \texttt{for} operator is much more powerful that it seems. We can simulate some operations of the original \lang\ semantics. Assume the allowed expressions are
%
% \begin{tabular}{lcll}
% $e$ & $:=$ & $V\in \Mvar$ & (matrix variable)\\
%  & $|$ & $e^T$ & (transpose)\\
%  & $|$ & $e_1 \cdot e_2$ & (matrix multiplication)\\
%  & $|$ & $e_1 + e_2$ & (matrix addition)\\
%  & $|$ & $\text{apply}[f](e_1,\ldots ,e_n)$ & (application of $f\in \Fun$)\\
%  & $|$ & $\ffor{v}{X}{e}$ & (canonical for loop, with $v, X \in \Mvar$).
% \end{tabular}
%
% Where function application $\text{apply}[f]$ works only on scalars. This is
%
% \begin{itemize}
% \item $\ttype(\text{apply}[f](e_1,\ldots ,e_n)) = (1,1)$, whenever $\ttype(e_1)= \cdots =\ttype(e_n)=(1,1)$, and $f:\mathbb{C}^n\mapsto \mathbb{C}$; and is undefined otherwise.
% \item $\sem{\text{apply}[f](e_1,\ldots ,e_n)}{\I}$ is a $1\times 1$ matrix whose only entry has the value $f(\sem{e_1}{\I},\ldots ,\sem{e_n}{\I})$.
% \end{itemize}
%
% \subsection{Minimal Core}
% We have the following:
%
% \begin{itemize}
% \item {\em Pointwise function application.} Note that
%
% \begin{align*}
% f(e_1,\ldots, e_n)&=\\
% &\texttt{for }v,X.\quad X + \\
% &\quad \texttt{for }w,Y.\quad Y + \\
% &\quad \quad v\left[ \text{apply}[f](v^Te_1w, \ldots, v^Te_nw) \right] w^T
% \end{align*}
%
% \item {\em Conjugate transpose.} Let $\texttt{conj}(a) := \overline{a}$ be the conjugate function. Then we can simulate the conjugate transpose of expression $e$ as $\texttt{conj}(e^T)$.
% \item {\em One vector.} Using the function $f(x) := 1$, we can define $$\ones(e) := f(\texttt{for } v,X. X + e\cdot v).$$
% \item {\em Multiplying a matrix by a scalar.} Let $c$ be a scalar ($1\times 1$ matrix) and $f_{\times}(a, b) := a \times b$ . Then, for an expression $e$, $c\odot e := f_{\times}(\ones(e)\cdot c \cdot \ones(e^T)^T, e)$.
% \item {\em Diagonal of a vector.} The operator $\diag(e)$ can be defined as:
% $$\diag(e) := \texttt{for } v, X. X + (v^T\cdot e) \odot vv^T.$$
% \end{itemize}
%
% \section{Order!!}
%
% \noindent{\bf Order.} The \texttt{for} operator assumes that canonical vectors come in some particular order, however, it does not give us an immediate access to this order. An interesting question is whether, using the \texttt{for} loops, we can define a \langfor expression which allow us to define the order of canonical vectors. Next we show that this is indeed possible. To begin with, we can easily obtain the last canonical vector using the expression $$v_{max} := \ffor{v}{X}{v}$$
%
% The fundamental property of iteration we use here is that the result variable is always initiated with the null matrix. Therefore the loop above will simply keep on storing the current canonical vector before returning the final one. The ability to do this sort of manipulation was one of the reasons why we initiate $A_0$ in the semantics of \texttt{for} to null matrix.
%
% To define an order relation for canonical vectors, notice that the following matrix:
% \[
% Z_{eq} = \begin{bmatrix}
%     1 & 1 & \cdots &  1 \\
%     0 & \ddots & \ddots & \vdots \\
%     \hdotsfor{3} & 1 \\
%     0 & \cdots & \cdots & 1
% \end{bmatrix},
% \]
% has the property that $e_i^*\cdot Z_{eq} \cdot e_j$, for two canonical vectors $e_i,e_j$ of the same dimension, is equal to one if and only $i\leq j$, and is zero otherwise. $Z_{eq}$ can easily de defined in \langfor as follows:
% $$Z_{\text{eq}}=\ffor{v}{X}{X + \left[ (Xv_{max}) + v \right]v^* + vv^*_{max}},$$
% where $v_{max}$ is as defined above. The intuition behind this expression is that using the last canonical vector $v_{max}$, we have access to the final column of $X$ (via the product $X\cdot v_{max}$), to which we add the current canonical vector $v$, thus constructing $Z_{eq}$ by filling it column by column. By defining $$\lleq{u}{v} := u^*\cdot Z_{eq} \cdot v,$$
% we obtain an order relation that allows us to discern whether one canonical vector comes before the other in the order given by \texttt{for}. If we want a strict order, we can just use $Z_< := Z_{eq} - I$.
%
% Interestingly, we can also define the predecessor relation between canonical vectors. For this, we require the following matrix:
% %\[
% %S := \begin{bmatrix}
% %    0 & 1         & 0         & \cdots &  0 \\
% %    0 & \ddots & 1         & \cdots & 0 \\
% %    \vdots & \vdots & \ddots & \ddots & \vdots \\
% %    \hdotsfor{3} & \ddots & 1 \\
% %    0 & \cdots & \cdots & \cdots & 0
% %\end{bmatrix}.
% %\]
% \[
% S = \begin{bmatrix}
%     0 & 1 & \cdots &  0 \\
%     0 & \ddots & \ddots & \vdots \\
%     \hdotsfor{3} & 1 \\
%     0 & \cdots & \cdots & 0
% \end{bmatrix},
% \]
% Using this matrix, we have that for a canonical vector $e_i$:
% \[
%   			S\cdot e_i=\begin{cases}
%                e_{i-1}, \text{ if } i > 1 \\
%               \mathbf{0}, \text{ if } i = 1
%             \end{cases}
% 		\]
% where $\mathbf{0}$ is a vector of zeros of the same type as $e_i$. Notice also that $\ones(v)^*\cdot S \cdot v$ is equal to zero, for a canonical vector $v$, if and only if $v = e_1$ is the first canonical vector, and zero otherwise. To define the first canonical vector in the order given by \texttt{for}, we can then write:
% $$v_{min} := \ffor{v}{X}{\mmin{v}\cdot v},$$
% where the expression $\mmin{v}$ is defined as $\mmin{v} := 1 - \ones(v)^*\cdot S \cdot v$, and, when evaluated over canonical vectors, will result in $1$ if and only if $v=e_1$ is the first canonical vector. Finally, we denote that $S$ can be defined using the following \langfor expression:
% \begin{multline*}
% S:= \texttt{for }v,X.\quad X + \\
% \left[ (1 - v^*v_{max})vv_{max}^* - (Xv_{max}) v_{max}^* + (Xv_{max})v^*\right].
% \end{multline*}
%
%
%
%



%\section{Order}\label{sec:order}
%%!TEX root = /Users/fgeerts/Documents/MLforloops/pods/main.tex
%
%
% \noindent{\bf Order.}
With the introduction of \texttt{for} loops we not only extend \lang\ with bounded recursion, we also introduce order information. Indeed, the semantics of the \texttt{for} operator assumes that the canonical vectors $b_1,b_2,\ldots$
are accessed in this order. It implies, among other things, that \langfor\ expressions are not permutation-invariant.
We can, for example, return the bottom right-most entry in a matrix. Indeed, consider the expression $v_{max} := \ffor{v}{X}{v}$ which, for it to well-typed requires both $v$ and $X$ to be of type $(\alpha,1)$. Then, $\sem{v_{max}}{\I}=b_n^n$, for $n=\dom(\alpha)$, simply because the $X=\mathbf{0}$ initially, but $X$ will be overwritten by $b_1^n,b_2^n,\ldots,b_n^n$, in this order. Hence, at the end of the evaluation $b_n^n$ is returned.
To extract the bottom right-most entry we now simply use $v_{max}^T\cdot M\cdot v_{max}$.


%
% assumes that canonical vectors come in some particular order.



 however, it does not give us an immediate access to this order. An interesting question is whether, using the \texttt{for} loops, we can define a \langfor expression which allow us to define the order of canonical vectors. Next we show that this is indeed possible. 
 
 \floris{I am here. Not sure how insightful details about the encodings are. Should we just state that we can do it?}
 % To begin with, we can easily obtain the last canonical vector using the expression $$v_{max} := \ffor{v}{X}{v}$$
%
% The fundamental property of iteration we use here is that the result variable is always initiated with the null matrix. Therefore the loop above will simply keep on storing the current canonical vector before returning the final one. The ability to do this sort of manipulation was one of the reasons why we initiate $A_0$ in the semantics of \texttt{for} to null matrix.

To define an order relation for canonical vectors, notice that the following matrix:
\[
Z_{eq} = \begin{bmatrix}
    1 & 1 & \cdots &  1 \\
    0 & \ddots & \ddots & \vdots \\
    \hdotsfor{3} & 1 \\
    0 & \cdots & \cdots & 1 
\end{bmatrix},
\] 
has the property that $e_i^*\cdot Z_{eq} \cdot e_j$, for two canonical vectors $e_i,e_j$ of the same dimension, is equal to one if and only $i\leq j$, and is zero otherwise. $Z_{eq}$ can easily de defined in \langfor as follows:
$$Z_{\text{eq}}=\ffor{v}{X}{X + \left[ (Xv_{max}) + v \right]v^* + vv^*_{max}},$$
where $v_{max}$ is as defined above. The intuition behind this expression is that using the last canonical vector $v_{max}$, we have access to the final column of $X$ (via the product $X\cdot v_{max}$), to which we add the current canonical vector $v$, thus constructing $Z_{eq}$ by filling it column by column. By defining $$\lleq{u}{v} := u^*\cdot Z_{eq} \cdot v,$$
we obtain an order relation that allows us to discern whether one canonical vector comes before the other in the order given by \texttt{for}. If we want a strict order, we can just use $Z_< := Z_{eq} - I$.

Interestingly, we can also define the predecessor relation between canonical vectors. For this, we require the following matrix:
%\[
%S := \begin{bmatrix}
%    0 & 1         & 0         & \cdots &  0 \\
%    0 & \ddots & 1         & \cdots & 0 \\
%    \vdots & \vdots & \ddots & \ddots & \vdots \\
%    \hdotsfor{3} & \ddots & 1 \\
%    0 & \cdots & \cdots & \cdots & 0 
%\end{bmatrix}.
%\]
\[
S = \begin{bmatrix}
    0 & 1 & \cdots &  0 \\
    0 & \ddots & \ddots & \vdots \\
    \hdotsfor{3} & 1 \\
    0 & \cdots & \cdots & 0
\end{bmatrix},
\] 
Using this matrix, we have that for a canonical vector $e_i$:
\[
  			S\cdot e_i=\begin{cases}
               e_{i-1}, \text{ if } i > 1 \\
              \mathbf{0}, \text{ if } i = 1
            \end{cases}
		\]
where $\mathbf{0}$ is a vector of zeros of the same type as $e_i$. Notice also that $\ones(v)^*\cdot S \cdot v$ is equal to zero, for a canonical vector $v$, if and only if $v = e_1$ is the first canonical vector, and zero otherwise. To define the first canonical vector in the order given by \texttt{for}, we can then write:
$$v_{min} := \ffor{v}{X}{\mmin{v}\cdot v},$$
where the expression $\mmin{v}$ is defined as $\mmin{v} := 1 - \ones(v)^*\cdot S \cdot v$, and, when evaluated over canonical vectors, will result in $1$ if and only if $v=e_1$ is the first canonical vector. Finally, we denote that $S$ can be defined using the following \langfor expression:
\begin{multline*}
S:= \texttt{for }v,X.\quad X + \\
\left[ (1 - v^*v_{max})vv_{max}^* - (Xv_{max}) v_{max}^* + (Xv_{max})v^*\right].
\end{multline*}





\section{Algorithms in Linear Algebra}\label{sec:queries}
\floris{I did not do a pass over this yet...}
The real power of \langfor comes from the fact that \texttt{for} loops allow us to express many classical linear algebra algorithms. Here we illustrate this by showing how to compute the $LU$ factorization of a matrix using Gaussian eliminations, which is one of the most commonly used matrix algorithms. Recall that $A$ is said to be LU factorizable if there exists matrices $T_1,\ldots, T_{n}$ where $T_i=E_{n}^{(i)}\cdots E_{i+1}^{(i)}$ for some elementary matrices $E_{j}^{(i)}=I+\alpha_{ij}\cdot e_{i}e_{j}^{*}$ such that $T_{n}\cdots T_1A=U$ holds, where $U$ is an upper triangular matrix.

Now, assume that $A$ is a square matrix which allows $LU$ factorization without row pivoting (we deal with this case later on). This means that after reducing the column $i$ of $A$ (i.e. we make all of the entries below the diagonal zero in the column $i$), we will end up with a matrix which has the element in position $i+1,i+1$ different from zero, for each $i$ strictly less than the dimension of $A$. 

To reduce the first column of $A$ it suffices to multiply $A$ from the left with the matrix $T_1 := I + c_1\cdot e_1^*$, where the vector $I$ is the identity matrix, $e_1$ is the first canonical vector, and $c_1$ is the vector 
\[
c_1 :=
\begin{bmatrix}
    0 \\
    \alpha_{21} \\
    \vdots \\
    \alpha_{n1}
\end{bmatrix},
\]
with $\alpha_{j1} := -\frac{A_{j1}}{A_{11}}$ is the number with which we need to multiply the first row of $A$ in order to reduce the row $j$ of $A$. More generally, if $A$ is reduced up to column $i-1$, reducing the $i$th column is achieved by computing $T_i \cdot A$, where $T_i := I + c_i\cdot e_i^*$, with $c_i$ being the vector with zeros up to position $i$, and with the value $-\frac{A_{ji}}{A_{ii}}$ in position $j > i$. To express the matrices $T_i$ in \langfor we will use the following expression:
$$\ccol{A}{y} := \ffor{v}{X}{(y^*\cdot Z_{<} \cdot v)(v^*\cdot A \cdot y)v + X},$$
which, when $y$ is interpreted by the $j$th canonical vector, computes the vector which has zeroes in positions $1\ldots j$, and values $A_{ij}$ in positions $i>j$. Intuitively, the product $y^*\cdot Z_{<}\cdot v$ makes all positions up to $j$ equal zero, and $v^*\cdot A\cdot y$ extracts the element $A_{ij}$ of $A$ when $v$ is interpreted by the $i$th canonical vector, and $y$ by the $j$th one.



Using this expression, we can now compute $T_i$s by writing:
$$\red{A}{y} := I + f_/(\ccol{A}{y},-(y^*\cdot A\cdot y)\cdot \ones(y))y^*,$$
where $f_/$ is the division function. Notice that $f_/(\ccol{A}{y},-(y^*\cdot A\cdot y)\cdot \ones(y))$ in fact computes the vectors $c_i$ defined above. The Gaussian elimination can now be performed by the following expression:
$$
U(A) :=  \left( \initf{I}{v}{X}{\red{X\cdot A}{v}\cdot X} \right) \cdot A,
$$
which computes the $U$ from the $LU$ factorization of $A$, whenever $A$ can be $LU$ factorized without row interchange. Notice that the latter assumption is crucial, since it will ensure that the element of $A$ in position $ii$ is different from zero after each \texttt{reduce} step. As explained previously, the inner \texttt{for} loop in fact computes the product of matrices $T_{n}\cdots T_1$, which in fact amounts to the matrix $L^{-1}$ from the $LU$ factorization of $A$. Given that each $T_i$ is easily invertible, we can also recover $L$.

As stated previously, the construction above works under the assumption that no row pivoting is needed when performing the Gaussian eliminations, giving us:
\begin{proposition}\label{prop:gauss}
There is a \langfor expression $e(X)$ such that by interpreting $X$ with a matrix $A$ will result in the matrix $U$ from the $LU$ factorization of $A$ whenever $A$ is $LU$ factorizable.
\end{proposition}

Now, since $A^{-1}=U^{-1}L^{-1}$ and using that $U$ is upper triangular, we can obtain the determinant and inverse of $A$ and thus


\thomas{I don't know how much of the proof goes here or if it's ok as it is.}


 \begin{proposition}\label{prop:determinant}
There is a \langfor expression $e(X)$ such that by interpreting $X$ with a matrix $A$ will result in the $1\times 1$ matrix with $\texttt{det}(A)$ in its only entry whenever $A$ is $LU$ factorizable.
\end{proposition}

\begin{proposition}\label{prop:inverse}
There is a \langfor expression $e(X)$ such that by interpreting $X$ with a matrix $A$ will result in the matrix $A^{-1}$ whenever $A$ is $LU$ factorizable.
\end{proposition}

If $A$ needs row interchange to be $LU$ factorizable (this is, $PA$ is $LU$ factorizable) we say that $A$ is $PALU$ factorizable, where the factorization is $PA=LU$. To accomplish this, we need something extra.

 \begin{proposition}\label{prop:palu}
There is a \langfor expression $e(X)$ such that by interpreting $X$ with a matrix $A$ will result in the matrix $U$ from the $PALU$ factorization of $A$ whenever $A$ is $PALU$ factorizable if and only if there exists a \langfor expression $e'(X)$ such that by interpreting $X$ with a vector $A$ outputs a canonical vector $e_k$ where $A_k$ is the first non zero entry of $A$.
\end{proposition}


\section{Expressiveness of for loops}\label{sec:circuits}
%To avoid nasty technical stuff, we will try to connect with circuits:
%\begin{itemize}
%\item Upper bound (AC?)
%\item Lower bound (transitive closure?)
%\end{itemize}

% OTHER IDEAS ABOUT PRESENTATION
%OPTION 1:
% LINK WITH TMS FIRST (NO CIRCUITS)!
% NOW INTRODUCE CIRCUITS
% USING THIS LINK, ASSUME THAT FOR EVALUATING A CIRCUIT INFO ABOUT A GATE IS AVAILABLE THROUGH AN EFFICIENT TM (NO UNIFORMITY)
% STATE THE RESULT ABOUT SIMULATION OF A SINGLE CIRCUIT IN ML
% NOW ESTABLISH A LINK WITH UNIFORM CIRCUITS -- JUST SAY THAT WE ALREADY KNOW HOW TO SIMULATE THE MACHINE, SO ALL GOOD
% NOW GO IN THE OTHER DIRECTION WITH A MORE EXPRESSIVE CLASS OF CIRCUITS

%OPTION 2:
% DEFINE ARITHMETIC CIRCUITS AND EVALUATE A SINGLE CIRCUIT
% LINK WITH TMS WHEN ASSUMING THAT TMS ARE GIVEN FOR CIRCUIT DATA
% ASSUME THAT FOR EVALUATING A CIRCUIT INFO ABOUT A GATE IS AVAILABLE THROUGH AN EFFICIENT TM (NO UNIFORMITY)
% NOW GO IN THE OTHER DIRECTION WITH A MORE EXPRESSIVE CLASS OF CIRCUITS
% NOW ESTABLISH A LINK WITH UNIFORM CIRCUITS


In this section we explore the expressive power of $\langfor$. Given that \langfor\ expressions compute functions over matrices whose entries are not only boolean, but can in fact be arbitrary elements from $\RR$, a natural candidate for comparison is the class of arithmetic circuits \cite{allender}. As we show in the remainder of this section, \langfor\ actually captures the expressive power of arithmetic circuits. To show this result, we first recall the definition of arithmetic circuits.
%
%In order to derive this result, we first look at the connection of \langfor\ and Turing machines. When observing a \langfor\ expression over some schema, perhaps the most crucial characteristic is the fact that depending on the instance, the number of iterations that a for loop does changes, as does the size of the matrix computed by our expression. This is analogous to how the number of steps a Turing machine takes changes depending on the input size, and gives some intuition on why \langfor\ expressions might be able to simulate Turing machines. Indeed, what we show is that \langfor\ expressions can actually  simulate Turing machines that use linear space and run in polynomial time. Formally, we prove the following.
%
%\begin{theorem}
%\label{th-tm-ml}
%Let $T$ be a Turing machine with $\ell$ read only input tapes, a work tape, and an output tape. 
%\end{theorem}
%
%Next we move to comparison with arithmetic circuits. 

An {\em arithmetic circuit} $\Phi$ over a set $X=\{x_1,\ldots,x_n\}$ of variables is a directed
acyclic labelled graph. The vertices of $\Phi$ are called {\em gates} and denoted by $g_1,\ldots,g_m$;
the edges in $\Phi$ are called {\em wires}. The children of a gate $g$ correspond to all gates
$g'$ such that $(g,g')$ is an edge. The parents of $g$ correspond to all gates $g'$ 
such that $(g,g')$ is an edge. The {\em in-degree}, or a {\em fan-in}, of a gate $g$ refers to its number of children,
the {\em out-degree} to its number of parents. We will not assume any restriction on the in-degree of a gate, and will thus consider circuits with unbounded fan-in. Gates with in-degree $0$ are called {\em input gates}
and are labelled by either a variable in $X$ or a constant $0$ or $1$. All other gates
are labelled by either $+$ or $\times$, and are referred to as {\em sum gates} or {\em product gates}, respectively.
Gates with out-degree $0$ are called {\em output gates}. When talking about arithmetic circuits, one usually focuses on circuits with $n$ input gates and a single output gate.

The {\em size} of $\Phi$, denoted by $|\Phi|$, is its number of gates. The {\em depth} of $\Phi$, denoted
by $\mathsf{depth}(\Phi)$, is the length of the longest directed path from the any output gate to any of the input gates. The {\em degree} of a gate is defined inductively: a leaf node has degree 1, a plus node has a degree equal to the maximum of  degrees of its children, and a product node has a degree equal to the sum of the degrees of its children. When $\Phi$ has a single output gate, the degree of $\Phi$, denoted by $\mathsf{degree}(\Phi)$, is defined as the degree of its output gate. If $\phi$ has a single output gate and its input gates take values from $\RR$, then $\Phi$ corresponds to a polynomial in $\RR[X]$ in a natural way. In this case, the {\em degree} of $\Phi$ equals the degree of the polynomial corresponding to $\Phi$.
If $v_1,\ldots ,v_n$ are values in $\RR$, then we define the result of the circuit on this input as the value computed by the corresponding polynomial, and denote this value with $\Phi_k(v_1,\ldots ,v_k)$.

%\domagoj{Should we already present the result without talking about uniformity?}

% MAYBE OVERCOMPLICATING
%In order to talk about efficient evaluation of arithmetic circuits in terms of complexity classes defined by Turing machines, we need a notion of uniformity of circuit families. An {\em arithmetic circuit family} is a set of arithmetic circuits $\{\Phi_n\mid n=1,2,\ldots\}$ where $\Phi_n$ has $n$ input variables. An arithmetic circuit family is {\em uniform} if there exist \logspace-Turing machines\footnote{There are several versions of this definition that give a different amount of resources to the Turing machine. See \cite{allender} for an in-depth discussion on the subject.} $M_1$ and $M_2$, such that:
%\begin{itemize}
%\item On input $1^n$, the machine $M_1$ returns an encoding of the arithmetic circuit $\Phi_n$ for each $n$;
%\item On input $1^n$, and an encoding of a gate $g$, the machine $M_2$ outputs the relevant information about $g$ (e.g. whether it is a sum or a product gate, the list of its children, whether it is an output gate, etc.).
%\end{itemize}
%
%The idea here is that the machine $M_1$ gives an overview of the circuit itself, while the machine $M_2$ allows us to efficiently evaluate the circuits by traversing the circuit graph in a depth-first fashion, starting from the output node. We observe that uniform arithmetic circuit families are necessarily of polynomial size.
%
%SHOULD EXPLAIN BRIEFLY HOW TO EVALUATE A CIRCUIT!

%In order to talk about efficient evaluation of arithmetic circuits in terms of complexity classes defined by Turing machines, we need a notion of uniformity of circuit families. 
In order to talk about efficient evaluation of arithmetic circuits, we need a way to compute their output algorithmically. Namely, we need a way to access their components such as the children of some gate, the value of an input gate, the operation  carried out by the gate, etc. The most general way to do this is via Turing machines. Additionally, this will allow us to handle inputs of arbitrary size, similarly as when working with Turing machines. This idea is captured by the notion of uniform circuit families. 

An {\em arithmetic circuit family} is a set of arithmetic circuits $\{\Phi_n\mid n=1,2,\ldots\}$ where $\Phi_n$ has $n$ input variables and a single output variable. An arithmetic circuit family is {\em uniform} if there exists a \logspace-Turing machine\footnote{There are several versions of this definition that give a different amount of resources to the Turing machine. See \cite{AllenderJMV98} for an in-depth discussion on the subject.}, which on input $1^n$, returns an encoding of the arithmetic circuit $\Phi_n$ for each $n$.
% Additionally, we assume that there is a \logspace-machine which, when started on input $1^n$, and an encoding of a gate $g$, outputs the relevant information about $g$ in $\Phi_n$ (e.g. whether it is a sum or a product gate, the list of its children, whether it is an output gate, etc.).
We observe that uniform arithmetic circuit families are necessarily of polynomial size, however, their degree can grow exponentially. A circuit family 
$\{\Phi_n\mid n=1,2,\ldots\}$ is said to be of polynomial degree if $\mathsf{degree}(\Phi_n)\in O(p(n))$, for some polynomial $p(n)$. Similarly, a circuit family is of logarithmic depth, whenever $\mathsf{depth}(\Phi_n)\in O(logn)$. We can now show that \langfor\ subsumes uniform arithmetic circuit families that are of polynomial degree and logarithmic depth. 

%\domagoj{The thing about the second machine is crucial, but it might be a bit confusing here, as we only need it to explain how one actually computes the output of a circuit. Because of this I'm thinking of maybe presenting the connection without uniformity.}

\begin{theorem}
\label{th-circuits-ml}
For any uniform arithmetic circuit family $\{\Phi_n\mid n=1,2,\ldots\}$ of polynomial degree and logarithmic depth there is a \langfor\ schema $\Sch$ and an expression $e_\Phi$ using a matrix variable $v$, with $\ttype(v)=(\alpha,1)$ and $\ttype(e) = (1,1)$, such that for any input $v_1,\ldots ,v_n$ to the circuit $\Phi_n$:
\begin{itemize}
\item If $\I = (\dom,\conc)$ is a \lang\ instance such that $\dom(\alpha) = n$ and $\conc(v) = (v_1 \ldots v_n)^*$
\item Then $\sem{e}{\I} = \Phi_n(v_1,\ldots ,v_n)$.
\end{itemize}
\end{theorem}

It is important to note that the expression $e_\Phi$ does not change depending on the input size, meaning that it is uniform in the same sense as the circuit family being generated by a single Turing machine. The different input sizes for a \langfor\ instance are handled by the typing mechanism of the language. The expression $e_\Phi$ on input $(v_1,\ldots ,v_n)^*$ basically simulates depth-first evaluation of the arithmetic circuit $\Phi_n$. 

In order to do this, we consider an algorithm for evaluating $\Phi_n$ on input $(v_1,\ldots ,v_n)$ that maintains  two stacks: the gates-stack that tracks the current gate being evaluated, and the values-stack that stores the value that is being computed for this gate. The idea behind having two stacks is that whenever the number of items on the gates-stack is higher by one than the number of items on the values-stack, we know that we are processing a fresh gate, and we have to initialize its current value (to 0 if it is a plus gate, and to 1 if it is a product gate), and push it to the values-stack. We then proceed by processing the children of the head of the gates-stack one by one, and aggregate the results using sum if we are working with a plus gate, and product otherwise. 

In order to store these two stacks, we can use vectors of size $n$, where $n$ is the number of inputs. Here we crucially depend on the fact that the circuit is of logarithmic depth, and therefore the size of the two stacks is bounded by $n$ (apart from the portion before the asymptotic bound kicks-in, which can be hard coded into the formula). 

...

Now that we know that arithmetic circuits can be simulated using \langfor\ expressions, it is natural to ask whether the same holds in the other direction. That is, we are asking whether for each \langfor\ expression $e$ over some schema $\Sch$ there is a uniform family of arithmetic circuits computing precisely the same result depending on the input size. In order to show this, we need to extend the 

EXTEND THE CIRCUITS...


\section{Restricting the power of for loops}\label{sec:restrict}
% !TeX spellcheck = en_US

\newcommand{\hprod}{\circ}

\subsection{Sumation matlang and relational algebra}

When defining the identity matrix and several other expressions, we actually only update $X$ by adding some matrix to it. This restricted form of the $\texttt{for}$ loop proved to be useful throughout the paper, and we will therefore introduce it a special operator. That is, we define:
$$\Sigma v. e := \ffor{v}{X}{X + e}.$$
We define the subfragment of \langfor, called \langsum, to consist of the $\Sigma$ operator plus the ``core'' operators in \lang, namely, transposition, matrix multiplication and addition, scalar multiplication, and pointwise function applications.

Apart from defining the identity matrix with \langsum, the sum quantifier also allows for computing the trace of a matrix $A$ using the expression $tr(A) := \Sigma v. v^*\cdot A \cdot v$. Interestingly enough, this restricted version of \texttt{for} already allows us to capture the \lang\ operators that are not present in the syntax of \langsum. More precisely, we have:
\smallskip

\noindent {\em Function application.} Notice that in \lang, a function is applied pointwise to matrices of arbitrary size, while \langsum only allows functions that process matrices of size $1\times 1$. Using the summation operator we can lift this condition, and allow applying a function $f:\mathbb{\RR}^n \mapsto \mathbb{\RR}$ on expressions $e_1,\ldots ,e_n$ of arbitrary (but equal) type by writing 
$$\Sigma x_i \Sigma x_j. f(x_i^T\cdot e_1\cdot x_j, \ldots ,x_i^T\cdot e_n\cdot x_j) \cdot x_i\cdot x_j^T,$$
which simply reconstructs the matrix obtained by applying $f$ to every position of $e_1$ through $e_n$, by using the fact that for two canonical vectors $b_i^m$ and $b_j^n$, the product $b_i^m \cdot (b_j^n)^*$ defines a $m\times n$ matrix whose only non-zero entry is in the position $ij$.
\smallskip

\noindent {\em One vector.} We can define $\ones(e) := \Sigma v.\, v.$ where $\ttype(v) = (\alpha, 1)$ and $\ttype(e) = (\alpha, \beta)$ for some $\beta$. 
\smallskip

\noindent {\em Diagonal of a vector.} The operator $\diag(e)$ can be defined as:
$$\diag(e) := \Sigma v. (v^T\cdot e) \cdot vv^T.$$

Furthermore, one can easily check that the 4-clique expression of Example~\ref{ex:fourcliques} can be defined in \langsum. Therefore, we conclude the following result. 
\begin{corollary}
\lang\ is strictly subsumed by \langsum.
\end{corollary}

%To show that the inclusion here is strict, we illustrate how one can detect whether a undirected graph has a four clique, which it is not definable in \lang~\cite{BrijderGBW19}. For this, we define an expression $f(u,v) := 1 - u^*\cdot v$. Notice that when $u$ and $v$ are interpreted by two canonical vector of the same dimension, we have that:
%\[
%  			f(u,v)=1-u^*v=\begin{cases}
%               0 \text{ if } u=v \\
%               1 \text{ if } u\neq v
%            \end{cases}
%		\]
%
%To distinguish four-cliques, we will need to determine whether we are dealing with four different nodes. For this, we will utilize the function $$g(u,v,w,r)=f(u,v)\cdot f(u,w)\cdot f(u,r)\cdot f(v,w)\cdot f(v,r)\cdot f(w,r),$$
%which, when evaluated over four canonical vectors of the same dimension, will give us 1 if and only if the four vectors are distinct. With this at hand, we can now define:		
%\begin{multline*}
%\texttt{4-clique}(A) := \ssum v_1.\ssum v_2. \ssum v_3. \ssum v_4.\\ (v_1^*Av_2)(v_1^*Av_3)(v_1^*Av_4)(v_2^*Av_3)(v_2^*Av_4)(v_3^*Av_4) \cdot
%\\g(v_1,v_2,v_3,v_4).
%\end{multline*}
%
%When $A$ is an adjacency matrix of an undirected graph $G$, then we have that $\texttt{4-clique}(A)$ is different from zero if and only if $G$ has a four-clique. Using this, and the fact that \lang\ is subsumed by First Order Logic with aggregates that uses only three variables \cite{matlang}, we immediately obtain the following:
%
%\begin{corollary}
%There is a  \langfor expression that is not expressible in \lang.
%\end{corollary}

What operations over matrices can be defined with \langsum that is beyond \lang? In~\cite{brijder2019matrices}, it was shown that \lang\ was strictly included in the relational algebra of $K$-relations~\cite{GreenKT07}, called Annotated Relational Algebra (ARA) in~\cite{brijder2019matrices}.
Then a natural idea is to compare the expressive power of \langsum with ARA. For making this comparison clear, in the following we give the formal definition of ARA~\cite{GreenKT07} to then see how to connect both formalism.

\newcommand{\ddom}{\mathbb{D}}
\newcommand{\fdom}{\operatorname{dom}}
\newcommand{\att}{\mathbb{A}}
\newcommand{\tuples}{\mathbf{tuples}}
\newcommand{\supp}{\operatorname{supp}}
\newcommand{\cJ}{\mathcal{J}}
\newcommand{\cR}{\mathcal{R}}
\newcommand{\adom}{\mathbf{adom}}

\newcommand{\ksum}{\oplus}
\newcommand{\kprod}{\odot}
\newcommand{\bigksum}{\bigoplus}
\newcommand{\bigkprod}{\bigodot}
\newcommand{\kzero}{\mymathbb{0}}
\newcommand{\kone}{\mymathbb{1}}

\newcommand{\row}{\mathsf{row}}
\newcommand{\rows}{\mathsf{rows}}
\newcommand{\col}{\mathsf{col}}
\newcommand{\cols}{\mathsf{cols}}
\newcommand{\arae}{Q}


Let $\ddom$ be a data domain and $\att$ a set of attributes. A relational signature is a finite subset of $\att$. A relational schema is a function $\cR$ on finite set of symbols $\fdom(\cR)$ such that $\cR(R)$ is a relation signature for each $R \in \fdom(\cR)$. To simplify the notation, from now on we write $R$ to denote both the symbol $R$ and the relational signature $\cR(R)$.
Furthermore, we write $R \in \cR$ to say that $R$ is a symbol of $\cR$. 
For $R \in \cR$, an $R$-tuple is a function $t: R \rightarrow \ddom$. We denote by $\tuples(R)$ the set of all $R$-tuples. Given $X \subseteq R$, we denote by $t[X]$ the restriction of $t$ to the set $X$.

A semiring $(K, \ksum, \kprod, \kzero, \kone)$ is an algebraic structure where $K$ is a non-empty set, $\ksum$ and $\kprod$ are binary operations over $K$, and $\kzero, \kone \in K$. Furthermore,  $\ksum$ and $\kprod$ are associative operations, $\kzero$ and $\kone$ are the identities of $\ksum$ and $\kprod$ respectively, $\ksum$ is a commutative operation, $\kprod$ distributes over $\ksum$, and $\kzero$ annihilates $K$ (i.e. $\kzero \kprod k = k \kprod \kzero = \kzero$). As usual, we assume that all semirings in this paper are commutative, namely, $\kprod$ is also commutative. We use $\bigksum_X$ or $\bigkprod_X$ for the $\ksum$- or $\kprod$-operation over all elements in $X$, respectively. Typical examples of semirings are the reals $(\RR, +, \times, 0,1)$, the natural numbers $(\RR, +, \times, 0,1)$, and the boolean semiring $(\{0,1\}, \vee, \wedge, 0, 1)$. 

Fix a semiring $(K, \ksum, \kprod, \kzero, \kone)$ and a relational schema $\cR$. A $K$-relation of $R \in \cR$ is a function $r: \tuples(R) \rightarrow K$ such that the support  $\supp(r) = \{t \in \tuples(R) \mid r(t) \neq \kzero\}$ is finite. 
A $K$-instance $\cJ$ of $\cR$ is a function that assigns relational signatures of $\cR$ to $K$-relations. Given $R \in \cR$, we denote by $R^\cJ$ the $K$-relation associated to $R$. Recall that $R^\cJ$ is a function and then $R^\cJ(t)$ is the value in $K$ assign to $t$. 
Given a $K$-relation $r$ we denote by $\adom(r)$ the active domain of $r$ defined as $\adom(r) = \{t(a) \mid t \in \supp(r) \wedge a \in R\}$.
Then the active domain of an $K$-instance $\cJ$ of $\cR$ is defined as $\adom(\cJ) = \bigcup_{R \in \cR} \adom(R^\cJ)$. 

An ARA expression $\arae$ over $\cR$ is given by the following syntax:
$$
\begin{array}{rcl}
\arae & := & R \ \mid \ \arae \cup \arae \ \mid \  \pi_X(\arae) \ \mid \  \sigma_X(\arae) \ \mid \ \rho_f(\arae) \ \mid \ \arae \bowtie \arae
\end{array}
$$
where $R \in \cR$, $X \subseteq \att$ is finite, and $f: X \rightarrow Y$ is a one to one mapping with $Y \subseteq \att$. One can extend the schema $\cR$ to any expression over $\cR$ recursively as follows: $\cR(R) = R$, $\cR(\arae \cup \arae') = \cR(\arae)$, $\cR(\pi_X(\arae)) = X$, $\cR(\sigma_X(\arae)) = \cR(\arae)$, $\cR(\rho_f(\arae)) = X$ where $f:X \rightarrow Y$, and $\cR(\arae \bowtie \arae') = \cR(\arae) \cup \cR(\arae')$ for every expressions $\arae$ and $\arae'$.
We further assume that any expression $\arae$ satisfies the following syntactic restrictions: $\cR(\arae') = \cR(\arae'')$ whenever $\arae = \arae' \cup \arae''$, $X \subseteq \cR(\arae')$ whenever $\arae = \pi_X(\arae')$ or $\arae = \sigma_X(\arae')$, and $Y = \cR(\arae')$ whenever $\arae = \rho_f(\arae')$ with $f: X \rightarrow Y$.

Given an ARA expression $\arae$ and a $K$-instance $\cJ$ of $\cR$, we define the semantics $\ssem{\arae}{\cJ}$ as a $K$-relation of $\cR(\arae)$ as follows. For $X \subseteq \att$, let $\operatorname{Eq}_X(t) = \kone$ when $t(a) = t(b)$ for every $a, b \in X$, and $\operatorname{Eq}_X(t) = \kzero$ otherwise. For every tuple $t \in \cR(\arae)$:
$$
\begin{array}{ll}
\text{if $\arae = R$, then} & \ssem{\arae}{\cJ}(t) = R^\cJ(t) \\
\text{if $\arae = \arae_1 \cup \arae_2$, then} & \ssem{\arae}{\cJ}(t) = \ssem{\arae_1}{\cJ}(t) \ksum \ssem{\arae_2}{\cJ}(t)  \\
\text{if $\arae = \pi_X(\arae')$, then} & \ssem{\arae}{\cJ}(t) = \bigksum_{t': t'[X] = t} \ssem{\arae'}{\cJ}(t') \\
\text{if $\arae = \sigma_X(\arae')$, then} & \ssem{\arae}{\cJ}(t) = 
\ssem{\arae'}{\cJ}(t) \kprod \operatorname{Eq}_X(t)  \\
\text{if $\arae = \rho_f(\arae')$, then} & \ssem{\arae}{\cJ}(t) = 
\ssem{\arae'}{\cJ}(t \circ f)  
\\
\text{if $\arae = \arae_1 \bowtie \arae_2$, then} & \ssem{\arae}{\cJ}(t) =  \ssem{\arae_1}{\cJ}(t[Y]) \kprod  \ssem{\arae_2}{\cJ}(t[Z])
\end{array}
$$
where $Y = \cR(\arae_1)$ and $Z = \cR(\arae_2)$. It is important to note that the $\bigksum$-operation on the semantics of $\pi_X(\arae')$ is well defined given that the support of $\ssem{\arae'}{\cJ}$ is always finite. 

We are ready for comparing \langsum with ARA. First of all, we need to extend \langsum from $\RR$ to any semiring. Indeed, one can easily verify that the semantics of \lang, \langfor and \langsum can be translated from $\RR$ to $K$ by switching from matrices over $(\RR, +, \times, 0, 1)$ to matrices over $(K, \ksum, \kprod, \kzero, \kone)$.
From now on we denote by  $\mtr{K}$ the set of all $K$-matrices. Similar than for \lang\ over $\RR$, given a \lang\ schema $\Sch$ a $K$-instance $\I$ over $\Sch$ is a pair $\I = (\dom,\conc)$, where $\dom : \DD \mapsto \NN$ assigns a value to each size symbol, and $\conc : \Mnam \mapsto \mtr{K}$ assigns a concrete $K$-matrix to each matrix variable. Then it is straightforward to extend the semantics of \lang, \langfor, and \langsum from $(\RR, +, \times, 0, 1)$ to $(K, \ksum, \kprod, \kzero, \kone)$ by switching $+$ with $\ksum$ and $\times$ with $\kprod$. 

The next step for comparing \langsum with ARA is to represent $K$-matrices as $K$-relations.
Let $\Sch=(\Mnam,\size)$ be a \lang\ schema. On the relational side
we have for each size symbol $\alpha\in\DD\setminus\{1\}$, attributes $\alpha$, $\row_\alpha$, and $\col_\alpha$ in $\att$. Furthermore, for each $V\in\Mnam$ and $\alpha \in \DD$ we denote
by $R_V$ and $R_\alpha$ its corresponding relation name, respectively. Then, given $\Sch$ we define the relational schema $\text{Rel}(\Sch)$ such that $\fdom(\text{Rel}(\Sch)) =  \{R_\alpha \mid \alpha\in\DD\} \cup \{R_V \mid V \in \Mnam\}$ where $\text{Rel}(\Sch)(R_\alpha) = \{\alpha\}$ and:
\[
\text{Rel}(\Sch)(R_V) = \begin{cases}
\lbrace\row_\alpha,\col_\beta \rbrace & \text{ if $ \size(V)=(\alpha,\beta)$} \\
\lbrace\row_\alpha \rbrace & \text{ if $ \size(V)=(\alpha,1)$} \\
\lbrace\col_\beta \rbrace  &
\text{ if $ \size(V)=(1,\beta)$} \\
\lbrace\rbrace & \text{ if $\size(V)=(1,1)$}.
\end{cases}
\]
Consider now a matrix instance $\I = (\dom,\conc)$ over $\Sch$.
Let $V\in\Mnam$ with $\size(V)=(\alpha,\beta)$ and let $\conc(V)$ be its corresponding $K$-matrix of dimension $\dom(\alpha)\times \dom(\beta)$.
To encode $\I$ as a $K$-instance in ARA, we use as data domain $\ddom = \mathbb{N} \setminus \{0\}$. Then we construct the $K$-instance $\text{Rel}(\I)$ such that for each $V\in\Mnam$ we define 
$R_V^{\text{Rel}(\I)}(t):=\conc(V)_{ij}$ whenever $t(\row_\alpha) = i \leq \dom(\alpha)$ and $t(\col_\beta) = j \leq \dom(\beta)$, and $\kzero$ otherwise. Furthermore, for each $\alpha \in \DD$ we define $R_\alpha^{\text{Rel}(\I)}(t):=\kone$ whenever $t(\alpha) \leq \dom(\alpha)$, and $\kzero$ otherwise. In other words, $R_\alpha$ and $R_\beta$ encodes the active domain of a matrix variable $V$ with $\size(V)=(\alpha,\beta)$. Given that the ARA framework of \cite{GreenKT07} represents the ``absence'' of a tuple in the relation with $\kzero$, we need to find a way to encode all entries of a matrix in ARA. For instance, we need to be able to encode a $\kzero$-matrix of dimension $(\alpha,\beta)$ in~ARA.

We are ready to state the first connection between \langsum and ARA by using the previous encoding.
\begin{proposition}
	For each \langsum expression $e$ over schema $\Sch$ such that $\Sch(e)=(\alpha,\beta)$ with $\alpha\neq 1\neq\beta$, there exists an ARA expression $\Phi(e)$ over relational schema $\text{Rel}(\Sch)$ such that $\text{Rel}(\Sch)(\Phi(e))=\{\row_\alpha,\row_\beta\}$ and 
	such that for any instance $\I$ over~$\Sch$,
	$$
	\sem{e}{\I}_{i,j}=\ssem{\Phi(e)}{\text{Rel}(\I)}(t)
	$$
	for tuple $t(\mathrm{row}_\alpha)=i$ and $t(\mathrm{col}_\beta)=j$. Similarly for when $e$ has schema $\Sch(e)=(\alpha,1)$, $\Sch(e)=(1,\beta)$ or $\Sch(e)=(1,1)$, then $\Phi(e)$ has schema $\text{Rel}(\Sch)(\Phi(e))=\{\mathrm{row}_\alpha\}$,
	$\text{Rel}(\Sch)(\Phi(e))=\{\mathrm{col}_\alpha\}$, or
	$\text{Rel}(\Sch)(\Phi(e))=\{\}$, respectively.
\end{proposition}
To translate ARA into \langsum, we must restrict our comparison to ARA over $K$-relations with at most two attributes. Given that linear algebra works over vector and matrices, it is reasonable to restrict to unary or binary relations as input. Note that this is only a restriction to the input relations and not to intermediate relations, namely, expressions can create relation signatures of arbitrary size from the binary input relations. Thus, from now we say that a relational schema $\cR$ is binary if $|R| \leq 2$ for every $R \in \cR$. We also make the assumption that there is an (arbitrary) order, denoted by $<$, on the attributes in $\att$. 
This is to identify which attributes correspond to rows and columns when moving to matrices. 
Then, given that relations will be  either unary or binary and there is an order in the attributes, we write $t = (v)$ or $t = (u,v)$ to denote a tuple over a unary or binary relation $R$, respectively, where $u$ and $v$ is the value of the first and second attribute with respect to $<$.

Consider a binary relational schema $\cR$. With each $R\in \cR$ we associate a matrix variable $V_R$ such that, if $R$ is a binary relational signature, then $V_R$ represents a (square) matrix, and, if not (i.e. $R$ is unary), then $V_R$ represents a vector. Formally, fix a symbol $\alpha \in \DD \setminus \{1\}$. Let $\text{Mat}(\cR)$ denote the \lang \ schema
$(\Mnam_\cR,\size_\cR)$ such that $\Mnam_\cR = \{ V_R \mid R \in \cR\}$ and $\size_\cR(V_R) = (\alpha, \alpha)$ whenever $|R| = 2$, and $\size_\cR(V_R) = (\alpha, 1)$ whenever $|R|=1$. 
Take now a $K$-instance $\cJ$ of $\cR$ and suppose that $\adom(\cJ) = \{d_1, \ldots, d_n\}$ is the active domain of $\cJ$ (i.e. the order over $\adom(\cJ)$ is arbitrary). Then we define the matrix instance $\text{Mat}(\cJ) = (\dom_\cJ,\conc_\cJ)$ such that $\dom_\cJ(\alpha) = n$, $\conc_\cJ(V_R)_{i,j} = R^{\cJ}((d_i, d_j))$ whenever $|R|=2$, and $\conc_\cJ(V_R)_{i} = R^{\cJ}((d_i))$ whenever $|R|=1$. 
Note that, although each $K$-relation can have different active domain, we encode them as square matrices by considering the active domain of the $K$-instance.

\begin{proposition}
	Let $\cR$ be a binary relational schema. For each ARA expression $\arae$ over $\cR$  such that $|\cR(\arae)| = 2$, there exists a \langsum  expression $\Psi(\arae)$ over \lang \ schema $\text{Mat}(\cR)$ such that for any $K$-instance $\cJ$ with $\adom(\cJ) = \{d_1, \ldots, d_n\}$ over $\cR$,
	$$
	\ssem{\arae}{\cJ}((d_i, d_j))=\sem{\Psi(\arae)}{\text{Mat}(\cJ)}_{i,j}.
	$$
	Similarly for when $|\cR(\arae)| = 1$, or $|\cR(\arae)| = 0$ respectively.
\end{proposition} 

It is important to remark that the $\arae$ of the previous result can have intermediate expressions that are not necessary binary given that the proposition only restricts that the input relation and the schema of $\arae$ must have arity at most two. 

Given the previous two propositions we derive the following conclusion which is the first characterization of relational algebra with a (sub)-fragment of linear algebra.
\begin{corollary}
	\langsum and ARA over binary relational schemas are equally expressive. 
\end{corollary}


\subsection{Hadamard product and weighted logics}

%!TEX root = ../main.tex
% !TeX spellcheck = en_US
% !TEX root = ../main.tex
Similarly to using sum, we can use other operations to update $X$ in the for-loop. The next natural choice is to consider products of matrices. In contrast to matrix sum, we have two options: either we can choose to use matrix product or to use the pointwise matrix product, also called the Hadamard product. We treat  matrix product in the next subsection and first explain here the connection of sum and Hadamard product operators to weighted logics.

For the rest of this section, fix a semiring $(K, \ksum, \kprod, \kzero, \kone)$. The Hadamard product over $K$-matrices can be defined as the pointwise application of $\kprod$ between two matrices of the same size. Formally, we define the expression $e \hadprod e'$ where $e, e'$ are expressions with respect to $\cS$ and $\ttype(e) = \ttype(e')$ for some schema $\Sch=(\Mnam,\size)$. Then the semantics of $e \hadprod e'$ is the pointwise application of $\kprod$, namely, $\sem{e \hadprod e'}{\I}_{ij} = \sem{e}{\I}_{ij} \kprod \sem{e'}{\I}_{ij}$ for any instance $\I$ of $\cS$. This enables us to define, similar as for  $\Sigma v$, the  pointwise-product quantifier $\qhadprod v$ as follows:
$$
\qhadprod v. \  e := \ffor{v}{X\!=\!e_{\kone}}{X \circ e}.
$$
where $e_\kone$ is the (easily definable) \langfor expression for the matrix with the same type as $X$ and all entries equal to the $\kone$-element of $K$ (i.e., we need to initialize $X$ accordingly with the $\kprod$-operator).
We call \langprod  the subfragment of \langfor that consists of \langsum \ extended with $\qhadprod v$.

\begin{example}
	Similar to the trace of a matrix, a useful function in linear algebra is to compute the product of the values on the diagonal. 
	Using the $\qhadprod v$ operator, this can be easily expressed:
	 $$
	 e_{\mathsf{dp}}(V) := \qhadprod v. \ v^T\cdot V \cdot v.$$
\end{example}

Clearly, the inclusion of this new operator extends the expressive power to \langsum. For example,  $\sem{e_{\mathsf{dp}}}{\I}$ can be an exponentially large number in the dimension $n$ of the input.
By contrast, one can easily show that all expressions in \langsum can only return numbers polynomial in  $n$. That is, \langprod is more expressive than \langsum and $\mathsf{RA}_{K}^+$. 

To measure the expressive power of \langprod, we use weighted logics~\cite{DrosteG05} (WL) as a yardstick. Weighted logics extend monadic second-order logic from the boolean semiring to any semiring $K$. Furthermore, it has been used extensively to characterize the expressive power of weighted automata in terms of logic~\cite{droste2009handbook}. We use here the first-order subfragment of weighted logics to suit our purpose and, moreover, we extend its semantics over weighted structures (similar as in~\cite{GradelV17}).

A relational vocabulary $\Gamma$ is a finite collection of relation symbols such that each $R \in \Gamma$ has an associated arity, denoted by $\arity(R)$.
A $K$-weighted structure over $\Gamma$ (or just structure) is a pair $\cA = (A, \{R^\cA\}_{R \in \Gamma})$ such that $A$ is a non-empty finite set (i.e. the domain) and, for each $R \in \Gamma$, $R^\cA: A^{\arity(R)} \rightarrow K$ is a function that associates to each tuple in $A^{\arity(R)}$ a weight in $K$.

Let $X$ be a set of first-order variables. A $K$-weighted logic (WL) formula $\varphi$ over $\Gamma$ is defined by the following syntax:
$$
\begin{array}{rcl}
\varphi & := & x = y \ \mid \ R(\bar{x}) \ \mid \ \varphi \ksum \varphi \ \mid \ \varphi \kprod \varphi \ \mid \ \Sigma x. \varphi \ \mid \ \Pi x. \varphi
\end{array}
$$ 
where $x, y \in X$, $R \in \Gamma$, and $\bar{x} = x_1, \ldots, x_k$ is a sequence of variables in $X$ such that $k=\arity(R)$. As usual, we say that $x$ is a free variable of $\varphi$, if $x$ is not below $\Sigma x$ or $\Pi x$ quantifiers (e.g. $x$ is free in $\Sigma y. R(x,y)$ but $y$ is not). 
Given that $K$ is fixed, from now on we talk about structures and formulas without mentioning $K$ explicitly.  

An assignment $\sigma$ over a structure $\cA = (A, \{R^\cA\}_{R \in \Gamma})$ is a function $\sigma: X \rightarrow A$. Given $x \in X$ and $a \in A$, we denote by $\sigma[x \mapsto a]$ a new assignment such that $\sigma[x \mapsto a](y) = a$ whenever $x = y$ and $\sigma[x \mapsto a](y) = \sigma(y)$ otherwise. For $\bar{x} = x_1, \ldots, x_k$,  we write $\sigma(\bar{x})$ to say $\sigma(x_1),\ldots, \sigma(x_k)$. Given a structure $\cA = (A, \{R^\cA\}_{R \in \Gamma})$ and an assignment $\sigma$, we define the semantics $\ssem{\varphi}{\cA}(\sigma)$ of $\varphi$ as follows:
$$
\begin{array}{ll}
\text{if $\varphi := x = y$, then} & \ssem{\varphi}{\cA}(\sigma) = 
\left\{
\begin{array}{ll}
\kone & \text{if $\sigma(x) = \sigma(y)$} \\
\kzero & \text{otherwise}
\end{array}
\right. \\
\text{if $\varphi := R(\bar{x})$, then} & \ssem{\varphi}{\cA}(\sigma) = R^\cA(\sigma(\bar{x})) \\
\text{if $\varphi := \varphi_1 \ksum \varphi_2$, then} & \ssem{\varphi}{\cA}(\sigma) = \ssem{\varphi_1}{\cA}(\sigma) \ksum \ssem{\varphi_2}{\cA}(\sigma)  \\
\text{if $\varphi := \varphi_1 \kprod \varphi_2$, then} & \ssem{\varphi}{\cA}(\sigma) = \ssem{\varphi_1}{\cA}(\sigma) \kprod \ssem{\varphi_2}{\cA}(\sigma)  \\
\text{if $\varphi := \Sigma x. \, \varphi'$, then} & \ssem{\varphi}{\cA}(\sigma) =  \bigksum_{a \in A} \ssem{\varphi'}{\cA}(\sigma[x \mapsto a]) \\
\text{if $\varphi := \Pi x. \, \varphi'$, then} & \ssem{\varphi}{\cA}(\sigma) =  \bigkprod_{a \in A} \ssem{\varphi'}{\cA}(\sigma[x \mapsto a])
\end{array}
$$
When $\varphi$ contains no free variables, we omit $\sigma$ and write $\ssem{\varphi}{\cA}$ instead of $\ssem{\varphi}{\cA}(\sigma)$.

For comparing the expressive power of \langprod with WL, we have to show how to encode \lang\ instances into structures and vice versa. For this, we make two assumptions to put both languages at the same level: (1) we restrict structures to relation symbols of arity at most two and (2) we restrict instances to square matrices. The first assumption is for the same reasons as when comparing \langsum with $\mathsf{RA}_K^+$, and the second assumption is to have a crisp translation between both languages. Indeed, understanding the relation of \langprod with WL for non-square matrices is slightly more complicated and we leave this for future work. 

Let $\Sch=(\Mnam,\size)$ be a schema of square matrices, that is, there exists an $\alpha$ such that $\size(V) \in \{1, \alpha\} \times \{1,\alpha\}$ for every $V \in \Mnam$.
Define the relational vocabulary $\text{WL}(\Sch) = \{R_V \mid V \in \Mnam\}$ such that $\arity(R_V) = 2$ if $\size(V) = (\alpha, \alpha)$, $\arity(R_V) = 1$ if $\size(V) \in \{(\alpha,1), (1,\alpha)\}$, and $\arity(R_V) = 0$ otherwise.
Then given a matrix instance $\I = (\dom,\conc)$ over $\Sch$ define the structure $\text{WL}(\I) = (\{1, \ldots, n\}, \{R_V^{\I}\} )$ such that $\dom(\alpha) = n$ and $R_V^{\I}(i, j) = \conc(V)_{i,j}$ if $\size(V) = (\alpha, \alpha)$, $R_V^{\I}(i) = \conc(V)_{i}$ if $\size(V) \in \{(\alpha,1), (1,\alpha)\}$, and $R_V^{\I} = \conc(V)$ if $\size(V) = (1,1)$.

To encode weighted structures into matrices and vectors, the story is similar as for $\mathsf{RA}_K^+$. Let $\Gamma$ be a relational vocabulary where $\arity(R) \leq 2$. 
Define $\text{Mat}(\Gamma) = (\Mnam_\Gamma,\size_\Gamma)$ such that $\Mnam_\Gamma = \{ V_{R} \mid R \in \Gamma\}$ and $\size_\Gamma(V_{R})$ is equal to $(\alpha, \alpha), (\alpha, 1)$, or $(1,1)$ if $\arity(R)=2$, $\arity(R)=1$, or $\arity(R)=0$, respectively, for some $\alpha \in \DD$. Similarly, let $\cA = (A, \{R^{\cA}\}_{R \in \Gamma})$ be a structure with $A = \{a_1, \ldots, a_n\}$, ordered arbitrarily.
Then we define the matrix instance $\text{Mat}(\cA) = (\dom,\conc)$ such that $\dom(\alpha) = n$, $\conc(V_{R})_{i,j} = R^{\cA}(a_i, a_j)$ if $\arity(R)=2$, $\conc(V_{R})_{i} = R^{\cA}(a_i)$ if $\arity(R)=1$, and $\conc(V_{R}) = R^{\cA}$ otherwise.

Let $\Sch$ be a \lang\ schema of square matrices and $\Gamma$ a relational vocabulary of relational symbols of arity at most $2$. We can then show the equivalence of \langprod and WL as follows. 
\begin{proposition} \label{prop:wl}
Weighted logics over $\Gamma$ and \langprod over $\Sch$ have the same expressive power. More specifically,
\begin{itemize}
	\item for each \langprod expression $e$ over $\Sch$ such that $\Sch(e)=(1,1)$, there exists a WL-formula $\Phi(e)$ over $\text{WL}(\Sch)$ such that for every instance $\I$ of~$\Sch$, 
	$
	\sem{e}{\I} = \ssem{\Phi(e)}{\text{WL}(\I)}
	$.
	\item for each WL-formula $\varphi$ over $\Gamma$ without free variables, there exists a \langprod expression $\Psi(\varphi)$ such that for any structure $\cA$ over~$\text{Mat}(\Gamma)$,
	$
	\ssem{\varphi}{\cA}=\sem{\Psi(\varphi)}{\text{Mat}(\cA)}
	$.
\end{itemize}	
\end{proposition}



\subsection{Matrix multiplication as a quantifier}

Analogously to the summation and Hadamard product with respect to canonical vectors, we can use the usual product of matrices. Formally, for an expression $e$ define:
$$
\sprod v.\,  e=\ffor {v}{X = I}{X\cdot e}.
$$
where $I$ is the identity matrix. 
Using the product operator we can express multiple interesting properties. To begin with, we can compute the product of diagonal elements of a matrix: 
$$
dp(A) := \sprod v. v^*\cdot A \cdot v.$$
Another property of interest is computing the transitive closure of a graph adjacency matrix $A$. It is well known the transitive closure of this matrix, denoted $tc(A)$ equals to the matrix consisting of non-zero entries of $(I + A)^n$, where $n$ is the dimension of $A$. Using the product operator we can define:
$$tc(A) := f_{>0}(\sprod v. (I + A)),$$
where $f_{>0}(x) := 1$ if $x>0$, and $f_{>0}(x) = 0$ otherwise, is used to make the result a zero-one matrix. Notice that the expression for $tc(A)$ ignores the canonical vectors, and simply multiplies the previous result with $(I + A)$, thus computing the desired value.

Using the combination of canonical sum and product, we can also define more general operators over matrices, such as the power sum operator, which, given a square matrix $A$, computes $I + A + A^2 + \cdots + A^n$. This operator, denoted by $ps(A)$ can be defined as:
$$ps(A) := \ssum v.\sprod w. \left( (w^* S_{<} v)\cdot (A-I) + I\right),$$
where $S_{<}$ is the (strict) order matrix defined in Section~\ref{sec:order}. The outer loop here defines which power we compute. That is, when $v$ is the $i$th canonical vector, we compute $A^i$. Computing $A^i$ is achieved via the inner product loop, which uses $w^*S_{<}v$ to determine whether $w$ comes before $v$ in the ordering of canonical vectors. When this is the case we multiply the current result by $A$, and when $w$ is greater than $v$, we use $I$ not to affect the already computed result.

Note that $\sprod v$ can define the quantifier $\qhadprod v$ and, further, it increases the expressive power of \langprod~.
\begin{proposition}
	Every expression in \langprod can be defined in \langsum extended with $\sprod v$ quantifier. Moreover, there exists an expression that uses the $\sprod v$ quantifier that cannot be defined in \langprod.
\end{proposition}

It is interesting that the $\sprod v$ give the power of transitive closure to \langsum in a natural way and without defining it explicitly. We leave the study of this operator and, in particular, to understand its expressibility for future work. 




% \section{Connecting Linear Algebra with Relational Algebra}
% We prove the ARA theorem here.

If we are short on pages, the proof sketch can make it here.

Perhaps discuss quantifier connection here and not in the previous section?

%\section{Applications}
%Here we show the real strength of our framework by connecting it to some well established areas:

\begin{itemize}
\item Graph query languages
\item Machine learning
\item Query optimization
\end{itemize}


\section{Conclusions}\label{sec:conclude}
Please, be the first one to give some nice conclusions!

%%
%% The next two lines define the bibliography style to be used, and
%% the bibliography file.
\bibliographystyle{ACM-Reference-Format}
\bibliography{biblio}

%%
%% If your work has an appendix, this is the place to put it.

\newpage 
\appendix

\onecolumn
\section*{Appendix}
%!TEX root = /Users/fgeerts/Documents/MLforloops/pods/main.tex

\section{Preliminaries}
We first introduce some additional notations and describe simplifications that will be used later in the appendix.
\subsection{Definitions}
We sometimes want to iterate over $k$ canonical vectors. We define the following shorthand notation:
\begin{align*}
  \ffor{v_1,\ldots, v_k}{X}{e(X,v_1,\ldots, v_n)}:= &\ffor{v_1}{X_1}{X_1 +} \\
  &\hspace{1em}\initf{X_1}{v_2}{X_2}{X_2 + } \\
  &\hspace{2em}\initf{X_2}{v_3}{X_3}{X_3 + } \\
  &\hspace{8em}\ddots \\
  &\hspace{4em}\initf{X_{k-1}}{v_k}{X_k}{ e(X_k,v_1,\ldots, v_k)}.
\end{align*}
To reference $\ell$ different vector variables $X_1,\ldots,X_\ell$ in every iteration and update them in different ways we define:
\begin{align*}
&\ffor{v}{X_1,\ldots, X_\ell}{\left( e_1(X_1,v), e_2(X_2,v), \ldots, e_l(X_\ell,v) \right)} :=\\
&\hspace{4em}\ffor{v}{X}{e_1(X\cdot e_{\mathsf{min}},v)\cdot e_{\mathsf{min}}^T + e_2(X\cdot e_{\mathsf{min} + 1},v)\cdot e_{\mathsf{min} + 1}^T + \ldots + e_\ell(X\cdot e_{\mathsf{max}},v)\cdot e_{\mathsf{max}}^T}
\end{align*}
We note that for the latter expression to be semantically correct both $v$, $X_i$ and $e_i$ for $ i=1,\ldots,\ell$ have to be of type $\alpha\times 1$, and $X$ has to be of type $\alpha\times\gamma$ as the expression inside. 
\floris{I believe that $\gamma$ should be $\alpha$? Also, this implies that we cannot loop over more than $\alpha$ different vector variables. Does this cause a problem later on?}
When evaluated on an instance $\I$,
$e_{\mathsf{min}}, e_{\mathsf{min} + i}$ evaluate to $b_1$ and $b_{1+i}$, respectively, and we show their defining expressions in section \ref{app:order}. Similarly for $e_{\mathsf{max}}=b_n$.
The combinations of both previous operators results in:
\begin{align*}
&\ffor{v_1,\ldots, v_k}{X_1,\ldots, X_\ell}{\left( e_1(X_1,v_1,\ldots, v_k), e_2(X_2,v_1,\ldots, v_k), \ldots, e_\ell(X_\ell,v_1,\ldots, v_k) \right)} :=\\
&\hspace{4em}\ffor{v_1,\ldots, v_k}{X}{e_1(X\cdot e_{\mathsf{min}},v)\cdot e_{\mathsf{min}}^T + e_2(X\cdot e_{\mathsf{min} + 1},v)\cdot e_{\mathsf{min} + 1}^T + \ldots + e_\ell(X\cdot e_{\mathsf{max}},v)\cdot e_{\mathsf{max}}^T}
\end{align*}
It is clear that this expression iterates over $k$ canonical vectors and references $\ell$ independent vectors updating each of them in their particular way.
\label{app:def}

\subsection{Simplifications}\label{app:simp}
When showing results based on induction of expressions in \langfor, it is often convenient to assume that function applications $f(e_1,\ldots,e_k)$ for $f\in\Fun_k$ are restricted to
the case when all expressions $e_1,\ldots,e_k$ have type $1\times 1$. This does not loose generality. Indeed,
for general function applications $f(e_1,\ldots,e_k)$, if we have $\ssum$, scalar product and function application on scalars (here denoted by $f_{1\times 1}$), we can simulate full function application, as follows:
 $$
f(e_1,\ldots, e_k) \coloneqq \Sigma v_i \Sigma v_j. f_{1\times 1}(v_i^T\cdot e_1\cdot v_j, \ldots ,v_i^T\cdot e_k\cdot v_j) \times v_i\cdot v_j^T.
$$

Furthermore, it also convenient at times to use the pointwise functions
$f_\odot^k:\RR^k\mapsto \RR:(x_1,\ldots,x_k)\mapsto x_1\times\cdots \cdot x_k$ and 
$f_\oplus^k:\RR^k\mapsto \RR:(x_1,\ldots,x_k)\mapsto x_1+\cdots + x_k$. In fact, it is readily observed that adding these functions does not extend the expressive power of \langfor:
\begin{lemma}
\label{lm-prod-sum}
We have that $\langforf{\emptyset} \equiv \langforf{\{f_\odot^k,f_\oplus^k \ | \ k\in \mathbf{N}\}}$.
\end{lemma}
In fact, this lemma also holds for the smaller fragments we consider.
%
We also observe that having $f_\odot^2:\RR^2\to\RR$ allows us to define scalar multiplication:
$$
e_1\times e_2 \coloneqq f_{\kprod}(\ones(e_2)^T\cdot e_1 \cdot \ones(e_2)^T, e_2).
$$
Conversely, $f_\odot^k$ can be expressed using scalar multiplication, as can be seen from our simulation of general function applications by pointwise function application on scalars.
%
% % assume \langfor to have functions that process only matrices of size $1\times 1$
% % and not scalar product.
%
% % \floris{I am confused by what follows below. I see that general function application can be
% % reduced to scalar function application. Similarly, matrix multiplication be reduced (do we use this?). BUT the expression for scalar multiplication used scalar multiplication?!?}
% \thomas{Resolved here, the following is used in the proofs of section 6. The reduction of matrix multiplication it's not used I believe}
% \begin{itemize}
% \item
%
%
% \item  This is useful when the function allowed is $\kprod$, since one can be computed using the other and viceversa.
% Note that, if $\ssum$ is present, multiplication function application on matrices of any size can be simulated as
% $$
% f_{\kprod}(e_1,\ldots, e_n) \coloneqq \Sigma x_i \Sigma x_j. (x_i^T\cdot e_1\cdot x_j)\times \ldots \times (x_i^T\cdot e_n\cdot x_j) \times x_i\cdot x_j^T,
% $$
% which also holds if $e_1,\ldots, e_n$ are scalars. If we have function application, we can simulate
% scalar product. Let $c$ a scalar:
% $$
% c\times A \coloneqq f_{\kprod}(e_{\ones}(A)^T\cdot c \cdot e_{\ones}(A)^T, A).
% $$
%
% Another interesting consequence of allowing for loops in \lang\ is that the pointwise product and sum can be defined without direct access to these operators. Namely, the following result tells us that we can always assume to have access to the product and the sum.
% This will be of crucial importance when comparing \langfor\ to arithmetic circuits in Section \ref{sec:circuits} and to annotated relations in Section \ref{sec:restrict}. Formally, we have:
%
%
%
%
Finally, a notational simplification is that when using scalars $a\in\RR$ in our expressions, we write sometimes
$a$ instead of $[a]$. For example,  $(1-e_{\ones}(v)^T\cdot v)$ stands for  $([1]-e_{\ones}(v)^T\cdot v)$.

%
% \end{itemize}



\section{Proofs of Section~\ref{sec:formatlang}}

\subsection{Order predicates}\label{app:order}
We detail how order information on canonical vectors can be obtained in \langfor.
We provide explicit expressions for the operators mentioned in Section~\ref{sec:formatlang}
and furthermore, we also define expressions for operators that will be used in our proofs.
% The expression in this section are intended to be used over canonical vectors.

To begin with, we can easily obtain the last canonical vector using the expression 
$$
e_{\mathsf{max}} := \ffor{v}{X}{v}.
$$ 
In other words, we simply overwrite $X$ with the current canonical vector in each iteration.
Hence, at the end, $X$ is assigned to the last canonical vector.

%
% The fundamental property of the iteration that we use here is that the result variable $X$ initiated with the null matrix.
As already mentioned in the main body of the paper,
to define an order relation for canonical vectors, we notice that the following matrix:
\[
S_{\leq} = \begin{bmatrix}
    1 & 1 & \cdots &  1 \\
    0 & \ddots & \ddots & \vdots \\
    \hdotsfor{3} & 1 \\
    0 & \cdots & \cdots & 1 
\end{bmatrix}.
\]
has the property that for two canonical vectors $b_i$ and $b_j$ of the same dimension, 
$$b_i^T\cdot S_{\leq} \cdot b_j=\begin{cases}1 & \text{if $i\leq j$}\\
0 &\text{otherwise}.
\end{cases}
$$
We observe that $S_{\leq}$ can be expressed in \langfor as 
follows:
$$
S_{\leq}:=\ffor{v}{X}{X + \bigl((X\cdot e_{\mathsf{max}}) + v \bigr)\cdot v^T + v\cdot e^T_{\mathsf{max}}},
$$
where $e_{\mathsf{max}}$ is as defined above. 
The intuition behind this expression is that by using the last canonical vector $b_n$, as returned by $e_{\mathsf{max}}$, we have access to the last column of $X$ (via the product $X\cdot e_{\mathsf{max}}$). We use this column such that after the $i$-th iteration, this column contains the $i$-th column of $S_{\leq}$. This is done by incrementing $X$ with $v\cdot e_{\mathsf{max}}^T$.
To construct $S_{\leq}$, in the $i$-th iteration we further increment $X$ with 
(i)~the current last column in $X$ (via $X\cdot e_{\mathsf{max}}\cdot v^T$) which holds
the $(i-1)$-th column of $S_{\leq}$; and (ii)~the current canonical vector (via $v\cdot v^T$). Hence, after iteration $i$, $X$ contains the first $i$ columns of $S_{\leq}$ and holds the $i$th column of $S_{\leq}$ in its last column. It is now readily verified that $X=S_{\leq}$ after the $n$th iteration.
%
%
% The intuition behind this expression is that by using the last canonical vector $e_{\mathsf{max}}$, we have access to the final column of $X$
% (via the product $X\cdot e_{\mathsf{max}}$), to which we add the current canonical vector $v$, thus
% constructing $S_{\leq}$ by filling it column by column.

By defining 
$$
\mathsf{succ}(u,v) := u^T\cdot S_{\leq} \cdot v,
$$
we obtain an order relation that allows us to discern whether one canonical vector comes before 
the other in the order given by $S_{\leq}$. If we want a strict order, we can just use the matrix
$S_< := S_{\leq} - e_{\mathsf{Id}}$, where $e_{\mathsf{Id}}$ is an expression in \langfor which returns the identity matrix (of appropriate dimension). Given this, we define
$$\mathsf{succ}^+(u,v) := u^T\cdot S_{<} \cdot v.$$
from which we can also derive 
$$
\mathsf{max}(u):=u^T\cdot e_{\mathsf{max}}.
$$
which is an expression that returns the last canonical vector.

Interestingly, we can also define the \textit{previous} relation between canonical vectors. 
For this, we require the following matrix:
\[
\mathsf{Prev} = \begin{bmatrix}
    0 & 1 & \cdots &  0 \\
    0 & \ddots & \ddots & \vdots \\
    \hdotsfor{3} & 1 \\
    0 & \cdots & \cdots & 0
\end{bmatrix},
\]
Using this matrix, we have that for a canonical vector $b_i$:
\[
\mathsf{Prev}\cdot b_i=\begin{cases}
               b_{i-1}, \text{ if } i > 1. \\
              \mathbf{0}, \text{ if } i = 1.
            \end{cases}
\]
where $\mathbf{0}$ is a vector of zeros of the same type as $b_i$. Notice also that $\ones(u)^T\cdot \mathsf{Prev} \cdot u$ is equal to zero, for a canonical vector $u$, if and only if $u = b_1$ is the first canonical vector, and zero otherwise.
Therefore the expression $\mathsf{min}(u)$ is defined as $$\mathsf{min}(u) := 1 - \ones(u)^T\cdot \mathsf{Prev} \cdot u,$$ and, when evaluated over canonical vectors, will result in $1$ if and only if $u=b_1$ is the first canonical vector.
To define the first canonical vector in the order given by \texttt{for}, we can then write:
$$e_{\mathsf{min}} := \ffor{v}{X}{X + \mathsf{min}(v)\times v},$$
Finally, we show that $\mathsf{Prev}$ can be defined using the following \langfor expression:
$$e_{\mathsf{Prev}}:= \ffor{v}{X}{X + \bigl((1 - \mathsf{max}(v))\times v\cdot e_{\mathsf{max}}^T - (X\cdot e_{\mathsf{max}})\cdot e_{\mathsf{max}}^T + (X\cdot e_{\mathsf{max}})\cdot v^T\bigr)}.$$
Here, $X$ is initialized as $\mathbf{0}$ and thus in the first iteration we put
 $b_1$ in the last column of $X$ (note that $X\cdot e_{\mathsf{max}}$ is also zero in the first iteration). Next, in iteration two, we add a matrix that has the stored vector $X\cdot e_{\mathsf{max}}$ (the previous canonical vector) in the column indicated by $v$ (the current canonical vector) and $v-X\cdot e_{\mathsf{max}}$ in the last column, to replace the vector stored. As a consequence, $b_2$ is now stored in the last column. In the last iteration, we have $b_{n-1}$ already in the last column, so no further update of $X$ is required.
 
To get the \textit{next} relation we simply do $e_{\mathsf{Next}} = e_{\mathsf{Prev}}^T$. We have that for a canonical vector $b_i$:
\[
{\mathsf{Next}}\cdot b_i=\begin{cases}
               b_{i+1}, \text{ if } i < n. \\
              \mathbf{0}, \text{ if } i = n.
            \end{cases}
\]
In this way, we also can obtain the following operators for a canonical vector $v$: 
$$\mathsf{prev}(v):=e_{\mathsf{Prev}}\cdot v.$$
$$\mathsf{next}(v):=e_{\mathsf{Next}}\cdot v.$$
More generally, we define 
\begin{align*}
    e_{\mathsf{getPrevMatrix}}(v)&:=\sprod w.  \mathsf{succ}(w,v)\times e_{\mathsf{Prev}} + (1 - \mathsf{succ}(w,v))\times e_{\mathsf{Id}}\\
    e_{\mathsf{getNextMatrix}}(v)&:=\sprod w. \mathsf{succ}(w,v)\times e_{\mathsf{Next}} + (1 - \mathsf{succ}(w,v))\times e_{\mathsf{Id}}
\end{align*}
expressions that, when $v$ is interpreted as canonical vector $b_i$, output $\mathsf{Prev}^i$ and $\mathsf{Next}^i$ respectively.
Note that
\[
\mathsf{Prev}^j\cdot b_i=\begin{cases}
               b_{i-j}, \text{ if } i > j. \\
              \mathbf{0}, \text{ if } i \leq j.
            \end{cases}
\]
and
\[
\mathsf{Next}^j\cdot b_i=\begin{cases}
               b_{i+j}, \text{ if } i + j \leq n. \\
              \mathbf{0}, \text{ if } i + j > n.
            \end{cases}
\]
Finally, define
$$
e_{\mathsf{min}+i}:=\underbrace{e_{\mathsf{getNextMatrix}}(\ldots e_{\mathsf{getNextMatrix}}}_{i \text{ times}}(e_{\mathsf{min}}))
$$
and
$$
e_{\mathsf{max}-i}:=\underbrace{e_{\mathsf{getPrevMatrix}}(\ldots e_{\mathsf{getPrevMatrix}}}_{i \text{ times}}(e_{\mathsf{max}}))
$$
We note that some these expressions were already used in Section~\ref{app:def}.
\subsection{Order as asset}\label{app:asset_order}
\floris{This section should be placed in the appropriate place, i.e., this should be part of proofs for the last section. Let's see how the paper evolves? What is the precise result shown here?}

We next consider the order information in the context of our restricted fragments in Section~\ref{sec:restrict}.

Assume we can only perform the restricted versions of $\ffor{v}{X}{e}$, this is, we can only do 
$\ssum$ and $\sprod$ defined in section \ref{sec:restrict}. Note that these operators also iterate over canonical vectors
in certain order. Since they are operations that \textit{aggregate} information, it doesn't seem
possible to access this order explicitly and compare canonical vectors, like in the full version of $\langfor$.
Let's see what we can do if this order is \textit{supplied}. 
The goal of this is to have a restriction in the recursion property of $\ffor{v}{X}{e}$, this is, 
in the iteration we don't allow access to the current result.
We say that an expression does not use the recursion property of $\langfor$ if it uses
only $\ssum$, $\sprod$ and has access to order.
An example of this are the expressions $e_{\mathbf{det}}(M)$ and $e_{\mathbf{inv}}(M)$ defined 
in section \ref{app:inverse}.

First, assume we have $e_{S_{<}}$ (and thus $e_{S_{\leq}}$) such that the following holds:

$$
b_i^T\cdot S_{<} \cdot b_j=\begin{cases}
               1, \text{ if } i < j.\\
              \mathbf{0}, \text{ if not.}
            \end{cases}
$$
As a consequence we
can compute $\mathsf{succ}$ and $\mathsf{succ}^+$. We can define
\begin{align*}
  e_{\mathsf{min}}&:=\ssum v. \left[ \sprod w. \mathsf{succ}(w,v)\right] \times v. \\
  e_{\mathsf{max}}&:=\ssum v. \left[ \sprod w. \left( 1-\mathsf{succ}(w,v) \right) \right] \times v.
\end{align*}
Note that, if we have $f_{>0}$ we can define
$$
e_{\mathsf{Pred}}:= e_{S_{<}}- f_{>0}(e_{S_{<}}^2)
$$
Also $e_{\mathsf{Next}}:=e_{\mathsf{Pred}}^T$.
We can now define $\mathsf{prev}(v)$ and $\mathsf{next}(v)$ as in the previous section. 
The same goes with $e_{\mathsf{getPrevMatrix}}(V)$, 
$e_{\mathsf{getNextMatrix}}(V)$, $e_{\mathsf{min}+i}$ and $e_{\mathsf{max}+i}$.

On the other hand, supose we have $e_{\mathsf{Prev}}$
(and thus $e_{\mathsf{Next}}$) such that the following holds:

$$
b_i^T\cdot \mathsf{Prev} \cdot b_j=\begin{cases}
               1, \text{ if } i = j-1.\\
              \mathbf{0}, \text{ if not.}
            \end{cases},
\hspace{2em}b_i^T\cdot \mathsf{Next} \cdot b_j=\begin{cases}
               1, \text{ if } i=j+1.\\
              \mathbf{0}, \text{ if not.}
            \end{cases}.
$$
We can also define $\mathsf{prev}(v)$ and $\mathsf{next}(v)$.
Note that we can compute
$$
e_{S_{\leq}}:=\ssum v. \left( v\cdot v^T + \ssum w. v\cdot\mathsf{next}(v)^T \right)
$$
Here, $(e_{S_{\leq}})_{ij}=1$ if and only if $b_i$ comes 
before according to $\mathsf{Pred}$ and $\mathsf{Next}$, or is equal to $b_j$. As a consequence we
can compute $\mathsf{succ}$ and $\mathsf{succ}^+$.

We can now compute everything as we have $S_{\leq}$ and thus $S_{<}$.

% $$
% b_i^T\cdot \mathsf{Prev} \cdot b_j=\begin{cases}
%                1, \text{ if } b_i \text{ comes immediately before } b_j.\\
%               \mathbf{0}, \text{ if not.}
%             \end{cases},
% \hspace{2em}b_i^T\cdot \mathsf{Next} \cdot b_j=\begin{cases}
%                1, \text{ if } b_i \text{ comes immediately after } b_j.\\
%               \mathbf{0}, \text{ if not.}
%             \end{cases}.
% $$


% The expression in this section are intended to be used over canonical vectors.
% To have access to order, we need a matrix $S_{\leq}$ such that the following holds for 
% canonical vectors $b_i$ and $b_j$:
% $$
% b_i^T\cdot S_{\leq} \cdot b_j=\begin{cases}
%                1, \text{ if } b_i \text{ comes before } b_j\\
%               \mathbf{0}, \text{ if not.}
%             \end{cases}
% $$
% Note that $S_{\leq}$ encodes \textit{total} order. 

% Using this matrix 


% We would also like to have access to 
% \textit{local} information, this is, know if a canonical vector comes immediately before another. 
% Let's call this matrix $\mathsf{Prev}$. 
% The following must hold:
% $$
% b_i^T\cdot \mathsf{Prev} \cdot b_j=\begin{cases}
%                1, \text{ if } b_i \text{ comes immediately before } b_j\\
%               \mathbf{0}, \text{ if not.}
%             \end{cases}
% $$

% Using this matrix, we have that for a canonical vector $b_i$:
% \[
% \mathsf{Prev}\cdot b_i=\begin{cases}
%                b_{i-1}, \text{ if } i > 1 \\
%               \mathbf{0}, \text{ if } i = 1
%             \end{cases}
% \]

% Note that $\mathsf{Prev}$ can be defined using the following \langfor expression:
% $$
% e_{\mathsf{Prev}}:= \texttt{for }v,X.\quad X + \left[ (1 - \mathsf{max}(v))\times ve_{\mathsf{max}}^T - (Xe_{\mathsf{max}})\cdot e_{\mathsf{max}}^T + (Xe_{\mathsf{max}})\cdot v^T\right].
% $$

% Here, $X$ starts as 0 and thus in turn $X\cdot e_{\mathsf{max}}$, so we initiate storing $b_1$ in the last column. Next, we add a matrix that has the stored vector $Xe_{\mathsf{max}}$ (the previous canonical vector) in the column indicated by $v$ (the current canonical vector) and $v-Xe_{\mathsf{max}}$ in the last column, to replace the vector stored.
% The last iteration does nothing (sums the zero matrix).

% To get the \textit{next} relation we simply do $e_{\mathsf{Next}} = e_{\mathsf{Prev}}^T$. We have that for a canonical vector $b_i$:
% \[
% \mathsf{Next}\cdot b_i=\begin{cases}
%                b_{i+1}, \text{ if } i < n \\
%               \mathbf{0}, \text{ if } i = n
%             \end{cases}
% \]


\section{Proofs of Section~\ref{sec:queries}}
We next provide more details about how to perform LU-decomposition (without and with pivoting)
and to compute the determinant and inverse of a matrix.
\subsection{LU-decomposition}
\newtheorem*{ALU}{Proposition~\ref{prop:gauss}}

We start with LU-decomposition without pivoting. We recall proposition \ref{prop:gauss}:
\begin{ALU}
  There exists $\langforf{f_/}$ expressions $e_L(V)$ and $e_U(V)$ such that
  $\sem{e_L}{\I}=L$ and $\sem{e_U}{\I}=U$ form an LU-decomposition of $A$,
  where $\conc(V)=A$ and $A$ is LU-factorizable.
\end{ALU}
\begin{proof}
	Let $A$ be an LU-factorizable matrix. We already explained how the expression 
	$e_U(V)$ is obtained in the main body of the paper, i.e., 
	$$
	e_{U}(V) :=  \left( \initf{e_{\mathsf{Id}}}{y}{X}{\red{X\cdot V}{y}\cdot X} \right) \cdot V.
	$$
	We recall that $e_U(A)=T_n\cdot\cdots\cdot T_1\cdot A$ with $L^{-1}=T_n\cdot\cdots\cdot T_1$. Let
	$$
	e_{L^{-1}}(V) :=  \initf{e_{\mathsf{Id}}}{y}{X}{\red{X\cdot V}{y}\cdot X}.
	$$
such that	$$
	e_{\mathsf{U}}(V) :=  e_{L^{-1}}(V) \cdot V.
	$$
	%
%
% 	 We therefore focus on the
% 	expression $e_{L}(V)$.
%
% 	 We define
% $$
% e_{L^{-1}}(V) :=  \initf{e_{\mathsf{Id}}}{y}{X}{\red{X\cdot V}{y}\cdot X}.
% $$
% Note that, in the iteration, if $V$ is $A$, $X$ is initially $A_0=I$ then
% \begin{align*}
%   A_i&=\red{A_{i-1}\cdot A}{b_i}\cdot A_{i-1} \\
%   &=T_iA_{i-1} \\
%   &=T_iT_{i-1}\cdots T_1A_0 \\
%   &=T_iT_{i-1}\cdots T_1.
% \end{align*}
% Such that $T_nT_{n-1}\cdots T_1A=U$. Note that because of the definition of $\ccol{\cdot}{\cdot}$ inside $\red{\cdot}{\cdot}$, we always get $T_n=I$.
% So we have
% $$
% e_{\mathsf{U}}(V) :=  e_{L^{-1}}(V) \cdot V.
% $$
It now suffices to observe that, since $T_n=I$,
\begin{align*}
  L^{-1}&=(I-c_1\cdot b_1^T)\cdots (I-c_{n-1}\cdot  b_{n-1}^T) \\
  &=I-c_1\cdot b_1^T-\cdots - c_{n-1}\cdot b_{n-1}^T
\end{align*}
and hence,
\begin{align*}
  L&=(I+c_1\cdot b_1^T)\cdots (I+c_{n-1}\cdot b_{n-1}^T) \\
  &=I+c_1\cdot b_1^T+\cdots + c_{n-1}\cdot b_{n-1}^T.
\end{align*}
As a consequence, to obtain $L$ from $L^{-1}$ we just need to multiply every entry below the diagonal by $-1$. Since both  $L$ and $L^{-1}$ are lower triangular, this can done 
by computing $L=-1\times L^{-1} + 2\times I$. Translated into \langfor, this means that we can define
$$
e_{L}(V) :=  -1\times e_{L^{-1}}(V) + 2\times e_{\mathsf{Id}},
$$
which concludes the proof of the proposition.
\end{proof}


\subsection{LU-decomposition with pivoting}
Here we prove proposition  \ref{prop:palu}. Recall that $A$ is said to be LU factorizable if there exists matrices $T_1,\ldots, T_{n}$ where $T_i=E_{n}^{(i)}\cdots E_{i+1}^{(i)}$ for $1\leq i < n$ and $T_n=E^{(n)}_n$ for some elementary matrices $E_{j}^{(i)}=I+\alpha_{ij}\cdot e_{i}e_{j}^{*}$ such that $T_{n}\cdots T_1A=U$ holds, where $U$ is an upper triangular matrix. Define $A_k=T_{k-1}A_{k-1}$ for $1< k\leq n+1$ and $A_1=A$. Keep in mind that $A_k$ is $A$ with its columns reduced up to index $k-1$ (so $A_{n+1}=U$). 

If $A$ needs no row interchange we compute $$L^{-1}=\left( \initf{I}{v}{X}{\red{X\cdot A}{v}\cdot X} \right),$$ where in step $v=e_k$ we do $T_k=\red{X\cdot A}{v}$ 


Now, let's assume that during the LU factorization process we need row interchange immediately before step $k$, $1\leq k\leq n$, so we now aim to reduce the $k$-th column of $A_k=T_{k-1}\cdots T_1A$, or $A_k=A$ if $k=1$, but now $A_k$ has a zero pivot. 

Let $P$ be the matrix that denotes the necessary row interchange. We aim to reduce the $k$-th column of $PA_{k}$ since $A_{k}$ has a zero pivot. So to compute $T_k$ we do $\red{P\cdot X\cdot A}{v}$ in the iteration. Furthermore, we need to apply the permutation $P$ to the current result, so the expression ends as $\initf{I}{v}{X}{\red{P\cdot X\cdot A}{v}\cdot P\cdot X}$.

The factorization results in $U=T_{n}\cdots T_kPT_{k-1}\cdots T_1A$. We now explain why $T_{n}\cdots T_kPT_{k-1}\cdots T_1 = L^{-1}P.$ The permutation matrix $P$ has the form $P = I - uu^*$ and denotes row interchange (if multiplied by right) of rows $i$ and $j$ if $u=(e_{i}-e_{j})$. Note that in this case $i,j>k-1$ since we are using $P$ before reducing the $k$-th column so we are interchanging rows of index strictly greater than $k-1$ (if $k=n$ there is one row to interchange so nothing happens). Now, for $T_{l}=I-c_le_l^*$ with $l\leq k-1$ we have that $e_l^*P$ because $e_l$ has zeroes in positions $i+(k-1)$ and $j+(k-1)$. Note that $P^2=I$, thus $PT_lP=P^2-Pc_le_l^*P=I-\widehat{c}_le_l^*.$ Where $\widehat{c}_l=Pc_l$. Now $$T_{n}\cdots T_kPT_{k-1}\cdots T_1=T_{n}\cdots T_kPT_{k-1}P^2T_{k-2}P^2\cdots P^2 T_1P^2=T_{n}\cdots T_k(PT_{k-1}P)(PT_{k-2}P)(P\cdots P)(PT_1P)P=T_{n}\cdots T_k\widehat{T}_{k-1}\cdots \widehat{T}_1P$$ and $L^{-1} = T_{n}\cdots T_k\widehat{T}_{k-1}\cdots \widehat{T}_1$.

The goal is to compute $P$ within the iteration. The output of the PALU algorithm is $L^{-1}P$.

Let $\nneq{\cdot}$ be the operator that receives an $n$ dimensional vector $a$ and outputs a canonical vector $e_k$ such that $a_k$ is the first non zero entry of $a$. 

We prove that we can compute $\nneq$ if and only if we can compute PALU.

First, let $A$ be a PALU factorizable matrix. If we have the function $\nneq$, then we compute $$P=P(A,X,v) = I - \left[ v - \nneq{ \ccoleq{XA}{v} } \right]\left[ v - \nneq{ \ccoleq{XA}{v} } \right]^*.$$ Here $\ccoleq{\cdot}{\cdot}$ is the same as $\ccol{\cdot}{\cdot}$ but uses $Z_{eq}$ instead of $Z_{<}$.

Thus $$\text{PALU}(A):=\initf{I}{v}{X}{\red{P(A,X,v)\cdot X\cdot A}{v}\cdot P(A,X,v)\cdot X}= L^{-1}P.$$

Now we show that if we have can do the PALU factorization, we can compute $\nneq$. Let $a$ be an $n$ dimensional vector such that there exists $k:1\leq k\leq n$ where $a_i=0$ for all $i<k$, this is, $k$ is the index of the first non zero entry of $a$ (it must exist, otherwise there is nothing to be proved). Let $\lbrace b_1, \ldots, b_n\rbrace$ be an $n$ dimensional basis. Then, without loss of generality, $a$ is a linear combination of $b_1, \ldots, b_k$, with $k \leq n$. Now, let $A = \left[ a\hspace{1em} b_2 \hspace{1em} \cdots \hspace{1em}  b_n \right].$ Note that $A$ is PALU factorizable since $\lbrace a, b_2, \ldots, b_n\rbrace$ is a linearly independent set of $\mathbb{R}^n$. Furthermore, if we run the PALU algorithm the factorization results in

\[
U=(L^{-1}P)A = \begin{bmatrix}
    a_k & \cdots &  \vdots \\
    0 & \ddots & \vdots \\
    \vdots & \cdots & \cdots 
\end{bmatrix}.
\]

So, if we have the PALU function such that $PALU(A)=(L^{-1}P)$, we can compute $\nneq$ in the following way. $$\nneq{a}=\ffor{v}{X}{X+\left( v^*ZX\cdot\dfrac{a^*v}{v_{min}^*\left[ \text{PALU}(A)\cdot A\right]v_{min}} \right)\odot v + min(v)\left( v_{max} + \dfrac{a^*v}{v_{min}^*\left[ \text{PALU}(A)\cdot A\right]v_{min}}(v-v_{max})\right)}$$

In the expression above $a^*v$ extracts an entry of $a$, which will be zero until $v=e_k$ and we normalize it with $a_k$ obtained from $v_{min}^*\left[\text{PALU}(A)\cdot A\right]v_{min}$, so we add $v$ only in this case and $X=e_k$ is set. From then on $v^*ZX$ will always be zero and we will end up with $\nneq{a}=e_k$ as expected. For initialization, we use $min(v)$ as usual and we add $v_{max}$ to set $X=v_{max}$ so $v^*ZX=1$ until something changes. If $a_1\neq 0$ then we add $v-v_{max}$ and thus set $X=e_1$ and $v^*ZX=0$ since then, so $\nneq{a}=e_1$, as expected.

Note that the expression depends on the normalization of $a_k$, and would not be possible if we didn't have $a_k$ from the PALU algorithm.

Thus we can do PALU if and only if we can compute $\nneq$.

\subsection{Determinant and inverse}\label{app:inverse}
To see why proposition \ref{prop:determinant} is true it suffices to do $$\texttt{det}(A)=dp(U(A))$$ assuming $A$ is $LU$ factorizable, since $U$ is upper diagonal.
Now, to prove \ref{prop:inverse} we will need to elaborate a bit more. Define 
\[
D_U = \begin{bmatrix}
    u_{11} & \cdots & \cdots &  \vdots \\
    0 & u_{22} & \cdots &  \vdots \\
    0 & \ddots & \vdots & \vdots \\
    \vdots & \cdots& \cdots & u_{nn}
\end{bmatrix}.
\]

This is, $D_U$ is the diagonal matrix of $U$. We compute this as $$ \getdiag{U} = \ssum v. (v^*Uv) \odot vv^*.$$
Let $U'=U-D_U$, then $$ U^{-1}=\left[ D_U+U' \right]^{-1}= \left[ D_U\left( I+D_U^{-1}U'\right) \right]^{-1} = \left( I+D_U^{-1}U'\right)^{-1}D_U^{-1} $$
Note that $$D_U^{-1}=\ssum v. f_{/}(1,v^*D_Uv)\odot vv^*=\ssum v. f_{/}(1,v^*Uv)\odot vv^*.$$
Where $f_{/}$ is the division function. In the last equality we take advantage of the fact that the diagonal of $U$ and $D_U$ are the same.
We now focus on calculating $\left( I+D_U^{-1}U'\right)^{-1}$. First, by construction, $D_U^{-1}U'$ is strictly upper triangular and thus nilpotent, such that $\left( D_U^{-1}U'\right)^n=0$, where $n$ is the dimension of $A$. Recall the following algebraic identity and apply it to $x=D_U^{-1}U'$. $$(1+x)\left( \sum_{i=0}^{m}(-x)^i \right)=1-(-x)^{m+1}$$
Choosing $m=n-1$ we have $$\left(I+D_U^{-1}U' \right)\left( \sum_{i=0}^{n-1}(-D_U^{-1}U')^i \right)=I- \left( -D_U^{-1}U'\right)^n =I. $$
So $$\left(I+D_U^{-1}U' \right)^{-1}=\sum_{i=0}^{n-1}(-D_U^{-1}U')^i=\sum_{i=0}^{n}(-D_U^{-1}U')^i.$$
In our language $$ps([-1]\odot D_U^{-1}U')=\sum_{i=0}^{n}(-D_U^{-1}U')^i=\left(I+D_U^{-1}U' \right)^{-1}.$$
We define $$\diaginverse{U}=\ssum v. f_{/}(1,v^*Uv)\odot vv^*.$$
Finally $$U^{-1}= ps\left([-1]\odot \left[\diaginverse{U}(U-\getdiag{U})\right] \right)\diaginverse{U}.$$
Recall that we have $L^{-1}=\initf{I}{v}{X}{\red{X\cdot A}{v}\cdot X}.$ So $A^{-1}=U^{-1}L^{-1}.$
$$\text{compute }L^{-1}\rightarrow\text{compute }U=L^{-1}A\rightarrow\text{compute }U^{-1}\rightarrow\text{compute }A^{-1}=U^{-1}L^{-1}$$


\section{Proofs of Section~\ref{sec:circuits}}

% \subsection{Arithmetic circuits (AC)}
% An arithmetic circuit $\Phi$ over a set $X=\{x_1,\ldots,x_n\}$ of variables is a directed
acyclic labelled graph. The vertices of $\Phi$ are called gates and denoted by $g_1,\ldots,g_m$;
the edges in $\Phi$ are called wires. The children of a gate $g$ correspond to all gates
$g'$ such that $(g,g')$ is an edge. The parents of $g$ correspond to all gates $g'$ 
such that $(g,g')$ is an edge. The in-degree of a gate $g$ refers to its number of children,
the out-degree to its number of parents. Gates with in-degree $0$ are called input gates
and are labelled by either a variable in $X$ or a constant $0$ or $1$. Every other gate
is labeled by either $+$ or $\times$ and are referred to sum or product gates, respectively.
Gates with out-degree $0$ are called output gates.

\floris{Restrictions on fan-in?}

The size of $\Phi$, denoted by $|\Phi|$, is its number of gates. The depth of a gate $g$, denoted
by $\mathsf{depth}(g)$, is the length of the longest directed path reaching $g$. The depth of $\Phi$
is the maximal depth of a gate in $\Phi$. An arithmetic circuit $\Phi$ corresponds to a polynomial in $\mathbb{N}[X]$ in a natural way. The degree of $\Phi$ is the degree of the polynomial corresponding to $\Phi$.

An arithmetic circuit family is a set of arithmetic circuits $\{\Phi_n\mid n=1,2,\ldots\}$ where $\Phi_n$ has $n$ input variables. An arithmetic circuit family is uniform if there exists a logspace-computable function
which on input $1^n$ returns an encoding of an arithmetic circuits $C_n$ for each $n=1,2$. We observe that
uniform arithmetic circuit families are necessarily of polynomial size.


\subsection{Linear space functions}
% \section{Logspace Stuff}
%
% \section*{Simulating Logspace Turing machine on input $o^n$.}
%
% We are given a logspace Turing Machine $T=\left(Q,\{0\},\ell,\rhd,\lhd,\Delta,q_0,q_m\right)$ where $Q=\{q_1,\ldots,q_m\}$ are the states, $\ell$ denotes the number of heads, $q_0$ and $q_m$ denote initial and final state, respectively, $\{0\}$ is the tape alphabet, and $\rhd$ and $\lhd$ are special symbols denoting the beginning and the end of the tape, respectively. Finally,
% $\Delta=(\Delta_Q,\Delta_1,\ldots,\Delta_\ell)$ is the transition function of $T$, where $\Delta_Q:Q\times \{\rhd,0,\lhd\}^\ell\mapsto Q$ and for $i\in[\ell]$,
% $\Delta_i:Q\times  \{\rhd,0,\lhd\}^\ell\mapsto\{\gets,\to\}$. In other words, when $T$ is in state $q$ and the $\ell$ heads read symbols $b_1,\ldots,b_\ell$, $\Delta_Q(q,b_1,\ldots,b_\ell)$ indicates to which state $T$ will transition, and moreover, $\Delta_i(q,b_1,\ldots,b_\ell)$ says in which direction (left or right) the $i$th head will move. We assume that when $T$ is in the initial state $q_0$ all heads point to the first position (i.e., they all read symbol $\rhd$).
%
% In our setting, the tape contents will always be of the form $w_n:=\rhd 0^n \lhd$ for some $n\in\mathbb{N}$. As usual, $T$ cannot move beyond the begin and end markers, $\rhd$ and $\lhd$, respectively. We assume that $T$ accepts or rejects the input $w_n$ using $\mathcal{O}(n^k)$ steps. In other words, there exists a constant $c$ such that $T$ runs in at most $cn^k$ steps. For simplicity, we assume that $T$ runs for at most $n^{k+1}$ steps. For technical reasons, that will become clear below, we assume that once $T$ reaches the final state $q_m$, $T$ will further transition but only to the state $q_m$. That is, $\Delta$ contains transitions that make $T$ loop in $q_m$.
%
% \begin{proposition}
% Given a logspace Turing machine $T$ with $m$ states, $\ell$ heads and which runs on $w_n=\rhd 0^n \lhd$ in time $n^{k+1}$, for $n\in\mathbb{N}$, there exists (i)~a $\mathsf{MATLANG}$
% schema $\mathcal{S}=(\mathcal{M},\textsf{size})$ where $\mathcal{M}$ consists matrix variables\footnote{We also need a finite number of auxiliary variables, these will be specified in the proof.} $X_1,\ldots,X_m,Y_1,\ldots,Y_\ell, v_1,\ldots,v_{k+1}$ and $w$ with
% $\mathsf{size}(V)=\alpha\times 1$ for all $V\in\mathcal{M}$ and $V\neq w$ and $\mathsf{size}(w)=1\times 1$; and (ii)~a $\mathsf{MATLANG}$ expression $e_T$ over $\mathcal{S}$ such that for the instance $I=(\mathcal{D},\textsf{mat})$ over $\mathcal{S}$ with $\mathcal{D}(\alpha)=n+2$ and $\mathsf{mat}(V)=\left(\begin{smallmatrix}0\\
% 0\\\vdots\\0\end{smallmatrix}\right)$ for all $V\in\mathcal{M}$ and
% $\mathsf{mat}(w)=[0]$, we have that
% $e_T(I)=[1]$ if $T$ accepts $w_n$ and  $e_T(I)=[0]$ otherwise.
% \end{proposition}
% \begin{proof}
% We start by explaining the semantics of the matrix variables in $\mathcal{M}$. The variables $X_1,\ldots,X_m,Y_1,\allowbreak\ldots,Y_\ell$ will be used inside for loops and will be updated using \textsf{MATLANG} expressions. Initially, all these matrix variables are instantiated with the zero column vector, as described by the instance $I$.
%
% With each state $q_i\in Q$ we associate matrix variable $X_i$.
% Then, $T$ is in state $q_i$ when
%  $X_i=\left(\begin{smallmatrix}1\\
% 0\\\vdots\\0\end{smallmatrix}\right)$, otherwise $X_i=\left(\begin{smallmatrix}0\\
% 0\\\vdots\\0\end{smallmatrix}\right)$.	Similarly, with each head $i\in[\ell]$ we
% associate matrix variable $Y_i$. When the $i$th head points at position $j$ in $w_n$,
% then $Y_i=\mathsf{e}_j$, i.e., it is the $j$th canonical column vector. We remark that since the dimensions is $n+2$ and $T$ cannot change the input word $w_n$, the $n+2$ canonical vectors suffice to indicate all positions in $w_n$ (which is of length $n+2$).
%
% The variables $v_1,\ldots,v_{k+1}$ represent $k+1$ canonical vectors  which are use to iterate in for loops. By iterating over then, we can perform $(n+2)^{k+1}$ iterations, which suffices for simulating the $n^{k+1}$ steps used by $T$ on input $w_n$.
%
% Finally, the variable $w$ is used for the output of $e_T$. It contains a scalar and will hold $1$ if $w_n$ is accepted by $T$ and $0$ otherwise.
%
% The expression $e_T$ uses some subexpressions in $\mathsf{MATLANG}$ which use some auxiliary variables.
% As a consequence, $e_T$ is an expression defined over an extended schema $\mathcal{S}'$. Hence, the instance $I$ in the statement of the Proposition is in fact an instance $I'$ of $\mathcal{S}'$ which
% coincides with $I$ on $\mathcal{S}$ and in which the auxiliary matrix variables are all instantiated with zero vectors or matrices, depending on their size.
%
% We list the used subexpressions next and explicitly denote the auxiliary matrix variables:
% \begin{itemize}
% 	% \item $\mathsf{max}(z,Z)$, an expression over auxiliary variables $z$ and $Z$ with $\mathsf{size}(z)=\mathsf{size}(Z)=\alpha\times 1$. On input $I'$ with
% 	% $\mathsf{mat}(z)=\mathsf{mat}(Z)$ the zero column vector of dimension $n+2$,
% 	%  $\mathsf{max}(I)=\mathbf{e}_{n+2}$.
% 	\item $\mathsf{pred}(z,Z,z',Z')$, and expression over auxiliary variables $z$, $z'$, $Z$ and $Z'$ with $\mathsf{size}(z)=\mathsf{size}(z')=\mathsf{size}(Z)=\alpha\times 1$ and $\mathsf{size}(Z')=\alpha\times\alpha$. On input $I'$ with
% 	$\mathsf{mat}(z)=\mathsf{mat}(z')=\mathsf{mat}(Z)$ the zero column vector of dimension $n+2$, and $\mathsf{mat}(Z')$ the zero $(n+2)\times (n+2)$ matrix,
% 	 $\mathsf{pred}(I')$ returns an $(n+2)\times (n+2)$ matrix such that
%
% 	 $$\mathsf{pred}(I')\mathbf{e}_i:=\begin{cases}
% 	 \mathbf{e}_{i-1} & \text{if $i>1$}\\
% 	 \mathbf{0} & \text{if $i=1$}.
% 	\end{cases}
% 	$$
% 	In other words, $\mathsf{pred}$ defines a predecessor relation among canonical vectors of dimension $n+2$.
% 	 \item $\mathsf{succ}(z,Z,z',Z')$, and expression over auxiliary variables $z$, $z'$, $Z$ and $Z'$ with $\mathsf{size}(z)=\mathsf{size}(z')=\mathsf{size}(Z)=\alpha\times 1$ and $\mathsf{size}(Z')=\alpha\times\alpha$. On input $I'$ with
% 	$\mathsf{mat}(z)=\mathsf{mat}(z')=\mathsf{mat}(Z)$ the zero column vector of dimension $n+2$, and $\mathsf{mat}(Z')$ the zero $(n+2)\times (n+2)$ matrix,
% 	 $\mathsf{succ}(I')$ returns an $(n+2)\times (n+2)$ matrix such that
%
% 	 $$\mathsf{succ}(I')\mathbf{e}_i:=\begin{cases}
% 	 \mathbf{e}_{i+1} & \text{if $i<n+2$}\\
% 	 \mathbf{0} & \text{if $i=n+2$}.
% 	\end{cases}
% 	$$
% 	In other words, $\mathsf{succ}$ defines a successor relation among canonical vectors.
% 	\item $\textsf{ismin}(v,z,Z,z',Z)$ with auxiliary variables $z$, $z'$, $Z$ and $Z'$ as before, and $v$ is one of the (vector) variables in $\mathcal{M}$. For an $(n+2)\times 1$ vector $\mathbf{v}$, on input $I'[v\gets \mathbf{v}]$	$$\mathsf{ismin}(I'[v\gets\mathbf{v}]):=\begin{cases} 1 & \text{if $\mathbf{v}=\mathbf{e}_1$}\\
% 		0 & \text{otherwise}.
% 		\end{cases}$$
% 	\item $\textsf{ismax}(v,z,Z,z',Z)$ with auxiliary variables $z$, $z'$, $Z$ and $Z'$ as before, and
% 	and $v$ is one of the (vector) variables in $\mathcal{M}$. For an $(n+2)\times 1$ vector $\mathbf{v}$, on input $I'[v\gets \mathbf{v}]$
%
% 	$$\mathsf{ismax}(I'[v\gets\mathbf{v}]):=\begin{cases} 1 & \text{if $\mathbf{v}=\mathbf{e}_{n+2}$}\\
% 		0 & \text{otherwise}.
% 		\end{cases}$$
% 	\item $\mathsf{min}(z,Z,z',Z',z'',Z'')$, an expressions with
% 	auxiliary variables $z$, $z'$, $z''$, $Z$, $Z'$ and $Z''$ with $\mathsf{size}(z)=\mathsf{size}(z')=\mathsf{size}(z'')=\mathsf{size}(Z)=\mathsf{size}(Z'')\alpha\times 1$ and $\mathsf{size}(Z')=\alpha\times\alpha$. On input $I'$ with
% 	matrix variables instantiated with zero vectors (or matrix for $Z'$),
%  	 $\mathsf{min}(I')=\mathbf{e}_1$.
% 	 		\item We additionally define, based on the previous expressions,
% 			$$\mathsf{test}_b(v,z,Z,z',Z'):=\begin{cases} \mathsf{ismin}(v,z,Z,z,Z') & \text{if $b=\rhd$}\\
%      \mathsf{ismax}(v,z,Z,z,Z') & \text{if $b=\lhd$}\\
% 	 (1-\mathsf{ismin}(v,z,Z,z',Z'))(1-\mathsf{ismax}(v,z,Z,z',Z')) & \text{if $b=0$}.
% 		\end{cases}.$$
% 	When evaluated on $I'[v\gets\mathbf{v}]$, $\mathsf{test}_b(I'[v\gets\mathbf{v}])$ will be $1$ when either
% 	$\mathbf{v}=\mathbf{e}_{1}$ and $b=\rhd$ (first position)
% 	$\mathbf{v}=\mathbf{e}_{n+2}$ and $b=\lhd$ (last position),
% 	$\mathbf{v}\neq \mathbf{e}_{1}$ and $\mathbf{v}\neq \mathbf{e}_{n+2}$  and $b=0$ (not first or last position). We use this expression below to check whether the heads are consistent with the symbols on the tape.
%
%
%  \item Finally, we define
%  $$
%  \mathsf{move}_d(z,Z,z',Z'):=\begin{cases}
%  e_{\mathsf{pred}}(z,Z,z',Z) & \text{if $d=\gets$}\\
%   e_{\mathsf{succ}}(z,Z,z',Z) & \text{if $d=\to$}.
%  \end{cases},
%  $$
%  $\mathsf{move}_d(I')$ will simply return the predecessor matrix when $d=\gets$ and the successor matrix when $d=\to$. This expression will be used to move the heads.
% \end{itemize}
% We thus see that we only need $z,z',z'',Z,Z',Z''$ as auxiliary variables and these can be re-used for every occurrence of the subexpressions in $e_T$. From now one, we omit the auxiliary variables from the description of $e_T$.
%
% We define the expression $e_T$, as follows\footnote{In the expression I uses $;w$ to indicate the output variable. That is, the for loops updates all instances for $X_1,\ldots,X_m,Y_1,\ldots,Y_\ell$ and $w$, but the result of the expression is only what is in the instance corresponding to $w$. We can simulate this again if we allow constant dimensional canonical basis vectors in $\mathsf{MATLANG}$.}:
% $$
% e_T:= \mathsf{for\,} v_1,\ldots,v_{k+1},X_1,\ldots,X_m,Y_1,\ldots,Y_\ell; w.(e_w,e_{X_1},\ldots,e_{X_m},e_{Y_1},\ldots,e_{Y_\ell}),
% $$
% with
% \begin{align*}\allowdisplaybreaks
% 	e_w&:=\mathsf{ismin}(X_m)\\
% 	e_{X_1}&:=\left(\prod_{j=1}^{k+1} \textsf{ismin}(v_i)\right)\cdot\mathsf{min}
% 	+ \sum_{\substack{q,b_1,\ldots,b_\ell\\
% 	\Delta_Q(q,b_1,\ldots,b_\ell)=q_1}} \!\!\!\!\!\!\!\!\! \textsf{ismin}(X_q)\left(\prod_{j=1}^\ell \mathsf{test}_{b_j}(Y_j)\right)\mathsf{min}\\
% 	e_{X_i}&:= \sum_{\substack{q,b_1,\ldots,b_\ell\\
% 	\Delta_Q(q,b_1,\ldots,b_\ell)=q_i}}\!\!\!\!\!\!\!\!\! \textsf{ismin}(X_q)\left(\prod_{j=1}^\ell \mathsf{test}_{b_j}(Y_j)\right)\mathsf{min} \quad \text{for $i\neq 1$}\\
% 	e_{Y_i}&:=\left(\prod_{j=1}^{k+1} \textsf{ismin}(v_i)\right)\cdot\mathsf{min}
% 	+\sum_{\substack{q,b_1,\ldots,b_\ell\\
% 	\Delta_i(q,b_1,\ldots,b_\ell)=d}}\!\!\!\!\!\!\!\!\! \textsf{ismin}(X_q)\left(\prod_{j=1}^\ell \mathsf{test}_{b_j}(Y_j)\right)\mathsf{move}_d\cdot Y_i
% \end{align*}
%
% The correctness of $e_T$ is now readily verified. We do this by induction on the number of iterations in the for loop. We note that initially, all variables are assigned zero vectors and values (for $w$).
%
% At the start of the run of $T$, we are in state $q_1$ and all heads point to the first position. We argue that after the first
% iterations, i.e., when $v_i=\mathbf{e}_1$ for $i\in[k+1]$, we indeed have that $X_1=\mathbf{e_1}$, $X_j=\mathbf{0}$ for $j\neq 1$, and $Y_j=\mathbf{e}_1$ for $j\in[\ell]$. Indeed, in the expression for $e_{X_1}$ the test $\prod_{j=1}^{k+1} \textsf{ismin}(v_i)$ will return $1$ and hence $X_1$ is replaced by $\mathsf{min}=\mathbf{e}_1$. Since all $X_i$ are initially zero, $\mathsf{ismin}(X_i)$ evaluate to zero for all $i\in[m]$ so the second term in $e_{X_1}$ adds the zero vector to $\mathbf{e}_1$ and thus $X_1$ remains $\mathbf{e_1}$.
% Similarly, $e_{X_j}$ for $j\neq 1$ will leave $X_j$ unchanged, so these remain zero vectors. For the head positions, a similar arguments shows that after the first iterations, all $Y_i$ are set of $\mathbf{e}_1$.
%
% We next assume that up to a certain iteration $\kappa-1$, the matrix variables correctly encode a configuration of $T$ on $w_n$ and furthermore, this configuration is reachable from the initial configuration. We next show that this remains to hold in the $\kappa$th iteration.
%
% By induction, there will be a single $X_i$ which is instantiated with $\mathbf{e}_1$. Let use assume that this $X_q$. All other $X_i$ are instantiated with the zero vector. Furthermore, $Y_j=\mathbf{e}_{i_j}$ for some $i_j\in[n+2]$.
%
% Suppose that
% $\Delta_Q(q,b_{i_1},\ldots,b_{i\ell})=p$ and
% $\Delta_j(q,b_{i_1},\ldots,b_{i\ell})=d_j$. Then, inspecting the expressions $e_{X_i}$, all $X_{q_i}$ with $q_i\neq p$ will be replaced by the zero vector. The reason is that for $X_i$ to be replaced by $\mathsf{min}$ (i.e., $\mathbf{e}_1$) when $T$ is in state $q$,
% there must be a transition $\Delta_Q(q,b_{i_1}',\ldots,b_{i_\ell}')=q_i$
% and such that the $b_{i_j}'$ corresponds to the positions encoded by
% the $Y_i$'s. In particular, $b_{i_1},\ldots,b_{i\ell}$ and
% $b_{i_1}',\ldots,b_{i_\ell}'$ must have $\rhd$ and $\lhd$ at the same positions, and since the only remaining symbol is $0$, $b_{i_1},\ldots,b_{i\ell}$ and $b_{i_1}',\ldots,b_{i\ell}'$ must agree.
% This in turn would imply that there are two possible states $p$ and $q_i$ from $q$ while reading $b_{i_1},\ldots,b_{i\ell}$. This is impossible since $T$ is deterministic.
%
% If we next consider the expressions $e_{Y_i}$ for $i\in[\ell]$, then a similar argument shows that at most one of the terms in the second part in $e_{Y_i}$ can replace $Y_i$ with $\mathsf{move}_d\cdot Y_i$. By defining of $\mathsf{move}_d$ in terms of the predecessor or successor matrix (depending on whether $d=\gets$ or $d=\to$, respectively), and given that $Y_i$ corresponds to a canonical vector, say $\mathbf{e}_{i_s}$, then $\mathsf{move}_d\cdot Y_i$ will replace $Y_i$
% with either $\mathbf{e}_{i_s-1}$ or  $\mathbf{e}_{i_s+1}$. We note that when $Y_i$ is $\mathbf{e}_1$ or $\mathbf{e}_{n+2}$, $d$ must necessarily be $\to$ or $\gets$, respectively, since $T$ does not move beyond the end markers.
%
% Hence, all combined we see that after the $\kappa$the iteration, $X_1,\ldots,X_m$ and $Y_1,\ldots,Y_\ell$ indeed correspond to the next configuration of $T$.
%
% We now remark that $w$ will be zero unless in one of the iterations populates $X_m$ with $\mathbf{e}_{1}$, i.e., the $T$ is in the final state. By assumption, $T$ will continue to be in the final state from that point on, and thus after perform our $(n+2)^k$, $w$ will remain $1$. If no final state is encountered, $w$ remains $0$, as desired.
% \end{proof}


We consider  deterministic Turing Machines  (TM) $T$ consisting of $\ell$ read-only input tapes, denoted by $R_1,\ldots,R_\ell$,
a work tape, denoted by $W$, and a write-only output tape, denoted by $O$. The TM $T$ has a set $Q$ of $m$
states, denoted by $q_0,\ldots,q_m$. We assume that $q_0$ is the initial state and $q_m$ is the accepting state.
The input and tape alphabet are $\Sigma=\{0,1\}$ and $\Gamma=\Sigma\cup\{\rhd,\lhd\}$, respectively. The special symbol $\rhd$ denotes the beginning of each of the tapes, the symbol $\lhd$ denotes the end of the $\ell$ input tapes. The transition function $\Delta$ is defined as usual, i.e., 
$\Delta:Q\times \Gamma^{\ell+2} \to Q\times \Gamma^{2}\times \{\leftarrow,\sqcup,\rightarrow\}^{\ell+2}$ such that $\Delta(q,(a_1,\ldots,a_{\ell},b,c))=\bigl(q',(b',c'),(\mathsf{d}_1,\ldots,\mathsf{d}_{\ell+2})\bigr)$ with $\mathsf{d}_i\in \{\leftarrow,\sqcup,\rightarrow\}$, means that when $T$ is in state $q$ and the $\ell+2$ heads on the tapes read symbols $a_1,\ldots,a_{\ell},b,c$, respectively, then $T$ transitions to state $q'$, writes $b',c'$ on the work and output tapes, respectively, at the position to which the work and output tapes' heads points at, and finally moves the heads on the tapes according $\mathsf{d}_1,\ldots,\mathsf{d}_{\ell+2}$. More specifically, $\leftarrow$  indicates a move to the left, 
$\rightarrow$ a move to the right, and finally, $\sqcup$ indicates that the head does not move.

We assume that $\Delta$ is defined such that it ensures that on none of the tapes, heads can move beyond the leftmost marker $\rhd$. Furthermore, the tapes $R_1,\ldots,R_\ell$ are treated as read-only and the heads on these tapes cannot move beyond the end markers $\lhd$. Similarly, $\Delta$ ensures that the output tape $O$ is write only, i.e., its head cannot move to the left.  We also assume that $\Delta$ does not change the occurrences of $\rhd$ or writes $\lhd$ on the work and output tape.

A configuration of $T$ is defined in the usual way. That is, a configuration of the input tapes is of the form
$\rhd w_1qw_2\lhd$ with $w_1,w_2\in\Sigma^*$ and represents that the current tape content is $\rhd w_1w_2\lhd$, $T$ is in state $q$ and the head is positioned on the first symbol of $w_2$. Similarly, configurations of the work and output tape are represented by $\rhd w_1qw_2$. A configuration of $T$ is consists of configurations for all tapes. Given two configurations $c_1$ and $c_2$, we say that $c_1$ yields $c_2$ if $c_2$ is the result of applying the transition function $\Delta$ of $T$ based on the information in $c_1$. As usual, we close this ``yields'' relation transitively.

Given $\ell$ input words $w_1,\ldots,w_\ell\in\Sigma^*$, we assume that the initial configuration of $T$ is given by
 $\bigl(q_0\rhd  w_1\lhd,q_0\rhd w_2\lhd,\ldots, q_0\rhd w_\ell\lhd,q_0\rhd, q_0\rhd \bigr)$ and an accepting configuration is assumed to be of the form $\bigl(\rhd q_m w_1\lhd,\rhd q_m w_2\lhd,\ldots, \rhd q_m w_\ell\lhd,\rhd q_m,\rhd q_m w\bigr)$ for some $w\in\Sigma^*$. We say that $T$ computes the function $f:(\Sigma^*)^{\ell}\to\Sigma^*$ if for every $w_1,\ldots,w_\ell\in\Sigma^*$, the initial configuration yields (transitively) an accepting configuration such that the configuration on the output tape is
 given by $\rhd q_m f(w_1,\ldots,w_\ell)$.

We assume that once $T$ reaches an accepting configuration it stays indefinitely in that configuration (i.e., it loops). We further assume that $T$ only reaches an accepting configuration when all its input
words have the same size. Furthermore, when all inputs have the same size, $T$ will reach an accepting configuration. 


We say that $T$ is a \textit{linear space machine} when it reaches an accepting configuration 
on inputs of size $n$ by using $\mathcal{O}(n)$ space on its work tape and additionally needs  $\mathcal{O}(n^k)$ steps to do so. A \textit{linear input-output function} is a function of the form $f=\bigcup_{n\geq 0} f_n:(\Sigma^n)^\ell\to\Sigma^n$. In other words, for every $\ell$ words of the same size $n$, $f$ returns a word of size $n$. We say that a linear input-output function is a \textit{linear space input-output function} if
there exists a linear space machine  $T$ that for every $n\geq 0$, on input $w_1,\ldots,w_\ell\in\Sigma^n$ the TM $T$ has
$f_n(w_1,\ldots,w_\ell)$ on its the output tape when (necessarily) reaching an accepting configuration.

% We say that a function
% $f:\underbrace{\Sigma^n\times\cdots \times \Sigma^n}_{\text{$\ell$ times}}\to \Sigma^n$ is computable by a linear space machine $T$ if when $T$ is run on input $w_1,\ldots,w_\ell$ it halts and has $f(w_1,\ldots,w_\ell)$ on its output tape. We say that $f$ is a \textit{linear space poly function} if it is computable by a linear space TM $T$ which in addition runs in polynomial time, i.e., it hals in at most
% $\mathcal{O}(n^k)$ steps for a certain $k$ on any inputs $w_1,\ldots,w_\ell$ of size $n$.
\begin{proposition}
Let $f=\bigcup_{n\geq 0}f_n:(\Sigma^n)^\ell\to \Sigma^n$ be a linear space input-ouput function computed by a linear space  machine $T$ with $m$ states, $\ell$ input tapes, which consumes $\mathcal{O}(n)$ space and runs in $\mathcal{O}(n^{k-1})$ time on inputs of size $n$. There exists (i)~a $\mathsf{MATLANG}$ 
schema $\mathcal{S}=(\mathcal{M},\textsf{size})$ where $\mathcal{M}$ consists matrix variables\footnote{We also need a finite number of auxiliary variables, these will be specified in the proof.} $Q_1,\ldots,Q_m,R_1,\ldots,R_\ell,H_1,\ldots,H_\ell,W_1,\ldots,W_s,H_{W_1},\ldots,H_{W_s},O,H_O, v_1,\ldots,v_{k}$  with
$\mathsf{size}(V)=\alpha\times 1$ for all $V\in\mathcal{M}$; and (ii)~a $\mathsf{MATLANG}$ expression $e_f$ over $\mathcal{S}$ such that for the instance $I=(\mathcal{D},\textsf{mat})$ over $\mathcal{S}$ with $\mathcal{D}(\alpha)=n$ and 
$$\mathsf{mat}(R_i)=\mathsf{vec}(w_i)\in \mathbb{R}^n\text{, for $i\in[\ell]$ and all other matrix variables instantiated with the zero vector in $\mathbb{R}^n$} $$
for words $w_1,\ldots,w_\ell\in\Sigma^n$ and such that $\mathsf{vec}(w_i)$ is the $n\times 1$-vector encoding the word $w_i$, we have that the $\mathsf{mat}(O)=\mathsf{vec}(f_n(w_1,\ldots,w_n))\in\mathbb{R}^n$ after evaluating $e_f(I)$.
\end{proposition}
\begin{proof}
	The expression $e_f$ we construct will simulate the TM $T$. To have some more control on the space and time consumption of $T$, let us first assume that $n$ is large enough, say larger than $n\geq N$, such that $T$ runs in $sn$ space and $cn^{k-1}\leq n^k$ time for constants $s$ and $c$. We deal with $n<N$ later on.

To simulate $T$ we need to encode states, tapes and head positions. The matrix variables in 
$\mathcal{M}$ mentioned in the proposition will take these roles. More specifically, the variables $R_1,\ldots,R_\ell$ will hold the input vectors, $W_1,\ldots,W_s$ will hold the contents of the work
tape, where $s$ is the constant mentioned earlier, and $O$ will hold the contents of the output tape. The vectors corresponding to the work and output tape are initially set to the zero vector. The vector for the input tape $R_i$ is set to $\mathsf{vec}(w_i)$, for $i\in[\ell]$.

 With each tape we associate a matrix variable encoding the position of the head. More specifically, $H_1,\ldots,H_\ell$ correspond to the input tape heads,
$H_{W_1},\ldots, H_{W_s}$ are the heads for the work tape, and $H_O$ is the head of the output tape. All these vectors are initialised with the zero vector. Later on, these vectors will be zero except for a single position, indicating the positions in the corresponding tapes the heads point to. For those positions $j$, $1<j<n$, the head vectors will carry value $1$.  When $j=1$ or $n$ and when it concerns positions for the input tape, the head vectors can carry value $1$ or $2$. We need to treat these border cases separately
because we only have $n$ positions available to store the input words, whereas the actual input tapes consist of $n+2$ symbols because of $\rhd$ and $\lhd$. So when, for example, $H_1$ has a $1$ in its first
entry, we interpret it as the head is pointing to the first symbol of the input word $w_1$. When $H_1$
has a $2$ in its first position, we interpret it as the head pointing to $\rhd$. The end marker $\lhd$ is
dealt with in the same way, by using value $1$ or $2$ in the last position of $H_1$. We use this encoding
for all input tapes, and also for the work tape $W_1$ and output tape $O$ with the exception that no end marker $\lhd$ is present.

 
% ,Y_1,\allowbreak\ldots,Y_\ell$ will be used inside for loops and will be updated using \textsf{MATLANG} expressions. Initially, all these matrix variables are instantiated with the zero column vector, as described by the instance $I$.

To encode the states, we use the variables $Q_1,\ldots,Q_m$. We will ensure that when $T$ is in state $q_i$ when
 $\mathsf{mat}(Q_i)=(1,0,\ldots,0)^t\in\mathbb{R}^n$, otherwise $\mathsf{mat}(Q_i)$ is the zero vector in $\mathbb{R}^n$.	

Finally, the variables $v_1,\ldots,v_{k}$ represent $k$ canonical vectors  which are use to iterate in for loops. By iterating over then, we can perform $n^{k}$ iterations, which suffices for simulating the $\mathcal{O}(n^{k-1})$ steps used by $T$ to reach an accepting configuration. 
% (We recall that $T$ loops when reaching an accepting configuration.)

With these matrix variables in place, we start by defining $e_f$. It will consists of two subexpressions
$e_f^{\geq N}$, for dealing with $n\geq N$, and $e_f^{<N}$, for dealing with $n<N$. We explain the expresion
$e_f^{\geq N}$ first.



In our  expressions we use subexpressions which we defined before. These subexpression require some auxiliary variables, as detailed below. As a consequence, $e_f$ will be an expressions defined over an extended schema $\mathcal{S}'$. Hence, the instance $I$ in the statement of the Proposition is  an instance $I'$ of $\mathcal{S}'$ which
coincides with $I$ on $\mathcal{S}$ and in which the auxiliary matrix variables are all instantiated with zero vectors or matrices, depending on their size.

\floris{The expressions below are used in other proofs as well, I just wanted to check how many auxiliary variables are needed. We can extract the list below and place it somewhere else.}
We next list the used subexpressions and explicitly denote the auxiliary matrix variables:
\begin{itemize}
	% \item $\mathsf{max}(z,Z)$, an expression over auxiliary variables $z$ and $Z$ with $\mathsf{size}(z)=\mathsf{size}(Z)=\alpha\times 1$. On input $I'$ with
	% $\mathsf{mat}(z)=\mathsf{mat}(Z)$ the zero column vector of dimension $n+2$,
	%  $\mathsf{max}(I)=\mathbf{e}_{n+2}$.
	\item $\mathsf{pred}(z,Z,z',Z')$, and expression over auxiliary variables $z$, $z'$, $Z$ and $Z'$ with $\mathsf{size}(z)=\mathsf{size}(z')=\mathsf{size}(Z)=\alpha\times 1$ and $\mathsf{size}(Z')=\alpha\times\alpha$. On input $I'$ with 
	$\mathsf{mat}(z)=\mathsf{mat}(z')=\mathsf{mat}(Z)$ the zero column vector of dimension $n$, and $\mathsf{mat}(Z')$ the zero $n\times n$ matrix,
	 $\mathsf{pred}(I')$ returns an $n\times n$ matrix such that 
	 
	 $$\mathsf{pred}(I')\mathbf{e}_i:=\begin{cases} 
	 \mathbf{e}_{i-1} & \text{if $i>1$}\\
	 \mathbf{0} & \text{if $i=1$}.
	\end{cases}
	$$
	In other words, $\mathsf{pred}$ defines a predecessor relation among canonical vectors of dimension $n$.
	 \item $\mathsf{succ}(z,Z,z',Z')$, and expression over auxiliary variables $z$, $z'$, $Z$ and $Z'$ with $\mathsf{size}(z)=\mathsf{size}(z')=\mathsf{size}(Z)=\alpha\times 1$ and $\mathsf{size}(Z')=\alpha\times\alpha$. On input $I'$ with 
	$\mathsf{mat}(z)=\mathsf{mat}(z')=\mathsf{mat}(Z)$ the zero column vector of dimension $n$, and $\mathsf{mat}(Z')$ the zero $n\times n$ matrix,
	 $\mathsf{succ}(I')$ returns an $n\times n$ matrix such that 
	 
	 $$\mathsf{succ}(I')\mathbf{e}_i:=\begin{cases} 
	 \mathbf{e}_{i+1} & \text{if $i<n$}\\
	 \mathbf{0} & \text{if $i=n$}.
	\end{cases}
	$$
	In other words, $\mathsf{succ}$ defines a successor relation among canonical vectors.
	\item $\textsf{ismin}(v,z,Z,z',Z)$ with auxiliary variables $z$, $z'$, $Z$ and $Z'$ as before, and $v$ is one of the (vector) variables in $\mathcal{M}$. For an $n\times 1$ vector $\mathbf{v}$, on input $I'[v\gets \mathbf{v}]$	$$\mathsf{ismin}(I'[v\gets\mathbf{v}]):=\begin{cases} 1 & \text{if $\mathbf{v}=\mathbf{e}_1$}\\
		0 & \text{otherwise}.
		\end{cases}$$
	\item $\textsf{ismax}(v,z,Z,z',Z)$ with auxiliary variables $z$, $z'$, $Z$ and $Z'$ as before, and 
	and $v$ is one of the (vector) variables in $\mathcal{M}$. For an $n\times 1$ vector $\mathbf{v}$, on input $I'[v\gets \mathbf{v}]$
	
	$$\mathsf{ismax}(I'[v\gets\mathbf{v}]):=\begin{cases} 1 & \text{if $\mathbf{v}=\mathbf{e}_{n}$}\\
		0 & \text{otherwise}.
		\end{cases}$$
	\item $\mathsf{min}(z,Z,z',Z',z'',Z'')$, an expressions with
	auxiliary variables $z$, $z'$, $z''$, $Z$, $Z'$ and $Z''$ with $\mathsf{size}(z)=\mathsf{size}(z')=\mathsf{size}(z'')=\mathsf{size}(Z)=\mathsf{size}(Z'')=\alpha\times 1$ and $\mathsf{size}(Z')=\alpha\times\alpha$. On input $I'$ with 
	matrix variables instantiated with zero vectors (or matrix for $Z'$),
 	 $\mathsf{min}(I')=\mathbf{e}_1$. 
	 		
\end{itemize}
We thus see that we only need $z,z',z'',Z,Z',Z''$ as auxiliary variables and these can be re-used whenever $e_f$ calls these functions. From now one, we omit the auxiliary variables from the description of $e_f$.


Let us first define $e_f^{\geq N}$. Since we want to simulate $T$ we need to be able to check which transitions of $T$ can be applied based on a current configuration. More precisely,
suppose that we want to check whether $\delta(q_i,(a_1,\ldots,a_{\ell},b,c))$ is applicable, then we need to check whether $T$ is in state $q_i$, we can do this by checking 
$\mathsf{ismin}(Q_i)$, and whether the heads on the tapes read symbols $a_1,\ldots,a_{\ell},b,c$. We check the latter by the following expressions.
For the input tapes $R_i$ we define
$$
\mathsf{test\_inp}^i_b:=\begin{cases}
(1-\mathsf{ismin}(1/2\cdot H_i))\cdot(1-\mathsf{ismax}(1/2\cdot H_i))\cdot(1- R_i^t\cdot H_i) & \text{if $b=0$}\\
(1-\mathsf{ismin}(1/2\cdot H_i))\cdot(1-\mathsf{ismax}(1/2\cdot H_i))\cdot(R_i^t\cdot H_i) & \text{if $b=1$}\\
\mathsf{ismin}(1/2\cdot H_i) & \text{if $b=\rhd$}\\
\mathsf{ismax}(1/2\cdot H_i) & \text{if $b=\lhd$},\\
\end{cases}
$$
which return $1$ if and only if either $b\in\{0,1\}$ is the value in $\mathsf{mat}(R_i)$ at the position encoded by $\mathsf{mat}(H_i)$, or when $b=\rhd$ and $\mathsf{mat}(H_i)$ is the vector $(2,0,\ldots,0)\in\mathbb{R}^n$, or when $b=\lhd$ and $\mathsf{mat}(H_i)$ is the vector $(0,0,\ldots,2)\in\mathbb{R}^n$. Similarly, for the output tape we define
$$
\mathsf{test\_out}_b:=\begin{cases}
(1-\mathsf{ismin}(1/2\cdot H_O))\cdot(1- O^t\cdot H_O) & \text{if $b=0$}\\
(1-\mathsf{ismin}(1/2\cdot H_O))\cdot(O^t\cdot H_O) & \text{if $b=1$}\\
\mathsf{ismin}(1/2\cdot H_O) & \text{if $b=\rhd$},\\
\end{cases}
$$
and for the work tapes $W_1,\ldots,W_s$ we define
$$
\mathsf{test\_work}^i_b:=\begin{cases}
(1-\mathsf{ismin}(1/2\cdot H_{W_i}))\cdot(1- W_i^t\cdot H_{W_i})) & \text{if $b=0$}\\
(1-\mathsf{ismin}(1/2\cdot H_{W_i}))\cdot (W_i^t\cdot H_{W_i}) & \text{if $b=1$}\\
\mathsf{ismin}(1/2\cdot H_{W_i}) & \text{if $b=\rhd$ and $i=1$}.\\
\end{cases}
$$
We then combine all these expressions into a single expression for $q_i\in Q$, $a_1,\ldots,a_\ell,b,c\in\Gamma$:
$$
\mathsf{isconf}_{q_i,a_1,\ldots,a_\ell,b,c}:=
\mathsf{ismin}(Q_i)\cdot \left(\prod_{j=1}^{\ell} \mathsf{test\_inp}_{a_j}^j\right)
\cdot\left(\sum_{j=1}^s \mathsf{test}_b^j\right)\cdot \mathsf{test\_outp}_{c}.
$$
This expression will return $1$ if and only if the vectors representing the tapes, head positions and states are such that $\mathsf{Q_i}$ is the first canonical vector (and thus $T$ is in state $q_i$), the heads point to entries in the tape vectors storing the symbols $a_1,\ldots,a_{\ell}, b,c$ or they point to the first (or last for input tapes) positions but have value $2$ (when the symbols are $\rhd$ or $\lhd$). 

To ensure that at the beginning of the simulation of $T$ by $e_f^{\geq N}$ we correctly encode that we are in the initial configuration, we thus need to initialise all vectors $\mathsf{mat}(H_1),\mathsf{mat}(H_2),\ldots, \mathsf{mat}(H_\ell), \mathsf{mat}(H_{W_1}),\mathsf{mat}(H_O)$ with the vector $(2,0,0,\ldots,0)\in\mathbb{R}$ since all heads read the symbol $\rhd$. Similarly, we have to initialise $\mathsf{Q_1}$ with the first canonical vector since $T$ is in state $q_0$.

We furthermore need to be able to correctly adjust head positions. We do this by means of the predecessor and successor expressions described above. 
A consequence of our encoding is that we need to treat the border cases (corresponding to $\rhd$ and $\lhd$) differently. More specifically, for the input tapes $R_i$ and heads $H_i$ we define 
$$
\mathsf{move\_inp}^i_{\mathsf{d}}:=
\begin{cases}
2\cdot \mathsf{ismin}(H_i)\cdot H_i + 1/2\cdot\mathsf{ismax}(1/2\cdot H_i)\cdot H_i  + (1-\mathsf{ismin}(H_i))(1-\mathsf{ismax}(1/2H_i))\cdot e_{\mathsf{pred}}\cdot H_i  
& \text{if $\mathsf{d}=\leftarrow$}\\
2\cdot \mathsf{ismax}(H_i)\cdot H_i + 1/2\cdot\mathsf{ismin}(1/2\cdot H_i)\cdot H_i  + (1-\mathsf{ismin}(1/2\cdot H_i))(1-\mathsf{ismax}(H_i))\cdot e_{\mathsf{succ}}\cdot H_i  
 & \text{if $\mathsf{d}=\rightarrow$}\\
H_i & \text{if $\mathsf{d}=\sqcup$}. 
\end{cases}
$$
In other words, we shift to the previous (or next) canonical vector when $\mathsf{d}$ is $\leftarrow$ or $\rightarrow$, respectively, unless we need to move to or from the position that will hold $\rhd$ or $\lhd$. In those case we readjust $\mathsf{mat}(H_i)$ (which will either $(1,0,\ldots,0)$, $(2,0,\ldots,0)$, $(0,\ldots,0,1)$ or $(0,\ldots,0,2)$) by either dividing or multiplying with $2$. In this way we can correctly infer whether or not the head points to the begin and end markers. For the output tape we proceed in a similar way, but only taking into account the begin marker and recall that we do not have moves to the left:
$$
\mathsf{move\_outp}_{\mathsf{d}}:=
\begin{cases}
1/2\cdot\mathsf{ismin}(1/2\cdot H_O)\cdot H_O  + (1-\mathsf{ismin}(1/2\cdot H_O))\cdot e_{\mathsf{succ}}\cdot H_O  
 & \text{if $\mathsf{d}=\rightarrow$}\\
H_O & \text{if $\mathsf{d}=\sqcup$}. 
\end{cases}
$$
Since we represent the work tape by $s$ vectors $W_1,\ldots,W_s$ we need to ensure that only one of the head vectors $H_{W_i}$ has a non-zero value and that by moving left or right, we need to appropriately update the right head vector. We do this as follows. We first consider the work tapes $W_i$ for $i\neq 1,s$ and define
$$
\mathsf{move\_work}^i_{\mathsf{d}}:=
\begin{cases}
	-\mathsf{ismin}(H_{W_i})\cdot H_{W_i} + (1-\mathsf{ismin}(H_{W_i}))\cdot e_{\mathsf{pred}}\cdot H_{W_i} + \mathsf{ismin}(H_{W_{i+1}})\cdot\mathsf{max} & \text{if $\mathsf{d}=\leftarrow$}\\
		-\mathsf{ismax}(H_{W_i})\cdot H_{W_i} + (1-\mathsf{ismax}(H_{W_i}))\cdot e_{\mathsf{succ}}\cdot H_{W_i} + \mathsf{ismax}(H_{W_{i-1}})\cdot\mathsf{min} & \text{if $\mathsf{d}=\rightarrow$}\\
	H_{W_i} & \text{if $\mathsf{d}=\sqcup$}. 	
\end{cases}
$$
In other words, we set the $H_{W_i}$ to zero when a move brings us to either $W_{i-1}$ or $W_{i+1}$, we
move the successor or predecessor when staying within $W_i$, or initialise $H_{W_i}$ with the first or last canonical vector when moving from $W_{i-1}$ to $W_i$ (right move) or from $W_{i+1}$ to $W_i$ (left move).
For $i=s$ we can ignore the parts in the previous expression that involve $W_{s+1}$ (which does not exist):
$$
\mathsf{move\_work}^s_{\mathsf{d}}:=
\begin{cases}
	-\mathsf{ismin}(H_{W_s})\cdot H_{W_i} + (1-\mathsf{ismin}(H_{W_s}))\cdot e_{\mathsf{pred}}\cdot H_{W_s}  & \text{if $\mathsf{d}=\leftarrow$}\\
		-\mathsf{ismax}(H_{W_s}) \cdot H_{W_s} + (1-\mathsf{ismax}(H_{W_s}))\cdot e_{\mathsf{succ}}\cdot H_{W_s} + \mathsf{ismax}(H_{W_{s-1}})\cdot \mathsf{min} & \text{if $\mathsf{d}=\rightarrow$}\\
	H_{W_s} & \text{if $\mathsf{d}=\sqcup$}. 	
\end{cases}
$$
For $i=1$, we can ignore the part involving $W_{0}$ (which does not exist) but have to take $\rhd$ into account:
$$
\mathsf{move\_work}^1_{\mathsf{d}}:=
\begin{cases}
	2\cdot \mathsf{ismin}(H_{W_1})\cdot H_{W_i} + (1-\mathsf{ismin}(H_{W_1}))\cdot e_{\mathsf{pred}}\cdot H_{W_1} + \mathsf{ismin}(H_{W_{2}})\cdot\mathsf{max} & \text{if $\mathsf{d}=\leftarrow$}\\
		1/2\cdot\mathsf{ismin}(1/2\cdot H_{W_1})\cdot H_{W_1} + (1-\mathsf{ismax}(1/2\cdot H_{W_1}))\cdot e_{\mathsf{succ}}\cdot H_{W_1}  & \text{if $\mathsf{d}=\rightarrow$}\\
	H_{W_1} & \text{if $\mathsf{d}=\sqcup$}. 	
\end{cases}
$$
A final ingredient for defining $e_f^{\geq N}$ are expressions which update the work and output tape.
To define these expression, we need the position and symbol to put on the tape. For the output tape we define
$$
\mathsf{write\_outp}_b:=\begin{cases}
\mathsf{ismin}(1/2\cdot H_O)\cdot O & \text{if $b=\rhd$}\\
(1-\mathsf{ismin}(1/2\cdot H_O))\cdot\left((1-O^t\cdot H_O)\cdot O + (O^t\cdot H_O)\cdot (O-H_O)\right) &\text{if $b=0$}\\
(1-\mathsf{ismin}(1/2\cdot H_O))\cdot\left((1-O^t\cdot H_O)\cdot (O+H_O) + (O^t\cdot H_O)\cdot O\right) &\text{if $b=1$}\\
\end{cases}
$$
and similarly for the work tapes $i\neq 1$:
$$
\mathsf{write\_work}_b^i:=\begin{cases}
W_i & \text{if $b=\rhd$}\\
(1-W_i^t\cdot H_{W_i})\cdot W_i + (W_i^t\cdot H_{W_i})\cdot (W_i-H_{W_i}) &\text{if $b=0$}\\
(1-W_i^t\cdot H_{W_i})\cdot (W_i+H_{W_i}) + (W_i^t\cdot H_{W_i})\cdot W_i &\text{if $b=1$},
\end{cases}
$$
and for  $W_1$ we have to take care again of the begin marker:
$$
\mathsf{write\_work}_b^1:=\begin{cases}
\mathsf{ismin}(1/2\cdot H_{W_1})\cdot W_1 & \text{if $b=\rhd$}\\
(1-\mathsf{ismin}(1/2\cdot H_{W_1})\cdot\left((1-W_1^t\cdot H_{W_1})\cdot W_1 + (W_1^t\cdot H_{W_1})\cdot (W_1-H_{W_1})\right) &\text{if $b=0$}\\
(1-\mathsf{ismin}(1/2\cdot H_{W_1})\cdot\left((1-W_1^t\cdot H_{W_1})\cdot (W_1+H_{W_1}) + (W_1^t\cdot H_{W_1})\cdot W_1\right) &\text{if $b=1$}.
\end{cases}
$$




% We additionally define, based on the previous expressions,
% 			$$\mathsf{test}_b(v,z,Z,z',Z'):=\begin{cases} \mathsf{ismin}(v,z,Z,z,Z') & \text{if $b=\rhd$}\\
%      \mathsf{ismax}(v,z,Z,z,Z') & \text{if $b=\lhd$}\\
% 	 (1-\mathsf{ismin}(v,z,Z,z',Z'))(1-\mathsf{ismax}(v,z,Z,z',Z')) & \text{if $b=0$}.
% 		\end{cases}.$$
% 	When evaluated on $I'[v\gets\mathbf{v}]$, $\mathsf{test}_b(I'[v\gets\mathbf{v}])$ will be $1$ when either
% 	$\mathbf{v}=\mathbf{e}_{1}$ and $b=\rhd$ (first position)
% 	$\mathbf{v}=\mathbf{e}_{n+2}$ and $b=\lhd$ (last position),
% 	$\mathbf{v}\neq \mathbf{e}_{1}$ and $\mathbf{v}\neq \mathbf{e}_{n+2}$  and $b=0$ (not first or last position). We use this expression below to check whether the heads are consistent with the symbols on the tape.
%
%
% Finally, we define
%  $$
%  \mathsf{move}_d(z,Z,z',Z'):=\begin{cases}
%  e_{\mathsf{pred}}(z,Z,z',Z) & \text{if $d=\gets$}\\
%   e_{\mathsf{succ}}(z,Z,z',Z) & \text{if $d=\to$}.
%  \end{cases},
%  $$
%  $\mathsf{move}_d(I')$ will simply return the predecessor matrix when $d=\gets$ and the successor matrix when $d=\to$. This expression will be used to move the heads.
% We define the expression $e_T$, as follows\footnote{In the expression I uses $;w$ to indicate the output variable. That is, the for loops updates all instances for $X_1,\ldots,X_m,Y_1,\ldots,Y_\ell$ and $w$, but the result of the expression is only what is in the instance corresponding to $w$. We can simulate this again if we allow constant dimensional canonical basis vectors in $\mathsf{MATLANG}$.}:

We are now finally ready to define $e_f^{\geq N}$:
\begin{multline*}
e_f^{\geq N}:= \mathsf{for\,} v_1,\ldots,v_{k},Q_1,\ldots,Q_m,H_1,\ldots,H_\ell,W_1,\ldots,W_s, H_{W_1},\ldots,H_{W_s},O,H_O . \\
(e_{Q_1},\ldots,e_{Q_m},e_{H_1},\ldots,e_{H_\ell},e_{W_1},\ldots,e_{W_s},e_{H_{W_1}},\ldots,e_{H_{W_s}},e_{O}, e_{H_O})
\end{multline*}
with expressions (we use $\star$ below to mark irrelevant information in the transitions):
 \allowdisplaybreaks
\begin{align*}
	e_{Q_1}&:=\left(\prod_{j=1}^{k} \textsf{ismin}(v_i)\right)\cdot\mathsf{min}
	+ \sum_{\substack{(q_i,a_1,\ldots,a_\ell,b,c)\\
	\Delta(q_i,a_1,\ldots,a_\ell,b,c)=(q_1,\star)}} \!\!\!\!\!\!\!\!\! \mathsf{isconf}_{q_i,a_1,\ldots,a_\ell,b,c}\cdot \mathsf{min} \\
	e_{Q_j}&:=\sum_{\substack{(q_i,a_1,\ldots,a_\ell,b,c)\\
	\Delta(q_i,a_1,\ldots,a_\ell,b,c)=(q_j,\star)}} \!\!\!\!\!\!\!\!\! \mathsf{isconf}_{q_i,a_1,\ldots,a_\ell,b,c}\cdot \mathsf{min}
	 \quad \text{for $j\neq 1$}\\
	e_{H_i}&:=2\left(\prod_{j=1}^{k} \textsf{ismin}(v_i)\right)\cdot\mathsf{min}
	+\sum_{\substack{(q,a_1,\ldots,a_\ell,b,d)\\
	\Delta(q,a_1,\ldots,a_\ell,b,c)=(\star,\mathsf{d_i},\star)}}\!\!\!\!\!\!\!\!\! \mathsf{isconf}_{q,a_1,\ldots,a_\ell,b,c}\cdot\mathsf{move\_inp}^i_{\mathsf{d}_i}\\
	e_{H_{W_i}}&:=2\left(\prod_{j=1}^{k} \textsf{ismin}(v_i)\right)\cdot\mathsf{min}
	+\sum_{\substack{(q,a_1,\ldots,a_\ell,b,d)\\
	\Delta(q,a_1,\ldots,a_\ell,b,c)=(\star,\mathsf{d_{\ell+1}},\star)}}\!\!\!\!\!\!\!\!\! \mathsf{isconf}_{q,a_1,\ldots,a_\ell,b,c}\cdot\mathsf{move\_work}_{\mathsf{d}_{\ell+1}}^i\\
	e_{H_O}&:=2\left(\prod_{j=1}^{k} \textsf{ismin}(v_i)\right)\cdot\mathsf{min}
	+\sum_{\substack{(q,a_1,\ldots,a_\ell,b,d)\\
	\Delta(q,a_1,\ldots,a_\ell,b,c)=(\star,\mathsf{d}_{\ell+2})}}\!\!\!\!\!\!\!\!\! \mathsf{isconf}_{q,a_1,\ldots,a_\ell,b,c}\cdot\mathsf{move\_outp}_{\mathsf{d}_{\ell+2}}\\
	e_{W_i}&:=\sum_{\substack{(q,a_1,\ldots,a_\ell,b,d)\\
	\Delta(q,a_1,\ldots,a_\ell,b,c)=(\star,b',c',\star)}}\!\!\!\!\!\!\!\!\! \mathsf{isconf}_{q,a_1,\ldots,a_\ell,b,c}\cdot\mathsf{write\_work}_{b'}^i\\
	e_{O}&:=\sum_{\substack{(q,a_1,\ldots,a_\ell,b,d)\\
	\Delta(q,a_1,\ldots,a_\ell,b,c)=(\star,b',c',\star)}}\!\!\!\!\!\!\!\!\! \mathsf{isconf}_{q,a_1,\ldots,a_\ell,b,c}\cdot\mathsf{write\_outp}_{c'}.
\end{align*}

The correctness of $e_f^{\geq N}$ should be clear from the construction (one can formally verify this by
induction on the number of iterations). We next explain how the border cases $n<N$ can be dealt with.
For each $n<N$ and every possible input words
$w_1,\ldots,w_\ell$ of size $n$, we define a MATLANG expression which check whether
$\mathsf{mat}(R_i)=\mathsf{vec}(w_i)$ for each $i\in[\ell]$. This can be easily done since $n$ can be regarded as a constant. For example, to check whether $\mathsf{mat}(R_i)=(0,1,1)^t$ we simply write
$$
(1- R_i^t\cdot \mathsf{min})\cdot (R_i^t\cdot e_{\mathsf{succ}}\cdot\min)\cdot (1- R_i^t\cdot e_{\mathsf{succ}}\cdot e_{\mathsf{succ}}\cdot \min)\cdot (1- \mathbf{1}(R_i)^t\cdot e_{\mathsf{succ}}\cdot e_{\mathsf{succ}}\cdot e_{\mathsf{succ}}\cdot \min)
$$
which will evaluate to $1$ if and only if $\mathsf{mat}(R_i)=(0,1,1)^t$. We note that the factor is in place to check that the dimension $\mathsf{mat}(R_i)$ is three.
  We denote by
$e_{n,w}^i$ the expression which evaluates to $1$ if and only if $\mathsf{mat}(R_i)=\mathsf{vec}(w)$
for $|w|=n$.
We can similarly
write any word $w$ of fixed size in the matrix variable $O$. For example, suppose that $w=101$
then we write 
$$
O+ \mathsf{min}+  e_{\mathsf{succ}}\cdot e_{\mathsf{succ}}\cdot\mathsf{min}.
$$
We write $e_{n,w}$ be the expression which write $w$ of size $|w|=n$ in matrix variable $O$.
Then, the expressions
$$
e_{n,w_1,\ldots,w_n,w}:=e_{n,w_1}^1\cdot\cdots\cdot e_{nw_{\ell}}^\ell\cdot e_{n,w}
$$
will write $w$ in $O$ if and only if $\mathsf{mat}(R_i)=\mathsf{vec}(w_i)$ for $i\in[\ell]$.
We now simply take the disjunction over all words $w_1,\ldots,w_\ell\in\Sigma^n$ and $w=f_n(w_1,\ldots,w_\ell)\in\Sigma^n$:
$$
e_n:=\sum_{w_1,\ldots,w_\ell\in\Sigma^n} e_{n,w_1,\ldots,w_\ell,f_n(w_1,\ldots,w_\ell)},
$$
which correctly evaluates $f_n$. We next take a further disjunction by letting ranging from 
$n=0,\ldots, N-1$:
$$
e_f^{<N}:=\sum_{n=0}^{N-1} e_n
$$
Since every possible input is covered and only a unique expression $ e_{n,w_1,\ldots,w_\ell,f_n(w_1,\ldots,w_\ell)}$ will be triggered $e_f^{<N}$ will correctly
evaluate $f$ on inputs smaller than $N$.

Our final expression $e_f$ is now given by
$$
e_f:=e_f^{<N} + \mathsf{dim\_is\_greater\_than_N}\cdot e_f^{\geq 0}
$$
where $\mathsf{dim\_is\_greater\_than_N}$ is the expression
$\mathbf{1}(R_i)^t\cdot\underbrace{e_{\mathsf{succ}}\cdot\cdots\cdot e_{\mathsf{succ}}}_{\text{$N$ times}}$ which will evaluate to $1$ if an only if the input dimension is larger or equal than $N$.
\end{proof}


\subsection{Circuit evaluation}
Let $f(A)$ be a function over a $n\times 1$ vector $A$ such that $f$ can be computed by a $L-$uniform arithmetic circuit of log-depth. Each gate of the circuit that encodes $f$ has an $\texttt{id}\in\lbrace 0,1 \rbrace^n$. From now on, when we write $g$ for a gate of the circuit, we mean the $\texttt{id}$ encoding $g$.
Let $n^k$ be a polynomial such that the number of wires $W(n)\leq n^k$ for $n$ big enough. Further, we assume that $2W(n)\leq n^k$. We need this because the for-matlang simulation of the circuit is in a depth first search way, so $2W(n)$ wires will be traversed.
Then we have that:
\begin{itemize}
	\item the number of gates is bounded by $n^k$.
	\item we need at most $k\log (n)$ bits to store the $id$ of a gate.
	\item the depth of the circuit is at most $k'\log (n)$ for some $k'$.
\end{itemize}

So, let $n_0$ and $k$ such that $\forall n\geq n_0:$

\begin{align*}[right=\empheqrbrace (\star)]
    2W(n)&\leq n^k \\
	k \ceil{\log (n)} &\leq n-3 \\
	k' \ceil{\log(n)} &\leq n
\end{align*}

We know $n_0$ and $k$ exist. Let $n\geq n_0$. Towards the end, we will deal with the case when $n<n_0$.

Let $g$ be a gate. The children of $g$ are denoted by $g_1,\ldots, g_l$.
\begin{center}
\begin{tikzpicture}[level distance=1.5cm,
  level 1/.style={sibling distance=1.5cm},
  every node/.style = {
  	shape=circle,
    draw,
    align=center,
    top color=white,
    bottom color=white
    }]
  \node {\( g \)}
    child {node { \( g_1 \) }}
    child {node { \( \cdots \) }}
    child {node { \( g_l \) }};
\end{tikzpicture}
\end{center}

For example, a circuit that encodes the function $f(a_1,a_2,a_3,a_4)=a_1a_2 +a_3a_4$ is 

\begin{center}
\begin{tikzpicture}[level distance=1.5cm,
  level 1/.style={sibling distance=3cm},
  level 2/.style={sibling distance=1.5cm},
  every node/.style = {
  	shape=circle,
    draw,
    align=center,
    top color=white,
    bottom color=white
    }
  ]
  \node { \( + \) }
    child { node { \( \times \) }
      child { node { \( a_1 \) } }
      child {node { \( a_2 \) } }
    }
    child { node { \( \times \) }
      child { node { \( a_3 \) } }
      child { node { \( a_4 \) } }
    };
\end{tikzpicture}
\end{center}

We can simulate the polynomial $x^2+xy$ by doing $f(A)$ where $A=[x \hspace{1ex} x \hspace{1ex} x \hspace{1ex} y]^*$. The main idea is to traverse the circuit top down in a depth first search way and store visited gates in a stack and its corresponding current values in another stack, and aggregate in the iterations according to the gate type.

For a stack $S$, the operations are standard:

\begin{itemize}
	\item $\push{S}{s}$: pushes $s$ into $S$.
	\item $\pop{S}$: pops the top element.
	\item $\getsize{S}$: the length of the stack.
	\item $\gettop{S}$: the top element in the stack.
\end{itemize}

For the pseudo-code, $\cG$ and $\cV$ denote stacks of gates and values, respectively. The property that holds during the simulation is that the value in $\cV[i]$ is the value that $\cG[i]$ currently outputs. The algorithm ends with $\cG=\left[ g_{\texttt{root}}\right]$ and $\cV=\left[ v_{\texttt{root}}\right]$ after traversing the circuit, and returns $v_{\texttt{root}}$

During the evaluation algorithm there will be two possible configurations of $\cG$ and $\cV$.

\begin{enumerate}
	\item $\getsize{\cG} = \getsize{\cV} + 1$: this means that $\gettop{\cG}$ is a gate that we visit for the first time and we need to initialize its value.
	
	\item $\getsize{\cG} = \getsize{\cV}$: here $\gettop{\cV}$ is the value of evaluating the circuit in gate $\gettop{\cG}$. Therefore, we need to aggregate the value $\gettop{\cV}$ to the parent gate of $g$.
\end{enumerate}

We assume the circuit has input gates, $+, \times$-gates and allow constant $1$-gate.

The idea is to traverse the circuit top down in a depth first search way. For example, in the circuit $f(a_1,a_2,a_3,a_4)=a_1a_2 +a_3a_4$ above, we would initialize the output gate value as $0$ because it is a $+$ gate, so $\cG=\lbrace +\rbrace$, $\cV=\lbrace 0\rbrace$. Then stack the left $\times$ gate to $\cG$, stack its initial value (i.e. $1$) to $\cV$. Now stack $a_1$ to $\cG$ and its value (i.e. $a_1$) to $\cV$. Since we are on an input gate we pop the gate and value pair off of $\cG$ and $\cV$ respectively, aggregate $a_1$ to $\gettop{\cV}$ and continue by stacking the $a_2$ gate to $\cG$. We pop $a_2$ off of $\cV$ (and its gate off of $\cG$) and aggregate its value to $\gettop{\cV}$. We pop and aggregate the value of the left $\times$ gate to $\gettop{\cV}$ (the root value). Then continue with the right $\times$ gate branch similarly.

For the pseudo-code, we supply ourselves with the following functions:

\begin{itemize}
	\item[--] $\isplus{g}$: true if and only if $g$ is a $+$-gate.
	\item[--] $\isprod{g}$: true if and only if $g$ is a $\times$-gate.
	\item[--] $\isone{g}$: true if and only if $g$ is a $1$-gate.
	\item[--] $\isinput{g}$: true if and only if $g$ is an input gate.
	\item[--] $\getfirst{g}$: outputs the first child of $g$.
	\item[--] $\getinput{g}$: outputs $A[i]$ when $g$ is the $i$-th input.
	\item[--] $\isnotlast{g_1}{g_2}$: true if and only if $g_2$ is not the last child gate of $g_1$.
	\item[--] $\nextgate{g_1}{g_2}$: outputs the next child gate of $g_1$ after $g_2$.
	\item[--] $\getroot$: outputs the root gate of the circuit.
\end{itemize}

The corresponding $\lbrace 0,1 \rbrace^n\rightarrow\lbrace 0,1 \rbrace^n$ functions are:

\begin{itemize}
	\item[--] $\isplus{g}$: $1$ if and only if $g$ is a $+$-gate.
	\item[--] $\isprod{g}$: $1$ if and only if $g$ is a $\times$-gate.
	\item[--] $\isone{g}$: $1$ if and only if $g$ is a $1$-gate.
	\item[--] $\isinput{g}$: $1$ if and only if $g$ is an input gate.
	\item[--] $\getfirst{g}$: outputs the $\texttt{id}$ of the first child of $g$.
	\item[--] $\getinput{g}$: outputs $e_i$ where the $i$-th input gate of $A$ is encoded by $g$.
	\item[--] $\isnotlast{g_1}{g_2}$: $1$ if and only if $g_2$ is not the last child gate of $g_1$.
	\item[--] $\nextgate{g_1}{g_2}$: outputs the $\texttt{id}$ of the next child gate of $g_1$ after $g_2$.
	\item[--] $\getroot$: outputs the $\texttt{id}$ of the root gate of the circuit.
\end{itemize}

The previous functions are all definable by an $L$-transducer and can be defined from the $L$-transducer of $f$. Then, by proposition \ref{prop:transducer}, for each of these functions there is a \langfor expression that simulates them.

Now, we give the pseudo-code of the top-down evaluation. We define the functions $Initialize$ (algorithm \ref{alg:init_code}), $Aggregate$ (algorithm \ref{alg:agg_code}) and $Evaluate$ (algorithm \ref{alg:eval_code}). The main algorithm is $Evaluate$.

\begin{algorithm}
\caption{Initialize (pseudo-code)}\label{alg:init_code}
\begin{algorithmic}[1]
\Function{Initialize}{$\cG, \cV, A$}\Comment{The stacks and input. Here, $\getsize{\cG} =  \getsize{\cV} + 1$}
	\If{$\isplus{\gettop{\cG}}$}
		\State $\push{\cV}{0}$
		\State $\push{\cG}{\getfirst{\gettop{\cG}}}$
	\ElsIf{$\isprod{\gettop{\cG}}$}
		\State $\push{\cV}{1}$
		\State $\push{\cG}{\getfirst{\gettop{\cG}}}$
	\ElsIf{$\isone{\gettop{\cG}}$}
		\State $\push{\cV}{1}$
	\ElsIf{$\isinput{\gettop{\cG}}$}
		\State $\push{\cV}{A\left[ \getinput{\gettop{\cG}} \right]}$
	\EndIf
	\State \textbf{return} $\cG, \cV$
\EndFunction
\end{algorithmic}
\end{algorithm}

\begin{algorithm}
\caption{Aggregate (pseudo-code)}\label{alg:agg_code}
\begin{algorithmic}[1]
\Function{Aggregate}{$\cG, \cV$}\Comment{Here, $\getsize{\cG} =  \getsize{\cV}$}
	\State $g = \pop{\cG}$
	\State $v = \pop{\cV}$
	\If{$\isplus{\gettop{\cG}}$}
		\State $\gettop{\cV} = \gettop{\cV} + v$
	\ElsIf{$\isprod{\gettop{\cG}}$}
		\State $\gettop{\cV} = \gettop{\cV} \cdot v$
	\EndIf
	\If{$\isnotlast{\gettop{\cG}}{g}$}
		\State $\push{\cG}{\nextgate{\gettop{\cG}}{g}}$
	\EndIf
	\State \textbf{return} $\cG, \cV$
\EndFunction
\end{algorithmic}
\end{algorithm}

\begin{algorithm}
\caption{Evaluate (pseudo-code)}\label{alg:eval_code}
\begin{algorithmic}[1]
\Function{Evaluate}{$A$}\Comment{Input $ n\times 1$ vector $A$. Here, $\cG$ and $\cV$ are empty}
	\State $\push{\cG}{\getroot}$
	\While{$\getsize{\cG}\neq 1$ or $\getsize{\cV}\neq 1$}
		\If{$\getsize{\cG}\neq \getsize{V}$}
			\State $(\cG,\cV) := \texttt{Initialize}(\cG,\cV,A)$
		\Else
			\State $(\cG,\cV):= \texttt{Aggregate}(\cG,\cV)$
		\EndIf
	\EndWhile
	\State \textbf{return} $\gettop{\cV}$
\EndFunction
\end{algorithmic}
\end{algorithm}

The $Evaluate$ algorithm gives us the output of the circuit. Note that after each iteration it either holds that $\getsize{\cG} =  \getsize{\cV} + 1$ or $\getsize{\cG} =  \getsize{\cV}$. Furthermore, when we start we have $\getsize{\cG}=1$ and $\getsize{\cV}=0$. The condition $\getsize{\cG}= 1$ and $\getsize{\cV}=1$ holds only when we have traversed all the circuit, and the value in $\gettop{\cV}$ is the value that the root of the circuit outputs after its computation.

Next, we show how to encode this algorithm in $\langfor$.

Let $n_0\in\mathbb{N}$ be big enough for $\star$ to hold and let $n\geq k$. Hence, the number of gates (values) is bounded by $n^k$ and we need $k\log (n)$ bits to encode the id of each gate.

To simulate the two stacks $\cG$ and $\cV$ we keep a matrix $X$ of dimensions $n \times n$.

\begin{itemize}
	\item Column $n$ will store a canonical vector that marks the top of stack $V$ (values).
	\item Column $n-1$ will store a canonical vector that marks the top of stack $G$ (gates).
	\item Column $n-2$ is the stack of values where $X[1, n-2]$ is the bottom of the stack.
	\item Columns $1$ to $n-3$ are the stack of gates.
\end{itemize}

If we have $j$ gates in the stack and currently $\getsize{\cG}=\getsize{\cV}$ then $X$ would look like:

\[
X = \begin{bmatrix}
    \texttt{id}_1 & v_1 & 0 & 0 \\
    \texttt{id}_2 & v_2 & 0 & 0 \\
    \vdots & \vdots & \vdots & \vdots \\
    \texttt{id}_j & v_j & 1 & 1 \\
    0 & 0 & 0 & 0 \\
    \vdots & \vdots & \vdots & \vdots \\
     0 & 0 & 0 & 0
\end{bmatrix}.
\]

Since $n\geq n_0$, $(\star)$ holds and thus we never use more than $n-3$ bits to encode an $\texttt{id}$ and $j\leq n$ given that we never keep more gates than the depth of the tree. As a consequence, we never keep more than $n$ values either.

We make a series of definitions to make the notation more clear.

Let $e_i$ be the $i$-th canonical vector. $S$ and $P$ denote the successor and predecessor matrices respectively, such that

\[
  			S\cdot e_i=\begin{cases}
               e_{i+1} \text{ if } i\leq n \\
               \mathbf{0} \text{ otherwise }
            \end{cases}
\]

\[
  			P\cdot e_i=\begin{cases}
               e_{i-1} \text{ if } i\geq n \\
               \mathbf{0} \text{ otherwise }
            \end{cases}
\]

We write $e_{min}$ for the first canonical vector and $e_{max}$ for the last canonical vector. For any $i$ we write 
\begin{align*}
	e_{min+i} &= S^i\cdot e_{min} \\
	e_{max+i} &= P^i\cdot e_{max}
\end{align*}

We use the extra $\lbrace 0,1 \rbrace^n\rightarrow\lbrace 0,1 \rbrace^n$ functions that have a $\langfor$ translation:

\[
  			min(e)=\begin{cases}
               1 \text{ if } e=e_{min} \\
               0 \text{ otherwise }
             \end{cases}
\]

\[
  			max(e)=\begin{cases}
               1 \text{ if } e=e_{max} \\
               0 \text{ otherwise }
             \end{cases}
\]

\[
  			less(e_i,e_j)=\begin{cases}
               1 \text{ if } i\leq j \\
               0 \text{ otherwise }
             \end{cases}
\]

When used in $\langfor$ these functions output $[0]$ and $[1]$.

Now 
\begin{align*}
	e_{V}&:=e_{max-2} \\
	e_{G_{top}}&:=e_{max-1} \\
	e_{V_{top}}&:=e_{max}
\end{align*}

For a canonical vector, let $$\Iden{e_i}:=\ssum v. less(v,e_i)\cdot (v\cdot v^*).$$ This matrix has ones in the diagonal up to position $i$ marked by $e_{i}$. We define the following sub-matrices of $X$:
\begin{align*}
	V_{top} &:= X\cdot e_{V_{top}} \\
	V &:= \Iden{V_{top}} \cdot X \cdot e_v \\
 	G_{top} &:=X\cdot e_{G_{top}} \\
 	G &:= \Iden{G_{top}}\cdot X \cdot \Iden{e_{max-3}}
\end{align*}

For example, if we are in a step where $\getsize{\cG}=\getsize{\cV} + 1$ then

\[
X = \begin{bmatrix}
    \texttt{id}_1 & v_1 & 0 & 0 \\
    \texttt{id}_2 & v_2 & 0 & 0 \\
    \vdots & \vdots & \vdots & \vdots \\
    \texttt{id}_{j-1} & v_{j-1} & 0 & 1 \\
    \texttt{id}_j & 0 & 1 & 0 \\
    0 & 0 & 0 & 0 \\
    \vdots & \vdots & \vdots & \vdots \\
     0 & 0 & 0 & 0
\end{bmatrix}, 
G = \begin{bmatrix}
    \texttt{id}_1  \\
    \texttt{id}_2 \\
    \vdots   \\
    \texttt{id}_{j-1} \\
    \texttt{id}_j \\
    0 \\
    \vdots \\
     0 
\end{bmatrix}, 
V = \begin{bmatrix}
    v_1  \\
    v_2 \\
    \vdots   \\
    v_{j-1} \\
    0 \\
    0 \\
    \vdots \\
     0 
\end{bmatrix}, 
G_{top} = \begin{bmatrix}
    0  \\
    0 \\
    \vdots   \\
    0 \\
    1 \\
    0 \\
    \vdots \\
     0 
\end{bmatrix}, 
V_{top} = \begin{bmatrix}
    0  \\
    0 \\
    \vdots   \\
    1 \\
    0 \\
    0 \\
    \vdots \\
     0 
\end{bmatrix}
\]

Here, $V$ is a vector encoding the stack of values in $X$ and $G$ is a matrix encoding the stack of gates in $X$. Note that what is \textit{over} the top of the stacks is always set to zero due to $\Iden{G_{top}}$ and $\Iden{V_{top}}$.

To set the initial state (algorithm \ref{alg:eval_code} line 2) we define the $\langfor$ expression: $$\text{START}:= e_{min}\cdot \getroot^* + e_{min}\cdot e_{G_{top}}^*.$$
For the initialize step, we define the $\langfor$ expressions: INIT${\_}$PLUS (algorithm \ref{alg:init_code}, lines 2, 3, 4), INIT${\_}$PROD (algorithm \ref{alg:init_code}, lines 5, 6, 7), CONST (algorithm \ref{alg:init_code}, lines 8, 9) and INPUT (algorithm \ref{alg:init_code}, lines 10, 11):

\begin{align*}
	\text{INIT{\_}PLUS} &:= \isplus{G^*\cdot G_{top}}\odot \left[ G + S\cdot G_{top} \cdot \getfirst{G^*\cdot G_{top}}^*  + S\cdot G_{top}\cdot e_{G_{top}}^* +V\cdot e_{V} + S\cdot V_{top}\cdot e_{V_{top}}^* \right] \\
	\text{INIT{\_}PROD} &:= \isprod{G^*\cdot G_{top}}\odot \left[ G + S\cdot G_{top} \cdot \getfirst{G^*\cdot G_{top}}^* + S\cdot G_{top}\cdot e_{G_{top}}^* +(V + S\cdot v_{top})\cdot e_{V} + S\cdot V_{top}\cdot e_{V_{top}}^* \right] \\
	\text{CONST} &:= \isone{G^*\cdot G_{top}}\odot \left[ G + (V + S\cdot v_{top})\cdot e_{V} + S\cdot V_{top}\cdot e_{V_{top}}^* \right] \\
	\text{INPUT} &:= \isinput{G^*\cdot G_{top}}\odot \left[ G + \left(V + \left( A^* \cdot \getinput{G^*\cdot G_{top}} \cdot S\cdot V_{top} \right)\right)\cdot e_{V} + S\cdot V_{top}\cdot e_{V_{top}}^* \right]
\end{align*} 

Here, $G^*\cdot G_{top}$ is to get the current id in the top of the stack. In INIT${\_}$PLUS we get the current stack $G$, we add $S\cdot G_{top} \cdot \getfirst{G^*\cdot G_{top}}^*$ which is an $n\times n$ matrix with the first child of $G^*\cdot G_{top}$ in the next row. Then $S\cdot G_{top}\cdot e_{G_{top}}^*$ adds $S\cdot G_{top}$ to the $n-1$ column to mark the gate we added as the top. Next, we do the same with the values by adding $V\cdot e_{V} + S\cdot V_{top}\cdot e_{V_{top}}^*$.

The $\langfor$ expression equivalent to algorithm \ref{alg:init_code} is $$\text{INIT}:=\text{INIT{\_}PLUS}+\text{INIT{\_}PROD}+\text{CONST}+\text{INPUT}.$$

The idea is to return the matrix for the next iteration. Recall that here $\getsize{\cG}=\getsize{\cV} + 1$. So, when the operation is INPUT or CONST, if we start with

\[
\begin{bmatrix}
    \texttt{id}_1 & v_1 & 0 & 0 \\
    \texttt{id}_2 & v_2 & 0 & 0 \\
    \vdots & \vdots & \vdots & \vdots \\
    \texttt{id}_{j-1} & v_{j-1} & 0 & 1 \\
    \texttt{id}_j & 0 & 1 & 0 \\
    0 & 0 & 0 & 0 \\
    \vdots & \vdots & \vdots & \vdots \\
     0 & 0 & 0 & 0
\end{bmatrix}, \text{ then we return }
\begin{bmatrix}
    \texttt{id}_1 & v_1 & 0 & 0 \\
    \texttt{id}_2 & v_2 & 0 & 0 \\
    \vdots & \vdots & \vdots & \vdots \\
    \texttt{id}_{j-1} & v_{j-1} & 0 & 0 \\
    \texttt{id}_j & v_j & 1 & 1 \\
    0 & 0 & 0 & 0 \\
    \vdots & \vdots & \vdots & \vdots \\
     0 & 0 & 0 & 0
\end{bmatrix}.
\]

When the operation is INIT{\_}PLUS or INIT{\_}PROD, if we start with 

\[
\begin{bmatrix}
    \texttt{id}_1 & v_1 & 0 & 0 \\
    \texttt{id}_2 & v_2 & 0 & 0 \\
    \vdots & \vdots & \vdots & \vdots \\
    \texttt{id}_{j-1} & v_{j-1} & 0 & 1 \\
    \texttt{id}_j & 0 & 1 & 0 \\
    0 & 0 & 0 & 0 \\
    0 & 0 & 0 & 0 \\
    \vdots & \vdots & \vdots & \vdots \\
     0 & 0 & 0 & 0
\end{bmatrix}, \text{ then we return }
\begin{bmatrix}
    \texttt{id}_1 & v_1 & 0 & 0 \\
    \texttt{id}_2 & v_2 & 0 & 0 \\
    \vdots & \vdots & \vdots & \vdots \\
    \texttt{id}_{j-1} & v_{j-1} & 0 & 0 \\
    \texttt{id}_j & v_j & 0 & 1 \\
    \texttt{id}_{j+1} & 0 & 1 & 0 \\
    0 & 0 & 0 & 0 \\
    \vdots & \vdots & \vdots & \vdots \\
     0 & 0 & 0 & 0
\end{bmatrix}.
\]


For the aggregate expression (algorithm \ref{alg:agg_code}) we do the following. Let $$\pondIden{e_i}{c}=\ssum v. (v^*\cdot e_i)\cdot c\cdot v\cdot v^* + (1-v^*\cdot e_i)\cdot v \cdot v^*,$$ namely, it is the identity with $c$ in position $(i,i)$.

We define the expressions: AGG${\_}$PLUS (algorithm \ref{alg:agg_code}, lines 4, 5), AGG${\_}$PROD (algorithm \ref{alg:agg_code}, lines 6, 7),  IS${\_}$NOT${\_}$LAST (algorithm \ref{alg:agg_code}, lines 8, 9), IS${\_}$LAST and POP:

\begin{align*}
	\text{POP} &:= \Iden{P\cdot G_{top}}\cdot G + P\cdot V_{top}\cdot e_{V_{top}}^*  \\
	\text{AGG{\_}PLUS} &:= \isplus{G^* \cdot \left( P \cdot G_{top}\right)} \odot \left[ \left( \Iden{P\cdot V_{top}} \cdot V + \left( V^* \cdot V_{top} \right)\left( P\cdot V_{top} \right)\right) \cdot e_{V}^* \right] \\
	\text{AGG{\_}PROD} &:= \isprod{G^* \cdot \left( P \cdot G_{top}\right)} \odot \left[ \left( \pondIden{P\cdot V_{top}}{V^* \cdot V_{top}} \cdot \Iden{P\cdot V_{top}} \cdot V \right) \cdot e_{V}^* \right] \\
	\text{IS{\_}NOT{\_}LAST} &:= \isnotlast{G^* \cdot \left( P \cdot G_{top}\right)}{G^* \cdot G_{top}} \odot \left[  G_{top} \cdot \nextgate{G^* \cdot \left( P\cdot G_{top} \right) }{G^* \cdot G_{top}} + G_{top}\cdot e_{G_{top}}^* \right] \\
	\text{IS{\_}LAST} &:= \left( 1 - \isnotlast{G^* \cdot \left( P \cdot G_{top}\right)}{G^* \cdot G_{top}} \right)\odot \left[ \left( P\cdot G_{top} \right) \cdot e_{G_{top}}^* \right]
\end{align*}

The $\langfor$ expression equivalent to algorithm \ref{alg:agg_code} is $$\text{AGG}:=\text{POP} + \text{AGG{\_}PLUS}+\text{AGG{\_}PROD}+\text{IS{\_}NOT{\_}LAST}+\text{IS{\_}LAST}.$$

The $Evaluate$ method (algorithm \ref{alg:eval_code}) is defined as follows:

\begin{align*}
	\text{EVAL}&[A]= \\
	&e_{min}^* \cdot \ffor{X}{v_1, \ldots, v_k}: \big\lbrace \\
	&\left( \sprod_{i=1}^k min(v_i)\right) \odot START + \\
	&\left( 1- \sprod_{i=1}^k min(v_i)\right) \odot \left( \left(1 - min(G_{top})\cdot min(V_{top}) \right) \odot \left[ \left( 1 - G_{top}^*\cdot V_{top} \right) \odot \text{INIT} + \left(  G_{top}^*\cdot V_{top} \right) \odot \text{AGG} \right] + min(G_{top})\odot min(V_{top})\odot X\right) \\ 
	&\big\rbrace \cdot e_{V}
\end{align*}

Note that the $ \texttt{for}$-expression does the evaluation. The final output is in $X[1,max-2]$, we extract this value by multiplying the final result as $e_{min}^*\cdot [\texttt{for}(\ldots )]\cdot e_{V}$.

Finally, we need to take care of all $n<n_0$, where $(\star)$ does not necessarily hold. For any $i$, let: $$\text{Eval}[i,A]:= \text{ the } 1\times 1 \text{ matrix with the value of the polynomial } f(A) \text{ when } n=i.$$

Then we define: $$f(A)=\ssum_{i=0}^{n_0-1}(e_{min+1}\cdot e_{max}^*)\odot \text{EVAL}[i,A] + \left( (S^{n_0}\cdot e_{min})^*\cdot \ones (e_{min}) \right)\odot \text{EVAL}[A].$$ Above, $(e_{min+1}\cdot e_{max}^*)$ checks if the dimension is equal to $i$, and $(S^{n_0}\cdot e_{min})^*\cdot \ones (e_{min})$ checks if the dimension is greater or equal than $n_0$.














\subsection{From MATLANG to uniform ACs}
We connect MATLANG with arithmetic circuits as follows.

We first show that for every MATLANG expression $e$ we can associate a uniform arithmetic circuit family $\{\Phi_n^e\mid n=1,2,\ldots\}$ such that when $I$ is matrix instance of dimension $n$ (or $n\times n$), we can obtain $e(I)$ by evaluating $\Phi_n$ on $I$. \floris{The connection between the dimensions of instances etc and the ``$n$'' in the arithmetic circuits needs to be made precise. E.g., if we have $n^2$ matrices, we need more than $n$ variables...}

For a circuit $\Phi$, $\Phi[i]$ denotes its $i$-th output gate. Also, when we write $a \oplus b$ we mean 

\begin{center}
\begin{tikzpicture}[level distance=1.5cm,
  level 1/.style={sibling distance=1.5cm},
  every node/.style = {
  	shape=circle,
    draw,
    align=center,
    top color=white,
    bottom color=white
    }]
  \node {\( + \)}
    child {node { \( a \) }}
    child {node { \( b \) }};
\end{tikzpicture}
\end{center}

When we write $\bigoplus_{l=1}^n a_l$ we mean 

\begin{center}
\begin{tikzpicture}[level distance=1.5cm,
  level 1/.style={sibling distance=1.5cm},
  every node/.style = {
  	shape=circle,
    draw,
    align=center,
    top color=white,
    bottom color=white
    }]
  \node {\( + \)}
    child {node { \( a_1 \) }}
    child {node { \( \cdots \) }}
    child {node { \( a_n \) }};
\end{tikzpicture}
\end{center}

Same with $\otimes$.

Let $e$ be a \langfor expression. If $e=V$ we have
\begin{itemize}
	\item When $V$ is $1\times 1$ then $\Phi^e_1$ has the one input/output gate (\textit{not necessary, covered in second item}).
	\item When $V$ is $n\times 1$ or $1\times n$ then $\Phi^e_n$ has $n$ input/output gates.
	\item When $V$ is $n\times n$ then $\Phi^e_n$ has $n^2$ input/output gates. Here, $V_{ij}=\Phi_n(V)\left[ j+n(i-1)\right]$ and $i,j=1,\ldots, n$ (entries listed row by row).
\end{itemize}

If $e=e'^T$ then $\Phi^e_n=\Phi^{e'}_n$. 

If $e=e_1 + e_2$ we have

\begin{itemize}
	\item When $e$ is $1\times 1$ then $\Phi^e_1$ is $\Phi^{e_1}_1 \oplus \Phi^{e_2}_1$.
	\item When $e$ is $n\times 1$ or $1\times n$ then $\Phi^e_n$ has $n$ output gates, where gate $k$ is $\Phi^{e_1}_n[k] \oplus \Phi^{e_2}_n[k]$.
	\item When $V$ is $n\times n$ then $\Phi^e_1$ has $n^2$ output gates, where gate $k$ is $\Phi^{e_1}_n[k] \oplus \Phi^{e_2}_n[k]$.
\end{itemize}

If $e=f(e_1, \ldots, e_k)$ we have

\begin{itemize}
	\item When $e$ is $1\times 1$ (only case necessary) then $\Phi^e_1$ is 
	
\begin{center}
\begin{tikzpicture}[level distance=1.5cm,
  level 1/.style={sibling distance=1.5cm},
  every node/.style = {
  	shape=circle,
    draw,
    align=center,
    top color=white,
    bottom color=white
    }]
  \node {\( f \)}
    child {node { \( \Phi^{e_1}_1 \) }}
    child {node { \( \cdots \) }}
    child {node { \( \Phi^{e_k}_1 \) }};
\end{tikzpicture}
\end{center}

\end{itemize}

If $e=e_1\cdot e_2$ we have

\begin{itemize}
	\item When $e_1,e_2$ are $1\times 1$ then $\Phi^e_1$ is $\Phi^{e_1}_1 \otimes \Phi^{e_2}_1$.
	\item When $e_1$ is $1\times 1$ and $e_2$ is $1\times n$ then $\Phi^e_n$ has $n$ output gates, where output gate $i$ is $\Phi^{e_1}_1 \otimes \Phi^{e_2}_n[i]$.
	\item When $e_1$ is $n\times 1$ and $e_2$ is $1\times 1$ then $\Phi^e_n$ has $n$ output gates, where output gate $i$ is $\Phi^{e_1}_n[i] \otimes \Phi^{e_2}_1$.
	\item When $e_1$ is $n\times 1$ and $e_2$ is $1\times n$ then $\Phi^e_n$ has $n^2$ output gates, where output gate $k=j + n(i-1)$ is $\Phi^{e_1}_n[i] \otimes \Phi^{e_2}_n[j]$. Note that $k=1,\ldots, n^2$ and $i,j=1,\ldots, n$.
	\item When $e_1$ is $1\times n$ and $e_2$ is $n\times 1$ then $\Phi^e_n$ has one output gate $$\bigoplus_{l=1}^n \left( \Phi^{e_1}_n[l] \otimes \Phi^{e_2}_n[l] \right).$$
	\item When $e_1$ is $1\times n$ and $e_2$ is $n\times n$ then $\Phi^e_n$ has $n$ output gates, where gate $j$ is $$\bigoplus_{i=1}^n \left( \Phi^{e_1}_n[j] \otimes \Phi^{e_2}_n[j + n(i-1)] \right).$$
	\item When $e_1$ is $n\times n$ and $e_2$ is $n\times 1$ then $\Phi^e_n$ has $n$ output gates, where gate $i$ is $$\bigoplus_{j=1}^n \left( \Phi^{e_1}_n[j+n(i-1)] \otimes \Phi^{e_2}_n[j] \right).$$
	\item When $e_1$ is $n\times n$ and $e_2$ is $n\times n$ then $\Phi^e_n$ has $n^2$ output gates, where gate $k=j+n(i-1)$ is $$\bigoplus_{l=1}^n \left( \Phi^{e_1}_n[i] \otimes \Phi^{e_2}_n[j+n(l-1)] \right).$$ Note that $k=1,\ldots, n^2$ and $i,j=1,\ldots, n$.
\end{itemize}

If $e=\ffor{X}{v}e'(\cI, X, v)$ then $$\Phi^{e}_n=\Phi^{e'}_n\left( \cI, \Phi^{e'}_n \left( \cI, \cdots \Phi^{e'}_n\left( \cI, \Phi^{e'}_n\left( \cI, 0, v_1\right), v_2\right)\cdots, v_{n-1} \right), v_n \right).$$

Note that every circuit adds a constant number of layers except when $e=\ffor{X}{v}e'(\cI, X, v)$. This means that the depth still is polynomial. When $e=\ffor{X}{v}e'(\cI, X, v)$ we have that the depth of the circuit is $n\cdot p(n)$, where the depth of $e'(\cI, X, v)$ is $p(n)$, so it also remains polynomial.

As a consequence, blah blah....

In general, uniform arithmetic circuit family $\{\Phi_n^e\mid n=1,2,\ldots\}$ is not necessarily of polynomial degree. Indeed, consider it suffices to consider the MATLANG expression which computes
$f_n(x)=x^{2^n}$. 
As uniform arithmetic circuit families of polynomial degree have nice properties, e.g., they can be assumed to be of logarithmic depth, we next want to zoom in such families. In what follows we therefore limit ourselves to MATLANG expressions $e$ such that $\{\Phi_n^e\mid n=1,2,\ldots\}$ is a family of polynomial degree arithmetic circuits.

\begin{itemize}
	\item Checking whether a MATLANG expression $e$ corresponds to a family of of polynomial degree arithmetic circuits is undecidable. 
\end{itemize}

Let $e$ be a ``nice'' MATLANG expression, i.e.,  $\{\Phi_n^e\mid n=1,2,\ldots\}$ is a family of polynomial degree arithmetic circuits which can be assumed to be logarithmic depth. 
% \begin{itemize}
% 	\item There exists a MATLANG expression that can
% \end{itemize}

\subsection{Undecidability}
%!TEX root = /Users/fgeerts/Documents/MLforloops/pods/main.tex
\newtheorem*{Undec}{Proposition~\ref{prop-undec}}
Let $e$ be a \langfor expression over a matrix schema $\mathcal{S}=(\mathcal{M},\textsf{size})$ and let $V_1,\ldots, V_k$ be
the variables of $e$, each of type $(\alpha,\alpha)$, $(1,\alpha)$, $(\alpha,1)$ or $(1,1)$. We know from Theorem~\ref{th-ml-to-circuits}
that there exists a uniform arithmetic circuit family $\{\Phi_n \mid n=1,2,\ldots\}$
such that $\sem{e}(\I)=\Phi_n(A_1,\ldots,A_k)$ for any instance $\I$ such that
$\mathcal{D}(\alpha)=n$ and $\conc(V_i)=A_i$ for $i=1,\ldots,k$. We are interested in deciding
whether there exists such a  uniform arithmetic circuit family $\{\Phi_n \mid n=1,2,\ldots\}$
of polynomial degree, i.e., such that $\mathsf{degree}(\Phi_n)=\mathcal{O}(p(n))$ for some polynomial $p(x)$. If such a circuit family exists, we call $e$ of polynomial degree.

{\bf IN PROGRESS}

\begin{Undec}
	Given a \langfor expression $e$ over a schema $\Sch$, it is undecidable to check whether $e$ is of polynomial degree.
\end{Undec}
\begin{proof}
We show undecidability based on the following undecidable problem:
$$
\{ \langle M\rangle\mid \text{$M$ is a deterministic TM which halts on the empty input}\}
$$
Consider a TM $M$ described by $(Q,\Gamma=\{0,1\},q_0,q_m,\Delta)$
with $Q=\{q_1,\ldots,q_m\}$ its states, $q_1$ being the initial state and $q_m$ being
the halting state, $\Gamma$ is the tape alphabet, and $\Delta$ is a transition function
from $Q\times \Gamma\to Q\times\Gamma\times \{\leftarrow,\sqcup,\rightarrow\}$. The simulation
of linear space TM, as given in the proof of Proposition~\ref{prop:transducer} can be easily modified to
any TM $M$ provided that we limit the execution of $M$ to exactly $n$ steps. Let $e_M$ denote this expression. Similarly
as in the linear space TM simulation, we have relation $Q_1,\ldots,Q_m$ encoding the
states, a single relation $T$ encoding the tape and relation $H_T$ encoding the position
of the tape. We note that we do not have any input and our tape has length $n$ when the matrix variables are assigned $n$ as size. By contrast to the linear space 
simulation, we also use a single vector $v$ (instead of $k$ such vectors) 
to simulate $n$ steps of $M$. We modify the expression such that it returns $1$ if $M$
halts in at most $n$ steps, and $0$ if $M$ did not halt yet after $n$ steps.

As a consequence, when $M$ does not halt, $e_M$ will always return $0$
for any $n$. When $M$ halts, there will be an $n$ such that $e_M$ returns $1$.
It now suffices to consider the matlang expression
$$
d_M:=e_M\cdot e_{\mathsf{exp}}
$$
where $e_{\mathsf{exp}}$ computes the functions $x^{2^n}$. 
\floris{this expression needs to be detailed....}
Then, when $M$ does not halt we can clearly compute $d_M$ with a constant circuit ``0''
for any $n$, otherwise, the circuit needed will be of exponential degree
for at least one $n$. 
\floris{Is the latter clear?}
\end{proof}


%
% 3/ Based on e_M(n) we now consider the ML expression
%
%  d_M(n):=e_M(n) Pow(n)
%
% with Pow(n)=2^n (or any other ML expression which cannot
% be computed by a poly degree circuit). Then, d_M(n) will be of
% poly degree if M does not halt in n steps. And since we
% want to know whether it is poly degree for all n (uniform circuits)
% M should not halt. Hence, the poly degree problem cannot be decidable.



\section{Proofs of Section~\ref{sec:restrict}}

\subsection{From \langsum to ARA}
% !TeX spellcheck = en_US
%!TEX root = ../main.tex
\newtheorem*{SUMTOARA}{Proposition~\ref{prop:sum_to_ara}}

We prove proposition \ref{prop:sum_to_ara}.

\begin{SUMTOARA}
  For each \langsum expression $e$ over schema $\Sch$ such that $\Sch(e)=(\alpha,\beta)$ with $\alpha\neq 1\neq\beta$, there exists a \rak  expression $\arae(e)$ over relational schema $\text{Rel}(\Sch)$ such that $\text{Rel}(\Sch)(\arae(e))=\{\row_\alpha,\row_\beta\}$ and 
	such that for any instance $\I$ over~$\Sch$,
	$$
	\sem{e}{\I}_{i,j}=\ssem{\arae(e)}{\text{Rel}(\I)}(t)
	$$
	for tuple $t(\mathrm{row}_\alpha)=i$ and $t(\mathrm{col}_\beta)=j$. Similarly for when $e$ has schema $\Sch(e)=(\alpha,1)$, $\Sch(e)=(1,\beta)$ or $\Sch(e)=(1,1)$, then $\arae(e)$ has schema $\text{Rel}(\Sch)(\arae(e))=\{\mathrm{row}_\alpha\}$,
	$\text{Rel}(\Sch)(\arae(e))=\{\mathrm{col}_\alpha\}$, or
	$\text{Rel}(\Sch)(\arae(e))=\{\}$, respectively.
\end{SUMTOARA}

\begin{proof}
We start from a matrix schema $\Sch=(\Mnam,\size)$, where $\Mnam\subset \Mvar$ is a finite set of matrix variables, 
and $\size: \Mvar \mapsto \DD\times \DD$ is a function that maps each matrix variable to a pair of size symbols. 
On the relational side we have for each size symbol $\alpha\in\DD\setminus\{1\}$, attributes $\alpha$, $\row_\alpha$, 
and $\col_\alpha$ in $\att$. We also reserve some special attributes $\gamma_1,\gamma_2,\ldots$ whose role will become clear shortly.
% We will treat matrix variables that are used to iterate over canonical vectors differently from other matrix variables. To make the distinction clear we use capital $V\in \Mnam $ to indicate ``normal'' matrix variables and lower case $v_i\in\Mnam$ for the matrix variables used for iteration in for-loops.
For each $V\in\Mnam$ and $\alpha \in \DD$ we denote
by $R_V$ and $R_\alpha$ its corresponding relation name, respectively. 

Then, given $\Sch$ we define the relational 
schema $\text{Rel}(\Sch)$ such that $\fdom(\text{Rel}(\Sch)) =  \{R_\alpha \mid \alpha\in\DD\} \cup \{R_V \mid V \in \Mnam\}$
% \cup \{R_v \mid v \in \Mnam\}$
where $\text{Rel}(\Sch)(R_\alpha) = \{\alpha\}$ and for all $V\in\Mnam$:
\[
\text{Rel}(\Sch)(R_V) = \begin{cases}
\lbrace\row_\alpha,\col_\beta \rbrace & \text{ if $ \size(V)=(\alpha,\beta)$} \\
\lbrace\row_\alpha \rbrace & \text{ if $ \size(V)=(\alpha,1)$} \\
\lbrace\col_\beta \rbrace  &
\text{ if $ \size(V)=(1,\beta)$} \\
\lbrace\rbrace & \text{ if $\size(V)=(1,1)$}.
\end{cases}
\]
% Furthermore, $\text{Rel}(\Sch)(R_{v_i})=\lbrace \row_\alpha,\col_\beta\rbrace$ if
% $\size(v_i)=(\alpha,1)$ and $\text{Rel}(\Sch)(R_{v_i})=\lbrace \rbrace$ if $\size(v_i)=(1,1)$.
% For each $V\in\Mnam$ and $\alpha \in \DD$ we denote
% by $R_V$ and $R_\alpha$ its corresponding relation name, respectively. Then, given $\Sch$ we define the relational
% schema $\text{Rel}(\Sch)$ such that $\fdom(\text{Rel}(\Sch)) =  \{R_\alpha \mid \alpha\in\DD\} \cup \{R_V \mid V \in \Mnam\}$
% where $\text{Rel}(\Sch)(R_\alpha) = \{\alpha\}$ and:
% \[
% \text{Rel}(\Sch)(R_V) = \begin{cases}
% \lbrace\row_\alpha,\col_\beta \rbrace & \text{ if $ \size(V)=(\alpha,\beta)$} \\
% \lbrace\row_\alpha \rbrace & \text{ if $ \size(V)=(\alpha,1)$} \\
% \lbrace\col_\beta \rbrace  &
% \text{ if $ \size(V)=(1,\beta)$} \\
% \lbrace\rbrace & \text{ if $\size(V)=(1,1)$}.
% \end{cases}
% \]

Next, for a matrix instance $\I = (\dom,\conc)$ over $\Sch$,
let $V\in\Mnam$ with $\size(V)=(\alpha,\beta)$ and let $\conc(V)$ be its corresponding $K$-matrix of dimension $\dom(\alpha)\times \dom(\beta)$.
The $K$-instance in $\mathsf{RA}_{K}^+$ according to $\I$ is $\text{Rel}(\I)$ with data domain $\ddom = \mathbb{N} \setminus \{0\}$. For each $V\in\Mnam$ we define 
$R_V^{\text{Rel}(\I)}(t):=\conc(V)_{ij}$ whenever $t(\row_\alpha) = i \leq \dom(\alpha)$ and $t(\col_\beta) = j \leq \dom(\beta)$, and $\kzero$ otherwise. 
Also, for each $\alpha \in \DD$ we define $R_\alpha^{\text{Rel}(\I)}(t):=\kone$ whenever $t(\alpha) \leq \dom(\alpha)$, and $\kzero$ otherwise.
If $\size(V)=(\alpha,1)$ then $R_V^{\text{Rel}(\I)}(t):=\conc(V)_{i1}$ whenever $t(\row_\alpha) = i \leq \dom(\alpha)$ and $\kzero$ otherwise.
Similarly, if $\size(V)=(1,\beta)$ then $R_V^{\text{Rel}(\I)}(t):=\conc(V)_{1j}$ whenever $t(\col_\beta) = j \leq \dom(\beta)$ and $\kzero$ otherwise.
If $\size(V)=(1,1)$ then $R_V^{\text{Rel}(\I)}(()):=\conc(V)_{11}$.

% When it comes to ``iterator'' variables $v_i\in\Mnam$, we define
% $R_{v_i}^{\text{Rel}(\I)}(t):=\kone$  whenever $t(\row_\alpha) = t(\col_\alpha)$ and
% $R_{v_i}^{\text{Rel}(\I)}(t):=\kzero$  whenever $t(\row_\alpha)=i\neq t(\col_\alpha)=j$
% with $t(\row_\alpha)=i$ and $t(\col_\alpha) = j$ with $i,j\leq \dom(\alpha)$ when $\size(v_i)=(\alpha,1)$, and $R_{v_i}^{\text{Rel}(\I)}(()):=\kone$ when $\size(v_i)=(1,1)$.
% In other words, we instantiate iterator variables with the identity matrix of appropriate dimensions.

Let $e$ be a \langsum expression. In the following we need to distinguish between matrix variables $v$
that occur in $e$ as part of a sub-expression $\ssum v. (\cdot)$, i.e., those variables that will be used to iterate over by means of canonical vectors, and those that are not. To make this distinction clear, we use $v_1,v_2,\ldots$ for those ``iterator'' variables, and capital $V$ for the other variables occurring in $e$. For simplicity, we assume that each occurrence of $\ssum$ has its own iterator variable associated with it. 

We define free (iterator) variables, as follows.
$\mathsf{free}(V):=\emptyset$, $\mathsf{free}(v):=\{v\}$, $\mathsf{free}(e^T):=\mathsf{free}(e)$, $\mathsf{free}(e_1+e_2):=\mathsf{free}(e_1)\cup \mathsf{free}(e_2)$, $\mathsf{free}(e_1\cdot e_2):=\mathsf{free}(e_1)\cup \mathsf{free}(e_2)$,
 $\mathsf{free}(f_\odot(e_1,\ldots,e_k)):=\mathsf{free}(e_1)\cup\cdots \cup \mathsf{free}(e_k)$, and $\mathsf{free}(e=\ssum V. e_1)=\mathsf{free}(e_1)\setminus\{v\}$. We will explicitly denote the free variables in an expression $e$ by writing $e(v_1,\ldots,v_k)$.

We now use the following induction hypotheses:
\begin{itemize}
	\item If $e(v_1,\ldots,v_k)$ is of type $(\alpha,\beta)$ then there exists a
	\rak expression $\arae$ such that $\text{Rel}(\Sch)(\arae(e))=\{\row_\alpha,\col_\beta,\gamma_1,\ldots,\gamma_k\}$
	and such that 
	$$
	\ssem{\arae(e)}{\text{Rel}(\I)}(t)=\sem{e}{\I[v_1\gets b_{i_1},\ldots,v_k\gets b_{i_k}]}_{i,j}
	$$
	for tuple $t(\mathrm{row}_\alpha)=i$, $t(\mathrm{col}_\beta)=j$ and $t(\gamma_s)=i_s$ for $s=1,\ldots, k$.
	\item If $e(v_1,\ldots,v_k)$ is of type $(\alpha,1)$ then there exists a
	\rak expression $\arae$ such that $\text{Rel}(\Sch)(\arae(e))=\{\row_\alpha,\gamma_1,\ldots,\gamma_k\}$
	and such that 
	$$
	\ssem{\arae(e)}{\text{Rel}(\I)}(t)=\sem{e}{\I[v_1\gets b_{i_1},\ldots,v_k\gets b_{i_k}]}_{i,1}
	$$
	for tuple $t(\mathrm{row}_\alpha)=i$,  and $t(\gamma_s)=i_s$ for $s=1,\ldots, k$.
	And similarly for when $e$ is type $(1,\beta)$.
	\item If $e(v_1,\ldots,v_k)$ is of type $(1,1)$ then there exists a
	\rak expression $\arae$ such that $\text{Rel}(\Sch)(\arae(e))=\{\gamma_1,\ldots,\gamma_k\}$
	and such that 
	$$
	\ssem{\arae(e)}{\text{Rel}(\I)}(t)=\sem{e}{\I[v_1\gets b_{i_1},\ldots,v_k\gets b_{i_k}]}_{1,1}
	$$
	for tuple $t(\gamma_s)=i_s$ for $s=1,\ldots, k$.
\end{itemize}
Clearly, this suffices to show the proposition since we there consider expressions $e$ for which $\mathsf{free}(e)=\emptyset$, in which case the above statements reduce to the one given in the proposition.


The proof is by induction on the structure of \langsum expressions. In line with the simplifications in Section~\ref{app:simp}, it suffices to consider pointwise function application with $f_\odot$ instead of scalar multiplication. (We also note that we can express the one-vector operator in \langsum, so scalar multiplication can be expressed using $f_\odot$ in \langsum).

Let $e$ be a \langsum expression.
\begin{itemize}
  \item If $e=V$ then $\arae (e):=R_V$.
  \item If $e=v_p$ then $\arae (e):=\sigma_{\lbrace \row_\alpha,\gamma_p\rbrace}\bigl(\rho_{\row_\alpha\to \alpha}(R_\alpha)\bowtie \rho_{\gamma_p\to \alpha}(R_\alpha)\bigr)$ when  $v_p$ is of type $(\alpha,1)$. It is here that we introduce the attribute $\gamma_p$ associated with iterator variable $v_p$.
 We note that 
$$ \ssem{\arae(v_p)}{\text{Rel}(\I)}(t)=\sem{v_p}{\I[v_p\gets b_{j}]}_{i,1}=(b_j)_{i,1}
$$
for $t(\mathrm{row}_\alpha)=i$ and $t[\gamma_p]=j$. Indeed, $(b_j)_{i,1}=\kone$ if $j=i$
and this holds when $t(\mathrm{row}_\alpha)=t[\gamma_p]=j$, and $(b_j)_{i,1}=\kzero$ if $j\neq i$
and this also holds when $t(\mathrm{row}_\alpha)\neq t[\gamma_p]=j$.

    % $\arae (e):=R_V^{\text{Rel}(\I)}$.

  \item If $e(v_1,\ldots,v_k)=(e_1(v_1,\ldots,v_k))^T$ with $\Sch (e_1)=(\alpha, \beta)$ then \[
\arae(e) :=
\begin{cases}
\rho_{\mathrm{row}_\alpha \to \mathrm{col}_\alpha,\mathrm{col}_\beta \to \mathrm{row}_\beta}\bigl(\arae(e_1)\bigr) & \text{if } \alpha \neq 1 \neq \beta; \cr
\rho_{\mathrm{row}_\alpha \to \mathrm{col}_\alpha}\bigl(\arae(e_1)\bigr) & \text{if } \alpha \neq 1 = \beta; \cr
\rho_{\mathrm{col}_\beta \to \mathrm{row}_\beta}\bigl(\arae(e_1)\bigr) & \text{if } \alpha = 1 \neq \beta; \cr
\arae(e_1) & \text{if } \alpha = 1 = \beta.
\end{cases}
\]
 % \item If $e:=e_{\ones}(e')$ where $\Sch(e')=(\alpha, \beta)$ and $e_{\ones}(\cdot)$ is the $\mathsf{ones}$ operator
 %  then $\arae(e):=\rho_{\alpha\rightarrow\row_\alpha}(\mathsf{Eq}_{\alpha})$
\floris{There is an issue here since $e_1$ and $e_2$ can have different free iterator variables. I think we can ensure that both have the same by introducing them somehow, or alternative, since all operators are linear, by pushing $+$ all the way down? Not sure.}
\item If $e=e_1(v_1,\ldots,v_k)+e_2(v_1,\ldots,v_k)$ with $\Sch (e_1)=\Sch (e_2)=(\alpha, \beta)$ then $\arae (e):=\arae (e_1)\cup \arae (e_2)$. We assume here that $e_1$ and $e_2$ have the same free variables. This is without loss of generality. Indeed, as an example, suppose that we have $e_1(v_1,v_2)$
and $e_2(v_2,v_3)$. Then, we can replace $e_1$ by  $e_1(v_1,v_2,v_3)=(v_3^T\cdot v_3)\times e_1(v_1,v_2)$
and similarly, $e_2$ by $e_2(v_1,v_2,v_3)=(v_1^T\cdot v_1)\times e_2(v_2,v_3)$, where in addition we replace scalar multiplication with its simulation using $f_{\odot}$ and the ones vector, as mentioned earlier. 
	% \item If $e=e_1(v_{1}},\ldots,v_{i_p})+e_2(v_{j_1},\ldots,v_{j_q})$ with $\Sch (e_1)=\Sch (e_2)=(\alpha, \beta)$
	% where $\text{Rel}(\Sch)(\arae(e_1))=\lbrace\row_\alpha,\col_\beta, \gamma_{i_{1}}, \ldots, \gamma_{i_{p}} \rbrace$
	% and $\text{Rel}(\Sch)(\arae(e_2))=\lbrace\row_\alpha,\col_\beta, \gamma_{j_{1}}, \ldots, \gamma_{j_{q}} \rbrace$
	% then, if $t[X]$ is tuple $t$ restricted to set $X$, we define
	% $$
	% \ssem{\arae(e)}{\text{Rel}(\I)}(t)=\ssem{\arae(e_1)}{\text{Rel}(\I)}(t[\text{Rel}(\Sch)(\arae(e_1))]) + \ssem{\arae(e_2)}{\text{Rel}(\I)}(t[\text{Rel}(\Sch)(\arae(e_2))]).
	% $$
	% Here $\text{Rel}(\Sch)(\arae(e))=\lbrace\row_\alpha,\col_\beta, \gamma_{i_{1}}, \ldots, \gamma_{i_{p}}, \gamma_{j_{1}}, \ldots, \gamma_{j_{q}} \rbrace$ (if one $\gamma_{k}$ is repited, we leave just one).
	%
  \item If $e=f_\odot(e_1,\ldots, e_k)$ with $\Sch(e_i)=\Sch(e_j)$ for all $i,j\in[1,k]$, then $\arae(e):=\arae(e_1)\Join \cdots \Join\arae(e_k)$.

  \item If $e=e_1\cdot e_2$ with $\Sch (e_1)=(\alpha, \gamma)$ and $\Sch (e_2)=(\gamma, \beta)$, we have two cases. If $\gamma = 1$ then $\arae (e):=\arae (e_1)\Join \arae (e_2)$.
If $\gamma\neq 1$ then
$$
\arae (e) := \pi_{\lbrace \row_{\alpha},\col_{\beta}, \gamma_1,\ldots,\gamma_k \rbrace}\left(\rho_{C\to \col_\gamma}(\arae (e_1))\Join \rho_{C\to \row_\gamma}(\arae (e_2)) \right),
$$
when $\text{Rel}(\Sch)(\arae(e_1))=\{\row_\alpha,\col_\gamma,\gamma_1',\ldots,\gamma_{\ell}'\}$,
$\text{Rel}(\Sch)(\arae(e_2))=\{\row_\gamma,\col_\beta,\gamma_1'',\ldots,\gamma_{\ell}''\}$ and $\{\gamma_1,\ldots,\gamma_k\}=\{\gamma_1',\ldots,\gamma_k',\gamma_1'',\ldots,\gamma_m''\}$.


%   \item If $e=\ssum V. e_1$ where $\Sch(e_1)=(\alpha,\beta)$ and $\Sch(V)=(\gamma,1)$. Then we can just do $\pi_{\text{Rel}(\Sch)(R_V)}\left( \arae (e_1) \right)$.
% Note that when the $\row_\gamma$ attribute in $\arae (e_1)$ is instantiated with with tuple $t$ such that $t(\row_\alpha)=i$, $t(\col_\beta)=j$ and $t(\row_\gamma)=k$,
% then the expression evaluates to $e_1(\I [V\leftarrow e_{k}^{\dom(\gamma)}])_{ij}$. 
% Hence, by projecting in attributes $\lbrace \row_\alpha, \col_\beta\rbrace$ we range over all $k$ and sum up all $K$-values for each entry.
  \item If $e(v_1,\ldots,v_{p-1},v_{p+1},\ldots,v_k)=\ssum v_p. e_1(v_1,\ldots,v_k)$ where $\Sch(e_1)=(\alpha,\beta)$ and $\Sch(V)=(\gamma,1)$. Then we do 
  $$
  \arae (e):=\pi_{\text{Rel}(\Sch)(\arae(e_1))\setminus\{\gamma_p\}} \arae (e_1).
  $$
 Indeed, by induction we know that 
 $$
\ssem{\arae(e_1)}{\text{Rel}(\I)}(t)=\sem{e}{\I[v_1\gets b_{i_1},\ldots,v_k\gets b_{i_k}]}_{i,j}
$$
for tuple $t(\mathrm{row}_\alpha)=i$, $t(\mathrm{col}_\beta)=j$ and $t(\gamma_s)=i_s$ for $s=1,\ldots, k$.
Hence, for $t(\mathrm{row}_\alpha)=i$, $t(\mathrm{col}_\beta)=j$ and $t(\gamma_s)=i_s$ for $s=1,\ldots, k$ and $s\neq p$,
$$
\ssem{\arae(e_1)}{\text{Rel}(\I)}(t):=\bigksum_{i_p=1,\ldots,\dom(\gamma)} \sem{e_1}{\I[v_1\gets b_{i_1},\ldots,v_k\gets b_{i_k}]}_{i,j},$$
which precisely corresponds to 
$$
\sem{\ssum v_p. e_1(v_1,\ldots,v_k)}{\I[v_1\gets b_{i_1},\ldots,v_{p-1}\gets b_{p-1},v_{p+1}\gets b_{p+1},\ldots,v_k\gets b_k]}_{i,j}.
$$
 
 %  % $$
 %  % \pi_{\lbrace \row_{\alpha}, \col_{\beta} \rbrace}\left( \arae (e_1)\left[ R_{V}^{\text{Rel}(\I)}\leftarrow R_{\mathsf{Id}}^{\gamma} \right] \right).
 %  % $$
 %  $$
 %  \pi_{\lbrace \row_{\alpha}, \col_{\beta} \rbrace}\left( \arae (e_1)\left[ R_{V}\leftarrow R_{\mathsf{Id}}^{\gamma} \right] \right).
 %  $$
 %
 %
 %  Here $R_{\mathsf{Id}}^{\gamma}$ is an expression that, when evaluated, results identity of size $\dom(\gamma)\times\dom(\gamma)$ and which
 %  is defined as
 %  % $$
 % %  R_{\mathsf{Id}}^{\gamma}:= \sigma_{\lbrace \row_{\gamma},\col_{\gamma} \rbrace }\left( \rho_{\gamma\rightarrow\row_\gamma}\left( R_\gamma^{\text{Rel}(\I)} \right) \Join \rho_{\gamma\rightarrow\col_\gamma}\left( R_\gamma^{\text{Rel}(\I)} \right)\right)
 % %  $$
 %  $$
 %  R_{\mathsf{Id}}^{\gamma}:= \sigma_{\lbrace \row_{\gamma},\col_{\gamma} \rbrace }\left( \rho_{\gamma\rightarrow\row_\gamma}\left( R_\gamma \right) \Join \rho_{\gamma\rightarrow\col_\gamma}\left( R_\gamma\right)\right).
 %  $$
 %  Furthermore, $\arae (e_1)\left[ R_{V}\leftarrow R_{\mathsf{Id}}^{\gamma} \right]$
 %  means that every occurrence of $R_V$ in $e_1$ is replaced by  $R_{\mathsf{Id}}^{\gamma}$.
 %
 %  Note that when $\arae (e_1)\left[ R_{V}\leftarrow R_{\mathsf{Id}}^{\gamma} \right]$ is instantiated with with tuple $t$ such that $t(\row_\alpha)=i$, $t(\col_\beta)=j$,
 %  $t(\row_\gamma)=k$ and $t(\col_\gamma)=k$ (other values of are zero),
 %  then the expression evaluates to $e_1(\I [V\leftarrow e_{k}^{\dom(\gamma)}])_{ij}$.
 %  Hence, by projecting in attributes $\lbrace \row_\alpha, \col_\beta\rbrace$ we range over all $k$ and sum up all $K$-values for each entry.
\end{itemize}
All other cases, when expressions have type $(\alpha,1)$, $(1,\beta)$ or $(1,1)$ can be dealt with in a similar way.
% Here, we only allow scalar $\kprod$ function application,
% which is equivalent to scalar product, as mentioned in section \ref{app:simp}.
% Note that we can access the zero values of matrices because the indexes are in the active domain of $\text{Rel}(\I)$ due to the existence of $R_\alpha^{\text{Rel}(\I)}(t)$, for each $\alpha \in \DD$.
\end{proof}

%
% \begin{proof}
% We start from a matrix schema $\Sch=(\Mnam,\size)$, where $\Mnam\subset \Mvar$ is a finite set of matrix variables,
% and $\size: \Mnam \mapsto \DD\times \DD$ is a function that maps each matrix variable to a pair of size symbols.
% On the relational side we have for each size symbol $\alpha\in\DD\setminus\{1\}$, attributes $\alpha$, $\row_\alpha$,
% and $\col_\alpha$ in $\att$.
% We will treat matrix variables that are used to iterate over canonical vectors differently from other matrix variables. To make the distinction clear we use capital $V\in \Mnam $ to indicate a `normal'' matrix variables and lower case $v\in\Mnam$ for the matrix variables used for iteration in for-loops.
% For each $V$ and $v_i$ $\in\Mnam$ and $\alpha \in \DD$ we denote
% by $R_V$, $R_v$ and $R_\alpha$ its corresponding relation name, respectively.
%
%
% Then, given $\Sch$ we define the relational
% schema $\text{Rel}(\Sch)$ such that $\fdom(\text{Rel}(\Sch)) =  \{R_\alpha \mid \alpha\in\DD\} \cup \{R_V \mid V \in \Mnam\} \cup \{R_v \mid v \in \Mnam\}$
% where $\text{Rel}(\Sch)(R_\alpha) = \{\alpha\}$ and for all $V\in\Mnam$:
% \[
% \text{Rel}(\Sch)(R_V) = \begin{cases}
% \lbrace\row_\alpha,\col_\beta \rbrace & \text{ if $ \size(V)=(\alpha,\beta)$} \\
% \lbrace\row_\alpha \rbrace & \text{ if $ \size(V)=(\alpha,1)$} \\
% \lbrace\col_\beta \rbrace  &
% \text{ if $ \size(V)=(1,\beta)$} \\
% \lbrace\rbrace & \text{ if $\size(V)=(1,1)$}.
% \end{cases}
% \]
% Furthermore, $\text{Rel}(\Sch)(R_V)=\lbrace \row_\alpha,\col_\alpha\rbrace$ if
% $\size(v)=(\alpha,1)$ and $\text{Rel}(\Sch)(R_V)=\lbrace \rbrace$ if $\size(v)=(1,1)$.
%
% Next, for a matrix instance $\I = (\dom,\conc)$ over $\Sch$,
% let $V\in\Mnam$ with $\size(V)=(\alpha,\beta)$ and let $\conc(V)$ be its corresponding $K$-matrix of dimension $\dom(\alpha)\times \dom(\beta)$.
% The $K$-instance in $\mathsf{RA}_{K}^+$ according to $\I$ is $\text{Rel}(\I)$ with data domain $\ddom = \mathbb{N} \setminus \{0\}$. For each $V\in\Mnam$ we define
% $R_V^{\text{Rel}(\I)}(t):=\conc(V)_{ij}$ whenever $t(\row_\alpha) = i \leq \dom(\alpha)$ and $t(\col_\beta) = j \leq \dom(\beta)$, and $\kzero$ otherwise.
% Also, for each $\alpha \in \DD$ we define $R_\alpha^{\text{Rel}(\I)}(t):=\kone$ whenever $t(\alpha) \leq \dom(\alpha)$, and $\kzero$ otherwise.
% If $\size(V)=(\alpha,1)$ then $R_V^{\text{Rel}(\I)}(t):=\conc(V)_{i1}$ whenever $t(\row_\alpha) = i \leq \dom(\alpha)$ and $\kzero$ otherwise.
% Similarly, if $\size(V)=(1,\beta)$ then $R_V^{\text{Rel}(\I)}(t):=\conc(V)_{1j}$ whenever $t(\col_\beta) = j \leq \dom(\beta)$ and $\kzero$ otherwise.
% If $\size(V)=(1,1)$ then $R_V^{\text{Rel}(\I)}(t):=\conc(V)_{11}$ only when $t(1) = \kone$ and $\kzero$ otherwise.
%
% \floris{The inductive proof needs to accommodate for more than two attributes?!
% This is needed for the summation case and this is how we did before with Jan's ARA syntax..
% }
% \thomas{Yes, I assume that the relations that represent the matrices have at most two attributes. But the ARA expressions can have more attributes, like in the summation case, where $e_1$ has attributes $\row_\alpha, \col_\beta, \row_\gamma$, and then I modify it to use $\col_\gamma$ also (I end up projecing on tow attributes though). This is what happens if the we sum over a variable, we replace that relation and project. If that variable is not instantiated by any $\ssum$ then it should be replaced with the relation that the instance constructed (first case). Not sure if this is what you mean.}
% The proof is by induction on the structure of \langsum expressions. Let $e$ be a \langsum expression such that $\Sch (e)=(\alpha, \beta)$.
% \begin{itemize}
%   \item If $e=V$ then $\arae (e):=R_V$.
%   % $\arae (e):=R_V^{\text{Rel}(\I)}$.
%
%   \item If $e=e_1^T$ with $\Sch (e_1)=(\alpha, \beta)$ then \[
% \arae(e) :=
% \begin{cases}
% \rho_{\mathrm{row}_\alpha \to \mathrm{col}_\alpha,\mathrm{col}_\beta \to \mathrm{row}_\beta}\bigl(\arae(e_1)\bigr) & \text{if } \alpha \neq 1 \neq \beta; \cr
% \rho_{\mathrm{row}_\alpha \to \mathrm{col}_\alpha}\bigl(\arae(e_1)\bigr) & \text{if } \alpha \neq 1 = \beta; \cr
% \rho_{\mathrm{col}_\beta \to \mathrm{row}_\beta}\bigl(\arae(e_1)\bigr) & \text{if } \alpha = 1 \neq \beta; \cr
% \arae(e_1) & \text{if } \alpha = 1 = \beta.
% \end{cases}
% \]
% \floris{Why is the next case included? The ones operator is expressible in \langsum, no?}
% \thomas{Yes, but I made it explicitly (like the function application) because it was easier than simulate the scalar product. As it says in section \ref{app:simp}, both of these operators can simulate scalar prod, so scalar product is doable in ARA since ones and $f_{\odot}$ are.}
%   \item If $e:=e_{\ones}(e')$ where $\Sch(e')=(\alpha, \beta)$ and $e_{\ones}(\cdot)$ is the $\mathsf{ones}$ operator
%   then $\arae(e):=\rho_{\alpha\rightarrow\row_\alpha}(\mathsf{Eq}_{\alpha})$
%
%   \item If $e=e_1+e_2$ with $\Sch (e_1)=\Sch (e_2)=(\alpha, \beta)$ then $\arae (e):=\arae (e_1)\cup \arae (e_2)$.
%
% \floris{Are we assuming scalar function application? The use of $f=\odot$ should be made clear. That is, we consider $\langsum$ but without function applications in the statement. Now apparently we need a pointwise function. This relates to an earlier comment and to the result that Damogoj wrote in the main part at some points. We need to bring this back into A2 and so that we can cite it here.}
% \thomas{Since we can do the ones operator in ARA, instead of simulating scalar product in ARA we simulate $f_{\odot}$. With this and the ones translation, we can simulate scalar prod.}
%   \item If $e=f(e_1,\ldots, e_k)$ with $\Sch(e_i)=\Sch(e_j)$ for all $i,j\in[1,k]$, with $f=\kprod$, then $\arae(e):=\arae(e_1)\Join \ldots \Join\arae(e_k)$.
%
%   \item If $e=e_1\cdot e_2$ with $\Sch (e_1)=(\alpha, \gamma)$ and $\Sch (e_2)=(\gamma, \beta)$, we have two cases. If $\gamma = 1$ then $\arae (e):=\arae (e_1)\Join \arae (e_2)$.
% If $\gamma\neq 1$ then
% $$
% \arae (e) := \pi_{\lbrace \row_{\alpha},\row_{\beta} \rbrace}\left( \sigma_{\lbrace \col_\gamma, \row_\gamma \rbrace} \left( \arae (e_1)\Join \arae (e_2) \right) \right)
% $$
%
% %   \item If $e=\ssum V. e_1$ where $\Sch(e_1)=(\alpha,\beta)$ and $\Sch(V)=(\gamma,1)$. Then we can just do $\pi_{\text{Rel}(\Sch)(R_V)}\left( \arae (e_1) \right)$.
% % Note that when the $\row_\gamma$ attribute in $\arae (e_1)$ is instantiated with with tuple $t$ such that $t(\row_\alpha)=i$, $t(\col_\beta)=j$ and $t(\row_\gamma)=k$,
% % then the expression evaluates to $e_1(\I [V\leftarrow e_{k}^{\dom(\gamma)}])_{ij}$.
% % Hence, by projecting in attributes $\lbrace \row_\alpha, \col_\beta\rbrace$ we range over all $k$ and sum up all $K$-values for each entry.
%   \item If $e=\ssum V. e_1$ where $\Sch(e_1)=(\alpha,\beta)$ and $\Sch(V)=(\gamma,1)$. Then we do
%   % $$
%   % \pi_{\lbrace \row_{\alpha}, \col_{\beta} \rbrace}\left( \arae (e_1)\left[ R_{V}^{\text{Rel}(\I)}\leftarrow R_{\mathsf{Id}}^{\gamma} \right] \right).
%   % $$
%   $$
%   \pi_{\lbrace \row_{\alpha}, \col_{\beta} \rbrace}\left( \arae (e_1)\left[ R_{V}\leftarrow R_{\mathsf{Id}}^{\gamma} \right] \right).
%   $$
%
%
%   Here $R_{\mathsf{Id}}^{\gamma}$ is an expression that, when evaluated, results identity of size $\dom(\gamma)\times\dom(\gamma)$ and which
%   is defined as
%   % $$
%  %  R_{\mathsf{Id}}^{\gamma}:= \sigma_{\lbrace \row_{\gamma},\col_{\gamma} \rbrace }\left( \rho_{\gamma\rightarrow\row_\gamma}\left( R_\gamma^{\text{Rel}(\I)} \right) \Join \rho_{\gamma\rightarrow\col_\gamma}\left( R_\gamma^{\text{Rel}(\I)} \right)\right)
%  %  $$
%   $$
%   R_{\mathsf{Id}}^{\gamma}:= \sigma_{\lbrace \row_{\gamma},\col_{\gamma} \rbrace }\left( \rho_{\gamma\rightarrow\row_\gamma}\left( R_\gamma \right) \Join \rho_{\gamma\rightarrow\col_\gamma}\left( R_\gamma\right)\right).
%   $$
%   Furthermore, $\arae (e_1)\left[ R_{V}\leftarrow R_{\mathsf{Id}}^{\gamma} \right]$
%   means that every occurrence of $R_V$ in $e_1$ is replaced by  $R_{\mathsf{Id}}^{\gamma}$.
%
%   Note that when $\arae (e_1)\left[ R_{V}\leftarrow R_{\mathsf{Id}}^{\gamma} \right]$ is instantiated with with tuple $t$ such that $t(\row_\alpha)=i$, $t(\col_\beta)=j$,
%   $t(\row_\gamma)=k$ and $t(\col_\gamma)=k$ (other values of are zero),
%   then the expression evaluates to $e_1(\I [V\leftarrow e_{k}^{\dom(\gamma)}])_{ij}$.
%   Hence, by projecting in attributes $\lbrace \row_\alpha, \col_\beta\rbrace$ we range over all $k$ and sum up all $K$-values for each entry.
% \end{itemize}
%
% Here, we only allow scalar $\kprod$ function application,
% which is equivalent to scalar product, as mentioned in section \ref{app:simp}.
% % Note that we can access the zero values of matrices because the indexes are in the active domain of $\text{Rel}(\I)$ due to the existence of $R_\alpha^{\text{Rel}(\I)}(t)$, for each $\alpha \in \DD$.
% \end{proof}
%
% % Consider a matrix instance $\I = (\dom,\conc)$ over a schema $\Sch$.
% % Let $V\in\Mnam$ with $\size(V)=(\alpha,\beta)$ and let $\conc(V)$ be its corresponding matrix of dimension $\dom(\alpha)\times \dom(\beta)$.
% % Given an instance $\I$ over $\Sch$, the domain asssignment $\mathbb{D}_{\I}$ is defined as
% % $\mathbb{D}_{\I}(\row_\alpha)=[1,\dom(\alpha)]$ and
% % $\mathbb{D}_{\I}(\col_\alpha)=[1,\dom(\alpha)]$.
% % We further  define the database instance $\text{Rel}_\Sch(\I)$  to consist of relations for each $V_R\in N$ defined as follows:
% % $\mathcal{T}_{\mathbb{D}_{\I}}(\text{Rel}(\Sch)(V_R)) \to K$ such that
% % $(\text{Rel}_{_\Sch}(I))(t):=\conc(V)_{ij}$ where (1) $t(\row_\alpha)=i$ if $\alpha\neq 1$ and equal to $1$ if $\alpha = 1$; and (2) $t(\col_\beta)=j$ if $\beta\neq 1$ and equal to $1$ if $\beta= 1$.
%
% % We next translate \lang$(\sum,\prod)$ expressions $e$ into \ARA expressions $\arae(e)$ by induction on the structure of $e$. The translation closely follows the translation given in~\cite{brijder2019matrices}, except that we  additionally need to consider summation, pointwise functions and the $\star$-projection.
%
% % \begin{itemize}
% % 	\item If $e=V$ then $\arae(e):=V_R$.
% % 	\item if $e=e_1^t$ where $\Sch(e_1)=(\alpha,\beta)$ then \[
% % \arae(e) :=
% % \begin{cases}
% % \rho_{\mathrm{col}_\alpha \to \mathrm{row}_\alpha,\mathrm{row}_\beta \to \mathrm{col}_\beta}\bigl(\arae(e_1)\bigr) & \text{if } \alpha \neq 1 \neq \beta; \cr
% % \rho_{\mathrm{col}_\alpha \to \mathrm{row}_\alpha}\bigl(\arae(e_1)\bigr) & \text{if } \alpha \neq 1 = \beta; \cr
% % \rho_{\mathrm{row}_\beta \to \mathrm{col}_\beta}\bigl(\arae(e_1)\bigr) & \text{if } \alpha = 1 \neq \beta; \cr
% % \arae(e_1) & \text{if } \alpha = 1 = \beta,
% % \end{cases}
% % \]
%
% % \item
% % 	If $e = e_1 \cdot e_2$ where $\Sch(e_1) = (\alpha,\gamma)$ and $\Sch(e_2) =(\gamma,\beta)$, then we consider two cases. If $\gamma = 1$, then $\arae(e) := \arae(e_1) \Join \arae(e_2)$. If $\gamma \neq 1$, then
% % 	$$
% % 	\arae(e) := \hat{\pi}_C\Bigr(\rho_{\mathrm{col}_\gamma\to C}\bigl(\arae(e_1)\bigr)\Join\rho_{\mathrm{row}_\gamma\to C}\bigl(\arae(e_2)\bigr)\Bigr).$$
% % 	% , where $\varphi_1(\mathrm{col}_\gamma) = \varphi_2(\mathrm{row}_\gamma) = C \notin \{\mathrm{row}_\alpha, \mathrm{col}_\beta\}$ and $\varphi_1$ and $\varphi_2$ are the identity otherwise.
% % 	% If $e=e_1(v_1,\ldots,v_k)\cdot e_2(u_1,\ldots,u_s)$ where $e_1$ is $n\times\gamma$, $e_2$ is $\gamma\times m$. Let $\rho:\row_\gamma\rightarrow C,\col_\gamma\rightarrow C.$ We have two cases:
% % 	% 	\begin{itemize}
% % 	% 		\item If $\gamma\neq 1$ then $E=\widehat{\pi}_C\left( \rho\left(E_1\right)\bowtie\rho\left( E_2\right)\right).$
% % 	% 		\item If $\gamma = 1$ then $E=E_1\bowtie E_2$.
% % 	% 	\end{itemize}
% % 	\item If $e=f(e_1,\ldots,e_k)$ with $\Sch(e_i)=(1,1)$ for all $i\in[1,k]$, then
% % 	$\arae(e):=\text{Apply}[f]\bigl(\arae(e_1),\ldots,\arae(e_k)\bigr)$.
% % 	% we have that $E=E_1\cup\cdots\cup E_s$ if $f$ is sum and $E=E_1\bowtie\cdots\bowtie E_s$ if $f$ is multiplication.
% % 	\item If $e=\ssum V.e_1$ where $\Sch(e_1)=(\alpha,\beta)$ and $\Sch(V)=(\gamma,1)$. Then,
% % 	in $\arae(e_1)$ we replace $V_R$ by $\arae_{Id}(V_R)$ which computes a binary
% % 	relation encoding the $\gamma\times\gamma$ idenity matrix. Intuitively, by selecting different
% % 	columns of the identity matrix we can extract all canonical $\gamma\times 1$ basis vectors.
% % 	More precisely, $\arae_{Id}(V_R)$ is  defined by
% % 	$$
% % 	\sigma_{\{\mathrm{row}_\gamma,C_V\}}\Bigr(\mathbf{1}(V_R) \Join \mathbf{1}\bigl(\rho_{\mathrm{row}_\gamma \to C_V}(V_R)\bigr)\Bigr)$$
% % 	if $\gamma \neq 1$ and
% %    $\mathbf{1}(V_R)$ if $\gamma = 1$.
% %    Then,
% % 	$$
% % 	\arae(e):=\hat{\pi}_{C_V}\bigl(\arae(e_1)[V_R\gets \arae_{Id}(V_R)])\bigr).
% % 	$$
%
% % 	Note that when the $C_{V}$ attribute in $\arae(e_1[V_R\gets \arae_{Id}(V_R)])$
% % 	is instantiated with a value $j$ in $[1,n_\gamma]$, then this expression evaluates $e_1(\I[V\gets e_j^\gamma])$
% % 	Hence, by projecting over $C_V$ we range over all $j\in[1,n_\gamma]$ and sum up all $K$-values for each entry.
% % \end{itemize}
%
% % \begin{proposition}
% % 	For each \lang$(\sum,\prod)$ expression $e$ over schema $\Sch$ such that $\Sch(e)=(\alpha,\beta)$ with $\alpha\neq 1\neq\beta$, there exists an \ARA expression $\arae(e)$ over schema $\text{Rel}(\Sch)$ such that $\text{Rel}(\Sch)(\arae(e))=\{\mathrm{row}_\alpha,\mathrm{col}_\beta\}$ and
% % 	such that for any instance $\I$ over $\Sch$,
% % 	$$
% % 	e(\I)_{i,j}=\arae(e)(\text{Rel}_{\Sch}(\I))(t)
% % 	$$
% % 	for tuple $t(\mathrm{row}_\alpha)=i$ and $t(\mathrm{col}_\beta)=j$ in $\text{Rel}_{\Sch}(\I)$. Similarly for when $e$ has schema $\Sch(e)=(\alpha,1)$, $\Sch(e)=(1,\beta)$ or $\Sch(e)=(1,1)$, then $\arae(e)$ has schema $\text{Rel}(\Sch)(\arae(e))=\{\mathrm{row}_\alpha\}$,
% % $\text{Rel}(\Sch)(\arae(e))=\{\mathrm{col}_\alpha\}$, or
% % $\text{Rel}(\Sch)(\arae(e))=\{\}$, respectively.
% % \end{proposition}
%
% % For the converse translation, i.e., from \ARA to \lang++, we need to impose some restrictions. More precisely, we only consider \ARA expression $\varphi$
% % that take as input relations of arity at most two and also have a schema of arity at most two. Note that intermediate expressions can create schemas of arbitraty size. We also make the assumption that there is an order, denoted by $<$, on the attributes in $\mathbf{att}$. In particular, we assume $A_1<A_2<\cdots$ for attributes $A_i\in\mathbf{att}$.
%
% % This is to identify which attributes correspond to rows and columns when moving the matrix setting.
%
%
% % Consider a database schema $\mathcal{R}$ on a finite set $N$ of relation names
% %  assigning a relation schema $\mathcal{R}(R)\subseteq\mathbf{att}$ to each $R \in N$. We assume that $\mathcal{R}(R)$ has size at most two. With each relation name $R\in N$ we associate a matrix name $M_R$. Let $\text{Mat}(\mathcal{R})$ denote the set
% %  $V_R$ of matrix names for $R\in\mathcal{R}$. Consider an injective function $s:\mathbf{att}\to \DD$ associating with each attribute a unique size symbol. Let $V_R\in \text{Mat}(\mathcal{R})$ and
% % define
% %  $$
% % \size(V_R)=\begin{cases}
% % (s(A_1),s(A_2)) & \text{if $\mathcal{R}(R)=\{A_1,A_2\}$, $A_1<A_2$}\\
% % (s(A),1) & \text{if $\mathcal{R}(R)=\{A\}$}\\
% % (1,1) & \text{if $\mathcal{R}(R)=\{\}$}.
% % \end{cases}
% %  $$
% % Let $\mathbb{D}$ be a domain assignment. We define
% % $\dom(\alpha)=|\mathbb{D}(A)|$ where $s(A)=\alpha$.
% % We only consider consecutive domain assignments, i.e.,
% % such that attribute values take values in $[1,||\mathbb{D}(A)|]$.
% % Consider
% % a relation $r:
% % \mathcal{T}_{\mathbb{D}}(X) \to K$ of $R$ with $X\subseteq\{A_1,A_2\}$  as determined by
% % $\mathcal{R}(R)$. We associate a matrix instance $\I = (\dom,\conc)$ as follows.
% % We define $\dom(\alpha)=|\mathbb{D}(A_1)|$ and
% % $\dom(\beta)=|\mathbb{D}(A_2)|$ where $s(A_1)=\alpha$ and $s(A_2)=\beta$. Furthermore,
% % $\conc(V_R)=\text{Mat}(r)$ such that
% % $(\text{Mat}(r))_{i,j} := r(t)$, where $t$ is (1) the tuple with $t(A_1) = i$ and $t(A_2) = j$ if $|X| = 2$; (2) the tuple with $t(A_1) = i$ and $j = 1$ if $X = {A_1}$,
% % (3) the tuple with $t(A_2) = j$ and $i = 1$ if $X = {A_2}$,
% % (3) the unique tuple of $r$ if $X=\emptyset$. Clearly, $\text{Mat}(r)$ is a matrix of dimension as
% % specified by $\mathbb{D}$.
%
%
% % %
% % % In the following, when $\varphi$ is
% % % an \ARA expression of schema $\mathcal{R}(\varphi)=\{A_1,\ldots,A_k\}$ with $A_1<A_2<A_3<\cdots < A_k$
% % %
% % We next translate \ARA expressions in to \lang$(\sum,\prod)$ expressions over an extended schema. More specifically, we extend
% % $\Mnam$ with matrix variables $V_A$ for attributes $A$ appearing the \ARA expression. We first simulate \ARA expression entry-wise.
%
% % \begin{lemma}
% % For every \ARA expression $\varphi$ over schema $\mathcal{R}$ with there is \lang$(\sum,\prod)$ expression $\Upsilon(\varphi)$ such that for every database instances $\I$ over $\mathcal{R}$ using consecutive domain assignment,
% % $$
% % \varphi(\I)(i_1,\ldots,i_p)=\Upsilon(\varphi)(\text{Mat}(\I)\cup \{e_{i_1}^{n_1},\ldots,e_{i_p}^{n_p})
% % $$
% % \end{lemma}
% % \begin{proof}
% % \begin{itemize}
% % \item If $\varphi=R$, then $\Upsilon(e):=V_{A_1}^t\cdot V_R\cdot V_{A_2}$ if $\mathcal{R}(R)=\{A_1,A_2\}$ with $A_1<A_2$;
% % $\Upsilon(e):=V_A^t\cdot V_R$ if $\mathcal{R}(R)=\{A\}$; and
% % $\Upsilon(e):=V_R$ if $\mathcal{R}(R)=\{\}$. We define $\text{Ext}()$
% % \item If $\varphi=\mathbf{1}(\psi)$  then $
% % \Upsilon(\varphi):=\text{Apply}[x\mapsto 1](\Upsilon(\psi))$.
% % \item If $\varphi=\psi_1\cup \psi_2$ then
% % $\Upsilon(\varphi):=\text{Apply}[(x,y)\mapsto x+y](\Upsilon(\psi_1),\Upsilon(\psi_2))$.
% % \item If $\varphi=\pi_{Y}(\psi)$ for $Y\subseteq \mathcal{R}(\psi)$ then
% % $$
% % \Upsilon(\varphi):=\sum_{\{V_A\mid A\in \mathcal{R}(\psi)\setminus Y\}}\Upsilon(\psi).
% %  $$
% % \item If $\varphi=\sigma_{Y}(\psi)$ with $Y\subseteq\mathcal{R}(\psi)$ then
% % $$
% %  \Upsilon(\varphi):=\Upsilon(\psi)\cdot\prod_{A,B\in Y} V_{A}^t\cdot V_{B}.
% % $$
% % \item If $\varphi=\rho_{X\mapsto Y}(\psi)$ then
% % $$\Upsilon(\varphi):=\Upsilon(\psi)[V_A\gets V_B\mid A\in X, B\in Y, A\mapsto B].$$
% % \item If $\varphi=\psi_1\bowtie \psi_2$ then
% % $\Upsilon(\varphi):=\Upsilon(\psi_1)\cdot \Upsilon(\psi_2)$.
% % \item If $\varphi=\pi_{Y}^\star(\psi)$ for $Y\subseteq \mathcal{R}(\psi)$.
% % \item If $\varphi=\text{Apply}[f](\psi_1,\ldots,\psi_k)$  then
% % $\Upsilon(\varphi):=\text{Apply}[f](\Upsilon(\psi_1),\ldots,\Upsilon(Y_k))$.
% % \end{itemize}
% % \end{proof}
% % As a consequence, when $\varphi$ is an \ARA expression such that $\mathcal{R}(\varphi)=\{A_1,A_2\}$ with $A_1<A_2$ we have that
% % $$
% % \varphi(\I)(t)=
% % \sum_{V_{A_1},V_{A_2}}\Upsilon(\varphi)\cdot V_{A_1}\cdot V_{A_2}^t.
% % $$
% % When $\mathcal{R}(\varphi)=\{A\}$ we have
% % $$
% % \varphi(\I)(t)=
% % \sum_{V_{A}}\Upsilon(\varphi)\cdot V_{A_1}.
% % $$
% % and
% % when $\mathcal{R}(\varphi)=\{\}$ we have
% % $$
% % \varphi(\I)(t)=\Upsilon(\varphi).
% % $$
%
%
%


\subsection{From ARA to \langsum}
Let $\cR$ be binary relational schema. For each $R\in \cR$ we associate a matrix variable 
$V_R$ such that, if $R$ is a binary relational signature, then $V_R$ represents a (square) matrix, 
if $R$ is unary, then $V_R$ represents a vector and if $|R|=0$ then $V_R$ represents a constant. Formally, 
fix a symbol $\arae \in \DD \setminus \{1\}$. Let $\text{Mat}(\cR)$ denote the \lang \ schema
$(\Mnam_\cR,\size_\cR)$ such that $\Mnam_\cR = \{ V_R \mid R \in \cR\}$ and $\size_\cR(V_R) = (\alpha, \alpha)$ 
whenever $|R| = 2$, $\size_\cR(V_R) = (\alpha, 1)$ whenever $|R|=1$ and $\size_\cR(V_R) = (1, 1)$ whenever $|R|=0$. 
Let $\cJ$ be the $K$-instance of $\cR$ and suppose that $\adom(\cJ) = \{d_1, \ldots, d_n\}$ is 
the active domain (with arbitrary order) of $\cJ$. 
Define the matrix instance $\text{Mat}(\cJ) = (\dom_\cJ,\conc_\cJ)$ such 
that $\dom_\cJ(\arae) = n$, $\conc_\cJ(V_R)_{i,j} = R^{\cJ}((d_i, d_j))$ whenever $|R|=2$, $\conc_\cJ(V_R)_{i} = R^{\cJ}((d_i))$ 
whenever $|R|=1$, 
\floris{This case relates to nullary relations. What does $R^{\cJ}$ mean?}
and $\conc_\cJ(V_R)_{1,1} = R^{\cJ}$ whenever $|R|=0$. 
Note that we consider the active domain of the whole $K$-instance.

We next translate \rak expressions in to \langsum expressions over an extended schema. More specifically, we extend 
$\Mnam$ with matrix variables $V_A\in\Mnam_\cR $ for attributes $A$ appearing the \rak expression. We first simulate a \rak expression entry-wise.

The proof is by induction on the structure of expressions.

\begin{lemma}\label{app:prop:point_ara}
For every \rak expression $\arae$ over schema $\mathcal{R}$ with there is \langsum expression $\Psi(\arae)$ such that for every database instances $\I$ over $\mathcal{R}$ using consecutive domain assignment, 
$$
\ssem{\arae}{\cJ}(i_1,\ldots,i_p)=\sem{\Psi(\arae)}{\text{Mat}(\cJ)\cup e_{i_1}^{d_1},\ldots,e_{i_p}^{d_p}}
$$
\end{lemma}
\begin{proof}
\begin{itemize}
\item If $\arae=R$, then $\Psi(e):=V_{A_1}^t\cdot V_R\cdot V_{A_2}$ if $\mathcal{R}(R)=\{A_1,A_2\}$ with $A_1<A_2$; 
$\Psi(e):=V_A^t\cdot V_R$ if $\mathcal{R}(R)=\{A\}$; and 
$\Psi(e):=V_R$ if $\mathcal{R}(R)=\{\}$.
\item If $\arae=\arae_1\cup \arae_2$ then
$\Psi(\arae):=\Psi(\arae_1) + \Psi(\arae_2)$.
\item If $\arae=\pi_{Y}(\arae')$ for $Y\subseteq \mathcal{R}(\arae')$ then
$$
\Psi(\arae):=\sum_{\{V_A\mid A\in \mathcal{R}(\arae')\setminus Y\}}\Psi(\arae').
 $$
\item If $\arae=\sigma_{Y}(\arae')$ with $Y\subseteq\mathcal{R}(\arae')$ then
$$
 \Psi(\arae):=\Psi(\arae')\cdot\prod_{A,B\in Y} V_{A}^t\cdot V_{B}.
$$
Here $\prod$ is just matrix multiplication.
\item If $\arae=\rho_{X\mapsto Y}(\arae')$ then
$$\Psi(\arae):=\Psi(\arae')[V_A\gets V_B\mid A\in X, B\in Y, A\mapsto B].$$
\item If $\arae=\arae_1\bowtie \arae_2$ then
$\Psi(\arae):=\Psi(\arae_1)\cdot \Psi(\arae_2)$.
\end{itemize}
\end{proof}

Now we can obtain proposition \ref{prop:ara_to_sum}.

\newtheorem*{ARATOSUM}{Proposition~\ref{prop:ara_to_sum}}

\begin{ARATOSUM}
  Let $\cR$ be a binary relational schema. For each $\mathsf{RA}_{K}^+$  expression $\arae$ over $\cR$  such that $|\cR(\arae)| = 2$, there exists a \langsum  expression $\Psi(\arae)$ over \lang \ schema $\text{Mat}(\cR)$ such that for any $K$-instance $\cJ$ with $\adom(\cJ) = \{d_1, \ldots, d_n\}$ over $\cR$,
	$$
	\ssem{\arae}{\cJ}((d_i, d_j))=\sem{\Psi(\arae)}{\text{Mat}(\cJ)}_{i,j}.
	$$
	Similarly for when $|\cR(\arae)| = 1$, or $|\cR(\arae)| = 0$ respectively.
\end{ARATOSUM}
\begin{proof}
As a consequence of lemma \ref{app:prop:point_ara}, when $\arae$ is a \rak expression 
such that $\mathcal{R}(\arae)=\{A_1,A_2\}$ with $A_1<A_2$ we have that 
$$
\arae(\I)(t)=
\sum_{V_{A_1},V_{A_2}}\Psi(\arae)\cdot V_{A_1}\cdot V_{A_2}^t.
$$
When $\mathcal{R}(\arae)=\{A\}$ we have
$$
\arae(\I)(t)=
\sum_{V_{A}}\Psi(\arae)\cdot V_{A_1}.
$$
And when $\mathcal{R}(\arae)=\{\}$ we have
$$
\arae(\I)(t)=\Psi(\arae).
$$

\end{proof}
 

% \newtheorem*{ARATOSUM}{Proposition~\ref{prop:ara_to_sum}}

% Here we present the proof of proposition \ref{prop:ara_to_sum}.

% \begin{ARATOSUM}
%   Let $\cR$ be a binary relational schema. For each $\mathsf{RA}_{K}^+$  expression $\arae$ over $\cR$  such that $|\cR(\arae)| = 2$, there exists a \langsum  expression $\Psi(\arae)$ over \lang \ schema $\text{Mat}(\cR)$ such that for any $K$-instance $\cJ$ with $\adom(\cJ) = \{d_1, \ldots, d_n\}$ over $\cR$,
% 	$$
% 	\ssem{\arae}{\cJ}((d_i, d_j))=\sem{\Psi(\arae)}{\text{Mat}(\cJ)}_{i,j}.
% 	$$
% 	Similarly for when $|\cR(\arae)| = 1$, or $|\cR(\arae)| = 0$ respectively.
% \end{ARATOSUM}

% \begin{proof}
% Let $\cR$ be binary relational schema. For each $R\in \cR$ we associate a matrix variable 
% $V_R$ such that, if $R$ is a binary relational signature, then $V_R$ represents a (square) matrix, 
% if $R$ is unary, then $V_R$ represents a vector and if $|R|=0$ then $V_R$ represents a constant. Formally, 
% fix a symbol $\arae \in \DD \setminus \{1\}$. Let $\text{Mat}(\cR)$ denote the \lang \ schema
% $(\Mnam_\cR,\size_\cR)$ such that $\Mnam_\cR = \{ V_R \mid R \in \cR\}$ and $\size_\cR(V_R) = (\alpha, \alpha)$ 
% whenever $|R| = 2$, $\size_\cR(V_R) = (\alpha, 1)$ whenever $|R|=1$ and $\size_\cR(V_R) = (1, 1)$ whenever $|R|=0$. 
% Let $\cJ$ be the $K$-instance of $\cR$ and suppose that $\adom(\cJ) = \{d_1, \ldots, d_n\}$ is 
% the active domain (with arbitrary order) of $\cJ$. 
% Define the matrix instance $\text{Mat}(\cJ) = (\dom_\cJ,\conc_\cJ)$ such 
% that $\dom_\cJ(\arae) = n$, $\conc_\cJ(V_R)_{i,j} = R^{\cJ}((d_i, d_j))$ whenever $|R|=2$, $\conc_\cJ(V_R)_{i} = R^{\cJ}((d_i))$ 
% whenever $|R|=1$, 
% \floris{This case relates to nullary relations. What does $R^{\cJ}$ mean?}
% and $\conc_\cJ(V_R)_{1,1} = R^{\cJ}$ whenever $|R|=0$. 
% Note that we consider the active domain of the whole $K$-instance.
% The proof is by induction on the structure of expressions.

% \floris{We need to introduce the vector variables $V_A$ for attributes $A$
% explicitly in the induction?!}

% \begin{itemize}
%   \item If $\arae = R$ then $\Psi (\arae):=V_R$. Note that if $|R|=2$ then 
%     $$\sem{\Psi(\arae)}{\text{Mat}(\cJ)}_{i,j}=\conc(V_R)_{i,j} = R^{\cJ}((d_i, d_j))=\ssem{\arae}{\cJ}((d_i, d_j)).$$ 
%     If $|R|=1$ then 
%     $$\sem{\Psi(\arae)}{\text{Mat}(\cJ)}_{i,j}=\conc(V_R)_{i,1} = R^{\cJ}((d_i))=\ssem{\arae}{\cJ}((d_i)).$$
%     And if $|R|=0$ then 
% 	\floris{Not sure about this:}
%     $\sem{\Psi(\arae)}{\text{Mat}(\cJ)}_{i,j}=\conc(V_R)_{1,1} = R^{\cJ}=\ssem{\arae}{\cJ}$.
%   \item If $\arae=\arae_1\cup\arae_2$ then $\Psi(\arae):=\Psi(\arae_1)+\Psi(\arae_2)$.
%   \item If $\arae = \pi_{Y}(\arae_1)$ for $Y\subseteq R(\arae_1)$. Let $\cV = \lbrace V_A:A\in R(\arae_1)\setminus Y \rbrace$. 
%     Then
%     $$
%     \Psi(\arae) = \sum_{V\in\cV} \Psi(\arae_1) = \ssum V_{A_1}. \ssum \cdots \ssum V_{A_{l}}. \Psi(\arae_1).
%     $$
% \floris{What is $V_{\mathsf{Eq}_Y}$??}
%   \item If $\arae = \sigma_Y(\arae_1)$ with $Y\subseteq R(\arae_1)$ then 
%     $\Psi(\arae):=f_{\kprod}\left( \Psi(\arae_1), V_{\mathsf{Eq}_Y} \right)$.
%   \item If $\arae = \rho_{X\rightarrow Y}(\arae_1)$ then $\Psi (\arae):= \Psi (\arae_1)\left[ V_A\gets V_B, A\in X, B\in Y, A \mapsto B \right]$.
%   \item If $\arae = \arae_1\Join \arae_2$, where $R(\arae)=R(\arae_1)=R(\arae_2)$ then 
%     $\Psi (\arae):=f_{\kprod}\left( \Psi (\arae_1), \Psi (\arae_2) \right)$
% \end{itemize}

% Note that for function application we only allow $\kprod$, 
% which we can assume to have because we can do scalar multiplication and $\ssum$ (see section \ref{app:simp}).

% \end{proof}

% For the converse translation, we need to impose some restrictions. More precisely, we only consider a \rak expression $\arae$
% that take as input relations of arity at most two and also have a schema of arity at most two. 
% Note that intermediate expressions can create schemas of arbitraty size. We also make the assumption that there is an order, denoted by $<$, on the attributes in $\mathbf{att}$. In particular, we assume $A_1<A_2<\cdots$ for attributes $A_i\in\mathbf{att}$.

% This is to identify which attributes correspond to rows and columns when moving the matrix setting. 

% Consider a database schema $\mathcal{R}$ on a finite set $N$ of relation names
%  assigning a relation schema $\mathcal{R}(R)\subseteq\mathbf{att}$ to each $R \in N$. We assume that $\mathcal{R}(R)$ has size at most two. With each relation name $R\in N$ we associate a matrix name $M_R$. Let $\text{Mat}(\mathcal{R})$ denote the set
%  $V_R$ of matrix names for $R\in\mathcal{R}$. Consider an injective function $s:\mathbf{att}\to \DD$ associating with each attribute a unique size symbol. Let $V_R\in \text{Mat}(\mathcal{R})$ and
% define
%  $$
% \size(V_R)=\begin{cases}
% (s(A_1),s(A_2)) & \text{if $\mathcal{R}(R)=\{A_1,A_2\}$, $A_1<A_2$}\\
% (s(A),1) & \text{if $\mathcal{R}(R)=\{A\}$}\\
% (1,1) & \text{if $\mathcal{R}(R)=\{\}$}.
% \end{cases}
%  $$
% Let $\mathbb{D}$ be a domain assignment. We define 
% $\dom(\alpha)=|\mathbb{D}(A)|$ where $s(A)=\alpha$.
% We only consider consecutive domain assignments, i.e.,
% such that attribute values take values in $[1,||\mathbb{D}(A)|]$.
% Consider 
% a relation $r:
% \mathcal{T}_{\mathbb{D}}(X) \to K$ of $R$ with $X\subseteq\{A_1,A_2\}$  as determined by
% $\mathcal{R}(R)$. We associate a matrix instance $\I = (\dom,\conc)$ as follows.
% We define $\dom(\alpha)=|\mathbb{D}(A_1)|$ and
% $\dom(\beta)=|\mathbb{D}(A_2)|$ where $s(A_1)=\alpha$ and $s(A_2)=\beta$. Furthermore,
% $\conc(V_R)=\text{Mat}(r)$ such that 
% $(\text{Mat}(r))_{i,j} := r(t)$, where $t$ is (1) the tuple with $t(A_1) = i$ and $t(A_2) = j$ if $|X| = 2$; (2) the tuple with $t(A_1) = i$ and $j = 1$ if $X = {A_1}$,
% (3) the tuple with $t(A_2) = j$ and $i = 1$ if $X = {A_2}$,
% (3) the unique tuple of $r$ if $X=\emptyset$. Clearly, $\text{Mat}(r)$ is a matrix of dimension as
% specified by $\mathbb{D}$. 


%
% In the following, when $\arae$ is
% an \rak expression of schema $\mathcal{R}(\arae)=\{A_1,\ldots,A_k\}$ with $A_1<A_2<A_3<\cdots < A_k$

% \newcommand{\MLm}{\mathsf{MATLANG}}
\newcommand{\ML}{$\MLm$\xspace}
\newcommand{\ARAm}{\mathsf{ARA}}
\newcommand{\ARA}{$\ARAm$\xspace}
\newcommand{\ARAC}{$(\ARAm+\zeta_k)(k)$\xspace}
\newcommand{\ARACTWO}{$(\ARAm+\zeta_2)(2)$\xspace}
\newcommand{\Rel}{\mathrm{Rel}}
\newcommand{\Mat}{\mathrm{Mat}}

% \DeclareMathOperator{\sdiff}{\triangle}
% By \emph{function} we will always mean a total function. For a function $f: X \to Y$ and $Z \subseteq X$, the \emph{restriction} of $f$ to $Z$, denoted by $f|_Z$, is the function $Z \to Y$ where $f|_Z(x) = f(x)$ for all $x \in Z$.

%
%
%
% From the outset, we also fix countable infinite sets $\mathbf{rel}$, $\mathbf{att}$, and $\mathbf{dom}$, the elements of which are called \emph{relation names}, \emph{attributes}, and \emph{domain elements}, respectively. We assume an equivalence relation $\sim$ on $\mathbf{att}$ that partitions $\mathbf{att}$ into an infinite number of equivalence classes that are each infinite. When $A \sim B$, we say that $A$ and $B$ are \emph{compatible}. Intuitively, $A \sim B$ will mean that $A$ and $B$ have the same set of domain values. A function $f: X \to Y$ with $X$ and $Y$ sets of attributes is called \emph{compatible} if $f(A) \sim A$ for all $A \in X$.
%
\subsection{Annotated-Relation Algebra (\ARA)} 
For completeness, we start by recalling the definition of the \ARA query language. We  here closely follow the exposition given in~\cite{brijder2019matrices}.

Let $\mathbf{att}$ and $\mathbf{dom}$ denote countable infinite sets of  \emph{attributes} and \emph{domain elements}, respectively. A notion of compatibility between attributes is assumed. More formally,
we assume that an equivalence relation $\sim$ on $\mathbf{att}$ is present which partitions $\mathbf{att}$ into an infinite number of equivalence classes that are each infinite. When $A \sim B$, we say that $A$ and $B$ are \emph{compatible}. Intuitively, $A \sim B$ will mean that $A$ and $B$ have the same set of domain values. A function $f: X \to Y$ with $X$ and $Y$ sets of attributes is called \emph{compatible} if $f(A) \sim A$ for all $A \in X$.

A \emph{relation schema} is a finite subset of $\mathbf{att}$. A \emph{database schema} is a function $\mathcal{R}$ on a finite set $N$ of relation names, assigning a relation schema $\mathcal{R}(R)$ to each $R \in N$.
% The \emph{arity} of a relation name $R$ is the cardinality $|\mathcal{R}(R)|$ of its schema. The \emph{arity} of $\mathcal{R}$ is the largest arity among relation names $R \in N$.

The \emph{(positive) Annotated-Relation Algebra}, abbreviated by \ARA, is defined as follows. With each expression $\varphi$ in \ARA one also assigns a relation schema $\mathcal{R}(\varphi)$, by extending the initial schema
$\mathcal{R}$. An \emph{\ARA expression} $\varphi$ over a database schema $\mathcal{R}$ is equal to 
\begin{itemize}
\item a relation name $R$ of $\mathcal{R}$;
\item $\mathbf{1}(\psi)$, where $\psi$ is an \ARA expression, and $\mathcal{R}(\varphi) := \mathcal{R}(\psi)$;
\item $\psi_1 \cup \psi_2$, where $\psi_1$ and $\psi_2$ are \ARA expressions with $\mathcal{R}(\psi_1) = \mathcal{R}(\psi_2)$, and $\mathcal{R}(\varphi) := \mathcal{R}(\psi_1)$;
\item $\pi_Y(\psi)$, where $\psi$ is an \ARA expression and $Y \subseteq \mathcal{R}(\psi)$, and $\mathcal{R}(\varphi) := Y$;
% \item $\pi_Y^\star(\psi)$, where $\psi$ is an \ARA expression and $Y \subseteq \mathcal{S}(\psi)$, and $\mathcal{S}(\varphi) := Y$;
\item $\sigma_{Y}(\psi)$, where $\psi$ is an \ARA expression, $Y \subseteq \mathcal{R}(\psi)$, the elements of $Y$ are mutually compatible, and $\mathcal{R}(\varphi) := \mathcal{R}(\psi)$;
\item $\rho_{\mathcal{R}(\psi) \mapsto Y}(\psi)$, where $\psi$ is an \ARA expression and $\mathcal{R}(\psi) \mapsto Y$ is a compatible one-to-one correspondence of attributes with $Y \subseteq \mathbf{att}$, and $\mathcal{R}(\varphi) := Y$; or
\item $\psi_1 \Join \psi_2$, where $\psi_1$ and $\psi_2$ are \ARA expressions, and $\mathcal{R}(\varphi) := \mathcal{R}(\psi_1) \cup \mathcal{R}(\psi_2)$.
\end{itemize}

We next define the semantics of \ARA expression.
A \emph{domain assignment} is a function $\mathbb{D}: \mathbf{att} \to
2^{\mathbf{dom}}$ such that $A \sim B$ implies
$\mathbb{D}(A) = \mathbb{D}(B)$. Let $X$ be a relation schema. A \emph{tuple} over
$X$ with respect to $\mathbb{D}$
is a function $t: X \to \mathbf{dom}$ such that
$t(A) \in \mathbb{D}(A)$ for all $A \in X$. We denote by
$\mathcal{T}_{\mathbb{D}}(X)$ the set of tuples over $X$ with respect to $\mathbb{D}$. Note that
$\mathcal{T}_{\mathbb{D}}(X)$ is finite.  A \emph{relation} $r$ over
$X$ with respect to $\mathbb{D}$ is a function $r:
\mathcal{T}_{\mathbb{D}}(X) \to K$ for a \emph{semiring} $(K,+,*,0,1)$. So a relation annotates every tuple
over $X$ with respect to $\mathbb{D}$ with a value from $K$.  If $\mathcal{R}$ is a
database schema, then an \emph{instance $\mathcal{I}$ of
$\mathcal{R}$ with respect to $\mathbb{D}$} is a function that assigns to every
relation name $R$ of $\mathcal{R}$ a relation $\mathcal{I}(R):
\mathcal{T}_{\mathbb{D}}(\mathcal{R}(R)) \to K$.

The semantics of \ARA expressions is defined, as follows.

\begin{description}


\item[One] For every relation schema $X$,
  we define $\mathbf{1}_X: \mathcal{T}_{\mathbb{D}}(X) \to K$ where $\mathbf{1}_X(t) = 1$ for every $t \in \mathcal{T}_{\mathbb{D}}(X)$. 


\item[Union] Let $r_1, r_2: \mathcal{T}_{\mathbb{D}}(X) \to K$. Define $r_1 \cup r_2: \mathcal{T}_{\mathbb{D}}(X) \to K$ as $(r_1 \cup r_2)(t) = r_1(t) + r_2(t)$.


\item[Projection] Let $r: \mathcal{T}_{\mathbb{D}}(X) \to K$ and $Y \subseteq X$. Define $\pi_{Y}(r): \mathcal{T}_{\mathbb{D}}(Y) \to K$ as
\[
(\pi_{Y}(r))(t) = \sum_{\substack{t' \in \mathcal{T}_{\mathbb{D}}(X),\\ t'|_{Y} = t}} \!\! r(t').
\]


\item[Selection] Let $r: \mathcal{T}_{\mathbb{D}}(X) \to K$ and $Y \subseteq X$ where the elements of $Y$ are mutually compatible. Define $\sigma_{Y}(r): \mathcal{T}_{\mathbb{D}}(X) \to K$ such that
\[
(\sigma_{Y}(r))(t) =
\begin{cases}
r(t) & \text{if } t(A)=t(B) \text{ for all } A, B \in Y;\cr
0    & \text{otherwise}.
\end{cases}
\]


\item[Renaming] Let $r: \mathcal{T}_{\mathbb{D}}(X) \to K$ and $\varphi: X \to Y$ a compatible one-to-one correspondence. We define $\rho_\varphi(r): \mathcal{T}_{\mathbb{D}}(Y) \to K$ as $\rho_\varphi(r)(t) = r(t \circ \varphi)$.

\item[Join] Let $r_1: \mathcal{T}_{\mathbb{D}}(X_1) \to K$ and $r_2: \mathcal{T}_{\mathbb{D}}(X_2) \to K$. Define $r_1 \Join r_2: \mathcal{T}_{\mathbb{D}}(X_1 \cup X_2) \to K$ as $(r_1 \Join r_2)(t) = r_1(t|_{X_1})*r_2(t|_{X_2})$.
\end{description}

The above operations provide semantics for \ARA in a natural manner. Formally, let $\mathcal{R}$ be a database schema, let $\varphi$ be an \ARA expression over $\mathcal{R}$, and let $\mathcal{I}$ be an instance of $\mathcal{R}$. The \emph{output} relation $\varphi(\mathcal{I})$ of $\varphi$ under $\mathcal{I}$ is defined as follows. If $\varphi = R$ with $R$ a relation name of $\mathcal{R}$, then $\varphi(\mathcal{I}) := \mathcal{I}(R)$. If $\varphi = \mathbf{1}(\psi)$, then $\varphi(\mathcal{I}) := \mathbf{1}_{\mathcal{S}(\psi)}$. If $\varphi = \psi_1 \cup \psi_2$, then $\varphi(\mathcal{I}) := \psi_1(\mathcal{I}) \cup \psi_2(\mathcal{I})$. If $\varphi = \pi_{X}(\psi)$, then $\varphi(\mathcal{I}) := \pi_{X}(\psi(\mathcal{I}))$. If $\varphi = \sigma_{Y}(\psi)$, then $\varphi(\mathcal{I}) := \sigma_{Y}(\psi(\mathcal{I}))$. If $\varphi = \rho_\varphi(\psi)$, then $\varphi(\mathcal{I}) := \rho_\varphi(\psi(\mathcal{I}))$. Finally, if $\varphi = \psi_1 \Join \psi_2$, then $\varphi(\mathcal{I}) := \psi_1(\mathcal{I}) \Join \psi_2(\mathcal{I})$.


\subsection{An extension of \ARA}
We extend \ARA with the following two operators:
\begin{itemize}
 \item $\pi_Y^\star(\psi)$, where $\psi$ is an \ARA expression and $Y \subseteq \mathcal{R}(\psi)$, and $\mathcal{R}(\varphi) := Y$;
 \item $\textsf{Apply}[f](\psi_1,\ldots,\psi_k)$, where $\psi_1,\ldots,\psi_k$ are \ARA expressions with $\mathcal{R}(\psi_1)=\cdots=\mathcal{R}(\psi_k)$, 
 $f$ is a function $K^k\to K$,
 and 
 $\mathcal{R}(\varphi)=\mathcal{R}(\psi_1)$.
\end{itemize}
The semantics of these operators is given by:
\begin{description}
\item[$\star$-Projection] Let $r: \mathcal{T}_{\mathbb{D}}(X) \to K$ and $Y \subseteq X$. Define $\pi_{Y}^\star(r): \mathcal{T}_{\mathbb{D}}(Y) \to K$ as
\[
(\pi_{Y}(r))(t) = \prod_{\substack{t' \in \mathcal{T}_{\mathbb{D}}(X),\\ t'|_{Y} = t}} \!\! r(t').
\]
\item[Function application] Let $r_{i}: \mathcal{T}_{\mathbb{D}}(X) \to K$ for $i=1,\ldots,k$. Define $\textsf{Apply}[f](r_1,\ldots,r_k): \mathcal{T}_{\mathbb{D}}(Y) \to K$ as
\[
(\textsf{Apply}[f](r_1,\ldots,r_k))(t) = f(r_1(t),\ldots,r_k(t)).
\]
\end{description}
Hence, if $\varphi=\pi^\star_{Y}(\psi)$ then 
$\varphi(\mathcal{I}):=\pi^\star_{Y}(\psi(\mathcal{I}))$, and
if $\varphi=\textsf{Apply}[f](\psi_1,\allowbreak \ldots,\psi_k)$ then we have
$\varphi(\mathcal{I}):=\textsf{Apply}[f](\psi_1(\mathcal{I}), \ldots,\psi_k(\mathcal{I}))$. We let $\Omega$ denote a set of pointwise functions that can be used in function applications and write \ARA$_\Omega$ to make this explicit.


\subsection{Upper bound on expressivity}
There is a straightforward translation from \ARA$_{\Omega}$ expressions into the relational algebra with aggregation $\text{ALG}_{\text{aggr}}(\Omega',\Theta)$ as defined in~\cite{LIBKIN2003}. Here, $\Omega'$ consists of the functions in $\Omega$ and complemented with the unary functions $1:K\to K:k\mapsto 1$, to deal with $\mathbf{1}$ operator and $0:K\to K:k\mapsto 0$ to deal with selection and binary functions $f_+:K^2\to K:(k,\ell)\mapsto k+\ell$ and $f_*:K^2\to K:(k,\ell)\mapsto k*\ell$.
Furthermore, $\Theta$ consist of aggregate functions corresponding to the semiring sum and product, lifted to multi-sets. More precisely,
$\Theta$ includes $f_+^1,f_+^2,f_+^3,\ldots$ such that $f_+^n$ maps
$n$-element multi-sets in $K$ to their sum, and 
 $f_*^1,f_*^2,f_*^3,\ldots$ such that $f_*^n$ maps
 $n$-element multi-sets in $K$ to their product.

The language $\text{ALG}_{\text{aggr}}(\Omega,\Theta)$ is defined over a ``pure'' relational schema in which attributes are typed. It is easy to see that with every \ARA schema $\mathcal{R}$ we can associate a relational schema encoding the same information. Intuitively, we have one attribute of type $\mathbf{dom}$ for each $A\in\mathcal{R}$ and a special attribute $\text{Val}$ of type $K$ which is to hold the semiring values.

Given this translation, it is known that every expression in  $\text{ALG}_{\text{aggr}}(\Omega,\Theta)$ corresponds to an expression in the finite rank fragment $\mathcal{L}_{\infty,\omega}^*(\textbf{Cnt})$ of infinitary logic with counting $\mathcal{L}_{\infty,\omega}(\textbf{Cnt})$~\cite{LIBKIN2003,Hella:2001}. 
Since this logic is local, \ARA inherits this locality. As an example, 
transitive closure and connectivity of graphcs cannot be expressed in \ARA$_{\Omega}$.


% \newcommand{\row}{\mathsf{row}}
\newcommand{\rows}{\mathsf{rows}}
\newcommand{\col}{\mathsf{col}}
\newcommand{\cols}{\mathsf{cols}}

We next show that \lang$(\Sigma)$ and \ARA are closely connected. To make this correspond formal we first establish a link between matrix schemas and relation schemas, and matrix instances and database instances.

We start from a matrix schema $\Sch=(\Mnam,\size)$, where $\Mnam\subset \Mvar$ is a finite set of matrix variables, and $\size: \Mvar \mapsto \DD\times \DD$ is a function that maps each matrix variable to a pair of size symbols. On the relational side
we have for each size symbol $\alpha\in\DD\setminus\{1\}$, attributes $\row_\alpha$ and $\col_\alpha$ in $\mathbf{att}$. Given
$\Sch$, the domain asssignment $\mathbb{D}_\Sch$ is defined as 
$\mathbb{D}_\Sch(\row_\alpha)=[1,\size(\alpha)]$ and 
$\mathbb{D}_\Sch(\col_\alpha)=[1,\size(\alpha)]$. We define the database  schema $\text{Rel}(\Sch)$ such that for each $V\in\Mnam$,
\[
	\text{Rel}(\Sch)(V) = \begin{cases}
		\lbrace\row_\alpha,\col_\beta \rbrace & \text{ if $ \size(V)=(\alpha,\beta)$} \\
		\lbrace\row_\alpha \rbrace & \text{ if $ \size(V)=(\alpha,1)$} \\
		\lbrace\col_\beta \rbrace  &
	 \text{ if $ \size(V)=(1,\beta)$} \\
		\lbrace\rbrace & \text{ if $\size(V)=(1,1)$}.
\end{cases}
\]
Consider a matrix instance $\I = (\dom,\conc)$ over a schema $\Sch$.
Let $V\in\Mnam$ with $\size(V)=(\alpha,\beta)$ and $\conc(V)$ be its corresponding matrix of dimension $\dom(\alpha)\times \dom(\beta)$.
We then define the database instance $\text{Rel}_\Sch(\I)$  to consist of relations for each $V\in\Mnam$ defined as follows:
$\mathcal{T}_{\mathbb{D}_{\Sch}}(\text{Rel}(\Sch)(V)) \to K$ such that
$(\text{Rel}_{_\Sch}(I))(t):=\conc(V)_{ij}$ where (1) $t(\row_\alpha)=i$ if $\alpha\neq 1$ and equal to $1$ if $\alpha = 1$; and (2) $t(\col_\beta)=j$ if $\beta\neq 1$ and equal to $1$ if $\beta= 1$.

We next translate \lang++ expressions $e$ into \ARA expressions $\Phi(e)$ by induction on the structure.
%
% \begin{itemize}
% 	\item If $e=V$ then $\Phi(e):=\text{Rel}(\Sch)(V)$.
% 	\item if $e=e^t$ then \[
% \Phi(e) :=
% \begin{cases}
% \rho_{\varphi}(\Phi(e')) & \text{if } \alpha \neq 1 \neq \beta; \cr
% \rho_{\mathrm{col}_\alpha \to \mathrm{row}_\alpha}(\Phi(e')) & \text{if } \alpha \neq 1 = \beta; \cr
% \rho_{\mathrm{row}_\beta \to \mathrm{col}_\beta}(\Phi(e')) & \text{if } \alpha = 1 \neq \beta; \cr
% e' & \text{if } \alpha = 1 = \beta,
% \end{cases}
% \]
% where $\varphi$ maps $\mathrm{col}_\alpha$ to $\mathrm{row}_\alpha$ and $\mathrm{row}_\beta$ to $\mathrm{col}_\beta$.
% 	\item
% 	If $e = e_1 \cdot e_2$ where $\mathcal{S}(e_1) = \alpha \times \gamma$ and $\mathcal{S}(e_2) = \gamma \times \beta$, then we consider two cases. If $\gamma = 1$, then $\Upsilon(e) := \Upsilon(e_1) \Join \Upsilon(e_2)$. If $\gamma \neq 1$, then $\Upsilon(e) := \zeta_{C,2}(\rho_{\varphi_1}(\Upsilon(e_1)),\rho_{\varphi_2}(\Upsilon(e_2)))$, where $\varphi_1(\mathrm{col}_\gamma) = \varphi_2(\mathrm{row}_\gamma) = C \notin \{\mathrm{row}_\alpha, \mathrm{col}_\beta\}$ and $\varphi_1$ and $\varphi_2$ are the identity otherwise.
% 	If $e=e_1(v_1,\ldots,v_k)\cdot e_2(u_1,\ldots,u_s)$ where $e_1$ is $n\times\gamma$, $e_2$ is $\gamma\times m$. Let $\rho:\row_\gamma\rightarrow C,\col_\gamma\rightarrow C.$ We have two cases:
% 		\begin{itemize}
% 			\item If $\gamma\neq 1$ then $E=\widehat{\pi}_C\left( \rho\left(E_1\right)\bowtie\rho\left( E_2\right)\right).$
% 			\item If $\gamma = 1$ then $E=E_1\bowtie E_2$.
% 		\end{itemize}
% 	\item If $e=f(e_1,\ldots,e_s)$ we have that $E=E_1\cup\cdots\cup E_s$ if $f$ is sum and $E=E_1\bowtie\cdots\bowtie E_s$ if $f$ is multiplication.
% 	\item If $e=\ssum v.e'(v,v_1,\ldots, v_k)$ and let $E'$ be the corresponding ARA expression of $e'$. Note that $E'$ is $E'(\row_n, \col_m,A,A_1,\ldots,A_k), E'(\row_n,A,A_1,\ldots,A_k), E'(\col_m,A,A_1,\ldots,A_k)$ or $E'(A,A_1,\ldots,A_k)$ depending on the dimensions of $e'$. Then we have that $E=\widehat{\pi}_A(E').$ Note that this implies that if $e=\ssum v_1\ssum v_2\cdots\ssum v_k.e'(v_1,\ldots,v_k)$ then $E=\widehat{\pi}_{A_1}\widehat{\pi}_{A_2}\cdots\widehat{\pi}_{A_k}E'.$
% \end{itemize}

\subsection{Weighted logics and \langprod}
% !TeX spellcheck = en_US
%!TEX root = ../main.tex

\newtheorem*{WL}{Proposition~\ref{prop:wl}}

We prove proposition \ref{prop:wl}:

\begin{WL}
  Weighted logics over $\Gamma$ and \langprod over $\Sch$ have the same expressive power. More specifically,
  \begin{itemize}
  	\item for each \langprod expression $e$ over $\Sch$ such that $\Sch(e)=(1,1)$, there exists a WL-formula $\Phi(e)$ over $\text{WL}(\Sch)$ such that for every instance $\I$ of~$\Sch$, 
  	$
  	\sem{e}{\I} = \ssem{\Phi(e)}{\text{WL}(\I)}
  	$.
  	\item for each WL-formula $\varphi$ over $\Gamma$ without free variables, there exists a \langprod expression $\Psi(\varphi)$ such that for any structure $\cA$ over~$\text{Mat}(\Gamma)$,
  	$
  	\ssem{\varphi}{\cA}=\sem{\Psi(\varphi)}{\text{Mat}(\cA)}
  	$.
  \end{itemize}	
\end{WL}

\begin{proof}
Both directions are proved by induction on the structure of expression.

\smallskip

\noindent \textbf{(\langprod to WL)} First, let $\Sch=(\Mnam,\size)$ be a schema of square matrices, that is, there exists an $\alpha$ such 
that $\size(V) \in \{1, \alpha\} \times \{1,\alpha\}$ for every $V \in \Mnam$.
Define the relational vocabulary $\text{WL}(\Sch) = \{R_V \mid V \in \Mnam\}$ such that $\arity(R_V) = 2$ 
if $\size(V) = (\alpha, \alpha)$, $\arity(R_V) = 1$ if $\size(V) \in \{(\alpha,1), (1,\alpha)\}$, and 
$\arity(R_V) = 0$ otherwise.
Then given a matrix instance $\I = (\dom,\conc)$ over $\Sch$ with  $\dom(\alpha) = n$ define the structure 
$\text{WL}(\I) = (\{1, \ldots, n\}, \{R_V^{\I}\} )$ such that 
$R_V^{\I}(i, j) = \conc(V)_{i,j}$ if $\size(V) = (\alpha, \alpha)$, $R_V^{\I}(i) = \conc(V)_{i}$ 
if $\size(V) \in \{(\alpha,1), (1,\alpha)\}$, and $R_V^{\I} = \conc(V)$ if $\size(V) = (1,1)$.

Similar than in the proof of Proposition~\ref{prop:sum_to_ara}, for each expression $e(v_1, \ldots, v_k)$ of type $(\alpha, \alpha)$ we must encode in WL the $\alpha$ and the vector variables $v_1, \ldots, v_k$. For this, we use variables $x_{\alpha}^\row$, $x_{\alpha}^\col$, and $x_{v_i}$ for each variable $v_1, \ldots, v_k$. Then we use the following inductive hypothesis (similar to Proposition~\ref{prop:sum_to_ara}):

\newcommand{\varphie}{\varphi_e}
\newcommand{\xr}{x_{\alpha}^\row}
\newcommand{\xc}{x_{\alpha}^\col}

\begin{itemize}
	\item If $e(v_1,\ldots,v_k)$ is of type $(\alpha,\alpha)$ then there exists a WL formula $\varphie(x_{\alpha}^\row,x_{\alpha}^\col, x_{v_1}, \ldots, x_{v_k})$ such that
	$$
	\ssem{\varphi(e)}{\text{WL}(\I)}(\sigma) \ = \ \sem{e}{\I[v_1\gets b_{i_1},\ldots,v_k\gets b_{i_k}]}_{i,j}
	$$
	for assignment $\sigma$ with $\sigma(\xr)=i$, $\sigma(\xc)=j$ and $\sigma(x_{v_s})=i_s$ for $s=1,\ldots, k$.
	
	\item If $e(v_1,\ldots,v_k)$ is of type $(\alpha,1)$ then there exists a WL formula $\varphie(x_{\alpha}^\row, x_{v_1}, \ldots, x_{v_k})$ such that
	$$
	\ssem{\varphi(e)}{\text{WL}(\I)}(\sigma) \ = \ \sem{e}{\I[v_1\gets b_{i_1},\ldots,v_k\gets b_{i_k}]}_{i}
	$$
	for assignment $\sigma$ with $\sigma(\xr)=i$ and $\sigma(x_{v_s})=i_s$ for $s=1,\ldots, k$.
	And similarly for when $e$ is type $(1,\alpha)$.
	
	\item If $e(v_1,\ldots,v_k)$ is of type $(1,1)$ then there exists a WL formula $\varphie( x_{v_1}, \ldots, x_{v_k})$ such that
	$$
	\ssem{\varphi(e)}{\text{WL}(\I)}(\sigma) \ = \ \sem{e}{\I[v_1\gets b_{i_1},\ldots,v_k\gets b_{i_k}]}
	$$
	for assignment $\sigma$ with $\sigma(x_{v_s})=i_s$ for $s=1,\ldots, k$.
\end{itemize}
If we prove the previous statement we are done, because the last bullet is what we want to show when $e$ has no free vector variables. 
Then rest of the proof is to go by induction on the structure of \langprod expressions.
For a WL-formula $\varphi$ and FO-variables $x,y$, we will write  $\varphi[x \mapsto y]$ the formula $\varphi$ when $x$ is replaced with $y$ all over the formula (syntactically).
Let $e$ be a \langprod expression.
\begin{itemize} \itemsep3mm
  \item If $e:=V$ and $\Sch(e)= (\alpha, \alpha)$ then $\varphie:=R_V(\xr, \xc)$. Similarly, $\Sch(e)$ it is of type $(\alpha,1)$, $(1, \alpha)$, or $(1,1)$, then $\varphie:=R_V(\xr)$, $\varphie:=R_V(\xc)$, and $\varphie:=R_V$, respectively.
  
  \item If $e:=v$ and $\Sch(v)= (\alpha,1)$ then $\varphie := \xr = x_v$. Similar, if $\Sch(v)= (1,\alpha)$ then $\varphie := \xc = x_v$.
  
  \item if $e:= e_1^T$ and $\Sch(e)=(\alpha,\alpha)$ then
  $$
  \varphie:= \varphi_{e_1}[\xr \mapsto \xc, \xc \mapsto \xr].
  $$
  Similarly, if $\Sch(e)$ is equal to $(\alpha,1)$ or $(1,\alpha)$ then $\varphie:=\varphi_{e_1}[\xr \mapsto \xc]$ and $\varphie:=\varphi_{e_1}[\xc \mapsto \xr]$, respectively.   


	\item If $e=e_1+e_2$ with $\Sch (e_1)=\Sch (e_2)$, then $\varphie:= \varphi_{e_1} \ksum \varphi_{e_2}$.
	
	\item If $e=f_\odot(e_1,\ldots, e_k)$ with $\Sch(e_i)=\Sch(e_j)$ for all $i,j\in[1,k]$, then $\varphie:= \varphi_{e_1} \kprod \varphi_{e_2} \cdots \kprod \varphi_{e_k}$.
	
	\item If $e=e_1\cdot e_2$ with $\Sch (e_1)=\Sch (e_2)=(\alpha, \alpha)$,  then $\varphie:= \Sigma y. \  \varphi_{e_1}[\xc \mapsto y] \kprod \varphi_{e_2}[\xr \mapsto y]$ where $y$ is a fresh variable not mentioned in $\varphi_{e_1}$ or $\varphi_{e_2}$. Instead, if $\Sch (e_1)= (\alpha', 1)$ and $\Sch (e_2)=(1, \alpha'')$ with $\alpha', \alpha'' \in \{\alpha, 1\}$, then $\varphie := \varphi_{e_1} \kprod \varphi_{e_2}$.
	
	\item If $e=\ssum v. e_1(v)$, then we define $\varphie := \Sigma x_{v}. \  \varphi_{e_1}(x_v)$.

  \item If $e=\qhadprod v. e_1(v)$, then $\varphie := \sprod x_{v}.\  \varphi_{e_1}(x_v)$.
\end{itemize}
For the construction, it is straightforward to check that the inductive hypothesis holds for all cases. 

\medskip
\noindent \textbf{(WL to \langprod)} We now encode weighted structures into matrices and vectors. Let $\Gamma$ be a relational vocabulary 
where $\arity(R) \leq 2$. 
Define $\text{Mat}(\Gamma) = (\Mnam_\Gamma,\size_\Gamma)$ such 
that $\Mnam_\Gamma = \{ V_{R} \mid R \in \Gamma\}$ and $\size_\Gamma(V_{R})$ is equal to 
$(\alpha, \alpha), (\alpha, 1)$, or $(1,1)$ if $\arity(R)=2$, $\arity(R)=1$, or $\arity(R)=0$, 
respectively, for some $\alpha \in \DD$. Similarly, let $\cA = (A, \{R^{\cA}\}_{R \in \Gamma})$ 
be a structure with $A = \{a_1, \ldots, a_n\}$, ordered arbitrarily.
Then we define the matrix instance $\text{Mat}(\cA) = (\dom,\conc)$ such that $\dom(\alpha) = n$, 
$\conc(V_{R})_{i,j} = R^{\cA}(a_i, a_j)$ if $\arity(R)=2$, $\conc(V_{R})_{i,1} = R^{\cA}(a_i)$ if $\arity(R)=1$, 
and $\conc(V_{R})_{1,1} = R^{\cA}$ otherwise.

Let $\varphi$ be a formula over $\Gamma$.
\begin{itemize}
  \item If $\varphi:=x=y$ then $\Psi(\varphi):= \ssum v.v\cdot v^T$.
  \item If $\varphi:=R$ then $\Psi(\varphi):=V_R$.
  \item If $\varphi = \varphi_1 \ksum \varphi_2$ then $\Psi(\varphi):=\Psi(\varphi_1) + \Psi(\varphi_2)$.
  \item If $\varphi = \varphi_1 \kprod \varphi_2$ then $\Psi(\varphi):=\Psi(\varphi_1) \circ \Psi(\varphi_2)$.
  \item If $\varphi = \ssum x. \varphi_1$ then $\Psi(\varphi):=\ssum v.\Psi(\varphi_1)$.
  \item If $\varphi = \qhadprod x. \varphi_1$ then $\Psi(\varphi):=\qhadprod v.\Psi(\varphi_1)$.
\end{itemize}

The last two translations hold easily due to the fact that there are no free variables.

\end{proof}






\end{document}
\endinput
%%
%% End of file `sample-sigconf.tex'.
