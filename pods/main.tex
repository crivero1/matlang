%%
%% This is file `sample-sigconf.tex',
%% generated with the docstrip utility.
%%
%% The original source files were:
%%
%% samples.dtx  (with options: `sigconf')
%% 
%% IMPORTANT NOTICE:
%% 
%% For the copyright see the source file.
%% 
%% Any modified versions of this file must be renamed
%% with new filenames distinct from sample-sigconf.tex.
%% 
%% For distribution of the original source see the terms
%% for copying and modification in the file samples.dtx.
%% 
%% This generated file may be distributed as long as the
%% original source files, as listed above, are part of the
%% same distribution. (The sources need not necessarily be
%% in the same archive or directory.)
%%
%% The first command in your LaTeX source must be the \documentclass command.
\documentclass[sigconf]{acmart}

\settopmatter{printacmref=false} % Removes citation information below abstract
\renewcommand\footnotetextcopyrightpermission[1]{} % removes footnote with conference information in first column
\pagestyle{plain} % removes running headers

%%
%% \BibTeX command to typeset BibTeX logo in the docs
\AtBeginDocument{%
  \providecommand\BibTeX{{%
    \normalfont B\kern-0.5em{\scshape i\kern-0.25em b}\kern-0.8em\TeX}}}

%% Rights management information.  This information is sent to you
%% when you complete the rights form.  These commands have SAMPLE
%% values in them; it is your responsibility as an author to replace
%% the commands and values with those provided to you when you
%% complete the rights form.
%\setcopyright{acmcopyright}
%\copyrightyear{2018}
%\acmYear{2018}
%\acmDOI{10.1145/1122445.1122456}
%
%%% These commands are for a PROCEEDINGS abstract or paper.
%\acmConference[Woodstock '18]{Woodstock '18: ACM Symposium on Neural
%  Gaze Detection}{June 03--05, 2018}{Woodstock, NY}
%\acmBooktitle{Woodstock '18: ACM Symposium on Neural Gaze Detection,
%  June 03--05, 2018, Woodstock, NY}
%\acmPrice{15.00}
%\acmISBN{978-1-4503-XXXX-X/18/06}

\usepackage{xspace}
\newcommand{\paths}{\text{PATHS}}

%%%%%%%%%%%%%%%%%%%%%%%%%%%%%%%%
%			COMMENTS
%%%%%%%%%%%%%%%%%%%%%%%%%%%%%%%%
\usepackage[textwidth=2cm,textsize=small]{todonotes}

\newcommand{\domagoj}[1]{\todo[inline, color=blue!15]{{\bf Domagoj:} #1}}
\newcommand{\cristian}[1]{\todo[inline, color=orange!30]{{\bf Cristian:} #1}}
\newcommand{\thomas}[1]{\todo[inline, color=green!30]{{\bf Thomas:} #1}}
\newcommand{\floris}[1]{\todo[inline, color=red!30]{{\bf Floris:} #1}}

%%% UNCOMMENT BELOW TO REMOVE COMMENTS
%\renewcommand{\domagoj}[1]{}
%\renewcommand{\cristian}[1]{}
%\renewcommand{\thomas}[1]{}
%\renewcommand{\floris}[1]{}

%% Macros comments
\newcommand{\tover}[1]{\textcolor{red}{#1}}
\newcommand{\td}[1]{\textcolor{blue}{[TODO: #1]}}

%% Macros logics
\newcommand{\NN}{\mathbb{N}}
\newcommand{\ZZ}{\mathbb{Z}}
\newcommand{\MM}{\mathbb{M}}
\newcommand{\SE}{\mathbb{S}}
\newcommand{\BB}{\mathbb{B}}
\newcommand{\RR}{\mathbb{R}}
\newcommand{\cF}{\mathcal{F}}
\newcommand{\cI}{\mathcal{I}}
\newcommand{\cM}{\mathcal{M}}
\newcommand{\cS}{\mathcal{S}}
\newcommand{\cX}{\mathcal{X}}
\newcommand{\cV}{\mathcal{V}}
\newcommand{\QFBILIA}{\textsf{QFBILIA}}
\newcommand{\nnf}{\textsf{f}}
\newcommand{\Cat}{\operatorname{Cat}}
\newcommand{\ssum}{\textstyle\sum}
\newcommand{\sprod}{\textstyle\prod}
%\newcommand{\for}{\textbf{for }}

\newcommand{\ite}{\textsf{ite}}
\newcommand{\limp}{\Rightarrow}
\newcommand{\flc}{\rightarrow}


\newcommand{\bsingle}{\textsf{bag}}
\newcommand{\bplus}{\oplus}
\newcommand{\bminus}{\ominus}

%% Macros tools
\newcommand{\spen}{\textsc{spen}}
\newcommand{\zzz}{\textsc{Z3}}

%% Environments
\newtheorem{mythm}{Theorem}[section]
\newtheorem{mydef}[mythm]{Definition}
\newtheorem{myprop}[mythm]{Proposition}
\newtheorem{mylem}[mythm]{Lemma}
\newtheorem{myex}[mythm]{Example}
\newtheorem{mycor}[mythm]{Corollary}

\newtheorem*{myrem}{Remark} %% based on amsthm
\newtheorem*{mynota}{Notation}
\newtheorem*{mylem*}{Lemma}
\newtheorem{myclaim}{Claim}
\newtheorem*{myclaim*}{Claim}
\newtheorem*{myprop*}{Proposition}
\newtheorem*{comp}{Efficiency Study}


\newenvironment{point}[1]
{\subsection*{#1}}%
{}

\newcommand{\flist}{\text{\sc FList}}
\newcommand{\set}{\text{\sc Set}}
\newcommand{\fset}{\text{\sc FSet}}
\newcommand{\B}{\text{\bf B}}
%\newcommand{\G}{\mathcal{G}}
\newcommand{\K}{\mathcal{K}}
\newcommand{\LOG}{\text{\sc Log}}
\newcommand{\length}{\text{\rm length}}
\newcommand{\BK}{\text{\sc BK}}
%\newcommand{\cP}{\mathcal{P}}
%\newcommand{\cV}{\mathcal{V}}
%\newcommand{\cC}{\mathcal{C}}
%\newcommand{\cS}{\mathcal{S}}
%\newcommand{\cA}{\mathcal{A}}
%\newcommand{\cR}{\mathcal{R}}
%\newcommand{\cQ}{\mathcal{Q}}
\newcommand{\cG}{\mathcal{G}}
\newcommand{\cT}{\mathcal{T}}
\newcommand{\pr}{\mathbf{Pr}}
\newcommand{\Dyck}{\mathcal{D}}
\newcommand{\expected}{\mathbf{E}}
\newcommand{\bv}{\mathbf{v}}
\newcommand{\bV}{\mathbf{V}}
\newcommand{\bs}{\mathbf{s}}
\newcommand{\bsigma}{\mathbf{\sigma}}
\newcommand{\bw}{\mathbf{w}}
\newcommand{\ba}{\mathbf{a}}
\newcommand{\bq}{\mathbf{q}}
\newcommand{\bx}{\mathbf{x}}
\newcommand{\by}{\mathbf{y}}

\newcommand{\bstring}{\{0,1\}^\ast}

\newcommand{\quot}[1]{#1/\!\equiv}

\newcommand{\then}{\,|\,}
\newcommand{\body}{q}
\newcommand{\bchain}{\text{bc}}

\newcommand{\owner}{\text{\rm owner}}
\newcommand{\pred}{\text{\rm pred}}
\newcommand{\mine}{\text{\rm mine}}
\newcommand{\suc}{\text{\rm succ}}


\newcommand{\bP}{\mathbf{P}}
\newcommand{\bB}{\mathbf{B}}
\newcommand{\bA}{\mathbf{A}}
\newcommand{\bR}{\mathbf{R}}
\newcommand{\bS}{\mathbf{S}}
\newcommand{\bH}{\mathbf{H}}
\newcommand{\bQ}{\mathbf{Q}}

\newcommand{\forkm}[1]{F^{#1}}
\newcommand{\mfork}{F_m}
\newcommand{\mgfork}{F_{m,g}}
\newcommand{\last}{\text{\rm last}}
\newcommand{\best}{\text{\rm best}}
\newcommand{\cho}{\text{\rm choose}}

\newcommand{\ie}{i.e.$\!$ }

\newcommand{\longest}{{\text{\rm longest}}}

\newcommand{\subbody}{{\text{\rm sub-state}}}

\newcommand{\df}{\text{\rm DF}}
\newcommand{\fg}{\text{\rm FG}}
\newcommand{\fr}{\text{\rm FR}}
\newcommand{\bdf}{\text{\rm {\bf DF}}}
\newcommand{\bfg}{\text{\rm {\bf FG}}}
\newcommand{\bfr}{\text{\rm {\bf FR}}}

\newcommand{\meet}{\text{\rm meet}}

\DeclareMathOperator*{\argmax}{argmax}


%\newcommand{\rpa}{r_p^\alpha}
\newcommand{\rpa}{r_p}

\newcommand{\ameet}{\text{\rm all-meet}}
\newcommand{\base}{\text{\rm base}}

\newcommand{\ucl}{\text{\rm up-cl}}
\newcommand{\dcl}{\text{\rm down-cl}}


\newcommand{\sem}[2]{\llbracket #1 \rrbracket(#2)}


\newcommand{\Mnam}{\mathcal{M}}
\newcommand{\Mvar}{\mathcal{V}}
\newcommand{\Fun}{\mathcal{F}}
\newcommand{\Mlang}{\text{MATLANG}}
\newcommand{\llet}{\texttt{let } M = e_1 \texttt{ in } e_2}
\newcommand{\ones}{\mathbf{1}}
\newcommand{\diag}{\texttt{diag}}
\newcommand{\apply}[1]{\texttt{apply}[#1]}

\newcommand{\I}{\mathcal{I}}
\newcommand{\Voc}{\mathcal{S}}
\newcommand{\Sch}{\mathcal{S}}

\newcommand{\mtr}[1]{\texttt{Matrices}[#1]}

\newcommand{\dom}{\mathcal{D}}
\newcommand{\conc}{\texttt{mat}}

\newcommand{\DD}{\texttt{Symb}}
\newcommand{\size}{\texttt{size}}

\newcommand{\ddim}{\texttt{dim}}
\newcommand{\ttype}{\texttt{type}_{\Sch}}
\newcommand{\type}{\texttt{type}}

\newcommand{\lang}{\texttt{MATLANG}}
\newcommand{\langfor}{\texttt{for}-\texttt{MATLANG} }

\newcommand{\ffor}[3]{\texttt{for}\, #1,#2 \texttt{.}\, #3}

%\newcommand{\initf}[4]{\texttt{for}[#1]\, #2,#3 \texttt{.}\, #4}
\newcommand{\initf}[4]{\texttt{for}\, #2,#3\!=\! #1 \texttt{.}\, #4}

\newcommand{\lleq}[2]{\texttt{leq}(#1,#2)}
\newcommand{\mmin}[1]{\texttt{min}(#1)}

\newcommand{\ccol}[2]{\texttt{col(}#1,#2\texttt{)}}
\newcommand{\red}[2]{\texttt{reduce(}#1,#2\texttt{)}}

%%
%% Submission ID.
%% Use this when submitting an article to a sponsored event. You'll
%% receive a unique submission ID from the organizers
%% of the event, and this ID should be used as the parameter to this command.
%%\acmSubmissionID{123-A56-BU3}

%%
%% The majority of ACM publications use numbered citations and
%% references.  The command \citestyle{authoryear} switches to the
%% "author year" style.
%%
%% If you are preparing content for an event
%% sponsored by ACM SIGGRAPH, you must use the "author year" style of
%% citations and references.
%% Uncommenting
%% the next command will enable that style.
%%\citestyle{acmauthoryear}

%%
%% end of the preamble, start of the body of the document source.
\begin{document}

%%
%% The "title" command has an optional parameter,
%% allowing the author to define a "short title" to be used in page headers.
\title{Query language for Linear Algebra}

%%
%% The "author" command and its associated commands are used to define
%% the authors and their affiliations.
%% Of note is the shared affiliation of the first two authors, and the
%% "authornote" and "authornotemark" commands
%% used to denote shared contribution to the research.
\author{Floris Geerts}
\email{floris.geerts@uantwerpen.be}
\affiliation{%
  \institution{University of Antwerp}
%  \streetaddress{P.O. Box 1212}
%  \city{Dublin}
%  \state{Ohio}
%  \postcode{43017-6221}
}

\author{Thomas Mu\~noz}
\affiliation{%
  \institution{PUC Chile}
%  \streetaddress{1 Th{\o}rv{\"a}ld Circle}
%  \city{Hekla}
%  \country{Iceland}
}
\email{tfmunoz@uc.cl}

\author{Cristian Riveros}
\affiliation{%
  \institution{PUC Chile and IMFD Chile}
%  \streetaddress{1 Th{\o}rv{\"a}ld Circle}
%  \city{Hekla}
%  \country{Iceland}
}
\email{cristian.riveros@uc.cl}

\author{Domagoj Vrgo\v{c}}
\affiliation{%
  \institution{PUC Chile and IMFD Chile}
%  \streetaddress{1 Th{\o}rv{\"a}ld Circle}
%  \city{Hekla}
%  \country{Iceland}
}
\email{dvrgoc@ing.puc.cl}

%%
%% By default, the full list of authors will be used in the page
%% headers. Often, this list is too long, and will overlap
%% other information printed in the page headers. This command allows
%% the author to define a more concise list
%% of authors' names for this purpose.
%\renewcommand{\shortauthors}{Trovato and Tobin, et al.}

%%
%% The abstract is a short summary of the work to be presented in the
%% article.
\begin{abstract}
Abstract
\end{abstract}

%%
%% The code below is generated by the tool at http://dl.acm.org/ccs.cfm.
%% Please copy and paste the code instead of the example below.
%%%
%\begin{CCSXML}
%<ccs2012>
% <concept>
%  <concept_id>10010520.10010553.10010562</concept_id>
%  <concept_desc>Computer systems organization~Embedded systems</concept_desc>
%  <concept_significance>500</concept_significance>
% </concept>
% <concept>
%  <concept_id>10010520.10010575.10010755</concept_id>
%  <concept_desc>Computer systems organization~Redundancy</concept_desc>
%  <concept_significance>300</concept_significance>
% </concept>
% <concept>
%  <concept_id>10010520.10010553.10010554</concept_id>
%  <concept_desc>Computer systems organization~Robotics</concept_desc>
%  <concept_significance>100</concept_significance>
% </concept>
% <concept>
%  <concept_id>10003033.10003083.10003095</concept_id>
%  <concept_desc>Networks~Network reliability</concept_desc>
%  <concept_significance>100</concept_significance>
% </concept>
%</ccs2012>
%\end{CCSXML}
%
%\ccsdesc[500]{Computer systems organization~Embedded systems}
%\ccsdesc[300]{Computer systems organization~Redundancy}
%\ccsdesc{Computer systems organization~Robotics}
%\ccsdesc[100]{Networks~Network reliability}
%
%%%
%%% Keywords. The author(s) should pick words that accurately describe
%%% the work being presented. Separate the keywords with commas.
%\keywords{datasets, neural networks, gaze detection, text tagging}
%%%
%% This command processes the author and affiliation and title
%% information and builds the first part of the formatted document.
\maketitle

\section{Introduction}
% !TeX spellcheck = en_US
%!TEX root = ../main.tex

Linear algebra-based algorithms have become a key component in data analytical workflows. As such, there is a growing interest in the database community to integrate linear algebra functionalities into relational database management systems \cite{Jermaine/17/LAonRA,2019Boehm,LARA_Berlin_2016,JankovLYCZJG19,Khamis0NOS18}. In particular, from a query language perspective, several proposals have recently been put forward to unify relational algebra and linear algebra. Two notable examples of this are: \lara~\cite{HutchisonHS17}, a minimalistic language in which a number of atomic operations on 
%so-called 
associative tables are proposed, and \lang, a query language for 
%manipulating 
matrices \cite{matlang-journal}.\looseness=-1

Both \lara and \lang have been studied by the database theory community, showing interesting connections to relational algebra and logic. For example, fragments of \lara are known to capture first-order logic with aggregation~\cite{BarceloH0S20}, and \lang has been recently shown to be equivalent to a restricted version of the (positive) relational algebra on $K$-relations, \rak~\cite{brijder2019matrices}, where $K$ denotes a semiring. On the other hand, 
%there are 
some standard constructions in linear algebra 
%that 
are out of reach for these languages. For instance, it was shown that under standard complexity-theoretic assumptions, \lara can not compute the inverse of a matrix or its determinant~\cite{BarceloH0S20}, and operations such as the transitive closure of a matrix are known to be inexpressible in \lang~\cite{matlang-journal}. Given that these are fundamental constructs in linear algebra, one might wonder how to extend \lara or \lang in order to allow expressing such properties.

One approach would be to add these constructions explicitly to the language. Indeed, this was done for \lang in~\cite{matlang-journal}, and \lara in ~\cite{BarceloH0S20}. In these works, the authors have extended the core language with the trace, the inverse, the determinant, or the eigenvectors operators and study the expressive power of the result. However, one can argue that there is nothing special in these operators, apart they have been used historically in linear algebra textbooks and they extend the expressibility of the core language. The question here is whether these new operators form a sound and natural choice to extend the core language, or are they just some particular queries that we would like to support. 

In this paper we take a more principled approach by studying what are the atomic operations needed to define standard linear algebra algorithms. Inspecting any linear algebra textbook, one sees that most linear algebra procedures heavily rely on the use of for-loops in which iterations happen over the dimensions of the matrices involved. To illustrate this, let us consider the example of computing the transitive closure of a graph. This can be done using a modification of the Floyd-Warshall algorithm~\cite{cormen}, which takes as its input an $n\times n$ adjacency matrix $A$ representing our graph, and operates according to the following pseudo-code:
%\vspace{-1ex}
\begin{tabbing}
\quad\texttt{for}\=\,  $k = 1..n$ \texttt{do}\\
\> \texttt{for}\=\,  $i = 1..n$ \texttt{do}\\
\> \> \texttt{for}\=\,  $j = 1..n$ \texttt{do}\\
\> \> \> $A[i,j] := A[i,j] + A[i,k] \cdot A[k,j]$
\end{tabbing}
%\vspace{-1ex}
After executing the algorithm, all of the non zero entries signify an edge in the (irreflexive) transitive closure graph.
%  (and in fact, the number of paths of length at most $n$ between the two nodes).
%
\cristian{I am not sure about this sentence. Depending on the order how is iterated, maybe you can compute more than $n$. Actually, if you don't add the identity to the original matrix, even the transitive closure will not work.}
%\cristian{This is not exactly true. If we use $\cdot$ as $+$ and $+$ as min, it works, namely if we use the min/plus semiring. Should we say it?}

% FOR THE REAL ALGORITHM WE TREAT ZERO AS INFINITY!!!
%\vspace{-1ex}
%\begin{tabbing}
%\quad\texttt{for}\=\,  $k = 1..n$ \texttt{do}\\
%\> \texttt{for}\=\,  $i = 1..n$ \texttt{do}\\
%\> \> \texttt{for}\=\,  $j = 1..n$ \texttt{do}\\
%\> \> \> \texttt{if} $A[i,j]$\=\, $> A[i,k] + A[k,j]$\\
%\> \> \> \> $A[i,j] := A[i,k]+ A[k,j]$
%\end{tabbing}
%%\vspace{-1ex}
%After executing the algorithm, all of the non zero entries signify an edge in the transitive closure graph (and also the length of the shortest path between the two nodes). 


By
%When 
examining standard linear algebra algorithms such as Gaussian elimination, $LU$-decomposition, computing the inverse of a matrix, or its determinant, we can readily see that this pattern continues. Namely, we observe that there are two main components to such algorithms: (i) the ability to iterate up to the matrix dimension; and (ii) the ability to access a particular position in our matrix. In order to allow this behavior in a query language, we propose to extend \lang with limited recursion in the form of for-loops, resulting in the language \langfor. To  simulate the two components of standard linear algebra algorithms in a natural way, we simulate a loop of the form \texttt{for}\, $i=1..n$ \texttt{do} by leveraging canonical vectors. In other words, we use the canonical vectors $b_1=(1,0,\ldots)$, $b_2=(0,1,\ldots)$, \ldots, to access specific rows and columns, and iterate over these vectors. In this way,
% As a consequence of this,
we obtain a language able to compute important linear algebra operators such as $LU$-decomposition, determinant, matrix inverse, among other things.

Of course, a natural question to ask now is whether this really results in a language suitable for linear algebra? We argue that the correct way to approach this question is to compare our language to arithmetic circuits, which have been shown to capture the vast majority of existing matrix algorithms, from basic ones such as computing the determinant and the inverse, to complex procedures such as discrete Fourier transformation, and Strassen's algorithm (see \cite{ShpilkaY10,allender} for an overview of the area), and can therefore be considered to effectively capture linear algebra. In the main technical result of this paper, we show that \langfor indeed computes the same class of functions over matrices as the ones computed by arithmetic circuit families of bounded degree.  As a consequence, \langfor inherits all expressiveness properties of circuits, and thus can simulate any linear algebra algorithm definable by circuits.

Having established that \langfor indeed provides a good basis for a linear algebra language, we move to a more fine-grained analysis of the expressiveness of its different fragments. For this, we aim to provide a connection with logical formalisms, similarly as was done by linking \lara and \lang to first-order logic with aggregates~\cite{BarceloH0S20,matlang-journal}. As we show, capturing different logics correspond to restricting how matrix variables are updated in each iteration of the for-loops allowed in \langfor. For instance, if we only allow to add some temporary result to a variable in each iteration (instead of rewriting it completely like in any programming language), we obtain a language, called \langsum, which is equivalent to \rak, directly extending an analogous result shown for \lang, mentioned earlier~\cite{brijder2019matrices}. We then study updating matrix variables based on another standard linear algebra operator, the Hadamard product, resulting in a fragment called \langprod, which we show to be equivalent to weighted logics~\cite{DrosteG05}. Finally, in \langmprod 
we 
%consider updating 
update the variables based on the standard matrix product, and link this fragment 
%of our language 
to the ones discussed previously.  

\smallskip
\noindent
\textbf{Contribution and outline.} 
% Our main contributions can be summarized as follows.
\begin{itemize}[leftmargin=0.5cm]
	\item After we recall \lang in Section~\ref{sec:matlang}, we show in Section~\ref{sec:formatlang}
	how for-loops can be added to \lang in a natural way. We also observe that
	\langfor strictly extends \lang. In addition, we discuss some design decisions behind the definition of \langfor, noting that our use of canonical vectors results in the availability of an order relation.
	
	\item In Section~\ref{sec:queries} we show that \langfor can compute important linear algebra algorithms in a natural way. We provide expressions in \langfor for LU decomposition (used to solve linear systems of equations), the determinant and matrix inversion.
	\item More generally, in Section~\ref{sec:circuits} we report our main technical contribution.
	 We show that every  uniform arithmetic circuits of polynomial degree correspond to a \langfor expression, and vice versa, when a \langfor expression has polynomial degree, then there is an equivalent uniform family of arithmetic circuits. As a consequence, \langfor inherits all expressiveness properties of such circuits.
	\item  Finally, in Section~\ref{sec:restrict} we generalize the semantics of \langfor to matrices with values in a semiring $K$, and show that two natural fragment of \langfor, \langsum, and \langprod, are equivalent to the (positive) relational algebra and weighted logics on binary $K$-relations, respectively. We also briefly comment on a minimal fragment of \langfor, based on \langmprod, that is able to compute matrix inversion.
\end{itemize}
Due to space limitations, most proofs are referred to the appendix.
%
% \floris{This is example is getting too complicated! Any suggestions?}
% % %
% \begin{example}
% Consider a linear system of equations $L\cdot x=a$ with $L$ a matrix of dimensions $n\times n$, $a$ a vector of dimension $n\times 1$, and $x$ a vector of variables of dimensions $n\times 1$. Furthermore, assume that $L$ is a non-singular lower triangular matrix, i.e., all entries above the diagonal are zero and all entries on the diagonal are non-zero. To solve the system for $x$, it suffices to apply forward substitution, i.e.,
% $$	x_1:= a_1, \quad
% x_i:= \frac{1}{L_{ii}}\left(a_i -\sum_{j=1}^{i-1}L_{ij}a_j\right) \quad i\in[2,n]$$
% To view this procedure as a query in \langfor we proceed as follows.
% We use a matrix variable $M$ to store $L$ and a vector variable $V$ to store $a$.
% Furthermore, we reserve two special vector variables $v$ and $w$ which will range over
% the canonical vectors of the same dimension as $L$ and $a$. More specifically, they range
% over $b_1=(1,0,\ldots,0)$, $b_2=(0,1,0,\ldots,0)$,\ldots, $b_n=(0,\ldots,0,1)$ \textit{in this order}. We further have a special variable $X$ to hold intermediate results. In this example,
% $X$ will hold the solution $x$ for the system of equations at the end of the evaluation. We also
% require a second variable $Y$ which will hold $\sum_{j=1}^{i-1}L_{ij}a_j$ for a given $i$. Finally,
% we also allow the use the division function $f_/(x,y)=x/y$.
%
% We need to more operators in this example:
% the operator $\mathsf{min}(v)$ such that $\mathsf{min}(v):=1$ if $v=b_1$ and $\mathsf{min}(v):=0$ otherwise, and the $\mathsf{pred}^+(v,w)$ such that $\mathsf{pred}^+(b_j,b_i)=1$
%  if $j<i$ and $\mathsf{pred}^+(b_j,b_i)=0$ otherwise. These identify the first canonical vector and a strict predecessor relation on canonical vectors, respectively, We show later in the paper that these can be expressed in \langfor. Given these, we consider the following \langfor\ expressions:
% \begin{tabbing}
% $e_1(M,V,v)$\=:=\texttt{for}$\,w,Y\,.\, Y + \mathsf{pred}^+(w,v) (v^T\cdot M\cdot w)(w^T\cdot V)\times v$\\
% $e(M,V)$\=:=\texttt{for}$\,v,X\,.\, X + (\mathsf{min}(v)(v^T\cdot V)\times v) +(1-\mathsf{min}(v))e_1(M,V,v)$
% \end{tabbing}
% \end{example}
% % %


\smallskip
\noindent
\textbf{Related work.} 
We already mentioned \lara~\cite{HutchisonHS17} and \lang~\cite{matlang-journal}
whose expressive power was further analyzed in~\cite{BarceloH0S20,brijder2019matrices,Geerts19,Geerts20}.
Extensions of \texttt{SQL} for matrix manipulations are reported in~\cite{Jermaine/17/LAonRA}. Most relevant
is~\cite{JankovLYCZJG19} in which a recursion mechanism is added to \texttt{SQL} which resembles for-loops.
The expressive power of this extension is unknown, however. Classical logics with aggregation~\cite{Hella:2001} and fixed-point logics with counting~\cite{GroheP17} can also be used for linear algebra. More generally, for the descriptive complexity of linear algebra we refer to~\cite{dghl_rank,holm_phd}. Most of these works require to encode real numbers inside relations, whereas we treat real numbers as atomic values. We refer to relevant papers related to arithmetic circuits and logical formalisms on semiring-annotated relations in the corresponding sections later in the paper.


% \subsection{old stuff}
% In another line of work~\cite{}, matrices are encoded as relational tables and an extensions of SQL is proposed to carry out matrix manipulations. In particular, \cite{} extends SQL with a limited form of recursion -- alike dynamic programming - such that linear algebra-based procedures for learning feed-forward neural networks can be declaratively specified. To our knowledge, the precise expressive power of the resulting language has not been characterized. For example, it is unclear whether matrix inversion can be expressed.
%
% Based on these works, a natural question arises: how to add a natural form of recursion to linear algebra-based query languages? Inspecting any linear algebra textbook, one sees that most linear algebra procedures heavily rely on the use of for-loops in which iterations happen over the dimensions of the matrices involved. We thus propose to extend \lang\ with limited recursion in the form of for-loops, resulting the language \langfor. To define this recursion in a natural way, we simulate a loop of the form \texttt{for i=1,...n do} by leveraging canonical vectors. In other words, we use the canonical vectors $b_1=(1,0,\ldots)$, $b_2=(0,1,\ldots)$, \ldots, to access specific rows and columns, and iterate over these vectors. As an almost direct consequence, the expressive power of \langfor goes well beyond \lang. It can check for cliques of any given size, compute the transitive closure of a graph, and as we will show, compute important linear algebra operators such as LU-decomposition, determinant, matrix inverse, among other things.
%
% More generally, we show that \langfor\ is closely related to arithmetic circuits and we show that anything computable by an arithmetic circuit of polynomial degree can be computed in \langfor, and vice versa, provided that \langfor expressions can be compiled, in a uniform way into arithmetic circuits. Since these circuits are often said to ``capture'' linear algebra, we see our results as as a justification for our language.
%
% Furthermore, the introduction of recursion to \lang has some interesting consequences. First of all, when consecutive iterations can only perform updates in an additive way, we show that \langfor and the annotated relation algebra are equivalent. Secondly, when iterations update in a multiplicative manner,
% \langfor is equivalent to weighted logics.
%

%
% \begin{itemize}
% \item Explain why this is important.
% \item Say what sort of things we would like to express.
% \item Give a quick tour of our minimal language (examples).
% \item Stress our concrete contributions.
% \item Relate to previous work \cite{matlang,BrijderGBW19,Geerts19,HutchisonHS17}.
% \end{itemize}
%
% \domagoj{A good example for the intro is all shortest paths via Floyd-Warshall.}
%
% \noindent
% $\ffor{e_k}{\Dist}{ }$
% \\
% \hspace*{0.5cm} $\ffor{e_i}{\Dist}{ }$
% \\
% \hspace*{1cm} $\ffor{e_j}{\Dist}{ }$
% \\
% \hspace*{1.5cm}
% $\texttt{curr} := e_i^*\cdot \Dist \cdot e_j$\\
% \hspace*{1.5cm}
% $\texttt{new} := e_i^*\cdot \Dist \cdot e_k + e_k^*\cdot \Dist\cdot e_j$\\
% \hspace*{1.5cm}
% $\Dist + \texttt{update}(\texttt{curr},\texttt{new})\times (e_i\cdot e_j^*)$
%
% where
%
% \[
%   			\texttt{update}(x,y)=\begin{cases}
%                0, \text{ if } x<=y \\
%                -x + y, \text{ if } x > y
%             \end{cases}.
% 		\]




\section{Preliminaries}
We start by recalling the matrix query language \lang, introduced in \cite{matlang,matlang-journal}, which serves as our starting point.

\smallskip
\noindent
\textbf{Syntax.}\,  Let $\Mvar = \{V_1, V_2, \ldots\}$ be a countably infinite set of \textit{matrix variables} and $\Fun=\bigcup_{k>1}\Fun_k$ with
$\Fun_k$ a set of \textit{functions} of the  form $f:\RR^k \to \RR$, where $\RR$ denotes the set of real numbers. The syntax of \lang\ expressions is defined by the following grammar\footnote{The original syntax also permits the operator $\llet$, which replaces every occurrence of $V$ in $e_2$ with the value of $e_1$. Since this is just syntactic sugar, we omit this operator. We also explicitly include matrix addition and scalar multiplication, although these can be simulated by pointwise function applications. Finally, we use transposition instead of conjugate transposition since we work with matrices over $\RR$.}:


\begin{tabular}{lcll}
$e$ & $::=$ & $V\in \Mvar$ & (matrix variable)\\
 & $|$ & $e^T$ & (transpose)\\ 
 & $|$ & $\ones(e)$ & (one-vector)\\ 
 & $|$ & $\diag(e)$ & (diagonalization of a vector)\\  
 & $|$ & $e_1 \cdot e_2$ & (matrix multiplication)\\   
 & $|$ & $e_1 + e_2$ & (matrix addition)\\   
 & $|$ & $a\times e$ & (scalar multiplication, $a\in\RR$)\\   
  & $|$ & $f(e_1,\ldots ,e_k)$ & (pointwise application of $f\in\Fun_k$).    
\end{tabular}
\vspace{1ex}

$\lang$ is parametrized by a collection of functions $\Fun$ but in the remainder of the paper we only make this dependence explicit, and write
\lang$(\Fun)$, for some set $\Fun$ of functions, when these functions are crucial
for some results to hold. 
When we simply write \lang, we mean that any function can be used (including not using any function at all).
%When we simply write \lang, we mean that the only functions used are binary multiplication and addition.
%\domagoj{Double check the last part with the rest.}



\smallskip
\noindent
\textbf{Schemas and typing.}\,
To define the semantics of \lang\ expressions we need a notion of schema and well-typedness of expressions. A \lang\ \textit{schema} $\Sch$ is a pair $\Sch=(\Mnam,\size)$, where $\Mnam\subset \Mvar$ is a finite set of matrix variables, and $\size: \Mnam \mapsto \DD\times \DD$ is a function that maps each matrix variable in $\Mnam$ to a pair of \textit{size symbols}. The $\size$ function helps us determine whether certain matrix operations, such as matrix multiplication, can be performed for matrices adhering to a schema. 
We denote size symbols by greek letters $\alpha,\beta,\gamma$. We also assume that $1\in \DD$. 
To help us determine whether a \lang\ expression can always be evaluated, we define the \textit{type} of an expression $e$, with respect to a schema $\Sch$, denoted by $\ttype(e)$, inductively as follows:
\begin{itemize}
\item $\ttype(V):= \size(V)$, for a matrix variable $V\in\Mnam$;
\item $\ttype(e^T):= (\beta,\alpha)$ if $\ttype(e)=(\alpha,\beta)$;
% , and undefined if $\ttype(e)$ is undefined;
\item $\ttype(\ones(e)):= (\alpha,1)$ if $\ttype(e)=(\alpha,\beta)$;
 % and undefined if $\ttype(e)$ is undefined;
\item $\ttype(\diag(e)):= (\alpha,\alpha)$, if $\ttype(e)=(\alpha,1)$;
% , and undefined otherwise;
\item $\ttype(e_1 \cdot e_2):= (\alpha,\gamma)$ if  $\ttype(e_1)=(\alpha,\beta)$, and $\ttype(e_2)=(\beta,\gamma)$;
 % and undefined otherwise;
\item $\ttype(e_1 + e_2):=(\alpha,\beta)$ if $\ttype(e_1)=\ttype(e_2)=(\alpha,\beta)$;
 % and undefined otherwise;
\item $\ttype(a\times e):=(\alpha,\beta)$ if $\ttype(e)=(\alpha,\beta)$; and
\item $\ttype(f(e_1,\ldots ,e_k)):= (\alpha,\beta)$, whenever $\ttype(e_1) = \cdots = \ttype(e_k) := (\alpha,\beta)$ and $f\in\Fun_k$.
 % and is undefined otherwise.
\end{itemize}
When $\Sch$ is clear from the context we simply write $\type(e)$. We call an expression \textit{well-typed} according to the schema $\Sch$, if it has a defined type. 
A well-typed expression can be evaluated regardless of the actual sizes of the matrices assigned to matrix variables, as we describe next.

% $\dim(M)$ gives the dimension of the matrix $M$, where $\dim(M)\in \mathbb{N}^2$. We
\smallskip
\noindent
\textbf{Semantics.}\, We use $\mtr{\RR}$ to denote the set of all real matrices and for 
$A\in\mtr{\RR}$, $\dim(A)\in\NN^2$ denotes its dimensions.
 % over some field $\mathbb{F}$.
A (\lang) \textit{instance} $\I$ over a schema $\Sch$ is a pair $\I = (\dom,\conc)$, where $\dom : \DD \mapsto \NN$ assigns a value to each size symbol (and thus in turn  dimensions to each matrix variable), and $\conc : \Mnam \mapsto \mtr{\RR}$ assigns a concrete matrix to each matrix variable $V\in \Mnam$, such that $\dim(\conc(V)) = \dom(\alpha)\times \dom(\beta)$ if $\size(V) = (\alpha,\beta)$. That is, an instance tells us the dimensions of each matrix variable, and also the concrete matrices assigned to the variable names in $\Mnam$. We assume that $\dom(1) = 1$, for every instance $\I$. If $e$ is a well-typed expression according to $\Sch$, then we denote by $\sem{e}{\I}$ the matrix obtained by evaluating $e$ over $\I$, and define it as follows:
\begin{itemize}
\item $\sem{V}{\I} := \conc(V)$, for $V\in \Mnam$;
\item $\sem{e^T}{\I} := \sem{e}{\I}^T$, where $A^T$ is the transpose of a matrix $A$;
\item $\sem{\ones(e)}{\I}$ is a $n\times 1$ vector with $1$ as all of its entries, where $\dim(\sem{e}{\I})=(n,m)$;
\item $\sem{\diag(e)}{\I}$ is a diagonal matrix with the vector $\sem{e}{\I}$ on its main diagonal, and zero in every other position;
\item $\sem{e_1\cdot e_2}{\I} := \sem{e_1}{\I} \cdot \sem{e_2}{\I}$;
\item $\sem{e_1+ e_2}{\I} := \sem{e_1}{\I} + \sem{e_2}{\I}$;
\item $\sem{a\times e}{\I} := a\times \sem{e}{\I}$; and
\item $\sem{f(e_1,\ldots ,e_k)}{\I}$ is a matrix $A$ of the same size as $\sem{e_1}{\I}$, and where $A_{ij}$ has the value $f(\sem{e_1}{\I}_{ij},\ldots ,\sem{e_k}{\I}_{ij})$.
\end{itemize}
This concludes the description and semantics of \lang. We next provide some simple example.

\begin{example}Consider the \lang$(f_\odot)$ expression with $f_\odot:\RR^2\to\RR:(x,y)\mapsto x\cdot y$:
$$\mathsf{cwalk}:= (\ones(V))^T\cdot f_{\odot}\bigl(V\cdot V, \diag(\ones(V)\bigr)\cdot\ones(V).$$
Let $\Sch$  consist of $\Mnam:=\{V\}$ and $\size(V):=(\alpha,\alpha)$ such that
matrices assigned to $V$ by instances $\I$ over $\Sch$ are square matrices.
It is readily verified that $\mathsf{cwalk}$ is well-typed and more specifically, $\ttype(\mathsf{cwalk})=(1,1)$, i.e., it returns an element of $\RR$ on any  instance $\I$. Let $\I$ be such that $\dom(\alpha)=n$ and  $\conc(V)$ is an adjacency matrix $A$ of an undirected graph $G$ consisting of $n$ vertices. Then, it is readily verified that $\sem{\mathsf{cwalk}}{\I}$ returns the number of paths of length two in $G$ which start in and end at the same vertex.\qed
\end{example}
Although \lang\ forms a solid basis for a matrix query language, it is limited in expressive power. Indeed, \lang\ is subsumed by first order logic with aggregates that uses only three variables \cite{matlang}. As consequence, no \lang\ expression exists that can compute the transitive closure of a graph (represented by its adjacency matrix) or can compute the inverse of a matrix. Furthermore, no \lang\ expression exists which detects four-cliques in a graph \cite{matlang}.
Also, $\lang$ is not expressive enough to perform classical linear algebra algorithms such as LU-decomposition (Gaussian elimination).
 Rather than extending \lang\ with specific linear algebra operators, such as matrix inversion, we next introduce a limited form of recursion in \lang.
As we will see shortly, this extension allows us to express many linear algebra algorithms, including matrix inversion and LU-decomposition.
\floris{The paragraph above needs to be modified as it may overlap too much with the yet-to-be-written Introduction.}

%FIRST TRY

%
%
% Recall the basics of \lang\ \cite{matlang}, and linear algebra. Perhaps stress where \lang\ falls short with respect to natural linear algebra questions.
%
% \bigskip

%
%Since the baseline for our study is the \lang\ language introduced in \cite{matlang}, here we briefly recap its syntax and semantics. Let $\Mnam = \{M_1, M_2,\ldots\}$ be a countably infinite set of {\em matrix names}, $\Mvar = \{V_1, V_2, \ldots\}$ a countably infinite set of {\em matrix variables}, and $\Fun$ a set of functions $f:\mathbb{C}^n \mapsto \mathbb{C}$, where $\mathbb{C}$ denotes the set of complex numbers. A {\em vocabulary} $\Voc$ is a triple $\Voc = (\Mnam', \Mvar, \Fun)$, where $\Mnam'\subset \Mnam$ is a finite subset of matrix names. An {\em $\Voc$-instance} $\I$ maps every $M\in \Mnam'$ to a concrete matrix, and assigns a dimension $(m,n)$, with $m,n\in \mathbb{N}$ to every matrix variable. That is, if $M\in \Mnam'$, then $\I(M)$ is a matrix over $\mathbb{C}$ of some dimension, and if $V\in \Mvar$, then $\ddim(\I(V)) = (m,n)$; that is, $V$ is a placeholder for a matrix of a specific dimension\footnote{Note that in \cite{matlang} the authors introduce the notion of abstract typing for vocabulary symbols. To simplify the notation, and stay closer to standard definitions of First order logic, we opt to assign matrix types to variables directly on an instance level.}.
%
%The syntax of \lang\ expressions over the vocabulary $\Voc$ is defined by the following grammar:
%
%\begin{tabular}{lcll}
%$e$ & $::=$ & $M\in \Mnam'$ & (matrix name)\\
% & $|$ & $V\in \Mvar$ & (matrix variable)\\
% & $|$ & $\llet$ & (local binding)\\
% & $|$ & $e^*$ & (conjugate transpose)\\ 
% & $|$ & $\ones(e)$ & (one-vector)\\ 
% & $|$ & $\diag(e)$ & (diagonalization of a vector)\\  
% & $|$ & $e_1 \cdot e_2$ & (matrix multiplication)\\   
% & $|$ & $\apply{f}(e_1,\ldots ,e_n)$ & (pointwise application of $f$).    
%\end{tabular}
%
%To define the semantics of a \lang\ expression $e$ over $\Voc$, we first need to know whether $e$ can be evaluated due to matrix dimension constraints, since, for example,  the product $M_1 \cdot M_2$ of two matrices is not always defined. To overcome this, we define the {\em type} of each expression $e$ with respect to an instance $\I$, denoted by $\ttype(e)^\I$ in Table \ref{tab-types}.
%
%\begin{table}
%\begin{tabular}{rcll}
%$\ttype(M)^\I$ & $=$ & $\dim(\I(M))$, for $M\in \Mnam'$\\
%$\ttype(V)^\I$ & $=$ & $\dim(\I(V))$, for $V\in \Mvar$\\
%%$\ttype(\llet)^\I$ & $=$ & this is a dumb operator\\
%$\ttype(e^*)^\I$ & $=$ & $(m,n)$, if $\ttype(e)^\I = (n,m)$\\%, and undefined otherwise\\
%$\ttype(\ones(e))^\I$ & $=$ & $(n,1)$, if $\ttype(e)^\I = (n,m)$\\%, and undefined otherwise\\
%$\ttype(\diag(e))^\I$ & $=$ & $(n,n)$, if $\ttype(e)^\I = (n,1)$\\
%$\ttype(e_1\cdot e_2)^\I$ & $=$ & $(n,k)$, if $\ttype(e_1)^\I = (n,m)$, and $\ttype(e_2)^\I = (m,k)$\\
%$\ttype(\apply{f}(e_1,\ldots ,e_n))^\I$ & $=$ & $(n,m)$, if $\ttype(e_1)^\I = \ldots = \ttype(e_k)^\I = (m,n)$, and $f:\mathbf{C}^n\mathbf{C}$\\
%\end{tabular}
%\label{tab-types}
%\caption{Type of \lang\ expression $e$ over an instance $\I$.}
%\end{table}










%
% Since the baseline for our study is the \lang\ language introduced in \cite{matlang}, here we briefly recap its syntax and semantics.




\section{Queries in Linear Algebra}
%!TEX root = ../main.tex
% !TeX spellcheck = en_US



To extend \lang\ with recursion, we take inspiration from classical linear algebra algorithms, such as those described in \cite{num}. Many of these algorithms are based on \textit{for loops} in which the termination condition for each loop is determined by the matrix dimension. We have seen how the transitive closure of a matrix can be computed using for loops in the Introduction. Here we add this ability to \lang, and show that the resulting language, called $\langfor,$ can compute properties outside of the scope of \lang. We see more advanced examples, such as Gaussian elimination and computing an inverse of a matrix, later in the paper. 

\subsection{Syntax and semantics of \langfor} The syntax of \langfor is defined just as for \lang\, but with an extra rule in the grammar:
\medskip

\begin{tabular}{lcll}
 $\ffor{v}{X}{e}$ & (canonical for loop, with $v, X \in \Mvar$). 
\end{tabular}

\medskip
\noindent Intuitively, $X$ is a matrix variable which is iteratively updated according to the expression $e$. We simulate iterations of the form ``\texttt{for} $i\in [1..n]$'' by letting $v$ loop over the \textit{canonical vectors} $b_1^n,\ldots,b_n^n$ of dimension $n$. Here,
%
% That is,
% for each $n\in \mathbb{N}$, we denote by $b_1^n,\ldots ,b_n^n$ the  canonical vectors of dimension $n$; that is,
$b_1^n = [1\ 0 \cdots 0]^T$, $b_2^n = [0\ 1\ 0 \cdots 0]^T$, etc. When $n$ is clear from the context we simply write $b_1,b_2,\ldots$. In addition, the expression $e$ in the rule above may depend on $v$. 

We next make the semantics precise and start by
declaring the type of loop expressions.
Given a schema $\Sch$, the type of a \langfor expression $e$, denoted $\ttype(e)$, is defined inductively as in \lang\, but with following extra rule:
\begin{itemize}
\item $\ttype(\ffor{v}{X}{e}) := (\alpha,\beta)$, if \\
$\ttype(e)=\ttype(X) =(\alpha,\beta)$ and $\ttype(v) = (\gamma,1)$.
\end{itemize}
We note that $\Sch$ now necessarily includes $v$ and $X$ as variables and assigns size symbols to them.
%Similarly as when defining \lang\, here we start with a core set of matrix operations (i.e sum, product, and the transpose of a matrix). In addition to this, we allow applying functions, and  the $\ffor{v}{X}{e}$ construct, which allows looping over the canonical vectors of a specified dimension, and updating the context of the variable $X$ in each iteration. The latter operator is inspired by classical Linear Algebra algorithms \cite{num}, which commonly use loops whose termination conditions are determined by the matrix dimension, and will allow us to express many properties of interest.
%A \langfor {\em schema} $\Sch$ is a pair $\Sch=(\Mnam,\size)$, where $\Mnam\subset \Mvar$ is a finite set of matrix variables, and $\size: \Mvar \mapsto \DD\times \DD$ is a function that maps each matrix variable to a pair of {\em size symbols}. Given a schema $\Sch$, the type of a \langfor expression, denoted $\ttype(e)$, is defined inductively as follows:
%\begin{itemize}
%\item $\ttype(V) = \size(V)$, for a matrix variable $V$,
%\item $\ttype(e^*) = (\beta,\alpha)$, if $\ttype(e)=(\alpha,\beta)$; and undefined if $\ttype(e)$ is undefined,
%\item $\ttype(e_1 \cdot e_2) = (\alpha,\gamma)$, whenever $\ttype(e_1)=(\alpha,\beta)$, and $\ttype(e_2)=(\beta,\gamma)$; and is undefined otherwise,
%\item $\ttype(e_1 + e_2) = \ttype(e_1)$, if $\ttype(e_1) = \ttype(e_2)$; and is undefined otherwise,
%\item $\ttype(f(e_1,\ldots ,e_n)) = (1,1)$, whenever $\ttype(e_1)= \cdots =\ttype(e_n)=(1,1)$, and $f:\mathbb{C}^n\mapsto \mathbb{C}$; and is undefined otherwise,
%\item $\ttype(f(e_1,\ldots ,e_n)) = (\alpha,\beta)$, whenever $\ttype(e_1) = \ldots = \ttype(e_k) = (\alpha,\beta)$, and $f:\mathbb{C}^n \mapsto  \mathbb{C}$; and is undefined otherwise, and,
%\item $\ttype(\ffor{v}{X}{e}) = \ttype(e)$, if $\ttype(X) = \ttype(e)$, and $\ttype(v) = (\gamma,1)$; and is undefined otherwise.
%\end{itemize}
%\cristian{Is this the same than in the previous section? Is there any difference?}
% To extend \lang\, we add $\ffor{v}{X}{e}$ construct, which allows looping over the canonical vectors of a specified dimension, and updating the context of the variable $X$ in each iteration. The latter operator is i
%
% A \langfor {\em schema} $\Sch$ is a pair $\Sch=(\Mnam,\size)$, where $\Mnam\subset \Mvar$ is a finite set of matrix variables, and $\size: \Mvar \mapsto \DD\times \DD$ is a function that maps each matrix variable to a pair of {\em size symbols}. 
We also remark that in the definition of the type of $\ffor{v}{X}{e}$, we require that $\ttype(X) = \ttype(e)$ as this expression updates the content of the variable $X$ in each iteration using the result of $e$. We further restrict the type of 
$v$ to be a vector, i.e., $\ttype(v)=(\gamma,1)$, since $v$ will be instantiated with canonical vectors.
% As we will see below, evaluating $e$ will depend on a canonical vector stored in $v$, and the current content of $X$. %Another difference from \lang\ is that we allow function application only on scalars (more precisely, on $1\times 1$ matrices).
%
A \langfor\ expression $e$ is well-typed over a schema $\Sch$ if its type is defined. 

For well-typed expressions we next define their semantics. This is done in an inductive way, just as for \lang. To define the semantics of $\ffor{v}{X}{e}$ over an instance $\I$, we need the following notation. Let $\I$ be an instance and $V\in \Mnam$. Then $\I[V := A]$ denotes an instance that coincides with $\I$, except that the value of the matrix variable $V$ is given by the matrix $A$. Assume that
$\ttype(v)= (\gamma,1)$, and $\ttype(e) = (\alpha,\beta)$ and $n := \dom(\gamma)$. Then, $\sem{\ffor{v}{X}{e}}{\I}$ is defined iteratively, as follows:
\begin{itemize}
\item Let $A_0 := \mathbf{0}$ be the zero matrix of size $\dom(\alpha)\times \dom(\beta)$.
\item For $i=1,\ldots n$, compute $A_i:= \sem{e}{\I[v := b^{n}_i, X:= A_{i-1}]}$.
\item Finally, set $\sem{\ffor{v}{X}{e}}{\I}:= A_{n}$.
\end{itemize}

%%
% A \langfor {\em instance} $\I$ over a schema $\Sch$, is a pair $\I = (\dom,\conc)$, where $\dom : \DD \mapsto \mathbb{N}$ assigns a value to each size symbol, and $\conc : \Mnam \mapsto \mtr{\mathbb{C}}$ assigns a concrete matrix to each matrix variable $M\in \Mnam$, such that $\dim(\conc(M)) = \dom(\alpha)\times \dom(\beta)$, where $\size(M) = (\alpha,\beta)$. As before, we assume that $\dom(1) = 1$, for every instance $\I$.
%  %(meaning that $e_1^n = \begin{bmatrix} 1 \\ 0 \\ \vdots \\ 0 \end{bmatrix}$, etc.).
% If $\I$ is an instance, $V$ a matrix variable such that $\size(V)= (\alpha,\beta)$, and $M$ a matrix of dimension $\dom(\alpha)\times \dom(\beta)$, then $\I[V := M]$ denotes an instance that coincides with $\I$, apart from the fact that the value of the matrix variable $V$ is the matrix $M$.
% %If $e$ is a well-typed expression according to $\Sch$, then we denote by $\sem{e}{\I}$ the matrix obtained by evaluating $e$ over $\I$, and define it as follows:
% %\begin{itemize}
% %\item $\sem{M}{\I} = \conc(M)$, for $M\in \Mnam$;
% %\item $\sem{e^*}{\I} = \sem{e}{\I}^*$, where $M^*$ is the conjugate transpose of a matrix $M$;
% %\item $\sem{e_1\cdot e_2}{\I} = \sem{e_1}{\I} \cdot \sem{e_2}{\I}$;
% %\item $\sem{e_1 + e_2}{\I} = \sem{e_1}{\I} + \sem{e_2}{\I}$;
% %\item $\sem{f(e_1,\ldots ,e_n)}{\I}$ is a $1\times 1$ matrix whose only entry has the value $f(\sem{e_1}{\I},\ldots ,\sem{e_n}{\I})$. Here we abuse the notation and use $\sem{e}{\I}$ to denote both a $1\times 1$ matrix, and a scalar from $\mathbb{C}$.
% %\item $\sem{f(e_1,\ldots ,e_n)}{\I}$ is a matrix $A$ of the same size as $\sem{e_1}{\I}$, and where $A_{ij}$ has the value $f(\sem{e_1}{\I}_{ij},\ldots ,\sem{e_n}{\I}_{ij})$.
% %\end{itemize}
% %\cristian{The same as with the type function, we can reuse the semantics of the previous section.}
%
% If $e$ is a well-typed expression according to $\Sch$, then we denote by $\sem{e}{\I}$ the matrix obtained by evaluating $e$ over $\I$, and define it as in \lang\, adding the new operator.
% \floris{Perhaps it is insightful to already give an easy example here, e.g., clique. Perhaps even better are the very simple expressions for diag and one vector. The text below is quite
% cryptic.}
%
% Notice that evaluating $\ffor{v}{X}{e}$ over an instance that already assigns values to $v$ and $X$, these values get overwritten immediately. Notice that if $e$ does not use the variable $v$, then it will have the same value in each iteration. Sometimes when we want to make it explicit that the expression $e$ uses matrix variables $X_1,\ldots ,X_n$, we write $e(X_1,\ldots ,X_n)$, where $X_1,\ldots ,X_n$ are all the matrix variables mentioned in $e$. Analogously with First Order Logic, we could define the notion of free and bound variables (i.e. the ones in the scope of a \texttt{for} operator), however, since we explicitly rewrite the value of these variables, this distinction will not play an important role.
%
%
% \cristian{I don't understand this comment of free and bound variables. What do you mean? I believe than similar than in first order variables, one can define this concept, but here it is not used.}
%
% %\subsection{Examples of \langfor expressions}




%%needs dimensions:
%Next we illustrate the versatility of the introduced language. To begin, we note that, unlike the original \lang\ proposal, we have at our disposals canonical vectors of arbitrary dimension. The most basic use of canonical vectors is for accessing a position $ij$ of some matrix $M$, a property which lies outside of the scope of \lang. For this, we can simply use the expression $(e_i^{n})^*\cdot M \cdot e_j^m$, where $M$ is a matrix of size $m\times n$, $e_i^n$ is the $i$th canonical vector of dimension $n$, and likewise, $e_j^m$ is the $j$th canonical vector of dimension $m$.

For better understanding how \langfor  works, we next provide some  examples.
We start by showing that the one-vector and $\diag$ operators are redundant
in \langfor.
% next present a series of properties that the language can express and introduce some operators that will be commonly used throughout the paper.
% , and show how the proposed language captures original \lang.
\begin{example}
We first show how the one-vector operator $\ones(e)$ can be expressed using \texttt{for} loops.
It suffices to consider the expression
$$e_{\ones}:=\ffor{v}{X}{X+v},$$
with $\ttype(v)=(\alpha,1)=\ttype(X)$ if $\ttype(e)=(\alpha,\beta)$. This expression is well-typed
and is of type $(\alpha,1)$. When evaluated over some instance $\I$ with $n=\dom(\alpha)$, $\sem{e_{\ones}}{\I}$ is defined as follows.
Initially, $A_0:=\mathbf{0}$. Then $A_i:=A_{i-1}+b_i^n$, i.e., the $i$th canonical vector is added to $A_{i-1}$.
Finally, $\sem{e_{\ones}}{\I}:=A_n$ and this now clearly coincides with $\sem{\ones(e)}{\I}$.\qed
\end{example}

% \begin{example}\label{ex:forml-identity}
% We first show how one can construct the identity matrix of a certain dimension. For this, it suffices to use the expression $$e_{\mathsf{Id}}:=\ffor{v}{X}{X + v\cdot v^T}.$$ For this expression to be well-typed, $v$ has to be a vector variable of type $\alpha\times 1$ and $X$ a matrix variable of type $(\alpha,\alpha)$. When evaluated over some instance $\I$, this loop starts by initializing $X$ as the zero matrix of dimension $n\times n$, where $n=\dom(\alpha)$, and then adds to $X$ the matrix $b_1^n\cdot (b_1^n)^T$ in the first iteration, the matrix $b_2^n\cdot (b_2^n)^T$ in the second iteration, and so on, until finally $b_n^n\cdot (b_n^n)^T$ is added. Hence, $X$ equals the identity matrix of dimension $n\times n$ in the end.\qed
% \end{example}
%
\begin{example}\label{ex:diag}
% Indeed, a minor variation of the expression $e_{\mathsf{Id}}$ suffices:
As another example, we show that the $\diag$ operator is redundant in \langfor.
Indeed, it suffices to consider the expression
$$e_{\mathsf{diag}}:=
\ffor{v}{X}{X + (v^T\cdot e) \times v\cdot v^T},$$ where $e$ is a \langfor\  expression of type $(\alpha,1)$. For this expression to be well-typed, $v$ has to be a vector variable of type $\alpha\times 1$ and $X$ a matrix variable of type $(\alpha,\alpha)$. Then, $\sem{e_{\mathsf{diag}}}{\I}$ is defined as follows.
Initially, $A_0$ is the zero matrix of dimension $n\times n$, where $n=\dom(\alpha)$. Then, in each iteration
$i\in[1..n]$, $A_{i}:=A_{i-1}+  (b_i^T\cdot\sem{e}{\I})\times (b_i^n\cdot (b_i^n)^T)$. In other words, $A_i$ is obtained by adding the matrix with value $(\sem{e}{\I})_i$ on position $(i,i)$ to $A_{i-1}$. Hence, $A_n=\sem{\diag(e)}{\I}$ and
thus $\sem{e_{\mathsf{diag}}}{\I}=\sem{\diag(e)}{\I}$.\qed%
 \end{example}
% That is, $A_i$ is obtained from $A_{i-1}$the $i$th entry of
% $\sem{e}{\I})^T$ is multiplied
%
% The difference with $e_{\mathsf{Id}}$ is that instead of the value $1$, the value
% $b_i^T\cdot\sem{e}{\I}=\sem{e}{\I}_{i}$ is put on position $i$ on the diagonal, when evaluating $e_{\mathsf{diag}}$ on $\I$.
%  One can verify that for the above expression to be well-typed, $v$ needs to be of type $(\alpha,1)$ and $X$ of type $(\alpha,\alpha)$. When evaluated over some instance $\I$, $X$ is initialised with the zero matrix of dimension $n\times n$, where $n=\dom(\alpha)$. Then, in iteration $i$, the $n\times n$-matrix
% $b_i^T\cdot \sem{e}{\I}\times (b_i\cdot b_i^T)$ is added. In other words, on the $i$th element of the diagonal, represented by $b_i\cdot b_i^T$, the value $b_i^T\cdot\sem{e}{\I}=\sem{e}{\I}_{i}$ is added. 
% That is, the expression $e_{\mathsf{diag}}$ evaluates to $\sem{\diag(e)}{\I}$.\qed

% It is also readily verified that the one-vector operator in \lang\ becomes redundant in \langfor.

%\floris{I commented out the part related to function applications, i.e., that function applications on scalars is sufficient. It is interesting but may be too much at this point? }

% Furthermore, general function applications can be simulated in \langfor\ by just assuming function applications $f(e_1,\ldots,e_k)$ for $f\in\Fun_k$ where $\ttype(e_i)=(1,1)$ for all $i\in[k]$, as is illustrated next.
% As before, we write \langfor$(\Fun)$ when \langfor\ expressions require certain function applications in some set $\Fun$.
%
% \begin{example}Consider the \langfor$(f)$ expression
% \begin{multline*}
% e=\texttt{for }v,X.\, X + \\
% \texttt{for }w,Y.\, Y + v\cdot f(v^T\cdot e_1\cdot w, \ldots, v^T\cdot e_k\cdot w)\cdot w^T.
% \end{multline*}
% In the expression above, we basically loop over all positions in $\sem{e_i}{\I}$, one by one, and apply the function $f$ on those positions.  One can see that $\sem{e}{\I}=\sem{f(e_1,\ldots,e_k)}{\I}$ for $e_i$'s of arbitrary type, but that $f$ is only
% applied on expressions of type $(1,1)$, such as $v^T\cdot e_i\cdot w$. \qed
% \end{example}
% %
%
% \smallskip
% \noindent
% \textbf{Core operators}. The previous example show that we can define \langfor\ using a small number of core operators, whilst still extending \lang. In particular, it suffices to define \langfor to consist of expressions of the form
% %
% %
% %
% % Note that the  \texttt{for} operator is much more powerful that it seems. We can simulate some operations of the original \lang\ semantics. Assume the allowed expressions are
%
% \begin{tabular}{lcll}
% $e$ & $:=$ & $V\in \Mvar$ & (matrix variable)\\
%  % & $|$ & $e^T$ & (transpose)\\
%  & $|$ & $e_1 \cdot e_2$ & (matrix multiplication)\\
%  % & $|$ & $e_1 + e_2$ & (matrix addition)\\
%  & $|$ & $\text{apply}[f](e_1,\ldots ,e_k)$ & (application of $f\in \Fun$)\\
%  & $|$ & $\ffor{v}{X}{e}$ & (canonical for loop, with $v, X \in \Mvar$).
% \end{tabular}
%
% Where function application $\text{apply}[f]$ works only on scalars. This is
%
% \begin{itemize}
% \item $\ttype(\text{apply}[f](e_1,\ldots ,e_n)) = (1,1)$, whenever $\ttype(e_1)= \cdots =\ttype(e_n)=(1,1)$, and $f:\mathbb{C}^n\mapsto \mathbb{C}$; and is undefined otherwise.
% \item $\sem{\text{apply}[f](e_1,\ldots ,e_n)}{\I}$ is a $1\times 1$ matrix whose only entry has the value $f(\sem{e_1}{\I},\ldots ,\sem{e_n}{\I})$.
% \end{itemize}

These examples illustrate that we can limit \langfor to consist of the following ``core'' operators: transposition, matrix multiplication and addition, scalar multiplication, pointwise function application, and for-loops. More specific, \langfor is defined by the following simplified syntax:
$$
e ::= V \ \mid \ e^T \ \mid \ e_1 \cdot e_2 \ \mid \ e_1 + e_2 \ \mid \ a\times e  \ \mid \  f(e_1,\ldots ,e_k) \ \mid \ \ffor{v}{X}{e}
$$

Similarly as for \lang, we write $\langforf{\Fun}$ for some set $\Fun$ of functions when these are required for the task at hand.

%  $f(e_1,\ldots,e_k)$ for $f\in\Fun_k$ where each $e_i$
% has type $(1,1)$.

%As a final example, we show how the Floyd-Warshall algorithm for computing the transitive closure of a matrix given in the Introduction can be defined using \langfor.
%\begin{example}\label{ex:floyd}
%The following expression in $\langf{f_>}$ computes the transitive closure of a $n\times n$ matrix $A$.
%\begin{tabbing}
%\texttt{for }\=$e_k,X_1.\, X_1 + $\\
%\> \texttt{for }\=$e_i,X_2.\, X_2 +$ \\
%\>\>\texttt{for }\=$e_j,X_3.\, X_3 +$ \\
%\>\>\>$(e_i^T\cdot A\cdot e_k \cdot e_k^T\cdot A\cdot e_j)\times e_i\cdot e_j^T$
%\end{tabbing}
%\end{example}


As a final example, we show that we can compute whether a graph contains a $4-\textsf{clique}$ using \langfor.
\begin{example}\label{ex:fourcliques}
To test for $4$-cliques it suffices to consider the expression
\begin{tabbing}
\texttt{for }\=$u,X_1.\, X_1 + $\\
\> \texttt{for }\=$v,X_2.\, X_2 +$ \\
\>\>\texttt{for }\=$w,X_3.\, X_3 +$ \\
\>\>\>\texttt{for }\=$x,X_4.\, X_4 +$ \\
\>\>\>\>$u^T\cdot V\cdot v \cdot u^T\cdot V\cdot w\cdot u^T\cdot V\cdot x \cdot $\\
\>\>\>\>$v^T\cdot V\cdot w \cdot v^T\cdot V\cdot x\cdot w^T\cdot V\cdot x \cdot g(u,v,w,x)$\\
\end{tabbing}
with $g(u,v,w,x)=f(u,v)\cdot f(u,w)\cdot f(u,x)\cdot f(v,w)\cdot f(v,x)\cdot f(w,x)$ and
$f(u,v)=1-u^T\cdot v$. Note that $f(b_i^n,b_j^n)=1$ if $i\neq j$ and $f(b_i^n,b_j^n)=0$ otherwise.
Hence, $g(b_i^n,b_j^n,b_k^n,b_\ell^n)=1$ if and only if all $i,j,k,l$ are pairwise different.
When evaluating the expression on an instance $\I$ such that $V$ is assigned to the adjacency 
matrix of a graph, the expression above evaluates to a non-zero value if and only if the graph
contains a four-clique.\qed
\end{example}
%
% % =\begin{cases}
% %                0 \text{ if } u=v \\
% %                1 \text{ if } u\neq v
% %             \end{cases}
% % 		\]
%
% To distinguish four-cliques, we will need to determine whether we are dealing with four different nodes. For this, we will utilize the function $$g(u,v,w,r)=f(u,v)\cdot f(u,w)\cdot f(u,r)\cdot f(v,w)\cdot f(v,r)\cdot f(w,r),$$

As mentioned previously, given \lang\ can not express 4-cliques \cite{BrijderGBW19}, we easily obtain the following.

\begin{proposition}
\label{cor-ml-fml}
For any collection of functions $\Fun$, 
$\langf{\Fun}$ is properly subsumed by $\langforf{\Fun}$.
\end{proposition} 







%\domagoj{This is used to simulate circuits, so I put in into a lemma. Perhaps just move to the appendix before that particular proof.}
%
%Another interesting consequence of allowing for loops in \lang\ is that the pointwise product and sum can be defined without direct access to these operators. Namely, the following result tells us that we can always assume to have access to the product and the sum.
%%This will be of crucial importance when comparing \langfor\ to arithmetic circuits in Section \ref{sec:circuits} and to annotated relations in Section \ref{sec:restrict}. Formally, we have:
%
%\begin{lemma}
%\label{lm-prod-sum}
%Let $f_\odot^k:\RR^k\mapsto \RR$ and $f_\oplus^k:\RR^k\mapsto \RR$ be the product and the sum function with $k$ attributes, respectively. Then it holds that $\langforf{\emptyset} \equiv \langforf{\{f_\odot^k,f_\oplus^k \ | \ k\in \mathbf{N}\}}$.
%\end{lemma}
%\cristian{This result is very ugly, why we want this? The interesting result is that you only need the function between constants to define the point-wise application of functions.}
%
%In what follows, we will sometimes abuse the notation and write $\langfor$ both when we are talking about $\langfor(\emptyset)$, and when referring to the language in general. On the other hand, when stating formal results, we will always make the set of functions explicit, and will use $\langfor$ to denote $\langfor(\emptyset)$.

\subsection{Design decisions behind \langfor}

\noindent\textbf{Loop Initialization.} As the reader may have observed, in the semantics of \texttt{for} loops we 
always initialise $A_0$ to the zero matrix~$\mathbf{0}$ (of appropriate dimensions). It is often convenient
to start the iteration given some concrete matrix  originating from the result of evaluation a \langfor\ expression $e_0$. To make this explicit, we write $\initf{e_0}{v}{X}{e}$ and its semantics is defined as above
with the difference that $A_0:=\sem{e_0}{\I}$. We observe, however, that $\initf{e_0}{v}{X}{e}$ can already
be expressed in \langfor. In other words, we do not loose generality by assuming an initialisation of $A_0$ by $\mathbf{0}$.
The key insight is that in \langfor\ we can check during evaluation whether or not
the current canonical vector $b_i^n$ is equal to the $b_1^n$. This is not entirely trivial to see and is due to fact that \texttt{for} loops iterate over the canonical vectors in a fixed order. We discuss this more in the next subsection, when we talk about the order. In particular, we can define a \langfor expression $\mmin$, which when evaluated on an instance, returns $1$ if its input vector is $b_1^n$, and returns $0$ otherwise. Given $\mmin$, consider now the
\langfor\ expression
 $$\ffor{v}{X}{\mmin{v}\cdot e(v,X/e_0) + (1-\mmin{v})\cdot e(v,X)},$$
 where we explicitly list $v$ and $X$ as matrix variables on which $e$ potentially depends on, and where
 $e(v,X/e_0)$ denotes the expression obtained by replacing every occurrence of $X$ in $e$ with $e_0$.
%
%  As mentioned previously, all of the iterations in \texttt{for} loops start by initializing $A_0$ to the null matrix. We have already seen that this property is useful for defining the extremal points of our ordering, however, sometimes we would like to start the iteration using some concrete matrix $B$. Using the operator $\mmin$, we can actually achieve this as follows. First, let $\ffor{v}{X}{e(v,X)}$ be the \texttt{for} loop that begins iterating from the null matrix. Here we write $e(v,X)$ to denote the variables that $e$ potentially (but not necessarily) uses. To start the iteration from $B$ it now suffices to use the following loop:
% $$\ffor{v}{X}{\mmin{v}\cdot e(v,X/B) + (1-\mmin{v})\cdot e(v,X)} \quad (\dag),$$
% where $e(v,X/B)$ denotes the expression obtained by replacing every occurrence of $X$ in $e$ with $B$.
When evaluating this expression on an instance $\I$, $A_0$ is initial set to the zero matrix, in the first iteration (when  $v=b_1^n$ and thus $\mmin{v}=1$)
we have $A_1=\sem{e}{\I[v:=b_1^n,X:=\sem{e_0}{\I}]}$, and for consecutive iterations (when only the part related to $1-\mmin{v}$ applies) $A_i$ is updated as before. Clearly, the result of this evaluation is equal to
$\sem{\initf{e_0}{v}{X}{e}}{\I}$. 

%As an illustration, we consider the transitive closure program from the beginning of this section. Similarly as for \lang, we write $\langforf{\Fun}$ for some set $\Fun$ of functions when these are required for the task at hand.
%%As an illustration, we consider the transitive closure program from the beginning of this section. Similarly as for \lang, we write \langfor$(\Fun)$ for some set $\Fun$ of functions when these are required for the task at hand. Analogously, we write $\langfor$ to denote $\langfor(\emptyset)$, but we might abuse the notation and write $\langfor$ to refer to the language in general.
%\begin{example}
% In $\langforf{f_>}$ we can express transitive closure with the following expression
%$$
%f_{>0}\left(\initf{e_{\mathsf{Id}}}{v}{X}{X\cdot (e_{\mathsf{Id}}+V)}\right),
%$$
%where $e_{\mathsf{Id}}$ is a \langfor\ expression constructing the identity matrix $I$. The expression $e_{\mathsf{Id}}$
%is just like the expression $e_{\mathsf{diag}}$ in Example~\ref{ex:diag}, but now only puts value $1$ on the diagonal.
% Observe that we start the loop using an initialisation by $I$. Hence, when evaluated on an instance $\I$ such that $V$ is an adjacency matrix $A$ of a graph, the expression above simply computes $(I+A)^n$, followed by the pointwise function application $f_{>0}$. In other words, it returns the adjacency matrix of the transitive closure of graph.\qed
%\end{example}

%As an illustration, we consider the Floyd-Warshall algorithm for computing the transitive closure of a matrix given in the Introduction. Similarly as for \lang, we write $\langforf{\Fun}$ for some set $\Fun$ of functions when these are required for the task at hand.

As an illustration, we consider the Floyd-Warshall algorithm given in the Introduction. 

%\begin{example}\label{ex:floyd}
%Consider the following expression in $\langf{f_>}$, where $f_>:\RR^2\to\RR$, is a function such that $f_>(x,y) = 1$ when $x>y$, and is zero otherwise:
%\begin{tabbing}
%\texttt{for\,}\=$e_k,\, X_1\!=\!A.\ \ X_1 \ + $\\
%\> \texttt{for\,}\=$e_i, \, X_2.\ \ X_2 \ +$ \\
%\>\>\texttt{for\,}\=$e_j,\, X_3.\ \ X_3 \ +$ \\
%\>\>\> $f_>(e_i^T\cdot X_1 \cdot e_j\ ,\ $\=$e_i^T\cdot X_1\cdot e_k + e_k^T\cdot X_1\cdot e_j)\ \cdot$\\
%\>\>\>\>$(e_i^T\cdot X_1\cdot e_k + e_k^T\cdot X_1\cdot e_j) \times e_i\cdot e_j^T$
%\end{tabbing}
%The expression $e_{FW}$ simulates the Floyd-Warshall algorithm by updating the matrix $A$, which is stored in the variable $X_1$. The inner sub-expression here constructs an $n\times n$ matrix that contains the distance of the path between $i$ and $j$ that goes through $k$ if and only if this distance is less than what we had previously here.. ZERO SHOULD BE INFINITY.
%$(i,j)$ if and only if one can reach the vertex $j$ from $i$ by going through $k$, and zero elsewhere. If an instance $\I$ assigns to $A$ the adjacency matrix of a graph, then $\sem{e_{FW}}{\I}$ contains the shortest distance between a vertex $i$ and the vertex $j$.
%% The transitive closure in then obtained using the $\langf{f_>}$ expression $f_>(e_{FW})$, where $f_{>0}:\RR\to\RR$ is such that $f_{>0}(x):=1$ if $x>0$ and $f_{>0}(x):=0$ otherwise.  
%\cristian{The same with the problem of the shortest path mentioned in the intro.}
%\qed
%\end{example}

\begin{example}\label{ex:floyd}
Consider the following expression:
\begin{tabbing}
$e_{FW} := $ \texttt{for\,}\=$v_k,\, X_1\!=\!A.\ \ X_1 \ + $\\
\> \texttt{for\,}\=$v_i, \, X_2.\ \ X_2 \ +$ \\
\>\>\texttt{for\,}\=$v_j,\, X_3.\ \ X_3 \ +$ \\
\>\>\>$(v_i^T\cdot X_1\cdot v_k \cdot v_k^T\cdot X_1\cdot v_j)\times v_i\cdot v_j^T$
\end{tabbing}
The expression $e_{FW}$ simulates the Floyd-Warshall algorithm by updating the matrix $A$, which is stored in the variable $X_1$. The inner sub-expression here constructs an $n\times n$ matrix that contains one in the position $(i,j)$ if and only if one can reach the vertex $j$ from $i$ by going through $k$, and zero elsewhere. If an instance $\I$ assigns to $A$ the adjacency matrix of a graph, then $\sem{e_{FW}}{\I}$ will be equal to the matrix produced by the algorithm given in the Introduction.
%contains a non zero value if and only if $j$ is reachable from $i$. 
%If we use the function $f_{>0}:\RR\to\RR$, with $f_{>0}(x):=1$ if $x>0$ and $f_{>0}(x):=0$ otherwise, then the $\langf{f_{>0}}$ expression $f_{>0}(e_{FW})$ computes precisely the same matrix as the algorithm from the Introduction.
% The transitive closure in then obtained using the $\langf{f_>}$ expression $f_>(e_{FW})$, where $f_{>0}:\RR\to\RR$ is such that $f_{>0}(x):=1$ if $x>0$ and $f_{>0}(x):=0$ otherwise.  
%\cristian{The same with the problem of the shortest path mentioned in the intro.}
\qed
\end{example}


\noindent\textbf{Order.} With the introduction of \texttt{for} loops we not only extend \lang\ with bounded recursion, we also introduce order information. Indeed, the semantics of the \texttt{for} operator assumes that the canonical vectors $b_1,b_2,\ldots$
are accessed in this order. It implies, among other things, that \langfor\ expressions are not permutation-invariant.
We can, for example, return the bottom right-most entry in a matrix. Indeed, consider the expression $e_{\mathsf{max}} := \ffor{v}{X}{v}$ which, for it to be well-typed, requires both $v$ and $X$ to be of type $(\alpha,1)$. Then, $\sem{e_{\mathsf{max}}}{\I}=b_n^n$, for $n=\dom(\alpha)$, simply because initially, $X=\mathbf{0}$, but $X$ will be overwritten by $b_1^n,b_2^n,\ldots,b_n^n$, in this order. Hence, at the end of the evaluation $b_n^n$ is returned.
To extract the bottom right-most entry from a matrix, we now simply use $e_{\mathsf{max}}^T\cdot V\cdot e_{\mathsf{max}}$.

Although the order is implicit in \langfor, we can explicitly use this order in \langfor expressions. More precisely, the order on canonical vectors is made accessible by
using the following matrix:
\[
S_{\leq} = \begin{bmatrix}
1 & 1 & \cdots &  1 \\
0 & 1 & \cdots & 1\\
\vdots & \vdots & \ddots & 1 \\
0 & 0 & \cdots & 1 
\end{bmatrix}.
\] 
We observe that $S_{\leq}$ has the property that $b_i^T\cdot S_{\leq} \cdot b_j=1$, for two canonical vectors $b_i$ and $b_j$ of the same dimension, if and only if $i\leq j$. Otherwise, $b_i^T\cdot S_{\leq} \cdot b_j=0$. 
Interestingly, we can build matrix $S_{\leq}$ with the following \langfor expression:
$$
e_{\leq}=\ffor{v}{X}{X + \left((X\cdot e_{\mathsf{max}}) + v \right)\cdot v^T + v\cdot e^T_{\mathsf{max}}},
$$
where $e_{\mathsf{max}}$ is as defined above. The intuition behind this expression is that by using the last canonical vector $b_n$, as returned by $e_{\mathsf{max}}$, we have access to the last column of $X$ (via the product $X\cdot e_{\mathsf{max}}$). We use this column such that after the $i$-th iteration, this column contains the $i$-th column of $S_{\leq}$. This is done by incrementing $X$ with $v\cdot e_{\mathsf{max}}^T$.
To construct $S_{\leq}$, in the $i$-th iteration we further increment $X$ with 
(i)~the current last column in $X$ (via $X\cdot e_{\mathsf{max}}\cdot v^T$) which holds
the $(i-1)$-th column of $S_{\leq}$; and (ii)~the current canonical vector (via $v\cdot v^T$). Hence, after iteration $i$, $X$ contains the first $i$ columns of $S_{\leq}$ and holds the $i$th column of $S_{\leq}$ in its last column. It is now readily verified that $X=S_{\leq}$ after the $n$th iteration.

It should be clear that if we can compute $S_{\leq}$ using $e_{\leq}$, then we can easily define the following predicates and vectors related with the order of cannonical vectors:
\begin{itemize}
	\item $\mathsf{succ}(u,v)$ such that $\mathsf{succ}(b_i^n,b_j^n)=1$ if $i\leq j$ and $0$ otherwise. Similarly, we can define
	$\mathsf{succ}^+(u,v)$ such that  $\mathsf{succ}^+(b_i^n,b_j^n)=1$ if $i < j$ and $0$ otherwise;
	\item $\mathsf{min}(u)$ such that  $\mathsf{min}(b_i^n)=1$ if $i=1$ and $\mathsf{min}(b_i^n)=0$ otherwise; 
	\item $\mathsf{max}(u)$ such that  $\mathsf{max}(b_i^n)=1$ if $i=n$ and $\mathsf{min}(b_i^n)=0$ otherwise; and
	\item $e_{\mathsf{min}}$ and $e_{\mathsf{max}}$ such that $\sem{e_{\mathsf{min}}}{\I}=b_1^n$ and 
	$\sem{e_{\mathsf{max}}}{\I}=b_n^n$, respectively.
\end{itemize}
The definitions of these expressions is not entirely trivial and are detailed in the appendix.

Having order information available results in \langfor to be quite expressive. We heavily rely on order information in the next sections to compute the inverse of matrices and more generally to simulate low complexity Turing machines and arithmetic circuits.

%%%% HERE PREVIOUS VERSION OF ORDER

%\noindent\textbf{Order.} With the introduction of \texttt{for} loops we not only extend \lang\ with bounded recursion, we also introduce order information. Indeed, the semantics of the \texttt{for} operator assumes that the canonical vectors $b_1,b_2,\ldots$
%are accessed in this order. It implies, among other things, that \langfor\ expressions are not permutation-invariant.
%We can, for example, return the bottom right-most entry in a matrix. Indeed, consider the expression $e_{\mathsf{max}} := \ffor{v}{X}{v}$ which, for it to be well-typed, requires both $v$ and $X$ to be of type $(\alpha,1)$. Then, $\sem{e_{\mathsf{max}}}{\I}=b_n^n$, for $n=\dom(\alpha)$, simply because initially, $X=\mathbf{0}$, but $X$ will be overwritten by $b_1^n,b_2^n,\ldots,b_n^n$, in this order. Hence, at the end of the evaluation $b_n^n$ is returned.
%To extract the bottom right-most entry from a matrix, we now simply use $e_{\mathsf{max}}^T\cdot V\cdot e_{\mathsf{max}}$.
%
%Interestingly, although the order is implicit in \langfor, we can explicitly use this order in \langfor expressions. More precisely, one can the define the following order predicates in \langfor\!:
%\begin{itemize}
%	\item $\mathsf{succ}(u,v)$ such that $\mathsf{succ}(b_i^n,b_j^n)=1$ if $i\leq j$ and $0$ otherwise. Similarly, we can define
%	$\mathsf{succ}^+(u,v)$ such that  $\mathsf{succ}^+(b_i^n,b_j^n)=1$ if $i < j$ and $0$ otherwise;
%	\item $\mathsf{min}(u)$ such that  $\mathsf{min}(b_i^n)=1$ if $i=1$ and $\mathsf{min}(b_i^n)=0$ otherwise; 
%	\item $\mathsf{max}(u)$ such that  $\mathsf{max}(b_i^n)=1$ if $i=n$ and $\mathsf{min}(b_i^n)=0$ otherwise; and
%	\item $e_{\mathsf{min}}$ and $e_{\mathsf{max}}$ such that $\sem{e_{\mathsf{min}}}{\I}=b_1^n$ and 
%	$\sem{e_{\mathsf{max}}}{\I}=b_n^n$, respectively.
%\end{itemize}
%The definitions of these expressions is not entirely trivial and are detailed in the appendix.
%We here only highlight that successor information on canonical vectors is made accessible by
%using the following matrix:
%\[
%S_{\leq} = \begin{bmatrix}
%1 & 1 & \cdots &  1 \\
%0 & 1 & \cdots & 1\\
%\vdots & \vdots & \ddots & 1 \\
%0 & 0 & \cdots & 1 
%\end{bmatrix}.
%\] 
%We observe that $S_{\leq}$ has the property that $b_i^T\cdot S_{\leq} \cdot b_j=1$, for two canonical vectors $b_i$ and $b_j$ of the same dimension, if and only if $i\leq j$. Otherwise, $b_i^T\cdot S_{\leq} \cdot b_j=0$. It should be clear that if we can compute $S_{\leq}$ using an expression $e_{\leq}$ in \langfor, then we can define
%$
%\mathsf{succ}(v,w):=v^T\cdot e_{\leq} \cdot w.
%$
%For $e_{\leq}$ we can take the following \langfor expression:
%$$
%e_{\leq}=\ffor{v}{X}{X + \left((X\cdot e_{\mathsf{max}}) + v \right)\cdot v^T + v\cdot e^T_{\mathsf{max}}},
%$$
%where $e_{\mathsf{max}}$ is as defined above. The intuition behind this expression is that by using the last canonical vector $b_n$, as returned by $e_{\mathsf{max}}$, we have access to the last column of $X$ (via the product $X\cdot e_{\mathsf{max}}$). We use this column such that after the $i$-th iteration, this column contains the $i$-th column of $S_{\leq}$. This is done by incrementing $X$ with $v\cdot e_{\mathsf{max}}^T$.
%To construct $S_{\leq}$, in the $i$-th iteration we further increment $X$ with 
%(i)~the current last column in $X$ (via $X\cdot e_{\mathsf{max}}\cdot v^T$) which holds
%the $(i-1)$-th column of $S_{\leq}$; and (ii)~the current canonical vector (via $v\cdot v^T$). Hence, after iteration $i$, $X$ contains the first $i$ columns of $S_{\leq}$ and holds the $i$th column of $S_{\leq}$ in its last column. It is now readily verified that $X=S_{\leq}$ after the $n$th iteration.
%
%Having order information available results in \langfor to be quite expressive. We heavily rely on order information in the next sections to compute the inverse of matrices and more generally to simulate low complexity Turing machines and arithmetic circuits.


%\subsection{Loop Initialization} As the reader may have observed, in the semantics of \texttt{for} loops we 
%always initialise $A_0$ to the zero matrix $\mathbf{0}$ (of appropriate dimensions). It is often convenient
%to start the iteration given some concrete matrix  originating from the result of evaluation a \langfor\ expression $e_0$. To make this explicit, we write $\initf{e_0}{v}{X}{e}$ and its semantics is defined as above
%with the difference that $A_0:=\sem{e_0}{\I}$. We observe, however, that $\initf{e_0}{v}{X}{e}$ can already
%be expressed in \langfor. In other words, we do not loose generality by assuming an initialisation of $A_0$ by $\mathbf{0}$.
%The key insight is that in \langfor\ we can check during evaluation whether or not
%the current canonical vector $b_i^n$ is equal to the $b_1^n$. This is not entirely trivial to see and is due to fact that \texttt{for} loops iterate over the canonical vectors in a fixed order. We discuss this more in the next section. In particular, we can define a \langfor expression $\mmin$, which when evaluated on an instance, returns $1$ if its input vector is $b_1^n$, and returns $0$ otherwise. Given $\mmin$, consider now the
%\langfor\ expression
% $$\ffor{v}{X}{\mmin{v}\cdot e(v,X/e_0) + (1-\mmin{v})\cdot e(v,X)},$$
% where we explicitly list $v$ and $X$ as matrix variables on which $e$ potentially depends on, and where
% $e(v,X/e_0)$ denotes the expression obtained by replacing every occurrence of $X$ in $e$ with $e_0$.
%%
%%  As mentioned previously, all of the iterations in \texttt{for} loops start by initializing $A_0$ to the null matrix. We have already seen that this property is useful for defining the extremal points of our ordering, however, sometimes we would like to start the iteration using some concrete matrix $B$. Using the operator $\mmin$, we can actually achieve this as follows. First, let $\ffor{v}{X}{e(v,X)}$ be the \texttt{for} loop that begins iterating from the null matrix. Here we write $e(v,X)$ to denote the variables that $e$ potentially (but not necessarily) uses. To start the iteration from $B$ it now suffices to use the following loop:
%% $$\ffor{v}{X}{\mmin{v}\cdot e(v,X/B) + (1-\mmin{v})\cdot e(v,X)} \quad (\dag),$$
%% where $e(v,X/B)$ denotes the expression obtained by replacing every occurrence of $X$ in $e$ with $B$.
%When evaluating this expression on an instance $\I$, $A_0$ is initial set to the zero matrix, in the first iteration (when  $v=b_1^n$ and thus $\mmin{v}=1$)
%we have $A_1=\sem{e}{\I[v:=b_1^n,X:=\sem{e_0}{\I}]}$, and for consecutive iterations (when only the part related to $1-\mmin{v}$ applies) $A_i$ is updated as before. Clearly, the result of this evaluation is equal to
%$\sem{\initf{e_0}{v}{X}{e}}{\I}$. 
%
%As an illustration, we consider the transitive closure program from the beginning of this section. Similarly as for \lang, we write \langfor$(\Fun)$ for some set $\Fun$ of functions when these are required for the task at hand.
%%As an illustration, we consider the transitive closure program from the beginning of this section. Similarly as for \lang, we write \langfor$(\Fun)$ for some set $\Fun$ of functions when these are required for the task at hand. Analogously, we write $\langfor$ to denote $\langfor(\emptyset)$, but we might abuse the notation and write $\langfor$ to refer to the language in general.
%\begin{example}
% We can express transitive closure in \langfor$(f_>)$. Indeed, consider the expression 
%$$
%f_{>0}\left(\initf{e_{\mathsf{Id}}}{v}{X}{X\cdot (e_{\mathsf{Id}}+V)}\right),
%$$
%where $e_{\mathsf{Id}}$ is a \langfor\ expression constructing the identity matrix $I$. The expression $e_{\mathsf{Id}}$
%is just like the expression $e_{\mathsf{diag}}$ in Example~\ref{ex:diag}, but now only puts value $1$ on the diagonal.
% Observe that we start the loop using an initialisation by $I$. Hence, when evaluated on an instance $\I$ such that $V$ is an adjacency matrix $A$ of a graph, the expression above simply computes $(I+A)^n$, followed by the pointwise function application $f_{>0}$. In other words, it returns the adjacency matrix of the transitive closure of graph.\qed
%\end{example}
% \smallskip
% \noindent
% \textbf{More examples}. We conclude by showing that the two examples at the beginning of this section can be
% expressed in \langfor. Indeed, for $4\mathsf{clique}$ it suffices to consider
%
% The modified \texttt{for} expression will now use $B$ as the assignment of $X$ in the very first iteration (as determined by $\mmin{v}$), and will proceed as before whenever $v$ is not equal to the first canonical vector, thus defining the desired result. Since starting iteration with a non null matrix is often useful, we will denote  by $\initf{B}{v}{X}{e}$ the expression $(\dag)$.

% \noindent{\bf For loops.}
% The biggest novelty of $\langfor$ is the for operator. To illustrate how this operator can be used, we first show how one can construct the identity matrix of needed dimension in \langfor.

%%NOT DOABLE IF NO ACCESS TO DIMENSION
%\noindent{\bf Elementary matrix operations.} Here we show the true value of adding canonical vectors to the language, by illustrating how they allow us to express elementary matrix operations \cite{linalg}:
%\begin{enumerate}
%\item {\bf Row switching.} To switch the row $i$ and row $j$ of an $m\times n$ matrix $M$, we can simply use the expression $T^{ij} \cdot M$, where:
%$$T^{ij} = I^m - e_i^m\cdot (e_i^m)^* - e_j^m\cdot (e_j^m)^* + e_i^m\cdot (e_j^m)^* + e_j^m\cdot (e_i^m)^*.$$
%\item {\bf Row multiplication.} Multiplying a row by a scalar $c$ is performed by 
%\item {\bf Row addition.}
%\end{enumerate}
%Analogously, we can perform these transformations of columns, by using the transposed matrix $M^*$, and the canonical vectors of appropriate dimension.

% To begin with, consider the expression $$v_{max} := \ffor{v}{X}{v}.$$
% we can easily obtain the last canonical vector using the expression
% \medskip
%
% \noindent{\bf Core operators}
% Note that the  \texttt{for} operator is much more powerful that it seems. We can simulate some operations of the original \lang\ semantics. Assume the allowed expressions are
%
% \begin{tabular}{lcll}
% $e$ & $:=$ & $V\in \Mvar$ & (matrix variable)\\
%  & $|$ & $e^T$ & (transpose)\\
%  & $|$ & $e_1 \cdot e_2$ & (matrix multiplication)\\
%  & $|$ & $e_1 + e_2$ & (matrix addition)\\
%  & $|$ & $\text{apply}[f](e_1,\ldots ,e_n)$ & (application of $f\in \Fun$)\\
%  & $|$ & $\ffor{v}{X}{e}$ & (canonical for loop, with $v, X \in \Mvar$).
% \end{tabular}
%
% Where function application $\text{apply}[f]$ works only on scalars. This is
%
% \begin{itemize}
% \item $\ttype(\text{apply}[f](e_1,\ldots ,e_n)) = (1,1)$, whenever $\ttype(e_1)= \cdots =\ttype(e_n)=(1,1)$, and $f:\mathbb{C}^n\mapsto \mathbb{C}$; and is undefined otherwise.
% \item $\sem{\text{apply}[f](e_1,\ldots ,e_n)}{\I}$ is a $1\times 1$ matrix whose only entry has the value $f(\sem{e_1}{\I},\ldots ,\sem{e_n}{\I})$.
% \end{itemize}
%
% \subsection{Minimal Core}
% We have the following:
%
% \begin{itemize}
% \item {\em Pointwise function application.} Note that
%
% \begin{align*}
% f(e_1,\ldots, e_n)&=\\
% &\texttt{for }v,X.\quad X + \\
% &\quad \texttt{for }w,Y.\quad Y + \\
% &\quad \quad v\left[ \text{apply}[f](v^Te_1w, \ldots, v^Te_nw) \right] w^T
% \end{align*}
%
% \item {\em Conjugate transpose.} Let $\texttt{conj}(a) := \overline{a}$ be the conjugate function. Then we can simulate the conjugate transpose of expression $e$ as $\texttt{conj}(e^T)$.
% \item {\em One vector.} Using the function $f(x) := 1$, we can define $$\ones(e) := f(\texttt{for } v,X. X + e\cdot v).$$
% \item {\em Multiplying a matrix by a scalar.} Let $c$ be a scalar ($1\times 1$ matrix) and $f_{\times}(a, b) := a \times b$ . Then, for an expression $e$, $c\odot e := f_{\times}(\ones(e)\cdot c \cdot \ones(e^T)^T, e)$.
% \item {\em Diagonal of a vector.} The operator $\diag(e)$ can be defined as:
% $$\diag(e) := \texttt{for } v, X. X + (v^T\cdot e) \odot vv^T.$$
% \end{itemize}
%
% \section{Order!!}
%
% \noindent{\bf Order.} The \texttt{for} operator assumes that canonical vectors come in some particular order, however, it does not give us an immediate access to this order. An interesting question is whether, using the \texttt{for} loops, we can define a \langfor expression which allow us to define the order of canonical vectors. Next we show that this is indeed possible. To begin with, we can easily obtain the last canonical vector using the expression $$v_{max} := \ffor{v}{X}{v}$$
%
% The fundamental property of iteration we use here is that the result variable is always initiated with the null matrix. Therefore the loop above will simply keep on storing the current canonical vector before returning the final one. The ability to do this sort of manipulation was one of the reasons why we initiate $A_0$ in the semantics of \texttt{for} to null matrix.
%
% To define an order relation for canonical vectors, notice that the following matrix:
% \[
% Z_{eq} = \begin{bmatrix}
%     1 & 1 & \cdots &  1 \\
%     0 & \ddots & \ddots & \vdots \\
%     \hdotsfor{3} & 1 \\
%     0 & \cdots & \cdots & 1
% \end{bmatrix},
% \]
% has the property that $e_i^*\cdot Z_{eq} \cdot e_j$, for two canonical vectors $e_i,e_j$ of the same dimension, is equal to one if and only $i\leq j$, and is zero otherwise. $Z_{eq}$ can easily de defined in \langfor as follows:
% $$Z_{\text{eq}}=\ffor{v}{X}{X + \left[ (Xv_{max}) + v \right]v^* + vv^*_{max}},$$
% where $v_{max}$ is as defined above. The intuition behind this expression is that using the last canonical vector $v_{max}$, we have access to the final column of $X$ (via the product $X\cdot v_{max}$), to which we add the current canonical vector $v$, thus constructing $Z_{eq}$ by filling it column by column. By defining $$\lleq{u}{v} := u^*\cdot Z_{eq} \cdot v,$$
% we obtain an order relation that allows us to discern whether one canonical vector comes before the other in the order given by \texttt{for}. If we want a strict order, we can just use $Z_< := Z_{eq} - I$.
%
% Interestingly, we can also define the predecessor relation between canonical vectors. For this, we require the following matrix:
% %\[
% %S := \begin{bmatrix}
% %    0 & 1         & 0         & \cdots &  0 \\
% %    0 & \ddots & 1         & \cdots & 0 \\
% %    \vdots & \vdots & \ddots & \ddots & \vdots \\
% %    \hdotsfor{3} & \ddots & 1 \\
% %    0 & \cdots & \cdots & \cdots & 0
% %\end{bmatrix}.
% %\]
% \[
% S = \begin{bmatrix}
%     0 & 1 & \cdots &  0 \\
%     0 & \ddots & \ddots & \vdots \\
%     \hdotsfor{3} & 1 \\
%     0 & \cdots & \cdots & 0
% \end{bmatrix},
% \]
% Using this matrix, we have that for a canonical vector $e_i$:
% \[
%   			S\cdot e_i=\begin{cases}
%                e_{i-1}, \text{ if } i > 1 \\
%               \mathbf{0}, \text{ if } i = 1
%             \end{cases}
% 		\]
% where $\mathbf{0}$ is a vector of zeros of the same type as $e_i$. Notice also that $\ones(v)^*\cdot S \cdot v$ is equal to zero, for a canonical vector $v$, if and only if $v = e_1$ is the first canonical vector, and zero otherwise. To define the first canonical vector in the order given by \texttt{for}, we can then write:
% $$v_{min} := \ffor{v}{X}{\mmin{v}\cdot v},$$
% where the expression $\mmin{v}$ is defined as $\mmin{v} := 1 - \ones(v)^*\cdot S \cdot v$, and, when evaluated over canonical vectors, will result in $1$ if and only if $v=e_1$ is the first canonical vector. Finally, we denote that $S$ can be defined using the following \langfor expression:
% \begin{multline*}
% S:= \texttt{for }v,X.\quad X + \\
% \left[ (1 - v^*v_{max})vv_{max}^* - (Xv_{max}) v_{max}^* + (Xv_{max})v^*\right].
% \end{multline*}
%
%
%
%



\section{Expressiveness of for loops}
%To avoid nasty technical stuff, we will try to connect with circuits:
%\begin{itemize}
%\item Upper bound (AC?)
%\item Lower bound (transitive closure?)
%\end{itemize}

% OTHER IDEAS ABOUT PRESENTATION
%OPTION 1:
% LINK WITH TMS FIRST (NO CIRCUITS)!
% NOW INTRODUCE CIRCUITS
% USING THIS LINK, ASSUME THAT FOR EVALUATING A CIRCUIT INFO ABOUT A GATE IS AVAILABLE THROUGH AN EFFICIENT TM (NO UNIFORMITY)
% STATE THE RESULT ABOUT SIMULATION OF A SINGLE CIRCUIT IN ML
% NOW ESTABLISH A LINK WITH UNIFORM CIRCUITS -- JUST SAY THAT WE ALREADY KNOW HOW TO SIMULATE THE MACHINE, SO ALL GOOD
% NOW GO IN THE OTHER DIRECTION WITH A MORE EXPRESSIVE CLASS OF CIRCUITS

%OPTION 2:
% DEFINE ARITHMETIC CIRCUITS AND EVALUATE A SINGLE CIRCUIT
% LINK WITH TMS WHEN ASSUMING THAT TMS ARE GIVEN FOR CIRCUIT DATA
% ASSUME THAT FOR EVALUATING A CIRCUIT INFO ABOUT A GATE IS AVAILABLE THROUGH AN EFFICIENT TM (NO UNIFORMITY)
% NOW GO IN THE OTHER DIRECTION WITH A MORE EXPRESSIVE CLASS OF CIRCUITS
% NOW ESTABLISH A LINK WITH UNIFORM CIRCUITS


In this section we explore the expressive power of $\langfor$. Given that \langfor\ expressions compute functions over matrices whose entries are not only boolean, but can in fact be arbitrary elements from $\RR$, a natural candidate for comparison is the class of arithmetic circuits \cite{allender}. As we show in the remainder of this section, \langfor\ actually captures the expressive power of arithmetic circuits. To show this result, we first recall the definition of arithmetic circuits.
%
%In order to derive this result, we first look at the connection of \langfor\ and Turing machines. When observing a \langfor\ expression over some schema, perhaps the most crucial characteristic is the fact that depending on the instance, the number of iterations that a for loop does changes, as does the size of the matrix computed by our expression. This is analogous to how the number of steps a Turing machine takes changes depending on the input size, and gives some intuition on why \langfor\ expressions might be able to simulate Turing machines. Indeed, what we show is that \langfor\ expressions can actually  simulate Turing machines that use linear space and run in polynomial time. Formally, we prove the following.
%
%\begin{theorem}
%\label{th-tm-ml}
%Let $T$ be a Turing machine with $\ell$ read only input tapes, a work tape, and an output tape. 
%\end{theorem}
%
%Next we move to comparison with arithmetic circuits. 

An {\em arithmetic circuit} $\Phi$ over a set $X=\{x_1,\ldots,x_n\}$ of variables is a directed
acyclic labelled graph. The vertices of $\Phi$ are called {\em gates} and denoted by $g_1,\ldots,g_m$;
the edges in $\Phi$ are called {\em wires}. The children of a gate $g$ correspond to all gates
$g'$ such that $(g,g')$ is an edge. The parents of $g$ correspond to all gates $g'$ 
such that $(g,g')$ is an edge. The {\em in-degree}, or a {\em fan-in}, of a gate $g$ refers to its number of children,
the {\em out-degree} to its number of parents. We will not assume any restriction on the in-degree of a gate, and will thus consider circuits with unbounded fan-in. Gates with in-degree $0$ are called {\em input gates}
and are labelled by either a variable in $X$ or a constant $0$ or $1$. All other gates
are labelled by either $+$ or $\times$, and are referred to as {\em sum gates} or {\em product gates}, respectively.
Gates with out-degree $0$ are called {\em output gates}. When talking about arithmetic circuits, one usually focuses on circuits with $n$ input gates and a single output gate.

The {\em size} of $\Phi$, denoted by $|\Phi|$, is its number of gates. The {\em depth} of $\Phi$, denoted
by $\mathsf{depth}(\Phi)$, is the length of the longest directed path from the any output gate to any of the input gates. The {\em degree} of a gate is defined inductively: a leaf node has degree 1, a plus node has a degree equal to the maximum of  degrees of its children, and a product node has a degree equal to the sum of the degrees of its children. When $\Phi$ has a single output gate, the degree of $\Phi$, denoted by $\mathsf{degree}(\Phi)$, is defined as the degree of its output gate. If $\phi$ has a single output gate and its input gates take values from $\RR$, then $\Phi$ corresponds to a polynomial in $\RR[X]$ in a natural way. In this case, the {\em degree} of $\Phi$ equals the degree of the polynomial corresponding to $\Phi$.
If $v_1,\ldots ,v_n$ are values in $\RR$, then we define the result of the circuit on this input as the value computed by the corresponding polynomial, and denote this value with $\Phi_k(v_1,\ldots ,v_k)$.

%\domagoj{Should we already present the result without talking about uniformity?}

% MAYBE OVERCOMPLICATING
%In order to talk about efficient evaluation of arithmetic circuits in terms of complexity classes defined by Turing machines, we need a notion of uniformity of circuit families. An {\em arithmetic circuit family} is a set of arithmetic circuits $\{\Phi_n\mid n=1,2,\ldots\}$ where $\Phi_n$ has $n$ input variables. An arithmetic circuit family is {\em uniform} if there exist \logspace-Turing machines\footnote{There are several versions of this definition that give a different amount of resources to the Turing machine. See \cite{allender} for an in-depth discussion on the subject.} $M_1$ and $M_2$, such that:
%\begin{itemize}
%\item On input $1^n$, the machine $M_1$ returns an encoding of the arithmetic circuit $\Phi_n$ for each $n$;
%\item On input $1^n$, and an encoding of a gate $g$, the machine $M_2$ outputs the relevant information about $g$ (e.g. whether it is a sum or a product gate, the list of its children, whether it is an output gate, etc.).
%\end{itemize}
%
%The idea here is that the machine $M_1$ gives an overview of the circuit itself, while the machine $M_2$ allows us to efficiently evaluate the circuits by traversing the circuit graph in a depth-first fashion, starting from the output node. We observe that uniform arithmetic circuit families are necessarily of polynomial size.
%
%SHOULD EXPLAIN BRIEFLY HOW TO EVALUATE A CIRCUIT!

%In order to talk about efficient evaluation of arithmetic circuits in terms of complexity classes defined by Turing machines, we need a notion of uniformity of circuit families. 
In order to talk about efficient evaluation of arithmetic circuits, we need a way to compute their output algorithmically. Namely, we need a way to access their components such as the children of some gate, the value of an input gate, the operation  carried out by the gate, etc. The most general way to do this is via Turing machines. Additionally, this will allow us to handle inputs of arbitrary size, similarly as when working with Turing machines. This idea is captured by the notion of uniform circuit families. 

An {\em arithmetic circuit family} is a set of arithmetic circuits $\{\Phi_n\mid n=1,2,\ldots\}$ where $\Phi_n$ has $n$ input variables and a single output variable. An arithmetic circuit family is {\em uniform} if there exists a \logspace-Turing machine\footnote{There are several versions of this definition that give a different amount of resources to the Turing machine. See \cite{AllenderJMV98} for an in-depth discussion on the subject.}, which on input $1^n$, returns an encoding of the arithmetic circuit $\Phi_n$ for each $n$.
% Additionally, we assume that there is a \logspace-machine which, when started on input $1^n$, and an encoding of a gate $g$, outputs the relevant information about $g$ in $\Phi_n$ (e.g. whether it is a sum or a product gate, the list of its children, whether it is an output gate, etc.).
We observe that uniform arithmetic circuit families are necessarily of polynomial size, however, their degree can grow exponentially. A circuit family 
$\{\Phi_n\mid n=1,2,\ldots\}$ is said to be of polynomial degree if $\mathsf{degree}(\Phi_n)\in O(p(n))$, for some polynomial $p(n)$. Similarly, a circuit family is of logarithmic depth, whenever $\mathsf{depth}(\Phi_n)\in O(logn)$. We can now show that \langfor\ subsumes uniform arithmetic circuit families that are of polynomial degree and logarithmic depth. 

%\domagoj{The thing about the second machine is crucial, but it might be a bit confusing here, as we only need it to explain how one actually computes the output of a circuit. Because of this I'm thinking of maybe presenting the connection without uniformity.}

\begin{theorem}
\label{th-circuits-ml}
For any uniform arithmetic circuit family $\{\Phi_n\mid n=1,2,\ldots\}$ of polynomial degree and logarithmic depth there is a \langfor\ schema $\Sch$ and an expression $e_\Phi$ using a matrix variable $v$, with $\ttype(v)=(\alpha,1)$ and $\ttype(e) = (1,1)$, such that for any input $v_1,\ldots ,v_n$ to the circuit $\Phi_n$:
\begin{itemize}
\item If $\I = (\dom,\conc)$ is a \lang\ instance such that $\dom(\alpha) = n$ and $\conc(v) = (v_1 \ldots v_n)^*$
\item Then $\sem{e}{\I} = \Phi_n(v_1,\ldots ,v_n)$.
\end{itemize}
\end{theorem}

It is important to note that the expression $e_\Phi$ does not change depending on the input size, meaning that it is uniform in the same sense as the circuit family being generated by a single Turing machine. The different input sizes for a \langfor\ instance are handled by the typing mechanism of the language. The expression $e_\Phi$ on input $(v_1,\ldots ,v_n)^*$ basically simulates depth-first evaluation of the arithmetic circuit $\Phi_n$. 

In order to do this, we consider an algorithm for evaluating $\Phi_n$ on input $(v_1,\ldots ,v_n)$ that maintains  two stacks: the gates-stack that tracks the current gate being evaluated, and the values-stack that stores the value that is being computed for this gate. The idea behind having two stacks is that whenever the number of items on the gates-stack is higher by one than the number of items on the values-stack, we know that we are processing a fresh gate, and we have to initialize its current value (to 0 if it is a plus gate, and to 1 if it is a product gate), and push it to the values-stack. We then proceed by processing the children of the head of the gates-stack one by one, and aggregate the results using sum if we are working with a plus gate, and product otherwise. 

In order to store these two stacks, we can use vectors of size $n$, where $n$ is the number of inputs. Here we crucially depend on the fact that the circuit is of logarithmic depth, and therefore the size of the two stacks is bounded by $n$ (apart from the portion before the asymptotic bound kicks-in, which can be hard coded into the formula). 

...

Now that we know that arithmetic circuits can be simulated using \langfor\ expressions, it is natural to ask whether the same holds in the other direction. That is, we are asking whether for each \langfor\ expression $e$ over some schema $\Sch$ there is a uniform family of arithmetic circuits computing precisely the same result depending on the input size. In order to show this, we need to extend the 

EXTEND THE CIRCUITS...


\section{Restricting the power of for loops}
% !TeX spellcheck = en_US

\newcommand{\hprod}{\circ}

\subsection{Sumation matlang and relational algebra}

When defining the identity matrix and several other expressions, we actually only update $X$ by adding some matrix to it. This restricted form of the $\texttt{for}$ loop proved to be useful throughout the paper, and we will therefore introduce it a special operator. That is, we define:
$$\Sigma v. e := \ffor{v}{X}{X + e}.$$
We define the subfragment of \langfor, called \langsum, to consist of the $\Sigma$ operator plus the ``core'' operators in \lang, namely, transposition, matrix multiplication and addition, scalar multiplication, and pointwise function applications.

Apart from defining the identity matrix with \langsum, the sum quantifier also allows for computing the trace of a matrix $A$ using the expression $tr(A) := \Sigma v. v^*\cdot A \cdot v$. Interestingly enough, this restricted version of \texttt{for} already allows us to capture the \lang\ operators that are not present in the syntax of \langsum. More precisely, we have:
\smallskip

\noindent {\em Function application.} Notice that in \lang, a function is applied pointwise to matrices of arbitrary size, while \langsum only allows functions that process matrices of size $1\times 1$. Using the summation operator we can lift this condition, and allow applying a function $f:\mathbb{\RR}^n \mapsto \mathbb{\RR}$ on expressions $e_1,\ldots ,e_n$ of arbitrary (but equal) type by writing 
$$\Sigma x_i \Sigma x_j. f(x_i^T\cdot e_1\cdot x_j, \ldots ,x_i^T\cdot e_n\cdot x_j) \cdot x_i\cdot x_j^T,$$
which simply reconstructs the matrix obtained by applying $f$ to every position of $e_1$ through $e_n$, by using the fact that for two canonical vectors $b_i^m$ and $b_j^n$, the product $b_i^m \cdot (b_j^n)^*$ defines a $m\times n$ matrix whose only non-zero entry is in the position $ij$.
\smallskip

\noindent {\em One vector.} We can define $\ones(e) := \Sigma v.\, v.$ where $\ttype(v) = (\alpha, 1)$ and $\ttype(e) = (\alpha, \beta)$ for some $\beta$. 
\smallskip

\noindent {\em Diagonal of a vector.} The operator $\diag(e)$ can be defined as:
$$\diag(e) := \Sigma v. (v^T\cdot e) \cdot vv^T.$$

Furthermore, one can easily check that the 4-clique expression of Example~\ref{ex:fourcliques} can be defined in \langsum. Therefore, we conclude the following result. 
\begin{corollary}
\lang\ is strictly subsumed by \langsum.
\end{corollary}

%To show that the inclusion here is strict, we illustrate how one can detect whether a undirected graph has a four clique, which it is not definable in \lang~\cite{BrijderGBW19}. For this, we define an expression $f(u,v) := 1 - u^*\cdot v$. Notice that when $u$ and $v$ are interpreted by two canonical vector of the same dimension, we have that:
%\[
%  			f(u,v)=1-u^*v=\begin{cases}
%               0 \text{ if } u=v \\
%               1 \text{ if } u\neq v
%            \end{cases}
%		\]
%
%To distinguish four-cliques, we will need to determine whether we are dealing with four different nodes. For this, we will utilize the function $$g(u,v,w,r)=f(u,v)\cdot f(u,w)\cdot f(u,r)\cdot f(v,w)\cdot f(v,r)\cdot f(w,r),$$
%which, when evaluated over four canonical vectors of the same dimension, will give us 1 if and only if the four vectors are distinct. With this at hand, we can now define:		
%\begin{multline*}
%\texttt{4-clique}(A) := \ssum v_1.\ssum v_2. \ssum v_3. \ssum v_4.\\ (v_1^*Av_2)(v_1^*Av_3)(v_1^*Av_4)(v_2^*Av_3)(v_2^*Av_4)(v_3^*Av_4) \cdot
%\\g(v_1,v_2,v_3,v_4).
%\end{multline*}
%
%When $A$ is an adjacency matrix of an undirected graph $G$, then we have that $\texttt{4-clique}(A)$ is different from zero if and only if $G$ has a four-clique. Using this, and the fact that \lang\ is subsumed by First Order Logic with aggregates that uses only three variables \cite{matlang}, we immediately obtain the following:
%
%\begin{corollary}
%There is a  \langfor expression that is not expressible in \lang.
%\end{corollary}

What operations over matrices can be defined with \langsum that is beyond \lang? In~\cite{brijder2019matrices}, it was shown that \lang\ was strictly included in the relational algebra of $K$-relations~\cite{GreenKT07}, called Annotated Relational Algebra (ARA) in~\cite{brijder2019matrices}.
Then a natural idea is to compare the expressive power of \langsum with ARA. For making this comparison clear, in the following we give the formal definition of ARA~\cite{GreenKT07} to then see how to connect both formalism.

\newcommand{\ddom}{\mathbb{D}}
\newcommand{\fdom}{\operatorname{dom}}
\newcommand{\att}{\mathbb{A}}
\newcommand{\tuples}{\mathbf{tuples}}
\newcommand{\supp}{\operatorname{supp}}
\newcommand{\cJ}{\mathcal{J}}
\newcommand{\cR}{\mathcal{R}}
\newcommand{\adom}{\mathbf{adom}}

\newcommand{\ksum}{\oplus}
\newcommand{\kprod}{\odot}
\newcommand{\bigksum}{\bigoplus}
\newcommand{\bigkprod}{\bigodot}
\newcommand{\kzero}{\mymathbb{0}}
\newcommand{\kone}{\mymathbb{1}}

\newcommand{\row}{\mathsf{row}}
\newcommand{\rows}{\mathsf{rows}}
\newcommand{\col}{\mathsf{col}}
\newcommand{\cols}{\mathsf{cols}}
\newcommand{\arae}{Q}


Let $\ddom$ be a data domain and $\att$ a set of attributes. A relational signature is a finite subset of $\att$. A relational schema is a function $\cR$ on finite set of symbols $\fdom(\cR)$ such that $\cR(R)$ is a relation signature for each $R \in \fdom(\cR)$. To simplify the notation, from now on we write $R$ to denote both the symbol $R$ and the relational signature $\cR(R)$.
Furthermore, we write $R \in \cR$ to say that $R$ is a symbol of $\cR$. 
For $R \in \cR$, an $R$-tuple is a function $t: R \rightarrow \ddom$. We denote by $\tuples(R)$ the set of all $R$-tuples. Given $X \subseteq R$, we denote by $t[X]$ the restriction of $t$ to the set $X$.

A semiring $(K, \ksum, \kprod, \kzero, \kone)$ is an algebraic structure where $K$ is a non-empty set, $\ksum$ and $\kprod$ are binary operations over $K$, and $\kzero, \kone \in K$. Furthermore,  $\ksum$ and $\kprod$ are associative operations, $\kzero$ and $\kone$ are the identities of $\ksum$ and $\kprod$ respectively, $\ksum$ is a commutative operation, $\kprod$ distributes over $\ksum$, and $\kzero$ annihilates $K$ (i.e. $\kzero \kprod k = k \kprod \kzero = \kzero$). As usual, we assume that all semirings in this paper are commutative, namely, $\kprod$ is also commutative. We use $\bigksum_X$ or $\bigkprod_X$ for the $\ksum$- or $\kprod$-operation over all elements in $X$, respectively. Typical examples of semirings are the reals $(\RR, +, \times, 0,1)$, the natural numbers $(\RR, +, \times, 0,1)$, and the boolean semiring $(\{0,1\}, \vee, \wedge, 0, 1)$. 

Fix a semiring $(K, \ksum, \kprod, \kzero, \kone)$ and a relational schema $\cR$. A $K$-relation of $R \in \cR$ is a function $r: \tuples(R) \rightarrow K$ such that the support  $\supp(r) = \{t \in \tuples(R) \mid r(t) \neq \kzero\}$ is finite. 
A $K$-instance $\cJ$ of $\cR$ is a function that assigns relational signatures of $\cR$ to $K$-relations. Given $R \in \cR$, we denote by $R^\cJ$ the $K$-relation associated to $R$. Recall that $R^\cJ$ is a function and then $R^\cJ(t)$ is the value in $K$ assign to $t$. 
Given a $K$-relation $r$ we denote by $\adom(r)$ the active domain of $r$ defined as $\adom(r) = \{t(a) \mid t \in \supp(r) \wedge a \in R\}$.
Then the active domain of an $K$-instance $\cJ$ of $\cR$ is defined as $\adom(\cJ) = \bigcup_{R \in \cR} \adom(R^\cJ)$. 

An ARA expression $\arae$ over $\cR$ is given by the following syntax:
$$
\begin{array}{rcl}
\arae & := & R \ \mid \ \arae \cup \arae \ \mid \  \pi_X(\arae) \ \mid \  \sigma_X(\arae) \ \mid \ \rho_f(\arae) \ \mid \ \arae \bowtie \arae
\end{array}
$$
where $R \in \cR$, $X \subseteq \att$ is finite, and $f: X \rightarrow Y$ is a one to one mapping with $Y \subseteq \att$. One can extend the schema $\cR$ to any expression over $\cR$ recursively as follows: $\cR(R) = R$, $\cR(\arae \cup \arae') = \cR(\arae)$, $\cR(\pi_X(\arae)) = X$, $\cR(\sigma_X(\arae)) = \cR(\arae)$, $\cR(\rho_f(\arae)) = X$ where $f:X \rightarrow Y$, and $\cR(\arae \bowtie \arae') = \cR(\arae) \cup \cR(\arae')$ for every expressions $\arae$ and $\arae'$.
We further assume that any expression $\arae$ satisfies the following syntactic restrictions: $\cR(\arae') = \cR(\arae'')$ whenever $\arae = \arae' \cup \arae''$, $X \subseteq \cR(\arae')$ whenever $\arae = \pi_X(\arae')$ or $\arae = \sigma_X(\arae')$, and $Y = \cR(\arae')$ whenever $\arae = \rho_f(\arae')$ with $f: X \rightarrow Y$.

Given an ARA expression $\arae$ and a $K$-instance $\cJ$ of $\cR$, we define the semantics $\ssem{\arae}{\cJ}$ as a $K$-relation of $\cR(\arae)$ as follows. For $X \subseteq \att$, let $\operatorname{Eq}_X(t) = \kone$ when $t(a) = t(b)$ for every $a, b \in X$, and $\operatorname{Eq}_X(t) = \kzero$ otherwise. For every tuple $t \in \cR(\arae)$:
$$
\begin{array}{ll}
\text{if $\arae = R$, then} & \ssem{\arae}{\cJ}(t) = R^\cJ(t) \\
\text{if $\arae = \arae_1 \cup \arae_2$, then} & \ssem{\arae}{\cJ}(t) = \ssem{\arae_1}{\cJ}(t) \ksum \ssem{\arae_2}{\cJ}(t)  \\
\text{if $\arae = \pi_X(\arae')$, then} & \ssem{\arae}{\cJ}(t) = \bigksum_{t': t'[X] = t} \ssem{\arae'}{\cJ}(t') \\
\text{if $\arae = \sigma_X(\arae')$, then} & \ssem{\arae}{\cJ}(t) = 
\ssem{\arae'}{\cJ}(t) \kprod \operatorname{Eq}_X(t)  \\
\text{if $\arae = \rho_f(\arae')$, then} & \ssem{\arae}{\cJ}(t) = 
\ssem{\arae'}{\cJ}(t \circ f)  
\\
\text{if $\arae = \arae_1 \bowtie \arae_2$, then} & \ssem{\arae}{\cJ}(t) =  \ssem{\arae_1}{\cJ}(t[Y]) \kprod  \ssem{\arae_2}{\cJ}(t[Z])
\end{array}
$$
where $Y = \cR(\arae_1)$ and $Z = \cR(\arae_2)$. It is important to note that the $\bigksum$-operation on the semantics of $\pi_X(\arae')$ is well defined given that the support of $\ssem{\arae'}{\cJ}$ is always finite. 

We are ready for comparing \langsum with ARA. First of all, we need to extend \langsum from $\RR$ to any semiring. Indeed, one can easily verify that the semantics of \lang, \langfor and \langsum can be translated from $\RR$ to $K$ by switching from matrices over $(\RR, +, \times, 0, 1)$ to matrices over $(K, \ksum, \kprod, \kzero, \kone)$.
From now on we denote by  $\mtr{K}$ the set of all $K$-matrices. Similar than for \lang\ over $\RR$, given a \lang\ schema $\Sch$ a $K$-instance $\I$ over $\Sch$ is a pair $\I = (\dom,\conc)$, where $\dom : \DD \mapsto \NN$ assigns a value to each size symbol, and $\conc : \Mnam \mapsto \mtr{K}$ assigns a concrete $K$-matrix to each matrix variable. Then it is straightforward to extend the semantics of \lang, \langfor, and \langsum from $(\RR, +, \times, 0, 1)$ to $(K, \ksum, \kprod, \kzero, \kone)$ by switching $+$ with $\ksum$ and $\times$ with $\kprod$. 

The next step for comparing \langsum with ARA is to represent $K$-matrices as $K$-relations.
Let $\Sch=(\Mnam,\size)$ be a \lang\ schema. On the relational side
we have for each size symbol $\alpha\in\DD\setminus\{1\}$, attributes $\alpha$, $\row_\alpha$, and $\col_\alpha$ in $\att$. Furthermore, for each $V\in\Mnam$ and $\alpha \in \DD$ we denote
by $R_V$ and $R_\alpha$ its corresponding relation name, respectively. Then, given $\Sch$ we define the relational schema $\text{Rel}(\Sch)$ such that $\fdom(\text{Rel}(\Sch)) =  \{R_\alpha \mid \alpha\in\DD\} \cup \{R_V \mid V \in \Mnam\}$ where $\text{Rel}(\Sch)(R_\alpha) = \{\alpha\}$ and:
\[
\text{Rel}(\Sch)(R_V) = \begin{cases}
\lbrace\row_\alpha,\col_\beta \rbrace & \text{ if $ \size(V)=(\alpha,\beta)$} \\
\lbrace\row_\alpha \rbrace & \text{ if $ \size(V)=(\alpha,1)$} \\
\lbrace\col_\beta \rbrace  &
\text{ if $ \size(V)=(1,\beta)$} \\
\lbrace\rbrace & \text{ if $\size(V)=(1,1)$}.
\end{cases}
\]
Consider now a matrix instance $\I = (\dom,\conc)$ over $\Sch$.
Let $V\in\Mnam$ with $\size(V)=(\alpha,\beta)$ and let $\conc(V)$ be its corresponding $K$-matrix of dimension $\dom(\alpha)\times \dom(\beta)$.
To encode $\I$ as a $K$-instance in ARA, we use as data domain $\ddom = \mathbb{N} \setminus \{0\}$. Then we construct the $K$-instance $\text{Rel}(\I)$ such that for each $V\in\Mnam$ we define 
$R_V^{\text{Rel}(\I)}(t):=\conc(V)_{ij}$ whenever $t(\row_\alpha) = i \leq \dom(\alpha)$ and $t(\col_\beta) = j \leq \dom(\beta)$, and $\kzero$ otherwise. Furthermore, for each $\alpha \in \DD$ we define $R_\alpha^{\text{Rel}(\I)}(t):=\kone$ whenever $t(\alpha) \leq \dom(\alpha)$, and $\kzero$ otherwise. In other words, $R_\alpha$ and $R_\beta$ encodes the active domain of a matrix variable $V$ with $\size(V)=(\alpha,\beta)$. Given that the ARA framework of \cite{GreenKT07} represents the ``absence'' of a tuple in the relation with $\kzero$, we need to find a way to encode all entries of a matrix in ARA. For instance, we need to be able to encode a $\kzero$-matrix of dimension $(\alpha,\beta)$ in~ARA.

We are ready to state the first connection between \langsum and ARA by using the previous encoding.
\begin{proposition}
	For each \langsum expression $e$ over schema $\Sch$ such that $\Sch(e)=(\alpha,\beta)$ with $\alpha\neq 1\neq\beta$, there exists an ARA expression $\Phi(e)$ over relational schema $\text{Rel}(\Sch)$ such that $\text{Rel}(\Sch)(\Phi(e))=\{\row_\alpha,\row_\beta\}$ and 
	such that for any instance $\I$ over~$\Sch$,
	$$
	\sem{e}{\I}_{i,j}=\ssem{\Phi(e)}{\text{Rel}(\I)}(t)
	$$
	for tuple $t(\mathrm{row}_\alpha)=i$ and $t(\mathrm{col}_\beta)=j$. Similarly for when $e$ has schema $\Sch(e)=(\alpha,1)$, $\Sch(e)=(1,\beta)$ or $\Sch(e)=(1,1)$, then $\Phi(e)$ has schema $\text{Rel}(\Sch)(\Phi(e))=\{\mathrm{row}_\alpha\}$,
	$\text{Rel}(\Sch)(\Phi(e))=\{\mathrm{col}_\alpha\}$, or
	$\text{Rel}(\Sch)(\Phi(e))=\{\}$, respectively.
\end{proposition}
To translate ARA into \langsum, we must restrict our comparison to ARA over $K$-relations with at most two attributes. Given that linear algebra works over vector and matrices, it is reasonable to restrict to unary or binary relations as input. Note that this is only a restriction to the input relations and not to intermediate relations, namely, expressions can create relation signatures of arbitrary size from the binary input relations. Thus, from now we say that a relational schema $\cR$ is binary if $|R| \leq 2$ for every $R \in \cR$. We also make the assumption that there is an (arbitrary) order, denoted by $<$, on the attributes in $\att$. 
This is to identify which attributes correspond to rows and columns when moving to matrices. 
Then, given that relations will be  either unary or binary and there is an order in the attributes, we write $t = (v)$ or $t = (u,v)$ to denote a tuple over a unary or binary relation $R$, respectively, where $u$ and $v$ is the value of the first and second attribute with respect to $<$.

Consider a binary relational schema $\cR$. With each $R\in \cR$ we associate a matrix variable $V_R$ such that, if $R$ is a binary relational signature, then $V_R$ represents a (square) matrix, and, if not (i.e. $R$ is unary), then $V_R$ represents a vector. Formally, fix a symbol $\alpha \in \DD \setminus \{1\}$. Let $\text{Mat}(\cR)$ denote the \lang \ schema
$(\Mnam_\cR,\size_\cR)$ such that $\Mnam_\cR = \{ V_R \mid R \in \cR\}$ and $\size_\cR(V_R) = (\alpha, \alpha)$ whenever $|R| = 2$, and $\size_\cR(V_R) = (\alpha, 1)$ whenever $|R|=1$. 
Take now a $K$-instance $\cJ$ of $\cR$ and suppose that $\adom(\cJ) = \{d_1, \ldots, d_n\}$ is the active domain of $\cJ$ (i.e. the order over $\adom(\cJ)$ is arbitrary). Then we define the matrix instance $\text{Mat}(\cJ) = (\dom_\cJ,\conc_\cJ)$ such that $\dom_\cJ(\alpha) = n$, $\conc_\cJ(V_R)_{i,j} = R^{\cJ}((d_i, d_j))$ whenever $|R|=2$, and $\conc_\cJ(V_R)_{i} = R^{\cJ}((d_i))$ whenever $|R|=1$. 
Note that, although each $K$-relation can have different active domain, we encode them as square matrices by considering the active domain of the $K$-instance.

\begin{proposition}
	Let $\cR$ be a binary relational schema. For each ARA expression $\arae$ over $\cR$  such that $|\cR(\arae)| = 2$, there exists a \langsum  expression $\Psi(\arae)$ over \lang \ schema $\text{Mat}(\cR)$ such that for any $K$-instance $\cJ$ with $\adom(\cJ) = \{d_1, \ldots, d_n\}$ over $\cR$,
	$$
	\ssem{\arae}{\cJ}((d_i, d_j))=\sem{\Psi(\arae)}{\text{Mat}(\cJ)}_{i,j}.
	$$
	Similarly for when $|\cR(\arae)| = 1$, or $|\cR(\arae)| = 0$ respectively.
\end{proposition} 

It is important to remark that the $\arae$ of the previous result can have intermediate expressions that are not necessary binary given that the proposition only restricts that the input relation and the schema of $\arae$ must have arity at most two. 

Given the previous two propositions we derive the following conclusion which is the first characterization of relational algebra with a (sub)-fragment of linear algebra.
\begin{corollary}
	\langsum and ARA over binary relational schemas are equally expressive. 
\end{corollary}


\subsection{Hadamard product and weighted logics}

%!TEX root = ../main.tex
% !TeX spellcheck = en_US
% !TEX root = ../main.tex
Similarly to using sum, we can use other operations to update $X$ in the for-loop. The next natural choice is to consider products of matrices. In contrast to matrix sum, we have two options: either we can choose to use matrix product or to use the pointwise matrix product, also called the Hadamard product. We treat  matrix product in the next subsection and first explain here the connection of sum and Hadamard product operators to weighted logics.

For the rest of this section, fix a semiring $(K, \ksum, \kprod, \kzero, \kone)$. The Hadamard product over $K$-matrices can be defined as the pointwise application of $\kprod$ between two matrices of the same size. Formally, we define the expression $e \hadprod e'$ where $e, e'$ are expressions with respect to $\cS$ and $\ttype(e) = \ttype(e')$ for some schema $\Sch=(\Mnam,\size)$. Then the semantics of $e \hadprod e'$ is the pointwise application of $\kprod$, namely, $\sem{e \hadprod e'}{\I}_{ij} = \sem{e}{\I}_{ij} \kprod \sem{e'}{\I}_{ij}$ for any instance $\I$ of $\cS$. This enables us to define, similar as for  $\Sigma v$, the  pointwise-product quantifier $\qhadprod v$ as follows:
$$
\qhadprod v. \  e := \ffor{v}{X\!=\!e_{\kone}}{X \circ e}.
$$
where $e_\kone$ is the (easily definable) \langfor expression for the matrix with the same type as $X$ and all entries equal to the $\kone$-element of $K$ (i.e., we need to initialize $X$ accordingly with the $\kprod$-operator).
We call \langprod  the subfragment of \langfor that consists of \langsum \ extended with $\qhadprod v$.

\begin{example}
	Similar to the trace of a matrix, a useful function in linear algebra is to compute the product of the values on the diagonal. 
	Using the $\qhadprod v$ operator, this can be easily expressed:
	 $$
	 e_{\mathsf{dp}}(V) := \qhadprod v. \ v^T\cdot V \cdot v.$$
\end{example}

Clearly, the inclusion of this new operator extends the expressive power to \langsum. For example,  $\sem{e_{\mathsf{dp}}}{\I}$ can be an exponentially large number in the dimension $n$ of the input.
By contrast, one can easily show that all expressions in \langsum can only return numbers polynomial in  $n$. That is, \langprod is more expressive than \langsum and $\mathsf{RA}_{K}^+$. 

To measure the expressive power of \langprod, we use weighted logics~\cite{DrosteG05} (WL) as a yardstick. Weighted logics extend monadic second-order logic from the boolean semiring to any semiring $K$. Furthermore, it has been used extensively to characterize the expressive power of weighted automata in terms of logic~\cite{droste2009handbook}. We use here the first-order subfragment of weighted logics to suit our purpose and, moreover, we extend its semantics over weighted structures (similar as in~\cite{GradelV17}).

A relational vocabulary $\Gamma$ is a finite collection of relation symbols such that each $R \in \Gamma$ has an associated arity, denoted by $\arity(R)$.
A $K$-weighted structure over $\Gamma$ (or just structure) is a pair $\cA = (A, \{R^\cA\}_{R \in \Gamma})$ such that $A$ is a non-empty finite set (i.e. the domain) and, for each $R \in \Gamma$, $R^\cA: A^{\arity(R)} \rightarrow K$ is a function that associates to each tuple in $A^{\arity(R)}$ a weight in $K$.

Let $X$ be a set of first-order variables. A $K$-weighted logic (WL) formula $\varphi$ over $\Gamma$ is defined by the following syntax:
$$
\begin{array}{rcl}
\varphi & := & x = y \ \mid \ R(\bar{x}) \ \mid \ \varphi \ksum \varphi \ \mid \ \varphi \kprod \varphi \ \mid \ \Sigma x. \varphi \ \mid \ \Pi x. \varphi
\end{array}
$$ 
where $x, y \in X$, $R \in \Gamma$, and $\bar{x} = x_1, \ldots, x_k$ is a sequence of variables in $X$ such that $k=\arity(R)$. As usual, we say that $x$ is a free variable of $\varphi$, if $x$ is not below $\Sigma x$ or $\Pi x$ quantifiers (e.g. $x$ is free in $\Sigma y. R(x,y)$ but $y$ is not). 
Given that $K$ is fixed, from now on we talk about structures and formulas without mentioning $K$ explicitly.  

An assignment $\sigma$ over a structure $\cA = (A, \{R^\cA\}_{R \in \Gamma})$ is a function $\sigma: X \rightarrow A$. Given $x \in X$ and $a \in A$, we denote by $\sigma[x \mapsto a]$ a new assignment such that $\sigma[x \mapsto a](y) = a$ whenever $x = y$ and $\sigma[x \mapsto a](y) = \sigma(y)$ otherwise. For $\bar{x} = x_1, \ldots, x_k$,  we write $\sigma(\bar{x})$ to say $\sigma(x_1),\ldots, \sigma(x_k)$. Given a structure $\cA = (A, \{R^\cA\}_{R \in \Gamma})$ and an assignment $\sigma$, we define the semantics $\ssem{\varphi}{\cA}(\sigma)$ of $\varphi$ as follows:
$$
\begin{array}{ll}
\text{if $\varphi := x = y$, then} & \ssem{\varphi}{\cA}(\sigma) = 
\left\{
\begin{array}{ll}
\kone & \text{if $\sigma(x) = \sigma(y)$} \\
\kzero & \text{otherwise}
\end{array}
\right. \\
\text{if $\varphi := R(\bar{x})$, then} & \ssem{\varphi}{\cA}(\sigma) = R^\cA(\sigma(\bar{x})) \\
\text{if $\varphi := \varphi_1 \ksum \varphi_2$, then} & \ssem{\varphi}{\cA}(\sigma) = \ssem{\varphi_1}{\cA}(\sigma) \ksum \ssem{\varphi_2}{\cA}(\sigma)  \\
\text{if $\varphi := \varphi_1 \kprod \varphi_2$, then} & \ssem{\varphi}{\cA}(\sigma) = \ssem{\varphi_1}{\cA}(\sigma) \kprod \ssem{\varphi_2}{\cA}(\sigma)  \\
\text{if $\varphi := \Sigma x. \, \varphi'$, then} & \ssem{\varphi}{\cA}(\sigma) =  \bigksum_{a \in A} \ssem{\varphi'}{\cA}(\sigma[x \mapsto a]) \\
\text{if $\varphi := \Pi x. \, \varphi'$, then} & \ssem{\varphi}{\cA}(\sigma) =  \bigkprod_{a \in A} \ssem{\varphi'}{\cA}(\sigma[x \mapsto a])
\end{array}
$$
When $\varphi$ contains no free variables, we omit $\sigma$ and write $\ssem{\varphi}{\cA}$ instead of $\ssem{\varphi}{\cA}(\sigma)$.

For comparing the expressive power of \langprod with WL, we have to show how to encode \lang\ instances into structures and vice versa. For this, we make two assumptions to put both languages at the same level: (1) we restrict structures to relation symbols of arity at most two and (2) we restrict instances to square matrices. The first assumption is for the same reasons as when comparing \langsum with $\mathsf{RA}_K^+$, and the second assumption is to have a crisp translation between both languages. Indeed, understanding the relation of \langprod with WL for non-square matrices is slightly more complicated and we leave this for future work. 

Let $\Sch=(\Mnam,\size)$ be a schema of square matrices, that is, there exists an $\alpha$ such that $\size(V) \in \{1, \alpha\} \times \{1,\alpha\}$ for every $V \in \Mnam$.
Define the relational vocabulary $\text{WL}(\Sch) = \{R_V \mid V \in \Mnam\}$ such that $\arity(R_V) = 2$ if $\size(V) = (\alpha, \alpha)$, $\arity(R_V) = 1$ if $\size(V) \in \{(\alpha,1), (1,\alpha)\}$, and $\arity(R_V) = 0$ otherwise.
Then given a matrix instance $\I = (\dom,\conc)$ over $\Sch$ define the structure $\text{WL}(\I) = (\{1, \ldots, n\}, \{R_V^{\I}\} )$ such that $\dom(\alpha) = n$ and $R_V^{\I}(i, j) = \conc(V)_{i,j}$ if $\size(V) = (\alpha, \alpha)$, $R_V^{\I}(i) = \conc(V)_{i}$ if $\size(V) \in \{(\alpha,1), (1,\alpha)\}$, and $R_V^{\I} = \conc(V)$ if $\size(V) = (1,1)$.

To encode weighted structures into matrices and vectors, the story is similar as for $\mathsf{RA}_K^+$. Let $\Gamma$ be a relational vocabulary where $\arity(R) \leq 2$. 
Define $\text{Mat}(\Gamma) = (\Mnam_\Gamma,\size_\Gamma)$ such that $\Mnam_\Gamma = \{ V_{R} \mid R \in \Gamma\}$ and $\size_\Gamma(V_{R})$ is equal to $(\alpha, \alpha), (\alpha, 1)$, or $(1,1)$ if $\arity(R)=2$, $\arity(R)=1$, or $\arity(R)=0$, respectively, for some $\alpha \in \DD$. Similarly, let $\cA = (A, \{R^{\cA}\}_{R \in \Gamma})$ be a structure with $A = \{a_1, \ldots, a_n\}$, ordered arbitrarily.
Then we define the matrix instance $\text{Mat}(\cA) = (\dom,\conc)$ such that $\dom(\alpha) = n$, $\conc(V_{R})_{i,j} = R^{\cA}(a_i, a_j)$ if $\arity(R)=2$, $\conc(V_{R})_{i} = R^{\cA}(a_i)$ if $\arity(R)=1$, and $\conc(V_{R}) = R^{\cA}$ otherwise.

Let $\Sch$ be a \lang\ schema of square matrices and $\Gamma$ a relational vocabulary of relational symbols of arity at most $2$. We can then show the equivalence of \langprod and WL as follows. 
\begin{proposition} \label{prop:wl}
Weighted logics over $\Gamma$ and \langprod over $\Sch$ have the same expressive power. More specifically,
\begin{itemize}
	\item for each \langprod expression $e$ over $\Sch$ such that $\Sch(e)=(1,1)$, there exists a WL-formula $\Phi(e)$ over $\text{WL}(\Sch)$ such that for every instance $\I$ of~$\Sch$, 
	$
	\sem{e}{\I} = \ssem{\Phi(e)}{\text{WL}(\I)}
	$.
	\item for each WL-formula $\varphi$ over $\Gamma$ without free variables, there exists a \langprod expression $\Psi(\varphi)$ such that for any structure $\cA$ over~$\text{Mat}(\Gamma)$,
	$
	\ssem{\varphi}{\cA}=\sem{\Psi(\varphi)}{\text{Mat}(\cA)}
	$.
\end{itemize}	
\end{proposition}



\subsection{Matrix multiplication as a quantifier}

Analogously to the summation and Hadamard product with respect to canonical vectors, we can use the usual product of matrices. Formally, for an expression $e$ define:
$$
\sprod v.\,  e=\ffor {v}{X = I}{X\cdot e}.
$$
where $I$ is the identity matrix. 
Using the product operator we can express multiple interesting properties. To begin with, we can compute the product of diagonal elements of a matrix: 
$$
dp(A) := \sprod v. v^*\cdot A \cdot v.$$
Another property of interest is computing the transitive closure of a graph adjacency matrix $A$. It is well known the transitive closure of this matrix, denoted $tc(A)$ equals to the matrix consisting of non-zero entries of $(I + A)^n$, where $n$ is the dimension of $A$. Using the product operator we can define:
$$tc(A) := f_{>0}(\sprod v. (I + A)),$$
where $f_{>0}(x) := 1$ if $x>0$, and $f_{>0}(x) = 0$ otherwise, is used to make the result a zero-one matrix. Notice that the expression for $tc(A)$ ignores the canonical vectors, and simply multiplies the previous result with $(I + A)$, thus computing the desired value.

Using the combination of canonical sum and product, we can also define more general operators over matrices, such as the power sum operator, which, given a square matrix $A$, computes $I + A + A^2 + \cdots + A^n$. This operator, denoted by $ps(A)$ can be defined as:
$$ps(A) := \ssum v.\sprod w. \left( (w^* S_{<} v)\cdot (A-I) + I\right),$$
where $S_{<}$ is the (strict) order matrix defined in Section~\ref{sec:order}. The outer loop here defines which power we compute. That is, when $v$ is the $i$th canonical vector, we compute $A^i$. Computing $A^i$ is achieved via the inner product loop, which uses $w^*S_{<}v$ to determine whether $w$ comes before $v$ in the ordering of canonical vectors. When this is the case we multiply the current result by $A$, and when $w$ is greater than $v$, we use $I$ not to affect the already computed result.

Note that $\sprod v$ can define the quantifier $\qhadprod v$ and, further, it increases the expressive power of \langprod~.
\begin{proposition}
	Every expression in \langprod can be defined in \langsum extended with $\sprod v$ quantifier. Moreover, there exists an expression that uses the $\sprod v$ quantifier that cannot be defined in \langprod.
\end{proposition}

It is interesting that the $\sprod v$ give the power of transitive closure to \langsum in a natural way and without defining it explicitly. We leave the study of this operator and, in particular, to understand its expressibility for future work. 




\section{Connecting Linear Algebra with Relational Algebra}
We prove the ARA theorem here.

If we are short on pages, the proof sketch can make it here.

Perhaps discuss quantifier connection here and not in the previous section?

\section{Applications}
Here we show the real strength of our framework by connecting it to some well established areas:

\begin{itemize}
\item Graph query languages
\item Machine learning
\item Query optimization
\end{itemize}


\section{Conclusions}
Please, be the first one to give some nice conclusions!

%%
%% The next two lines define the bibliography style to be used, and
%% the bibliography file.
\bibliographystyle{ACM-Reference-Format}
\bibliography{biblio}

%%
%% If your work has an appendix, this is the place to put it.

\newpage 
\appendix

\section{Appendix}

The real work begins here.

\end{document}
\endinput
%%
%% End of file `sample-sigconf.tex'.
