% \section{Logspace Stuff}
%
% \section*{Simulating Logspace Turing machine on input $o^n$.}
%
% We are given a logspace Turing Machine $T=\left(Q,\{0\},\ell,\rhd,\lhd,\Delta,q_0,q_m\right)$ where $Q=\{q_1,\ldots,q_m\}$ are the states, $\ell$ denotes the number of heads, $q_0$ and $q_m$ denote initial and final state, respectively, $\{0\}$ is the tape alphabet, and $\rhd$ and $\lhd$ are special symbols denoting the beginning and the end of the tape, respectively. Finally,
% $\Delta=(\Delta_Q,\Delta_1,\ldots,\Delta_\ell)$ is the transition function of $T$, where $\Delta_Q:Q\times \{\rhd,0,\lhd\}^\ell\mapsto Q$ and for $i\in[\ell]$,
% $\Delta_i:Q\times  \{\rhd,0,\lhd\}^\ell\mapsto\{\gets,\to\}$. In other words, when $T$ is in state $q$ and the $\ell$ heads read symbols $b_1,\ldots,b_\ell$, $\Delta_Q(q,b_1,\ldots,b_\ell)$ indicates to which state $T$ will transition, and moreover, $\Delta_i(q,b_1,\ldots,b_\ell)$ says in which direction (left or right) the $i$th head will move. We assume that when $T$ is in the initial state $q_0$ all heads point to the first position (i.e., they all read symbol $\rhd$).
%
% In our setting, the tape contents will always be of the form $w_n:=\rhd 0^n \lhd$ for some $n\in\mathbb{N}$. As usual, $T$ cannot move beyond the begin and end markers, $\rhd$ and $\lhd$, respectively. We assume that $T$ accepts or rejects the input $w_n$ using $\mathcal{O}(n^k)$ steps. In other words, there exists a constant $c$ such that $T$ runs in at most $cn^k$ steps. For simplicity, we assume that $T$ runs for at most $n^{k+1}$ steps. For technical reasons, that will become clear below, we assume that once $T$ reaches the final state $q_m$, $T$ will further transition but only to the state $q_m$. That is, $\Delta$ contains transitions that make $T$ loop in $q_m$.
%
% \begin{proposition}
% Given a logspace Turing machine $T$ with $m$ states, $\ell$ heads and which runs on $w_n=\rhd 0^n \lhd$ in time $n^{k+1}$, for $n\in\mathbb{N}$, there exists (i)~a $\mathsf{MATLANG}$
% schema $\mathcal{S}=(\mathcal{M},\textsf{size})$ where $\mathcal{M}$ consists matrix variables\footnote{We also need a finite number of auxiliary variables, these will be specified in the proof.} $X_1,\ldots,X_m,Y_1,\ldots,Y_\ell, v_1,\ldots,v_{k+1}$ and $w$ with
% $\mathsf{size}(V)=\alpha\times 1$ for all $V\in\mathcal{M}$ and $V\neq w$ and $\mathsf{size}(w)=1\times 1$; and (ii)~a $\mathsf{MATLANG}$ expression $e_T$ over $\mathcal{S}$ such that for the instance $I=(\mathcal{D},\textsf{mat})$ over $\mathcal{S}$ with $\mathcal{D}(\alpha)=n+2$ and $\mathsf{mat}(V)=\left(\begin{smallmatrix}0\\
% 0\\\vdots\\0\end{smallmatrix}\right)$ for all $V\in\mathcal{M}$ and
% $\mathsf{mat}(w)=[0]$, we have that
% $e_T(I)=[1]$ if $T$ accepts $w_n$ and  $e_T(I)=[0]$ otherwise.
% \end{proposition}
% \begin{proof}
% We start by explaining the semantics of the matrix variables in $\mathcal{M}$. The variables $X_1,\ldots,X_m,Y_1,\allowbreak\ldots,Y_\ell$ will be used inside for loops and will be updated using \textsf{MATLANG} expressions. Initially, all these matrix variables are instantiated with the zero column vector, as described by the instance $I$.
%
% With each state $q_i\in Q$ we associate matrix variable $X_i$.
% Then, $T$ is in state $q_i$ when
%  $X_i=\left(\begin{smallmatrix}1\\
% 0\\\vdots\\0\end{smallmatrix}\right)$, otherwise $X_i=\left(\begin{smallmatrix}0\\
% 0\\\vdots\\0\end{smallmatrix}\right)$.	Similarly, with each head $i\in[\ell]$ we
% associate matrix variable $Y_i$. When the $i$th head points at position $j$ in $w_n$,
% then $Y_i=\mathsf{e}_j$, i.e., it is the $j$th canonical column vector. We remark that since the dimensions is $n+2$ and $T$ cannot change the input word $w_n$, the $n+2$ canonical vectors suffice to indicate all positions in $w_n$ (which is of length $n+2$).
%
% The variables $v_1,\ldots,v_{k+1}$ represent $k+1$ canonical vectors  which are use to iterate in for loops. By iterating over then, we can perform $(n+2)^{k+1}$ iterations, which suffices for simulating the $n^{k+1}$ steps used by $T$ on input $w_n$.
%
% Finally, the variable $w$ is used for the output of $e_T$. It contains a scalar and will hold $1$ if $w_n$ is accepted by $T$ and $0$ otherwise.
%
% The expression $e_T$ uses some subexpressions in $\mathsf{MATLANG}$ which use some auxiliary variables.
% As a consequence, $e_T$ is an expression defined over an extended schema $\mathcal{S}'$. Hence, the instance $I$ in the statement of the Proposition is in fact an instance $I'$ of $\mathcal{S}'$ which
% coincides with $I$ on $\mathcal{S}$ and in which the auxiliary matrix variables are all instantiated with zero vectors or matrices, depending on their size.
%
% We list the used subexpressions next and explicitly denote the auxiliary matrix variables:
% \begin{itemize}
% 	% \item $\mathsf{max}(z,Z)$, an expression over auxiliary variables $z$ and $Z$ with $\mathsf{size}(z)=\mathsf{size}(Z)=\alpha\times 1$. On input $I'$ with
% 	% $\mathsf{mat}(z)=\mathsf{mat}(Z)$ the zero column vector of dimension $n+2$,
% 	%  $\mathsf{max}(I)=\mathbf{e}_{n+2}$.
% 	\item $\mathsf{pred}(z,Z,z',Z')$, and expression over auxiliary variables $z$, $z'$, $Z$ and $Z'$ with $\mathsf{size}(z)=\mathsf{size}(z')=\mathsf{size}(Z)=\alpha\times 1$ and $\mathsf{size}(Z')=\alpha\times\alpha$. On input $I'$ with
% 	$\mathsf{mat}(z)=\mathsf{mat}(z')=\mathsf{mat}(Z)$ the zero column vector of dimension $n+2$, and $\mathsf{mat}(Z')$ the zero $(n+2)\times (n+2)$ matrix,
% 	 $\mathsf{pred}(I')$ returns an $(n+2)\times (n+2)$ matrix such that
%
% 	 $$\mathsf{pred}(I')\mathbf{e}_i:=\begin{cases}
% 	 \mathbf{e}_{i-1} & \text{if $i>1$}\\
% 	 \mathbf{0} & \text{if $i=1$}.
% 	\end{cases}
% 	$$
% 	In other words, $\mathsf{pred}$ defines a predecessor relation among canonical vectors of dimension $n+2$.
% 	 \item $\mathsf{succ}(z,Z,z',Z')$, and expression over auxiliary variables $z$, $z'$, $Z$ and $Z'$ with $\mathsf{size}(z)=\mathsf{size}(z')=\mathsf{size}(Z)=\alpha\times 1$ and $\mathsf{size}(Z')=\alpha\times\alpha$. On input $I'$ with
% 	$\mathsf{mat}(z)=\mathsf{mat}(z')=\mathsf{mat}(Z)$ the zero column vector of dimension $n+2$, and $\mathsf{mat}(Z')$ the zero $(n+2)\times (n+2)$ matrix,
% 	 $\mathsf{succ}(I')$ returns an $(n+2)\times (n+2)$ matrix such that
%
% 	 $$\mathsf{succ}(I')\mathbf{e}_i:=\begin{cases}
% 	 \mathbf{e}_{i+1} & \text{if $i<n+2$}\\
% 	 \mathbf{0} & \text{if $i=n+2$}.
% 	\end{cases}
% 	$$
% 	In other words, $\mathsf{succ}$ defines a successor relation among canonical vectors.
% 	\item $\textsf{ismin}(v,z,Z,z',Z)$ with auxiliary variables $z$, $z'$, $Z$ and $Z'$ as before, and $v$ is one of the (vector) variables in $\mathcal{M}$. For an $(n+2)\times 1$ vector $\mathbf{v}$, on input $I'[v\gets \mathbf{v}]$	$$\mathsf{ismin}(I'[v\gets\mathbf{v}]):=\begin{cases} 1 & \text{if $\mathbf{v}=\mathbf{e}_1$}\\
% 		0 & \text{otherwise}.
% 		\end{cases}$$
% 	\item $\textsf{ismax}(v,z,Z,z',Z)$ with auxiliary variables $z$, $z'$, $Z$ and $Z'$ as before, and
% 	and $v$ is one of the (vector) variables in $\mathcal{M}$. For an $(n+2)\times 1$ vector $\mathbf{v}$, on input $I'[v\gets \mathbf{v}]$
%
% 	$$\mathsf{ismax}(I'[v\gets\mathbf{v}]):=\begin{cases} 1 & \text{if $\mathbf{v}=\mathbf{e}_{n+2}$}\\
% 		0 & \text{otherwise}.
% 		\end{cases}$$
% 	\item $\mathsf{min}(z,Z,z',Z',z'',Z'')$, an expressions with
% 	auxiliary variables $z$, $z'$, $z''$, $Z$, $Z'$ and $Z''$ with $\mathsf{size}(z)=\mathsf{size}(z')=\mathsf{size}(z'')=\mathsf{size}(Z)=\mathsf{size}(Z'')\alpha\times 1$ and $\mathsf{size}(Z')=\alpha\times\alpha$. On input $I'$ with
% 	matrix variables instantiated with zero vectors (or matrix for $Z'$),
%  	 $\mathsf{min}(I')=\mathbf{e}_1$.
% 	 		\item We additionally define, based on the previous expressions,
% 			$$\mathsf{test}_b(v,z,Z,z',Z'):=\begin{cases} \mathsf{ismin}(v,z,Z,z,Z') & \text{if $b=\rhd$}\\
%      \mathsf{ismax}(v,z,Z,z,Z') & \text{if $b=\lhd$}\\
% 	 (1-\mathsf{ismin}(v,z,Z,z',Z'))(1-\mathsf{ismax}(v,z,Z,z',Z')) & \text{if $b=0$}.
% 		\end{cases}.$$
% 	When evaluated on $I'[v\gets\mathbf{v}]$, $\mathsf{test}_b(I'[v\gets\mathbf{v}])$ will be $1$ when either
% 	$\mathbf{v}=\mathbf{e}_{1}$ and $b=\rhd$ (first position)
% 	$\mathbf{v}=\mathbf{e}_{n+2}$ and $b=\lhd$ (last position),
% 	$\mathbf{v}\neq \mathbf{e}_{1}$ and $\mathbf{v}\neq \mathbf{e}_{n+2}$  and $b=0$ (not first or last position). We use this expression below to check whether the heads are consistent with the symbols on the tape.
%
%
%  \item Finally, we define
%  $$
%  \mathsf{move}_d(z,Z,z',Z'):=\begin{cases}
%  e_{\mathsf{pred}}(z,Z,z',Z) & \text{if $d=\gets$}\\
%   e_{\mathsf{succ}}(z,Z,z',Z) & \text{if $d=\to$}.
%  \end{cases},
%  $$
%  $\mathsf{move}_d(I')$ will simply return the predecessor matrix when $d=\gets$ and the successor matrix when $d=\to$. This expression will be used to move the heads.
% \end{itemize}
% We thus see that we only need $z,z',z'',Z,Z',Z''$ as auxiliary variables and these can be re-used for every occurrence of the subexpressions in $e_T$. From now one, we omit the auxiliary variables from the description of $e_T$.
%
% We define the expression $e_T$, as follows\footnote{In the expression I uses $;w$ to indicate the output variable. That is, the for loops updates all instances for $X_1,\ldots,X_m,Y_1,\ldots,Y_\ell$ and $w$, but the result of the expression is only what is in the instance corresponding to $w$. We can simulate this again if we allow constant dimensional canonical basis vectors in $\mathsf{MATLANG}$.}:
% $$
% e_T:= \mathsf{for\,} v_1,\ldots,v_{k+1},X_1,\ldots,X_m,Y_1,\ldots,Y_\ell; w.(e_w,e_{X_1},\ldots,e_{X_m},e_{Y_1},\ldots,e_{Y_\ell}),
% $$
% with
% \begin{align*}\allowdisplaybreaks
% 	e_w&:=\mathsf{ismin}(X_m)\\
% 	e_{X_1}&:=\left(\prod_{j=1}^{k+1} \textsf{ismin}(v_i)\right)\cdot\mathsf{min}
% 	+ \sum_{\substack{q,b_1,\ldots,b_\ell\\
% 	\Delta_Q(q,b_1,\ldots,b_\ell)=q_1}} \!\!\!\!\!\!\!\!\! \textsf{ismin}(X_q)\left(\prod_{j=1}^\ell \mathsf{test}_{b_j}(Y_j)\right)\mathsf{min}\\
% 	e_{X_i}&:= \sum_{\substack{q,b_1,\ldots,b_\ell\\
% 	\Delta_Q(q,b_1,\ldots,b_\ell)=q_i}}\!\!\!\!\!\!\!\!\! \textsf{ismin}(X_q)\left(\prod_{j=1}^\ell \mathsf{test}_{b_j}(Y_j)\right)\mathsf{min} \quad \text{for $i\neq 1$}\\
% 	e_{Y_i}&:=\left(\prod_{j=1}^{k+1} \textsf{ismin}(v_i)\right)\cdot\mathsf{min}
% 	+\sum_{\substack{q,b_1,\ldots,b_\ell\\
% 	\Delta_i(q,b_1,\ldots,b_\ell)=d}}\!\!\!\!\!\!\!\!\! \textsf{ismin}(X_q)\left(\prod_{j=1}^\ell \mathsf{test}_{b_j}(Y_j)\right)\mathsf{move}_d\cdot Y_i
% \end{align*}
%
% The correctness of $e_T$ is now readily verified. We do this by induction on the number of iterations in the for loop. We note that initially, all variables are assigned zero vectors and values (for $w$).
%
% At the start of the run of $T$, we are in state $q_1$ and all heads point to the first position. We argue that after the first
% iterations, i.e., when $v_i=\mathbf{e}_1$ for $i\in[k+1]$, we indeed have that $X_1=\mathbf{e_1}$, $X_j=\mathbf{0}$ for $j\neq 1$, and $Y_j=\mathbf{e}_1$ for $j\in[\ell]$. Indeed, in the expression for $e_{X_1}$ the test $\prod_{j=1}^{k+1} \textsf{ismin}(v_i)$ will return $1$ and hence $X_1$ is replaced by $\mathsf{min}=\mathbf{e}_1$. Since all $X_i$ are initially zero, $\mathsf{ismin}(X_i)$ evaluate to zero for all $i\in[m]$ so the second term in $e_{X_1}$ adds the zero vector to $\mathbf{e}_1$ and thus $X_1$ remains $\mathbf{e_1}$.
% Similarly, $e_{X_j}$ for $j\neq 1$ will leave $X_j$ unchanged, so these remain zero vectors. For the head positions, a similar arguments shows that after the first iterations, all $Y_i$ are set of $\mathbf{e}_1$.
%
% We next assume that up to a certain iteration $\kappa-1$, the matrix variables correctly encode a configuration of $T$ on $w_n$ and furthermore, this configuration is reachable from the initial configuration. We next show that this remains to hold in the $\kappa$th iteration.
%
% By induction, there will be a single $X_i$ which is instantiated with $\mathbf{e}_1$. Let use assume that this $X_q$. All other $X_i$ are instantiated with the zero vector. Furthermore, $Y_j=\mathbf{e}_{i_j}$ for some $i_j\in[n+2]$.
%
% Suppose that
% $\Delta_Q(q,b_{i_1},\ldots,b_{i\ell})=p$ and
% $\Delta_j(q,b_{i_1},\ldots,b_{i\ell})=d_j$. Then, inspecting the expressions $e_{X_i}$, all $X_{q_i}$ with $q_i\neq p$ will be replaced by the zero vector. The reason is that for $X_i$ to be replaced by $\mathsf{min}$ (i.e., $\mathbf{e}_1$) when $T$ is in state $q$,
% there must be a transition $\Delta_Q(q,b_{i_1}',\ldots,b_{i_\ell}')=q_i$
% and such that the $b_{i_j}'$ corresponds to the positions encoded by
% the $Y_i$'s. In particular, $b_{i_1},\ldots,b_{i\ell}$ and
% $b_{i_1}',\ldots,b_{i_\ell}'$ must have $\rhd$ and $\lhd$ at the same positions, and since the only remaining symbol is $0$, $b_{i_1},\ldots,b_{i\ell}$ and $b_{i_1}',\ldots,b_{i\ell}'$ must agree.
% This in turn would imply that there are two possible states $p$ and $q_i$ from $q$ while reading $b_{i_1},\ldots,b_{i\ell}$. This is impossible since $T$ is deterministic.
%
% If we next consider the expressions $e_{Y_i}$ for $i\in[\ell]$, then a similar argument shows that at most one of the terms in the second part in $e_{Y_i}$ can replace $Y_i$ with $\mathsf{move}_d\cdot Y_i$. By defining of $\mathsf{move}_d$ in terms of the predecessor or successor matrix (depending on whether $d=\gets$ or $d=\to$, respectively), and given that $Y_i$ corresponds to a canonical vector, say $\mathbf{e}_{i_s}$, then $\mathsf{move}_d\cdot Y_i$ will replace $Y_i$
% with either $\mathbf{e}_{i_s-1}$ or  $\mathbf{e}_{i_s+1}$. We note that when $Y_i$ is $\mathbf{e}_1$ or $\mathbf{e}_{n+2}$, $d$ must necessarily be $\to$ or $\gets$, respectively, since $T$ does not move beyond the end markers.
%
% Hence, all combined we see that after the $\kappa$the iteration, $X_1,\ldots,X_m$ and $Y_1,\ldots,Y_\ell$ indeed correspond to the next configuration of $T$.
%
% We now remark that $w$ will be zero unless in one of the iterations populates $X_m$ with $\mathbf{e}_{1}$, i.e., the $T$ is in the final state. By assumption, $T$ will continue to be in the final state from that point on, and thus after perform our $(n+2)^k$, $w$ will remain $1$. If no final state is encountered, $w$ remains $0$, as desired.
% \end{proof}


We consider  deterministic Turing Machines  (TM) $T$ consisting of $\ell$ read-only input tapes, denoted by $R_1,\ldots,R_\ell$,
a work tape, denoted by $W$, and a write-only output tape, denoted by $O$. The TM $T$ has a set $Q$ of $m$
states, denoted by $q_0,\ldots,q_m$. We assume that $q_0$ is the initial state and $q_m$ is the accepting state.
The input and tape alphabet are $\Sigma=\{0,1\}$ and $\Gamma=\Sigma\cup\{\rhd,\lhd\}$, respectively. The special symbol $\rhd$ denotes the beginning of each of the tapes, the symbol $\lhd$ denotes the end of the $\ell$ input tapes. The transition function $\Delta$ is defined as usual, i.e., 
$\Delta:Q\times \Gamma^{\ell+2} \to Q\times \Gamma^{2}\times \{\leftarrow,\sqcup,\rightarrow\}^{\ell+2}$ such that $\Delta(q,(a_1,\ldots,a_{\ell},b,c))=\bigl(q',(b',c'),(\mathsf{d}_1,\ldots,\mathsf{d}_{\ell+2})\bigr)$ with $\mathsf{d}_i\in \{\leftarrow,\sqcup,\rightarrow\}$, means that when $T$ is in state $q$ and the $\ell+2$ heads on the tapes read symbols $a_1,\ldots,a_{\ell},b,c$, respectively, then $T$ transitions to state $q'$, writes $b',c'$ on the work and output tapes, respectively, at the position to which the work and output tapes' heads points at, and finally moves the heads on the tapes according $\mathsf{d}_1,\ldots,\mathsf{d}_{\ell+2}$. More specifically, $\leftarrow$  indicates a move to the left, 
$\rightarrow$ a move to the right, and finally, $\sqcup$ indicates that the head does not move.

We assume that $\Delta$ is defined such that it ensures that on none of the tapes, heads can move beyond the leftmost marker $\rhd$. Furthermore, the tapes $R_1,\ldots,R_\ell$ are treated as read-only and the heads on these tapes cannot move beyond the end markers $\lhd$. Similarly, $\Delta$ ensures that the output tape $O$ is write only, i.e., its head cannot move to the left.  We also assume that $\Delta$ does not change the occurrences of $\rhd$ or writes $\lhd$ on the work and output tape.

A configuration of $T$ is defined in the usual way. That is, a configuration of the input tapes is of the form
$\rhd w_1qw_2\lhd$ with $w_1,w_2\in\Sigma^*$ and represents that the current tape content is $\rhd w_1w_2\lhd$, $T$ is in state $q$ and the head is positioned on the first symbol of $w_2$. Similarly, configurations of the work and output tape are represented by $\rhd w_1qw_2$. A configuration of $T$ is consists of configurations for all tapes. Given two configurations $c_1$ and $c_2$, we say that $c_1$ yields $c_2$ if $c_2$ is the result of applying the transition function $\Delta$ of $T$ based on the information in $c_1$. As usual, we close this ``yields'' relation transitively.

Given $\ell$ input words $w_1,\ldots,w_\ell\in\Sigma^*$, we assume that the initial configuration of $T$ is given by
 $\bigl(q_0\rhd  w_1\lhd,q_0\rhd w_2\lhd,\ldots, q_0\rhd w_\ell\lhd,q_0\rhd, q_0\rhd \bigr)$ and an accepting configuration is assumed to be of the form $\bigl(\rhd q_m w_1\lhd,\rhd q_m w_2\lhd,\ldots, \rhd q_m w_\ell\lhd,\rhd q_m,\rhd q_m w\bigr)$ for some $w\in\Sigma^*$. We say that $T$ computes the function $f:(\Sigma^*)^{\ell}\to\Sigma^*$ if for every $w_1,\ldots,w_\ell\in\Sigma^*$, the initial configuration yields (transitively) an accepting configuration such that the configuration on the output tape is
 given by $\rhd q_m f(w_1,\ldots,w_\ell)$.

We assume that once $T$ reaches an accepting configuration it stays indefinitely in that configuration (i.e., it loops). We further assume that $T$ only reaches an accepting configuration when all its input
words have the same size. Furthermore, when all inputs have the same size, $T$ will reach an accepting configuration. 


We say that $T$ is a \textit{linear space machine} when it reaches an accepting configuration 
on inputs of size $n$ by using $\mathcal{O}(n)$ space on its work tape and additionally needs  $\mathcal{O}(n^k)$ steps to do so. A \textit{linear input-output function} is a function of the form $f=\bigcup_{n\geq 0} f_n:(\Sigma^n)^\ell\to\Sigma^n$. In other words, for every $\ell$ words of the same size $n$, $f$ returns a word of size $n$. We say that a linear input-output function is a \textit{linear space input-output function} if
there exists a linear space machine  $T$ that for every $n\geq 0$, on input $w_1,\ldots,w_\ell\in\Sigma^n$ the TM $T$ has
$f_n(w_1,\ldots,w_\ell)$ on its the output tape when (necessarily) reaching an accepting configuration.

% We say that a function
% $f:\underbrace{\Sigma^n\times\cdots \times \Sigma^n}_{\text{$\ell$ times}}\to \Sigma^n$ is computable by a linear space machine $T$ if when $T$ is run on input $w_1,\ldots,w_\ell$ it halts and has $f(w_1,\ldots,w_\ell)$ on its output tape. We say that $f$ is a \textit{linear space poly function} if it is computable by a linear space TM $T$ which in addition runs in polynomial time, i.e., it hals in at most
% $\mathcal{O}(n^k)$ steps for a certain $k$ on any inputs $w_1,\ldots,w_\ell$ of size $n$.
\begin{proposition}
Let $f=\bigcup_{n\geq 0}f_n:(\Sigma^n)^\ell\to \Sigma^n$ be a linear space input-ouput function computed by a linear space  machine $T$ with $m$ states, $\ell$ input tapes, which consumes $\mathcal{O}(n)$ space and runs in $\mathcal{O}(n^{k-1})$ time on inputs of size $n$. There exists (i)~a $\mathsf{MATLANG}$ 
schema $\mathcal{S}=(\mathcal{M},\textsf{size})$ where $\mathcal{M}$ consists matrix variables\footnote{We also need a finite number of auxiliary variables, these will be specified in the proof.} $Q_1,\ldots,Q_m,R_1,\ldots,R_\ell,H_1,\ldots,H_\ell,W_1,\ldots,W_s,H_{W_1},\ldots,H_{W_s},O,H_O, v_1,\ldots,v_{k}$  with
$\mathsf{size}(V)=\alpha\times 1$ for all $V\in\mathcal{M}$; and (ii)~a $\mathsf{MATLANG}$ expression $e_f$ over $\mathcal{S}$ such that for the instance $I=(\mathcal{D},\textsf{mat})$ over $\mathcal{S}$ with $\mathcal{D}(\alpha)=n$ and 
$$\mathsf{mat}(R_i)=\mathsf{vec}(w_i)\in \mathbb{R}^n\text{, for $i\in[\ell]$ and all other matrix variables instantiated with the zero vector in $\mathbb{R}^n$} $$
for words $w_1,\ldots,w_\ell\in\Sigma^n$ and such that $\mathsf{vec}(w_i)$ is the $n\times 1$-vector encoding the word $w_i$, we have that the $\mathsf{mat}(O)=\mathsf{vec}(f_n(w_1,\ldots,w_n))\in\mathbb{R}^n$ after evaluating $e_f(I)$.
\end{proposition}
\begin{proof}
	The expression $e_f$ we construct will simulate the TM $T$. To have some more control on the space and time consumption of $T$, let us first assume that $n$ is large enough, say larger than $n\geq N$, such that $T$ runs in $sn$ space and $cn^{k-1}\leq n^k$ time for constants $s$ and $c$. We deal with $n<N$ later on.

To simulate $T$ we need to encode states, tapes and head positions. The matrix variables in 
$\mathcal{M}$ mentioned in the proposition will take these roles. More specifically, the variables $R_1,\ldots,R_\ell$ will hold the input vectors, $W_1,\ldots,W_s$ will hold the contents of the work
tape, where $s$ is the constant mentioned earlier, and $O$ will hold the contents of the output tape. The vectors corresponding to the work and output tape are initially set to the zero vector. The vector for the input tape $R_i$ is set to $\mathsf{vec}(w_i)$, for $i\in[\ell]$.

 With each tape we associate a matrix variable encoding the position of the head. More specifically, $H_1,\ldots,H_\ell$ correspond to the input tape heads,
$H_{W_1},\ldots, H_{W_s}$ are the heads for the work tape, and $H_O$ is the head of the output tape. All these vectors are initialised with the zero vector. Later on, these vectors will be zero except for a single position, indicating the positions in the corresponding tapes the heads point to. For those positions $j$, $1<j<n$, the head vectors will carry value $1$.  When $j=1$ or $n$ and when it concerns positions for the input tape, the head vectors can carry value $1$ or $2$. We need to treat these border cases separately
because we only have $n$ positions available to store the input words, whereas the actual input tapes consist of $n+2$ symbols because of $\rhd$ and $\lhd$. So when, for example, $H_1$ has a $1$ in its first
entry, we interpret it as the head is pointing to the first symbol of the input word $w_1$. When $H_1$
has a $2$ in its first position, we interpret it as the head pointing to $\rhd$. The end marker $\lhd$ is
dealt with in the same way, by using value $1$ or $2$ in the last position of $H_1$. We use this encoding
for all input tapes, and also for the work tape $W_1$ and output tape $O$ with the exception that no end marker $\lhd$ is present.

 
% ,Y_1,\allowbreak\ldots,Y_\ell$ will be used inside for loops and will be updated using \textsf{MATLANG} expressions. Initially, all these matrix variables are instantiated with the zero column vector, as described by the instance $I$.

To encode the states, we use the variables $Q_1,\ldots,Q_m$. We will ensure that when $T$ is in state $q_i$ when
 $\mathsf{mat}(Q_i)=(1,0,\ldots,0)^t\in\mathbb{R}^n$, otherwise $\mathsf{mat}(Q_i)$ is the zero vector in $\mathbb{R}^n$.	

Finally, the variables $v_1,\ldots,v_{k}$ represent $k$ canonical vectors  which are use to iterate in for loops. By iterating over then, we can perform $n^{k}$ iterations, which suffices for simulating the $\mathcal{O}(n^{k-1})$ steps used by $T$ to reach an accepting configuration. 
% (We recall that $T$ loops when reaching an accepting configuration.)

With these matrix variables in place, we start by defining $e_f$. It will consists of two subexpressions
$e_f^{\geq N}$, for dealing with $n\geq N$, and $e_f^{<N}$, for dealing with $n<N$. We explain the expresion
$e_f^{\geq N}$ first.



In our  expressions we use subexpressions which we defined before. These subexpression require some auxiliary variables, as detailed below. As a consequence, $e_f$ will be an expressions defined over an extended schema $\mathcal{S}'$. Hence, the instance $I$ in the statement of the Proposition is  an instance $I'$ of $\mathcal{S}'$ which
coincides with $I$ on $\mathcal{S}$ and in which the auxiliary matrix variables are all instantiated with zero vectors or matrices, depending on their size.

\floris{The expressions below are used in other proofs as well, I just wanted to check how many auxiliary variables are needed. We can extract the list below and place it somewhere else.}
We next list the used subexpressions and explicitly denote the auxiliary matrix variables:
\begin{itemize}
	% \item $\mathsf{max}(z,Z)$, an expression over auxiliary variables $z$ and $Z$ with $\mathsf{size}(z)=\mathsf{size}(Z)=\alpha\times 1$. On input $I'$ with
	% $\mathsf{mat}(z)=\mathsf{mat}(Z)$ the zero column vector of dimension $n+2$,
	%  $\mathsf{max}(I)=\mathbf{e}_{n+2}$.
	\item $\mathsf{pred}(z,Z,z',Z')$, and expression over auxiliary variables $z$, $z'$, $Z$ and $Z'$ with $\mathsf{size}(z)=\mathsf{size}(z')=\mathsf{size}(Z)=\alpha\times 1$ and $\mathsf{size}(Z')=\alpha\times\alpha$. On input $I'$ with 
	$\mathsf{mat}(z)=\mathsf{mat}(z')=\mathsf{mat}(Z)$ the zero column vector of dimension $n$, and $\mathsf{mat}(Z')$ the zero $n\times n$ matrix,
	 $\mathsf{pred}(I')$ returns an $n\times n$ matrix such that 
	 
	 $$\mathsf{pred}(I')\mathbf{e}_i:=\begin{cases} 
	 \mathbf{e}_{i-1} & \text{if $i>1$}\\
	 \mathbf{0} & \text{if $i=1$}.
	\end{cases}
	$$
	In other words, $\mathsf{pred}$ defines a predecessor relation among canonical vectors of dimension $n$.
	 \item $\mathsf{succ}(z,Z,z',Z')$, and expression over auxiliary variables $z$, $z'$, $Z$ and $Z'$ with $\mathsf{size}(z)=\mathsf{size}(z')=\mathsf{size}(Z)=\alpha\times 1$ and $\mathsf{size}(Z')=\alpha\times\alpha$. On input $I'$ with 
	$\mathsf{mat}(z)=\mathsf{mat}(z')=\mathsf{mat}(Z)$ the zero column vector of dimension $n$, and $\mathsf{mat}(Z')$ the zero $n\times n$ matrix,
	 $\mathsf{succ}(I')$ returns an $n\times n$ matrix such that 
	 
	 $$\mathsf{succ}(I')\mathbf{e}_i:=\begin{cases} 
	 \mathbf{e}_{i+1} & \text{if $i<n$}\\
	 \mathbf{0} & \text{if $i=n$}.
	\end{cases}
	$$
	In other words, $\mathsf{succ}$ defines a successor relation among canonical vectors.
	\item $\textsf{ismin}(v,z,Z,z',Z)$ with auxiliary variables $z$, $z'$, $Z$ and $Z'$ as before, and $v$ is one of the (vector) variables in $\mathcal{M}$. For an $n\times 1$ vector $\mathbf{v}$, on input $I'[v\gets \mathbf{v}]$	$$\mathsf{ismin}(I'[v\gets\mathbf{v}]):=\begin{cases} 1 & \text{if $\mathbf{v}=\mathbf{e}_1$}\\
		0 & \text{otherwise}.
		\end{cases}$$
	\item $\textsf{ismax}(v,z,Z,z',Z)$ with auxiliary variables $z$, $z'$, $Z$ and $Z'$ as before, and 
	and $v$ is one of the (vector) variables in $\mathcal{M}$. For an $n\times 1$ vector $\mathbf{v}$, on input $I'[v\gets \mathbf{v}]$
	
	$$\mathsf{ismax}(I'[v\gets\mathbf{v}]):=\begin{cases} 1 & \text{if $\mathbf{v}=\mathbf{e}_{n}$}\\
		0 & \text{otherwise}.
		\end{cases}$$
	\item $\mathsf{min}(z,Z,z',Z',z'',Z'')$, an expressions with
	auxiliary variables $z$, $z'$, $z''$, $Z$, $Z'$ and $Z''$ with $\mathsf{size}(z)=\mathsf{size}(z')=\mathsf{size}(z'')=\mathsf{size}(Z)=\mathsf{size}(Z'')=\alpha\times 1$ and $\mathsf{size}(Z')=\alpha\times\alpha$. On input $I'$ with 
	matrix variables instantiated with zero vectors (or matrix for $Z'$),
 	 $\mathsf{min}(I')=\mathbf{e}_1$. 
	 		
\end{itemize}
We thus see that we only need $z,z',z'',Z,Z',Z''$ as auxiliary variables and these can be re-used whenever $e_f$ calls these functions. From now one, we omit the auxiliary variables from the description of $e_f$.


Let us first define $e_f^{\geq N}$. Since we want to simulate $T$ we need to be able to check which transitions of $T$ can be applied based on a current configuration. More precisely,
suppose that we want to check whether $\delta(q_i,(a_1,\ldots,a_{\ell},b,c))$ is applicable, then we need to check whether $T$ is in state $q_i$, we can do this by checking 
$\mathsf{ismin}(Q_i)$, and whether the heads on the tapes read symbols $a_1,\ldots,a_{\ell},b,c$. We check the latter by the following expressions.
For the input tapes $R_i$ we define
$$
\mathsf{test\_inp}^i_b:=\begin{cases}
(1-\mathsf{ismin}(1/2\cdot H_i))\cdot(1-\mathsf{ismax}(1/2\cdot H_i))\cdot(1- R_i^t\cdot H_i) & \text{if $b=0$}\\
(1-\mathsf{ismin}(1/2\cdot H_i))\cdot(1-\mathsf{ismax}(1/2\cdot H_i))\cdot(R_i^t\cdot H_i) & \text{if $b=1$}\\
\mathsf{ismin}(1/2\cdot H_i) & \text{if $b=\rhd$}\\
\mathsf{ismax}(1/2\cdot H_i) & \text{if $b=\lhd$},\\
\end{cases}
$$
which return $1$ if and only if either $b\in\{0,1\}$ is the value in $\mathsf{mat}(R_i)$ at the position encoded by $\mathsf{mat}(H_i)$, or when $b=\rhd$ and $\mathsf{mat}(H_i)$ is the vector $(2,0,\ldots,0)\in\mathbb{R}^n$, or when $b=\lhd$ and $\mathsf{mat}(H_i)$ is the vector $(0,0,\ldots,2)\in\mathbb{R}^n$. Similarly, for the output tape we define
$$
\mathsf{test\_out}_b:=\begin{cases}
(1-\mathsf{ismin}(1/2\cdot H_O))\cdot(1- O^t\cdot H_O) & \text{if $b=0$}\\
(1-\mathsf{ismin}(1/2\cdot H_O))\cdot(O^t\cdot H_O) & \text{if $b=1$}\\
\mathsf{ismin}(1/2\cdot H_O) & \text{if $b=\rhd$},\\
\end{cases}
$$
and for the work tapes $W_1,\ldots,W_s$ we define
$$
\mathsf{test\_work}^i_b:=\begin{cases}
(1-\mathsf{ismin}(1/2\cdot H_{W_i}))\cdot(1- W_i^t\cdot H_{W_i})) & \text{if $b=0$}\\
(1-\mathsf{ismin}(1/2\cdot H_{W_i}))\cdot (W_i^t\cdot H_{W_i}) & \text{if $b=1$}\\
\mathsf{ismin}(1/2\cdot H_{W_i}) & \text{if $b=\rhd$ and $i=1$}.\\
\end{cases}
$$
We then combine all these expressions into a single expression for $q_i\in Q$, $a_1,\ldots,a_\ell,b,c\in\Gamma$:
$$
\mathsf{isconf}_{q_i,a_1,\ldots,a_\ell,b,c}:=
\mathsf{ismin}(Q_i)\cdot \left(\prod_{j=1}^{\ell} \mathsf{test\_inp}_{a_j}^j\right)
\cdot\left(\sum_{j=1}^s \mathsf{test}_b^j\right)\cdot \mathsf{test\_outp}_{c}.
$$
This expression will return $1$ if and only if the vectors representing the tapes, head positions and states are such that $\mathsf{Q_i}$ is the first canonical vector (and thus $T$ is in state $q_i$), the heads point to entries in the tape vectors storing the symbols $a_1,\ldots,a_{\ell}, b,c$ or they point to the first (or last for input tapes) positions but have value $2$ (when the symbols are $\rhd$ or $\lhd$). 

To ensure that at the beginning of the simulation of $T$ by $e_f^{\geq N}$ we correctly encode that we are in the initial configuration, we thus need to initialise all vectors $\mathsf{mat}(H_1),\mathsf{mat}(H_2),\ldots, \mathsf{mat}(H_\ell), \mathsf{mat}(H_{W_1}),\mathsf{mat}(H_O)$ with the vector $(2,0,0,\ldots,0)\in\mathbb{R}$ since all heads read the symbol $\rhd$. Similarly, we have to initialise $\mathsf{Q_1}$ with the first canonical vector since $T$ is in state $q_0$.

We furthermore need to be able to correctly adjust head positions. We do this by means of the predecessor and successor expressions described above. 
A consequence of our encoding is that we need to treat the border cases (corresponding to $\rhd$ and $\lhd$) differently. More specifically, for the input tapes $R_i$ and heads $H_i$ we define 
$$
\mathsf{move\_inp}^i_{\mathsf{d}}:=
\begin{cases}
2\cdot \mathsf{ismin}(H_i)\cdot H_i + 1/2\cdot\mathsf{ismax}(1/2\cdot H_i)\cdot H_i  + (1-\mathsf{ismin}(H_i))(1-\mathsf{ismax}(1/2H_i))\cdot e_{\mathsf{pred}}\cdot H_i  
& \text{if $\mathsf{d}=\leftarrow$}\\
2\cdot \mathsf{ismax}(H_i)\cdot H_i + 1/2\cdot\mathsf{ismin}(1/2\cdot H_i)\cdot H_i  + (1-\mathsf{ismin}(1/2\cdot H_i))(1-\mathsf{ismax}(H_i))\cdot e_{\mathsf{succ}}\cdot H_i  
 & \text{if $\mathsf{d}=\rightarrow$}\\
H_i & \text{if $\mathsf{d}=\sqcup$}. 
\end{cases}
$$
In other words, we shift to the previous (or next) canonical vector when $\mathsf{d}$ is $\leftarrow$ or $\rightarrow$, respectively, unless we need to move to or from the position that will hold $\rhd$ or $\lhd$. In those case we readjust $\mathsf{mat}(H_i)$ (which will either $(1,0,\ldots,0)$, $(2,0,\ldots,0)$, $(0,\ldots,0,1)$ or $(0,\ldots,0,2)$) by either dividing or multiplying with $2$. In this way we can correctly infer whether or not the head points to the begin and end markers. For the output tape we proceed in a similar way, but only taking into account the begin marker and recall that we do not have moves to the left:
$$
\mathsf{move\_outp}_{\mathsf{d}}:=
\begin{cases}
1/2\cdot\mathsf{ismin}(1/2\cdot H_O)\cdot H_O  + (1-\mathsf{ismin}(1/2\cdot H_O))\cdot e_{\mathsf{succ}}\cdot H_O  
 & \text{if $\mathsf{d}=\rightarrow$}\\
H_O & \text{if $\mathsf{d}=\sqcup$}. 
\end{cases}
$$
Since we represent the work tape by $s$ vectors $W_1,\ldots,W_s$ we need to ensure that only one of the head vectors $H_{W_i}$ has a non-zero value and that by moving left or right, we need to appropriately update the right head vector. We do this as follows. We first consider the work tapes $W_i$ for $i\neq 1,s$ and define
$$
\mathsf{move\_work}^i_{\mathsf{d}}:=
\begin{cases}
	-\mathsf{ismin}(H_{W_i})\cdot H_{W_i} + (1-\mathsf{ismin}(H_{W_i}))\cdot e_{\mathsf{pred}}\cdot H_{W_i} + \mathsf{ismin}(H_{W_{i+1}})\cdot\mathsf{max} & \text{if $\mathsf{d}=\leftarrow$}\\
		-\mathsf{ismax}(H_{W_i})\cdot H_{W_i} + (1-\mathsf{ismax}(H_{W_i}))\cdot e_{\mathsf{succ}}\cdot H_{W_i} + \mathsf{ismax}(H_{W_{i-1}})\cdot\mathsf{min} & \text{if $\mathsf{d}=\rightarrow$}\\
	H_{W_i} & \text{if $\mathsf{d}=\sqcup$}. 	
\end{cases}
$$
In other words, we set the $H_{W_i}$ to zero when a move brings us to either $W_{i-1}$ or $W_{i+1}$, we
move the successor or predecessor when staying within $W_i$, or initialise $H_{W_i}$ with the first or last canonical vector when moving from $W_{i-1}$ to $W_i$ (right move) or from $W_{i+1}$ to $W_i$ (left move).
For $i=s$ we can ignore the parts in the previous expression that involve $W_{s+1}$ (which does not exist):
$$
\mathsf{move\_work}^s_{\mathsf{d}}:=
\begin{cases}
	-\mathsf{ismin}(H_{W_s})\cdot H_{W_i} + (1-\mathsf{ismin}(H_{W_s}))\cdot e_{\mathsf{pred}}\cdot H_{W_s}  & \text{if $\mathsf{d}=\leftarrow$}\\
		-\mathsf{ismax}(H_{W_s}) \cdot H_{W_s} + (1-\mathsf{ismax}(H_{W_s}))\cdot e_{\mathsf{succ}}\cdot H_{W_s} + \mathsf{ismax}(H_{W_{s-1}})\cdot \mathsf{min} & \text{if $\mathsf{d}=\rightarrow$}\\
	H_{W_s} & \text{if $\mathsf{d}=\sqcup$}. 	
\end{cases}
$$
For $i=1$, we can ignore the part involving $W_{0}$ (which does not exist) but have to take $\rhd$ into account:
$$
\mathsf{move\_work}^1_{\mathsf{d}}:=
\begin{cases}
	2\cdot \mathsf{ismin}(H_{W_1})\cdot H_{W_i} + (1-\mathsf{ismin}(H_{W_1}))\cdot e_{\mathsf{pred}}\cdot H_{W_1} + \mathsf{ismin}(H_{W_{2}})\cdot\mathsf{max} & \text{if $\mathsf{d}=\leftarrow$}\\
		1/2\cdot\mathsf{ismin}(1/2\cdot H_{W_1})\cdot H_{W_1} + (1-\mathsf{ismax}(1/2\cdot H_{W_1}))\cdot e_{\mathsf{succ}}\cdot H_{W_1}  & \text{if $\mathsf{d}=\rightarrow$}\\
	H_{W_1} & \text{if $\mathsf{d}=\sqcup$}. 	
\end{cases}
$$
A final ingredient for defining $e_f^{\geq N}$ are expressions which update the work and output tape.
To define these expression, we need the position and symbol to put on the tape. For the output tape we define
$$
\mathsf{write\_outp}_b:=\begin{cases}
\mathsf{ismin}(1/2\cdot H_O)\cdot O & \text{if $b=\rhd$}\\
(1-\mathsf{ismin}(1/2\cdot H_O))\cdot\left((1-O^t\cdot H_O)\cdot O + (O^t\cdot H_O)\cdot (O-H_O)\right) &\text{if $b=0$}\\
(1-\mathsf{ismin}(1/2\cdot H_O))\cdot\left((1-O^t\cdot H_O)\cdot (O+H_O) + (O^t\cdot H_O)\cdot O\right) &\text{if $b=1$}\\
\end{cases}
$$
and similarly for the work tapes $i\neq 1$:
$$
\mathsf{write\_work}_b^i:=\begin{cases}
W_i & \text{if $b=\rhd$}\\
(1-W_i^t\cdot H_{W_i})\cdot W_i + (W_i^t\cdot H_{W_i})\cdot (W_i-H_{W_i}) &\text{if $b=0$}\\
(1-W_i^t\cdot H_{W_i})\cdot (W_i+H_{W_i}) + (W_i^t\cdot H_{W_i})\cdot W_i &\text{if $b=1$},
\end{cases}
$$
and for  $W_1$ we have to take care again of the begin marker:
$$
\mathsf{write\_work}_b^1:=\begin{cases}
\mathsf{ismin}(1/2\cdot H_{W_1})\cdot W_1 & \text{if $b=\rhd$}\\
(1-\mathsf{ismin}(1/2\cdot H_{W_1})\cdot\left((1-W_1^t\cdot H_{W_1})\cdot W_1 + (W_1^t\cdot H_{W_1})\cdot (W_1-H_{W_1})\right) &\text{if $b=0$}\\
(1-\mathsf{ismin}(1/2\cdot H_{W_1})\cdot\left((1-W_1^t\cdot H_{W_1})\cdot (W_1+H_{W_1}) + (W_1^t\cdot H_{W_1})\cdot W_1\right) &\text{if $b=1$}.
\end{cases}
$$




% We additionally define, based on the previous expressions,
% 			$$\mathsf{test}_b(v,z,Z,z',Z'):=\begin{cases} \mathsf{ismin}(v,z,Z,z,Z') & \text{if $b=\rhd$}\\
%      \mathsf{ismax}(v,z,Z,z,Z') & \text{if $b=\lhd$}\\
% 	 (1-\mathsf{ismin}(v,z,Z,z',Z'))(1-\mathsf{ismax}(v,z,Z,z',Z')) & \text{if $b=0$}.
% 		\end{cases}.$$
% 	When evaluated on $I'[v\gets\mathbf{v}]$, $\mathsf{test}_b(I'[v\gets\mathbf{v}])$ will be $1$ when either
% 	$\mathbf{v}=\mathbf{e}_{1}$ and $b=\rhd$ (first position)
% 	$\mathbf{v}=\mathbf{e}_{n+2}$ and $b=\lhd$ (last position),
% 	$\mathbf{v}\neq \mathbf{e}_{1}$ and $\mathbf{v}\neq \mathbf{e}_{n+2}$  and $b=0$ (not first or last position). We use this expression below to check whether the heads are consistent with the symbols on the tape.
%
%
% Finally, we define
%  $$
%  \mathsf{move}_d(z,Z,z',Z'):=\begin{cases}
%  e_{\mathsf{pred}}(z,Z,z',Z) & \text{if $d=\gets$}\\
%   e_{\mathsf{succ}}(z,Z,z',Z) & \text{if $d=\to$}.
%  \end{cases},
%  $$
%  $\mathsf{move}_d(I')$ will simply return the predecessor matrix when $d=\gets$ and the successor matrix when $d=\to$. This expression will be used to move the heads.
% We define the expression $e_T$, as follows\footnote{In the expression I uses $;w$ to indicate the output variable. That is, the for loops updates all instances for $X_1,\ldots,X_m,Y_1,\ldots,Y_\ell$ and $w$, but the result of the expression is only what is in the instance corresponding to $w$. We can simulate this again if we allow constant dimensional canonical basis vectors in $\mathsf{MATLANG}$.}:

We are now finally ready to define $e_f^{\geq N}$:
\begin{multline*}
e_f^{\geq N}:= \mathsf{for\,} v_1,\ldots,v_{k},Q_1,\ldots,Q_m,H_1,\ldots,H_\ell,W_1,\ldots,W_s, H_{W_1},\ldots,H_{W_s},O,H_O . \\
(e_{Q_1},\ldots,e_{Q_m},e_{H_1},\ldots,e_{H_\ell},e_{W_1},\ldots,e_{W_s},e_{H_{W_1}},\ldots,e_{H_{W_s}},e_{O}, e_{H_O})
\end{multline*}
with expressions (we use $\star$ below to mark irrelevant information in the transitions):
 \allowdisplaybreaks
\begin{align*}
	e_{Q_1}&:=\left(\prod_{j=1}^{k} \textsf{ismin}(v_i)\right)\cdot\mathsf{min}
	+ \sum_{\substack{(q_i,a_1,\ldots,a_\ell,b,c)\\
	\Delta(q_i,a_1,\ldots,a_\ell,b,c)=(q_1,\star)}} \!\!\!\!\!\!\!\!\! \mathsf{isconf}_{q_i,a_1,\ldots,a_\ell,b,c}\cdot \mathsf{min} \\
	e_{Q_j}&:=\sum_{\substack{(q_i,a_1,\ldots,a_\ell,b,c)\\
	\Delta(q_i,a_1,\ldots,a_\ell,b,c)=(q_j,\star)}} \!\!\!\!\!\!\!\!\! \mathsf{isconf}_{q_i,a_1,\ldots,a_\ell,b,c}\cdot \mathsf{min}
	 \quad \text{for $j\neq 1$}\\
	e_{H_i}&:=2\left(\prod_{j=1}^{k} \textsf{ismin}(v_i)\right)\cdot\mathsf{min}
	+\sum_{\substack{(q,a_1,\ldots,a_\ell,b,d)\\
	\Delta(q,a_1,\ldots,a_\ell,b,c)=(\star,\mathsf{d_i},\star)}}\!\!\!\!\!\!\!\!\! \mathsf{isconf}_{q,a_1,\ldots,a_\ell,b,c}\cdot\mathsf{move\_inp}^i_{\mathsf{d}_i}\\
	e_{H_{W_i}}&:=2\left(\prod_{j=1}^{k} \textsf{ismin}(v_i)\right)\cdot\mathsf{min}
	+\sum_{\substack{(q,a_1,\ldots,a_\ell,b,d)\\
	\Delta(q,a_1,\ldots,a_\ell,b,c)=(\star,\mathsf{d_{\ell+1}},\star)}}\!\!\!\!\!\!\!\!\! \mathsf{isconf}_{q,a_1,\ldots,a_\ell,b,c}\cdot\mathsf{move\_work}_{\mathsf{d}_{\ell+1}}^i\\
	e_{H_O}&:=2\left(\prod_{j=1}^{k} \textsf{ismin}(v_i)\right)\cdot\mathsf{min}
	+\sum_{\substack{(q,a_1,\ldots,a_\ell,b,d)\\
	\Delta(q,a_1,\ldots,a_\ell,b,c)=(\star,\mathsf{d}_{\ell+2})}}\!\!\!\!\!\!\!\!\! \mathsf{isconf}_{q,a_1,\ldots,a_\ell,b,c}\cdot\mathsf{move\_outp}_{\mathsf{d}_{\ell+2}}\\
	e_{W_i}&:=\sum_{\substack{(q,a_1,\ldots,a_\ell,b,d)\\
	\Delta(q,a_1,\ldots,a_\ell,b,c)=(\star,b',c',\star)}}\!\!\!\!\!\!\!\!\! \mathsf{isconf}_{q,a_1,\ldots,a_\ell,b,c}\cdot\mathsf{write\_work}_{b'}^i\\
	e_{O}&:=\sum_{\substack{(q,a_1,\ldots,a_\ell,b,d)\\
	\Delta(q,a_1,\ldots,a_\ell,b,c)=(\star,b',c',\star)}}\!\!\!\!\!\!\!\!\! \mathsf{isconf}_{q,a_1,\ldots,a_\ell,b,c}\cdot\mathsf{write\_outp}_{c'}.
\end{align*}

The correctness of $e_f^{\geq N}$ should be clear from the construction (one can formally verify this by
induction on the number of iterations). We next explain how the border cases $n<N$ can be dealt with.
For each $n<N$ and every possible input words
$w_1,\ldots,w_\ell$ of size $n$, we define a MATLANG expression which check whether
$\mathsf{mat}(R_i)=\mathsf{vec}(w_i)$ for each $i\in[\ell]$. This can be easily done since $n$ can be regarded as a constant. For example, to check whether $\mathsf{mat}(R_i)=(0,1,1)^t$ we simply write
$$
(1- R_i^t\cdot \mathsf{min})\cdot (R_i^t\cdot e_{\mathsf{succ}}\cdot\min)\cdot (1- R_i^t\cdot e_{\mathsf{succ}}\cdot e_{\mathsf{succ}}\cdot \min)\cdot (1- \mathbf{1}(R_i)^t\cdot e_{\mathsf{succ}}\cdot e_{\mathsf{succ}}\cdot e_{\mathsf{succ}}\cdot \min)
$$
which will evaluate to $1$ if and only if $\mathsf{mat}(R_i)=(0,1,1)^t$. We note that the factor is in place to check that the dimension $\mathsf{mat}(R_i)$ is three.
  We denote by
$e_{n,w}^i$ the expression which evaluates to $1$ if and only if $\mathsf{mat}(R_i)=\mathsf{vec}(w)$
for $|w|=n$.
We can similarly
write any word $w$ of fixed size in the matrix variable $O$. For example, suppose that $w=101$
then we write 
$$
O+ \mathsf{min}+  e_{\mathsf{succ}}\cdot e_{\mathsf{succ}}\cdot\mathsf{min}.
$$
We write $e_{n,w}$ be the expression which write $w$ of size $|w|=n$ in matrix variable $O$.
Then, the expressions
$$
e_{n,w_1,\ldots,w_n,w}:=e_{n,w_1}^1\cdot\cdots\cdot e_{nw_{\ell}}^\ell\cdot e_{n,w}
$$
will write $w$ in $O$ if and only if $\mathsf{mat}(R_i)=\mathsf{vec}(w_i)$ for $i\in[\ell]$.
We now simply take the disjunction over all words $w_1,\ldots,w_\ell\in\Sigma^n$ and $w=f_n(w_1,\ldots,w_\ell)\in\Sigma^n$:
$$
e_n:=\sum_{w_1,\ldots,w_\ell\in\Sigma^n} e_{n,w_1,\ldots,w_\ell,f_n(w_1,\ldots,w_\ell)},
$$
which correctly evaluates $f_n$. We next take a further disjunction by letting ranging from 
$n=0,\ldots, N-1$:
$$
e_f^{<N}:=\sum_{n=0}^{N-1} e_n
$$
Since every possible input is covered and only a unique expression $ e_{n,w_1,\ldots,w_\ell,f_n(w_1,\ldots,w_\ell)}$ will be triggered $e_f^{<N}$ will correctly
evaluate $f$ on inputs smaller than $N$.

Our final expression $e_f$ is now given by
$$
e_f:=e_f^{<N} + \mathsf{dim\_is\_greater\_than_N}\cdot e_f^{\geq 0}
$$
where $\mathsf{dim\_is\_greater\_than_N}$ is the expression
$\mathbf{1}(R_i)^t\cdot\underbrace{e_{\mathsf{succ}}\cdot\cdots\cdot e_{\mathsf{succ}}}_{\text{$N$ times}}$ which will evaluate to $1$ if an only if the input dimension is larger or equal than $N$.
\end{proof}
