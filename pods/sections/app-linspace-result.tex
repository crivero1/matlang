We start by showing a crucial ingredient for making the correspondence between \langfor
and arithmetic circuits. More specifically, 
we show that any polynomial time Turing machine, working within linear space and producing linear space output, can be simulated in \langfor. 
For this proof and section only, we will denote the canonical vectors as
$\mathbf{e}_1, \ldots, \mathbf{e}_n$, since $b$ will be used to represent a value on a position of a tape.

We consider  deterministic Turing Machines (TM) $T$ consisting of $\ell$ read-only input tapes, 
denoted by $R_1,\ldots,R_\ell$,
a work tape, denoted by $W$, and a write-only output tape, denoted by $O$. The TM $T$ has a set $Q$ of $m$
states, denoted by $q_0,\ldots,q_m$. We assume that $q_0$ is the initial state and $q_m$ is the accepting state.
The input and tape alphabet are $\Sigma=\{0,1\}$ and $\Gamma=\Sigma\cup\{\rhd,\lhd\}$, respectively. The special 
symbol $\rhd$ denotes the beginning of each of the tapes, the symbol $\lhd$ denotes the end of the $\ell$ input tapes. 
The transition function $\Delta$ is defined as usual, i.e., 
$\Delta:Q\times \Gamma^{\ell+2} \to Q\times \Gamma^{2}\times \{\leftarrow,\sqcup,\rightarrow\}^{\ell+2}$ 
such that $\Delta(q,(a_1,\ldots,a_{\ell},b,c))=\bigl(q',(b',c'),(\mathsf{d}_1,\ldots,\mathsf{d}_{\ell+2})\bigr)$
with $\mathsf{d}_i\in \{\leftarrow,\sqcup,\rightarrow\}$, means that when $T$ is in state $q$ and the $\ell+2$ 
heads on the tapes read symbols $a_1,\ldots,a_{\ell},b,c$, respectively, then $T$ transitions to state $q'$,
writes $b',c'$ on the work and output tapes, respectively, at the position to which the work and output 
tapes' heads points at, and finally moves the heads on the tapes according 
$\mathsf{d}_1,\ldots,\mathsf{d}_{\ell+2}$. More specifically, $\leftarrow$  indicates a move to the left, 
$\rightarrow$ a move to the right, and finally, $\sqcup$ indicates that the head does not move.

We assume that $\Delta$ is defined such that it ensures that on none of the tapes, heads can move beyond 
the leftmost marker $\rhd$. Furthermore, the tapes $R_1,\ldots,R_\ell$ are treated as read-only and the heads 
on these tapes cannot move beyond the end markers $\lhd$. Similarly, $\Delta$ ensures that the output tape $O$ 
is write only, i.e., its head cannot move to the left.  We also assume that $\Delta$ does not change the 
occurrences of $\rhd$ or writes $\lhd$ on the work and output tape.

A configuration of $T$ is defined in the usual way. That is, a configuration of the input tapes is of the form
$\rhd w_1qw_2\lhd$ with $w_1,w_2\in\Sigma^*$ and represents that the current tape content is 
$\rhd w_1w_2\lhd$, $T$ is in state $q$ and the head is positioned on the first symbol of $w_2$. 
Similarly, configurations of the work and output tape are represented by $\rhd w_1qw_2$. 
A configuration of $T$ is consists of configurations for all tapes. Given two configurations 
$c_1$ and $c_2$, we say that $c_1$ yields $c_2$ if $c_2$ is the result of applying the transition 
function $\Delta$ of $T$ based on the information in $c_1$. As usual, we close this ``yields'' relation 
transitively.

Given $\ell$ input words $w_1,\ldots,w_\ell\in\Sigma^*$, we assume that the initial configuration of 
$T$ is given by
$\bigl(q_0\rhd  w_1\lhd,q_0\rhd w_2\lhd,\ldots, q_0\rhd w_\ell\lhd,q_0\rhd, q_0\rhd \bigr)$ and an 
accepting configuration is assumed to be of the form 
$\bigl(\rhd q_m w_1\lhd,\rhd q_m w_2\lhd,\ldots, \rhd q_m w_\ell\lhd,\rhd q_m,\rhd q_m w\bigr)$ for some
$w\in\Sigma^*$. We say that $T$ computes the function $f:(\Sigma^*)^{\ell}\to\Sigma^*$ if for every
$w_1,\ldots,w_\ell\in\Sigma^*$, the initial configuration yields (transitively) an accepting 
configuration such that the configuration on the output tape is
given by $\rhd q_m f(w_1,\ldots,w_\ell)$.

We assume that once $T$ reaches an accepting configuration it stays indefinitely in that configuration 
(i.e., it loops). We further assume that $T$ only reaches an accepting configuration when all its input
words have the same size. Furthermore, when all inputs have the same size, $T$ will reach an accepting 
configuration. 


We say that $T$ is a \textit{linear space machine} when it reaches an accepting configuration 
on inputs of size $n$ by using $\mathcal{O}(n)$ space on its work tape and additionally needs 
$\mathcal{O}(n^k)$ steps to do so. A \textit{linear input-output function} is a function of the form 
$f=\bigcup_{n\geq 0} f_n:(\Sigma^n)^\ell\to\Sigma^n$. In other words, for every $\ell$ words of the same 
size $n$, $f$ returns a word of size $n$. We say that a linear input-output function is a 
\textit{linear space input-output function} if
there exists a linear space machine  $T$ that for every $n\geq 0$, on input $w_1,\ldots,w_\ell\in\Sigma^n$ 
the TM $T$ has
$f_n(w_1,\ldots,w_\ell)$ on its the output tape when (necessarily) reaching an accepting configuration.

% We say that a function
% $f:\underbrace{\Sigma^n\times\cdots \times \Sigma^n}_{\text{$\ell$ times}}\to \Sigma^n$ is computable by a linear space machine $T$ if when $T$ is run on input $w_1,\ldots,w_\ell$ it halts and has $f(w_1,\ldots,w_\ell)$ on its output tape. We say that $f$ is a \textit{linear space poly function} if it is computable by a linear space TM $T$ which in addition runs in polynomial time, i.e., it hals in at most
% $\mathcal{O}(n^k)$ steps for a certain $k$ on any inputs $w_1,\ldots,w_\ell$ of size $n$.
\begin{proposition} \label{prop:transducer}
Let $f=\bigcup_{n\geq 0}f_n:(\Sigma^n)^\ell\to \Sigma^n$ be a linear space input-ouput function 
computed by a linear space  machine $T$ with $m$ states, $\ell$ input tapes, which consumes 
$\mathcal{O}(n)$ space and runs in $\mathcal{O}(n^{k-1})$ time on inputs of size $n$. 
There exists (i)~a $\mathsf{MATLANG}$ 
schema $\mathcal{S}=(\mathcal{M},\textsf{size})$ where $\mathcal{M}$ consists matrix 
variables\footnote{We also need a finite number of auxiliary variables, these will be specified 
in the proof.} 
$Q_1,\ldots,Q_m,R_1,\ldots,R_\ell,H_1,\ldots,H_\ell,W_1,\ldots,W_s,H_{W_1},\ldots,H_{W_s},O,H_O, v_1,\ldots,v_{k}$ 
with $\mathsf{size}(V)=\alpha\times 1$ for all $V\in\mathcal{M}$; and (ii)~a $\mathsf{MATLANG}$ 
expression $e_f$ over $\mathcal{S}$ such that for the instance 
$\I=(\mathcal{D},\textsf{mat})$ over $\mathcal{S}$ with $\mathcal{D}(\alpha)=n$ and 
$$\mathsf{mat}(R_i)=\mathsf{vec}(w_i)\in \mathbb{R}^n\text{, for $i\in[\ell]$ and all other matrix variables instantiated with the zero vector in $\mathbb{R}^n$} $$
for words $w_1,\ldots,w_\ell\in\Sigma^n$ and such that $\mathsf{vec}(w_i)$ is the $n\times 1$-vector 
encoding the word $w_i$, we have that  $\mathsf{mat}(O)=\mathsf{vec}(f_n(w_1,\ldots,w_n))\in\mathbb{R}^n$ 
after evaluating $e_f$ on $\I$.
\end{proposition}
\begin{proof}
	The expression $e_f$ we construct will simulate the TM $T$. To have some more control on the space 
	and time consumption of $T$, let us first assume that $n$ is large enough, say larger than $n\geq N$, 
	such that $T$ runs in $sn$ space and $cn^{k-1}\leq n^k$ time for constants $s$ and $c$. We deal with $n<N$ later on.

\floris{To connect with my earlier comment in B1. I guess we assume that $n$ is also large enough to update $m+\ell+2s+2$independent vectors??}
\thomas{I updated the definition of the operator so there should be no problem now.}
To simulate $T$ we need to encode states, tapes and head positions. The matrix variables in 
$\mathcal{M}$ mentioned in the proposition will take these roles. More specifically, the variables 
$R_1,\ldots,R_\ell$ will hold the input vectors, $W_1,\ldots,W_s$ will hold the contents of the work
tape, where $s$ is the constant mentioned earlier, and $O$ will hold the contents of the output tape. 
The vectors corresponding to the work and output tape are initially set to the zero vector. 
The vector for the input tape $R_i$ is set to $\mathsf{vec}(w_i)$, for $i\in[\ell]$.

 With each tape we associate a matrix variable encoding the position of the head. More specifically, 
 $H_1,\ldots,H_\ell$ correspond to the input tape heads,
$H_{W_1},\ldots, H_{W_s}$ are the heads for the work tape, and $H_O$ is the head of the output tape. 
All these vectors are initialised with the zero vector. Later on, these vectors will be zero except 
for a single position, indicating the positions in the corresponding tapes the heads point to. 
For those positions $j$, $1<j<n$, the head vectors will carry value $1$.  When $j=1$ or $n$ and when 
it concerns positions for the input tape, the head vectors can carry value $1$ or $2$. We need to treat 
these border cases separately
because we only have $n$ positions available to store the input words, whereas the actual input tapes 
consist of $n+2$ symbols because of $\rhd$ and $\lhd$. So when, for example, $H_1$ has a $1$ in its first
entry, we interpret it as the head is pointing to the first symbol of the input word $w_1$. When $H_1$
has a $2$ in its first position, we interpret it as the head pointing to $\rhd$. The end marker $\lhd$ is
dealt with in the same way, by using value $1$ or $2$ in the last position of $H_1$. We use this encoding
for all input tapes, and also for the work tape $W_1$ and output tape $O$ with the exception that no end 
marker $\lhd$ is present.

 
% ,Y_1,\allowbreak\ldots,Y_\ell$ will be used inside for loops and will be updated using \textsf{MATLANG} expressions. Initially, all these matrix variables are instantiated with the zero column vector, as described by the instance $I$.

To encode the states, we use the variables $Q_1,\ldots,Q_m$. We will ensure that when $T$ is in state 
$q_i$ when
 $\mathsf{mat}(Q_i)=[1,0,\ldots,0]^T\in\mathbb{R}^n$, otherwise $\mathsf{mat}(Q_i)$ is the zero 
 vector in $\mathbb{R}^n$.	

Finally, the variables $v_1,\ldots,v_{k}$ represent $k$ canonical vectors  which are used to iterate 
in for-loops. By iterating over then, we can perform $n^{k}$ iterations, 
which suffices for simulating the $\mathcal{O}(n^{k-1})$ steps used by $T$ to reach an accepting configuration. 
% (We recall that $T$ loops when reaching an accepting configuration.)

With these matrix variables in place, we start by defining $e_f$. It will consists of two subexpressions
$e_f^{\geq N}$, for dealing with $n\geq N$, and $e_f^{<N}$, for dealing with $n<N$. We explain the expression
$e_f^{\geq N}$ first.



In our  expressions we use subexpressions which we defined before in section \ref{app:order}. These subexpression 
require some auxiliary variables, as detailed below. As a consequence, $e_f$ will be an expressions 
defined over an extended schema $\mathcal{S}'$. Hence, the instance $\I$ in the statement of the Proposition 
is  an instance $\I'$ of $\mathcal{S}'$ which
coincides with $\I$ on $\mathcal{S}$ and in which the auxiliary matrix variables are all instantiated with 
zero vectors or matrices, depending on their size.

Now, we specify the finite auxiliary variables involved in the \langfor expression. These arise
when computing the following \langfor expressions defined 

% \floris{The expressions below are used in other proofs as well, I just wanted to check how many auxiliary 
% variables are needed. We can extract the list below and place it somewhere else.}
% We next list the used subexpressions and explicitly denote the auxiliary matrix variables:
\begin{itemize}
	% \item $ e_{\mathsf{max}}(z,Z)$, an expression over auxiliary variables $z$ and $Z$ with $\mathsf{size}(z)=\mathsf{size}(Z)=\alpha\times 1$. On input $I'$ with
	% $\mathsf{mat}(z)=\mathsf{mat}(Z)$ the zero column vector of dimension $n+2$,
	%  $ e_{\mathsf{max}}(I)=\mathbf{e}_{n+2}$.
	\item $e_{\mathsf{Prev}}(z,Z,z',Z')$, and expression over auxiliary variables $z$, $z'$, $Z$ and $Z'$ with 
	$\mathsf{size}(z)=\mathsf{size}(z')=\mathsf{size}(Z)=\alpha\times 1$ and 
	$\mathsf{size}(Z')=\alpha\times\alpha$. On input $\I'$ with 
	$\mathsf{mat}(z)=\mathsf{mat}(z')=\mathsf{mat}(Z)$ the zero column vector of dimension $n$, 
	and $\mathsf{mat}(Z')$ the zero $n\times n$ matrix,
	$\sem{e_{\mathsf{Prev}}}{\I'}$ returns the $n\times n$ matrix $\mathsf{Prev}$ such that 
	$$\mathsf{Prev}\cdot \mathbf{e}_i:=\begin{cases} 
	\mathbf{e}_{i-1} & \text{if $i>1$}\\
	\mathbf{0} & \text{if $i=1$}.
	\end{cases}
	$$
	% In other words, $e_{\mathsf{Prev}}$ defines a predecessor relation among canonical vectors of dimension $n$.
	\item $e_{\mathsf{Next}}(z,Z,z',Z')$, and expression over auxiliary variables $z$, $z'$, $Z$ and $Z'$ 
	with $\mathsf{size}(z)=\mathsf{size}(z')=\mathsf{size}(Z)=\alpha\times 1$ and 
	$\mathsf{size}(Z')=\alpha\times\alpha$. On input $\I'$ with 
	$\mathsf{mat}(z)=\mathsf{mat}(z')=\mathsf{mat}(Z)$ the zero column 
	vector of dimension $n$, and $\mathsf{mat}(Z')$ the zero $n\times n$ matrix,
	$\sem{e_{\mathsf{Next}}}{\I'}$ returns the $n\times n$ matrix $\mathsf{Next}$ such that 
	$$\mathsf{Next}\cdot \mathbf{e}_i:=\begin{cases} 
	\mathbf{e}_{i+1} & \text{if $i<n$}\\
	\mathbf{0} & \text{if $i=n$}.
	\end{cases}
	$$
	\item $\textsf{min}(v,z,Z,z',Z)$ with auxiliary variables $z$, $z'$, $Z$ and $Z'$ as before, 
	and $v$ is one of the (vector) variables in $\mathcal{M}$. For an $n\times 1$ vector $\mathbf{v}$, 
	on input $\I'[v\gets \mathbf{v}]$	$$\sem{\mathsf{min}}{\I'[v\gets\mathbf{v}]}:=\begin{cases} 1 & \text{if $\mathbf{v}=\mathbf{e}_1$}\\
		0 & \text{otherwise}.
		\end{cases}$$

	\item $\textsf{max}(v,z,Z,z',Z)$ with auxiliary variables $z$, $z'$, $Z$ and $Z'$ as before, and 
	and $v$ is one of the (vector) variables in $\mathcal{M}$. For an $n\times 1$ vector $\mathbf{v}$, 
	on input $\I'[v\gets \mathbf{v}]$
	
	$$\sem{\mathsf{max}}{\I'[v\gets\mathbf{v}]}:=\begin{cases} 1 & \text{if $\mathbf{v}=\mathbf{e}_{n}$}\\
		0 & \text{otherwise}.
		\end{cases}$$
	\item $e_{\mathsf{min}}(z,Z,z',Z',z'',Z'')$, an expressions with
	auxiliary variables $z$, $z'$, $z''$, $Z$, $Z'$ and $Z''$ with 
	$\mathsf{size}(z)=\mathsf{size}(z')=\mathsf{size}(z'')=\mathsf{size}(Z)=\mathsf{size}(Z'')=\alpha\times 1$ 
	and $\mathsf{size}(Z')=\alpha\times\alpha$. On input $\I'$ with 
	matrix variables instantiated with zero vectors (or matrix for $Z'$),
 	 $\sem{e_{\mathsf{min}}}{\I'}=\mathbf{e}_1$. 
	\item $e_{\mathsf{max}}(z,Z,z',Z',z'',Z'')$, an expressions with
	auxiliary variables $z$, $z'$, $z''$, $Z$, $Z'$ and $Z''$ with 
	$\mathsf{size}(z)=\mathsf{size}(z')=\mathsf{size}(z'')=\mathsf{size}(Z)=\mathsf{size}(Z'')=\alpha\times 1$ 
	and $\mathsf{size}(Z')=\alpha\times\alpha$. On input $\I'$ with 
	matrix variables instantiated with zero vectors (or matrix for $Z'$),
 	 $\sem{e_{\mathsf{max}}}{\I'}=\mathbf{e}_n$. 	 		
\end{itemize}
We thus see that we only need $z,z',z'',Z,Z',Z''$ as auxiliary variables and these can be re-used 
whenever $e_f$ calls these functions. From now one, we omit the auxiliary variables from the description 
of $e_f$.


Let us first define $e_f^{\geq N}$. Since we want to simulate $T$ we need to be able to check which 
transitions of $T$ can be applied based on a current configuration. More precisely,
suppose that we want to check whether $\delta(q_i,(a_1,\ldots,a_{\ell},b,c))$ is applicable, then we 
need to check whether $T$ is in state $q_i$, we can do this by checking 
$\mathsf{min}(Q_i)$, and whether the heads on the tapes read symbols $a_1,\ldots,a_{\ell},b,c$. We 
check the latter by the following expressions.
For the input tape $R_i$ we define
$$
\mathsf{test\_inp}^i_b:=\begin{cases}
(1-\mathsf{min}(1/2\cdot H_i))\cdot(1-\mathsf{max}(1/2\cdot H_i))\cdot(1- R_i^T\cdot H_i) & \text{if $b=0$}\\
(1-\mathsf{min}(1/2\cdot H_i))\cdot(1-\mathsf{max}(1/2\cdot H_i))\cdot(R_i^T\cdot H_i) & \text{if $b=1$}\\
\mathsf{min}(1/2\cdot H_i) & \text{if $b=\rhd$}\\
\mathsf{max}(1/2\cdot H_i) & \text{if $b=\lhd$},\\
\end{cases}
$$
which returns $1$ if and only if either $b\in\{0,1\}$ is the value in $\mathsf{mat}(R_i)$ at the 
position encoded by $\mathsf{mat}(H_i)$, or when $b=\rhd$ and $\mathsf{mat}(H_i)$ is the vector 
$(2,0,\ldots,0)\in\mathbb{R}^n$, or when $b=\lhd$ and $\mathsf{mat}(H_i)$ is the vector 
$(0,0,\ldots,2)\in\mathbb{R}^n$. Similarly, for the output tape we define
$$
\mathsf{test\_out}_b:=\begin{cases}
(1-\mathsf{min}(1/2\cdot H_O))\cdot(1- O^T\cdot H_O) & \text{if $b=0$}\\
(1-\mathsf{min}(1/2\cdot H_O))\cdot(O^T\cdot H_O) & \text{if $b=1$}\\
\mathsf{min}(1/2\cdot H_O) & \text{if $b=\rhd$},\\
\end{cases}
$$
and for the work tapes $W_1,\ldots,W_s$ we define
$$
\mathsf{test\_work}^i_b:=\begin{cases}
(1-\mathsf{min}(1/2\cdot H_{W_i}))\cdot(1- W_i^T\cdot H_{W_i})) & \text{if $b=0$}\\
(1-\mathsf{min}(1/2\cdot H_{W_i}))\cdot (W_i^T\cdot H_{W_i}) & \text{if $b=1$}\\
\mathsf{min}(1/2\cdot H_{W_i}) & \text{if $b=\rhd$ and $i=1$}.\\
\end{cases}
$$
We then combine all these expressions into a single expression for $q_i\in Q$, 
$a_1,\ldots,a_\ell,b,c\in\Gamma$:
$$
\mathsf{isconf}_{q_i,a_1,\ldots,a_\ell,b,c}:=
\mathsf{min}(Q_i)\cdot \left(\prod_{j=1}^{\ell} \mathsf{test\_inp}_{a_j}^j\right)
\cdot\left(\sum_{j=1}^s \mathsf{test\_work}_b^j\right)\cdot \mathsf{test\_out}_{c}.
$$
This expression will return $1$ if and only if the vectors representing the tapes, 
head positions and states are such that $\mathsf{Q_i}$ is the first canonical vector 
(and thus $T$ is in state $q_i$), the heads point to entries in the tape vectors storing 
the symbols $a_1,\ldots,a_{\ell}, b,c$ or they point to the first (or last for input tapes) 
positions but have value $2$ (when the symbols are $\rhd$ or $\lhd$). 

To ensure that at the beginning of the simulation of $T$ by $e_f^{\geq N}$ we correctly encode 
that we are in the initial configuration, we thus need to initialise all vectors 
$\mathsf{mat}(H_1),\mathsf{mat}(H_2),\ldots, \mathsf{mat}(H_\ell), \mathsf{mat}(H_{W_1}),\mathsf{mat}(H_O)$ 
with the vector $(2,0,0,\ldots,0)\in\mathbb{R}$ since all heads read the symbol $\rhd$. Similarly, 
we have to initialise $\mathsf{Q_1}$ with the first canonical vector since $T$ is in state $q_0$.

We furthermore need to be able to correctly adjust head positions. We do this by means of the predecessor 
and successor expressions described above. 
A consequence of our encoding is that we need to treat the border cases (corresponding to $\rhd$ and 
$\lhd$) differently. More specifically, for the input tapes $R_i$ and heads $H_i$ we define 
$$
\mathsf{move\_inp}^i_{\mathsf{d}}:=
\begin{cases}
2\times \mathsf{min}(H_i)\times H_i + 1/2\times\mathsf{max}(1/2\times H_i)\times H_i  + (1-\mathsf{min}(H_i))(1-\mathsf{max}(1/2 \times H_i))\times e_{\mathsf{Prev}}\cdot H_i  
& \text{if $\mathsf{d}=\leftarrow$}\\
2\times \mathsf{max}(H_i)\times H_i + 1/2\times\mathsf{min}(1/2\times H_i)\times H_i  + (1-\mathsf{min}(1/2\times H_i))(1-\mathsf{max}(H_i))\times e_{\mathsf{Next}}\cdot H_i  
 & \text{if $\mathsf{d}=\rightarrow$}\\
H_i & \text{if $\mathsf{d}=\sqcup$}. 
\end{cases}
$$
In other words, we shift to the previous (or next) canonical vector when $\mathsf{d}$ is $\leftarrow$ 
or $\rightarrow$, respectively, unless we need to move to or from the position that will hold $\rhd$ 
or $\lhd$. In those case we readjust $\mathsf{mat}(H_i)$ (which will either $(1,0,\ldots,0)$, $(2,0,\ldots,0)$, 
$(0,\ldots,0,1)$ or $(0,\ldots,0,2)$) by either dividing or multiplying with $2$. In this way we can 
correctly infer whether or not the head points to the begin and end markers. For the output tape we 
proceed in a similar way, but only taking into account the begin marker and recall that we do not have 
moves to the left:
$$
\mathsf{move\_outp}_{\mathsf{d}}:=
\begin{cases}
1/2\times\mathsf{min}(1/2\times H_O)\times H_O  + (1-\mathsf{min}(1/2\times H_O))\times e_{\mathsf{Next}}\times H_O  
 & \text{if $\mathsf{d}=\rightarrow$}\\
H_O & \text{if $\mathsf{d}=\sqcup$}. 
\end{cases}
$$
Since we represent the work tape by $s$ vectors $W_1,\ldots,W_s$ we need to ensure that only one 
of the head vectors $H_{W_i}$ has a non-zero value and that by moving left or right, we need to 
appropriately update the right head vector. We do this as follows. We first consider the work tapes 
$W_i$ for $i\neq 1,s$ and define
$$
\mathsf{move\_work}^i_{\mathsf{d}}:=
\begin{cases}
	-\mathsf{min}(H_{W_i})\times H_{W_i} + (1-\mathsf{min}(H_{W_i}))\times e_{\mathsf{Prev}}\cdot H_{W_i} + \mathsf{min}(H_{W_{i+1}})\times e_{\mathsf{max}} & \text{if $\mathsf{d}=\leftarrow$}\\
		-\mathsf{max}(H_{W_i})\times H_{W_i} + (1-\mathsf{max}(H_{W_i}))\times e_{\mathsf{Next}}\cdot H_{W_i} + \mathsf{max}(H_{W_{i-1}})\times e_{\mathsf{min}} & \text{if $\mathsf{d}=\rightarrow$}\\
	H_{W_i} & \text{if $\mathsf{d}=\sqcup$}. 	
\end{cases}
$$
In other words, we set the $H_{W_i}$ to zero when a move brings us to either $W_{i-1}$ or $W_{i+1}$, we
move the successor or predecessor when staying within $W_i$, or initialise $H_{W_i}$ with the first or 
last canonical vector when moving from $W_{i-1}$ to $W_i$ (right move) or from $W_{i+1}$ to $W_i$ (left move).
For $i=s$ we can ignore the parts in the previous expression that involve $W_{s+1}$ (which does not exist):
$$
\mathsf{move\_work}^s_{\mathsf{d}}:=
\begin{cases}
	-\mathsf{min}(H_{W_s})\times H_{W_i} + (1-\mathsf{min}(H_{W_s}))\times e_{\mathsf{Prev}}\cdot H_{W_s}  & \text{if $\mathsf{d}=\leftarrow$}\\
		-\mathsf{max}(H_{W_s}) \times H_{W_s} + (1-\mathsf{max}(H_{W_s}))\times e_{\mathsf{Next}}\cdot H_{W_s} + \mathsf{max}(H_{W_{s-1}})\times e_{\mathsf{min}} & \text{if $\mathsf{d}=\rightarrow$}\\
	H_{W_s} & \text{if $\mathsf{d}=\sqcup$}. 	
\end{cases}
$$
For $i=1$, we can ignore the part involving $W_{0}$ (which does not exist) but have to take $\rhd$ 
into account:
$$
\mathsf{move\_work}^1_{\mathsf{d}}:=
\begin{cases}
	2\times \mathsf{min}(H_{W_1})\times H_{W_i} + (1-\mathsf{min}(H_{W_1}))\times e_{\mathsf{Prev}}\cdot H_{W_1} + \mathsf{min}(H_{W_{2}})\times e_{\mathsf{max}} & \text{if $\mathsf{d}=\leftarrow$}\\
		1/2\times\mathsf{min}(1/2\times H_{W_1})\times H_{W_1} + (1-\mathsf{max}(1/2\times H_{W_1}))\times e_{\mathsf{Next}}\cdot H_{W_1}  & \text{if $\mathsf{d}=\rightarrow$}\\
	H_{W_1} & \text{if $\mathsf{d}=\sqcup$}. 	
\end{cases}
$$
A final ingredient for defining $e_f^{\geq N}$ are expressions which update the work and output tape.
To define these expression, we need the position and symbol to put on the tape. For the output tape we define
$$
\mathsf{write\_outp}_b:=\begin{cases}
\mathsf{min}(1/2\times H_O)\times O & \text{if $b=\rhd$}\\
(1-\mathsf{min}(1/2\times H_O))\times\left((1-O^T\cdot H_O)\times O + (O^T\cdot H_O)\times (O-H_O)\right) &\text{if $b=0$}\\
(1-\mathsf{min}(1/2\times H_O))\times\left((1-O^T\cdot H_O)\times (O+H_O) + (O^T\cdot H_O)\times O\right) &\text{if $b=1$}\\
\end{cases}
$$
and similarly for the work tapes $i\neq 1$:
$$
\mathsf{write\_work}_b^i:=\begin{cases}
W_i & \text{if $b=\rhd$}\\
(1-W_i^T\cdot H_{W_i})\times W_i + (W_i^T\cdot H_{W_i})\times (W_i-H_{W_i}) &\text{if $b=0$}\\
(1-W_i^T\cdot H_{W_i})\times (W_i+H_{W_i}) + (W_i^T\cdot H_{W_i})\times W_i &\text{if $b=1$},
\end{cases}
$$
and for  $W_1$ we have to take care again of the begin marker:
$$
\mathsf{write\_work}_b^1:=\begin{cases}
\mathsf{min}(1/2\times H_{W_1})\times W_1 & \text{if $b=\rhd$}\\
(1-\mathsf{min}(1/2\times H_{W_1})\times\left((1-W_1^T\cdot H_{W_1})\times W_1 + (W_1^T\cdot H_{W_1})\times (W_1-H_{W_1})\right) &\text{if $b=0$}\\
(1-\mathsf{min}(1/2\times H_{W_1})\times\left((1-W_1^T\cdot H_{W_1})\times (W_1+H_{W_1}) + (W_1^T\cdot H_{W_1})\times W_1\right) &\text{if $b=1$}.
\end{cases}
$$




% We additionally define, based on the previous expressions,
% 			$$\mathsf{test}_b(v,z,Z,z',Z'):=\begin{cases} \mathsf{min}(v,z,Z,z,Z') & \text{if $b=\rhd$}\\
%      \mathsf{max}(v,z,Z,z,Z') & \text{if $b=\lhd$}\\
% 	 (1-\mathsf{min}(v,z,Z,z',Z'))(1-\mathsf{max}(v,z,Z,z',Z')) & \text{if $b=0$}.
% 		\end{cases}.$$
% 	When evaluated on $I'[v\gets\mathbf{v}]$, $\mathsf{test}_b(I'[v\gets\mathbf{v}])$ will be $1$ when either
% 	$\mathbf{v}=\mathbf{e}_{1}$ and $b=\rhd$ (first position)
% 	$\mathbf{v}=\mathbf{e}_{n+2}$ and $b=\lhd$ (last position),
% 	$\mathbf{v}\neq \mathbf{e}_{1}$ and $\mathbf{v}\neq \mathbf{e}_{n+2}$  and $b=0$ (not first or last position). We use this expression below to check whether the heads are consistent with the symbols on the tape.
%
%
% Finally, we define
%  $$
%  \mathsf{move}_d(z,Z,z',Z'):=\begin{cases}
%  e_{\mathsf{Prev}}(z,Z,z',Z) & \text{if $d=\gets$}\\
%   e_{\mathsf{Next}}(z,Z,z',Z) & \text{if $d=\to$}.
%  \end{cases},
%  $$
%  $\mathsf{move}_d(I')$ will simply return the predecessor matrix when $d=\gets$ and the successor matrix when $d=\to$. This expression will be used to move the heads.
% We define the expression $e_T$, as follows\footnote{In the expression I uses $;w$ to indicate the output variable. That is, the for loops updates all instances for $X_1,\ldots,X_m,Y_1,\ldots,Y_\ell$ and $w$, but the result of the expression is only what is in the instance corresponding to $w$. We can simulate this again if we allow constant dimensional canonical basis vectors in $\mathsf{MATLANG}$.}:



We are now finally ready to define $e_f^{\geq N}$ :
\begin{multline*}
e_f^{\geq N}:= \mathsf{for\,} v_1,\ldots,v_{k},Q_1,\ldots,Q_m,H_1,\ldots,H_\ell,W_1,\ldots,W_s, H_{W_1},\ldots,H_{W_s},O,H_O . \\
(e_{Q_1},\ldots,e_{Q_m},e_{H_1},\ldots,e_{H_\ell},e_{W_1},\ldots,e_{W_s},e_{H_{W_1}},\ldots,e_{H_{W_s}},e_{O}, e_{H_O}).
\end{multline*}
We refer to section \ref{app:def} for the definition of this form of the for-loop. The expressions used are (we use $\star$ below to mark irrelevant information in the transitions):
 \allowdisplaybreaks
\begin{align*}
	e_{Q_1}&:=\left(\prod_{j=1}^{k} \textsf{min}(v_i)\right)\times e_{\mathsf{min}}
	+ \sum_{\substack{(q_i,a_1,\ldots,a_\ell,b,c)\\
	\Delta(q_i,a_1,\ldots,a_\ell,b,c)=(q_1,\star)}} \!\!\!\!\!\!\!\!\! \mathsf{isconf}_{q_i,a_1,\ldots,a_\ell,b,c}\times e_{\mathsf{min}} \\
	e_{Q_j}&:=\sum_{\substack{(q_i,a_1,\ldots,a_\ell,b,c)\\
	\Delta(q_i,a_1,\ldots,a_\ell,b,c)=(q_j,\star)}} \!\!\!\!\!\!\!\!\! \mathsf{isconf}_{q_i,a_1,\ldots,a_\ell,b,c}\times e_{\mathsf{min}}
	 \quad \text{for $j\neq 1$}\\
	e_{H_i}&:=2\left(\prod_{j=1}^{k} \textsf{min}(v_i)\right)\times e_{\mathsf{min}}
	+\sum_{\substack{(q,a_1,\ldots,a_\ell,b,d)\\
	\Delta(q,a_1,\ldots,a_\ell,b,c)=(\star,\mathsf{d_i},\star)}}\!\!\!\!\!\!\!\!\! \mathsf{isconf}_{q,a_1,\ldots,a_\ell,b,c}\times\mathsf{move\_inp}^i_{\mathsf{d}_i}\\
	e_{H_{W_i}}&:=2\left(\prod_{j=1}^{k} \textsf{min}(v_i)\right)\times e_{\mathsf{min}}
	+\sum_{\substack{(q,a_1,\ldots,a_\ell,b,d)\\
	\Delta(q,a_1,\ldots,a_\ell,b,c)=(\star,\mathsf{d_{\ell+1}},\star)}}\!\!\!\!\!\!\!\!\! \mathsf{isconf}_{q,a_1,\ldots,a_\ell,b,c}\times\mathsf{move\_work}_{\mathsf{d}_{\ell+1}}^i\\
	e_{H_O}&:=2\left(\prod_{j=1}^{k} \textsf{min}(v_i)\right)\times e_{\mathsf{min}}
	+\sum_{\substack{(q,a_1,\ldots,a_\ell,b,d)\\
	\Delta(q,a_1,\ldots,a_\ell,b,c)=(\star,\mathsf{d}_{\ell+2})}}\!\!\!\!\!\!\!\!\! \mathsf{isconf}_{q,a_1,\ldots,a_\ell,b,c}\times\mathsf{move\_outp}_{\mathsf{d}_{\ell+2}}\\
	e_{W_i}&:=\sum_{\substack{(q,a_1,\ldots,a_\ell,b,d)\\
	\Delta(q,a_1,\ldots,a_\ell,b,c)=(\star,b',c',\star)}}\!\!\!\!\!\!\!\!\! \mathsf{isconf}_{q,a_1,\ldots,a_\ell,b,c}\times\mathsf{write\_work}_{b'}^i\\
	e_{O}&:=\sum_{\substack{(q,a_1,\ldots,a_\ell,b,d)\\
	\Delta(q,a_1,\ldots,a_\ell,b,c)=(\star,b',c',\star)}}\!\!\!\!\!\!\!\!\! \mathsf{isconf}_{q,a_1,\ldots,a_\ell,b,c}\times\mathsf{write\_outp}_{c'}.
\end{align*}

The correctness of $e_f^{\geq N}$ should be clear from the construction (one can formally verify this by
induction on the number of iterations). We next explain how the border cases $n<N$ can be dealt with.
For each $n<N$ and every possible input words
$w_1,\ldots,w_\ell$ of size $n$, we define a \langfor expression which checks whether
$\mathsf{mat}(R_i)=\mathsf{vec}(w_i)$ for each $i\in[\ell]$. This can be easily done since $n$ 
can be regarded as a constant. For example, to check whether $\mathsf{mat}(R_i)=[0,1,1]^T$ we simply write
$$
(1- R_i^T\cdot e_{\mathsf{min}})\times (R_i^T\cdot e_{\mathsf{Next}}\cdot e_{\mathsf{min}})\times (1- R_i^T\cdot e_{\mathsf{Next}}\cdot e_{\mathsf{Next}}\cdot e_{\mathsf{min}})\times (1- e_{\ones}(R_i)^T\cdot e_{\mathsf{Next}}\cdot e_{\mathsf{Next}}\cdot e_{\mathsf{Next}}\cdot e_{\mathsf{min}})
$$
which will evaluate to $1$ if and only if $\mathsf{mat}(R_i)=[0,1,1]^T$. We note that the final factor is in 
place to check that the dimension of $\mathsf{mat}(R_i)$ is three.
  We denote by
$e_{n,w}^i$ the expression which evaluates to $1$ if and only if $\mathsf{mat}(R_i)=\mathsf{vec}(w)$
for $|w|=n$.
We can similarly
write any word $w$ of fixed size in the matrix variable $O$. For example, suppose that $w=101$
then we write 
$$
O+ e_{\mathsf{min}}+  e_{\mathsf{Next}}\cdot e_{\mathsf{Next}}\cdot e_{\mathsf{min}}.
$$
We write $e_{n,w}$ be the expression which write $w$ of size $|w|=n$ in matrix variable $O$.
Then, the expressions
$$
e_{n,w_1,\ldots,w_n,w}:=e_{n,w_1}^1\cdot\cdots\cdot e_{nw_{\ell}}^\ell\cdot e_{n,w}
$$
will write $w$ in $O$ if and only if $\mathsf{mat}(R_i)=\mathsf{vec}(w_i)$ for $i\in[\ell]$.
We now simply take the disjunction over all words 
$w_1,\ldots,w_\ell\in\Sigma^n$ and $w=f_n(w_1,\ldots,w_\ell)\in\Sigma^n$:
$$
e_n:=\sum_{w_1,\ldots,w_\ell\in\Sigma^n} e_{n,w_1,\ldots,w_\ell,f_n(w_1,\ldots,w_\ell)},
$$
which correctly evaluates $f_n$. We next take a further disjunction by letting ranging from 
$n=0,\ldots, N-1$:
$$
e_f^{<N}:=\sum_{n=0}^{N-1} e_n
$$
Since every possible input is covered and only a unique expression 
$ e_{n,w_1,\ldots,w_\ell,f_n(w_1,\ldots,w_\ell)}$ will be triggered $e_f^{<N}$ will correctly
evaluate $f$ on inputs smaller than $N$.

Our final expression $e_f$ is now given by
$$
e_f:=e_f^{<N} + \mathsf{dim\_is\_greater\_than_N}\times e_f^{\geq N}
$$
where $\mathsf{dim\_is\_greater\_than_N}$ is the expression
$e_{\ones}(R_i)^T\cdot\underbrace{e_{\mathsf{Next}}\cdot\cdots\cdot e_{\mathsf{Next}}}_{\text{$N$ times}}$ 
which will evaluate to $1$ if an only if the input dimension is larger or equal than $N$.
\end{proof}
