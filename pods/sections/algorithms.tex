%!TEX root = /Users/fgeerts/Documents/MLforloops/pods/main.tex
As already mentioned, our main motivation to introduce \texttt{for} loops is to be able
to express classical linear algebra algorithms in a natural way. We have see that \langfor\ is
quite expressive as it can check for cliques, compute the transitive closure, and can even
leverage a successor relation on canonical vectors. The big question is how expressive \langfor\
actually is. We will answer this in general in the next section by connecting \langfor with 
arithmetic circuits of polynomial degree. Through this connection, one can move back and forth between \langfor and arithmetic circuits, and as a consequence, anything computable by such a circuit can be
computed by \langfor as well. When it comes to specific linear algebra algorithms, the detour via circuits
can often be avoided. Indeed, in this section we illustrate that \langfor is able to
compute, in a natural way, the so-called LU decomposition of matrices, and consequently \langfor is expressive enough to compute matrix inversion and the determinant. 
% In fact, many of the algorithms presented in \cite{num} translate naturally in \langfor.

\subsection{LU decomposition}
%
%
% The real power of \langfor comes from the fact that \texttt{for} loops allow us to express many classical linear algebra algorithms. Here we illustrate this by showing how to compute the
A lower-upper (LU) decomposition factors a matrix $A$ as the product of a lower triangular matrix $L$ and upper triangular matrix $U$. This decomposition  underlies many linear algebra algorithms such as solving linear systems of equations, computing the inverse of a matrix and computing the determinant of a matrix, just to name a few. We next show that \langfor can compute a LU-decompositions of matrices.

\smallskip
\noindent
\textbf{LU decomposition by Gaussian elimination.} LU decomposition can be seen as matrix form of Gaussian elimination in which the columns of $A$
are reduced, one by one, to obtain the matrix $U$. The reduction of columns of $A$ is achieved
as follows. Consider the first column $[A_{11},\ldots,A_{n1}]^T$ of $A$ and  define 
\[
c_1 :=
\begin{bmatrix}
    0 \\
    \alpha_{21} \\
    \vdots \\
    \alpha_{n1}
\end{bmatrix},
\]
with $\alpha_{j1} := -\frac{A_{j1}}{A_{11}}$. Let $T_1=I+ c_1\cdot e_1^T$ and consider
$T_1\cdot A$. We note that the first column of $T_1\cdot A$ is now equal to $[A_{11},0,\ldots,0]^T$. That is,
all of its entries below the diagonal are zero. One then iteratively performs a similar computation, using $T_{i-1}\cdot\cdots\cdot T_1\cdot A$ and a matrix $T_i=I+c_i\cdot e_i^T$, to obtain a matrix whose $i$the column is zero below the diagonal (whilst preserving this property for all previous columns). At the end of this process, $T_n\cdot\cdots\cdot T_1\cdot A=U$ where $U$ is desired upper triangular matrix.
Furthermore, it is easily verified that each $T_i$ is invertible and by defining $L=T_1^{-1}\cdot\cdots\cdot T_n^{-1}$ one obtains a lower triangular matrix satisfying $A=L\cdot U$. Of course, the above procedure is only successful when the denominator used in the definition of the vector $c_i$ is non-zero. When this is the case we call a matrix $A$ \textit{LU-factorizable}. When such a denominator is zero in one of the reduction steps, one can remedy this situation by \textit{row pivoting}. That is, when the $i$th entry of the
$i$th row in $T_{i-1}\cdot\cdots\cdot T_1\cdot A$ is zero, one replaces the $i$th row by  $j$th row in this matrix, with $j>i$, provided that $i$the entry of the $j$th row is non-zero. If no such row exists, this implies that all elements below the diagonal are zero already in column $i$ and one can proceed with the next column. One can formulate this in matrix terms by stating that there exists a permutation matrix (which exchanges row) $P$ such that $P\cdot A=L\cdot U$. Any matrix $A$ is LU-factorizable \textit{with pivoting}.
 
 %
 % $LU$ factorization of a matrix using Gaussian eliminations, which is one of the most commonly used matrix algorithms. Recall that
% A matrix $A$ is said to be \textit{LU factorizable} if there exists matrices $T_1,\ldots, T_{n}$ where $T_i=E_{n}^{(i)}\cdots E_{i+1}^{(i)}$ for some \textit{elementary matrices} $E_{j}^{(i)}=I+\alpha_{ij} b_{i}\cdot b_{j}^{T}$ with $\alpha_{ij}\in \RR$, such that $T_{n}\cdots T_1\cdot A=U$ holds, where $U$ is an upper triangular matrix. The matrices $T_i$ are invertible and by defining $L=T_1^{-1}\cdots\cdots T_n^{-1}$ one obtains a lower triangular matrix $L$ such that $A=L\cdot U$. We recall that elementary matrices perform \textit{elementary row operations} such as, interchanging two rows, multiplying a row by an non-zero number, and multiply a row by a non-zero number and add the result to another row.
%
% Now, assume that $A$ is a square matrix which allows $LU$ factorization without row pivoting (we deal with this case later on). This means that after reducing the column $i$ of $A$ (i.e. we make all of the entries below the diagonal zero in the column $i$), we will end up with a matrix which has the element in position $i+1,i+1$ different from zero, for each $i$ strictly less than the dimension of $A$.
%
% To reduce the first column of $A$ it suffices to multiply $A$ from the left with the matrix $T_1 := I + c_1\cdot e_1^*$, where the vector $I$ is the identity matrix, $e_1$ is the first canonical vector, and $c_1$ is the vector
% \[
% c_1 :=
% \begin{bmatrix}
%     0 \\
%     \alpha_{21} \\
%     \vdots \\
%     \alpha_{n1}
% \end{bmatrix},
% \]
% % with $\alpha_{j1} := -\frac{A_{j1}}{A_{11}}$ is the number with which we need to multiply the first row of $A$ in order to reduce the row $j$ of $A$. More generally, if $A$ is reduced up to column $i-1$, reducing the $i$th column is achieved by computing $T_i \cdot A$, where $T_i := I + c_i\cdot e_i^*$, with $c_i$ being the vector with zeros up to position $i$, and with the value $-\frac{A_{ji}}{A_{ii}}$ in position $j > i$.To express the matrices $T_i$ in \langfor we will use the following expression:

\smallskip
\noindent
\textbf{Implementing LU decomposition in \langfor.} 
We first assume that the input matrices are LU-factorizable. We deal with general matrices later on.
To implement the above procedure, for each column $j$ we need to compute the vector $c_j$. We do this in two steps. First, we extract from our input matrix $M$ its $j$th column, but set all its upper diagonal entries to zero. This can be achieved by the following expression:
$$\ccol{M}{y} := \ffor{v}{X}{\mathsf{ssucc}(y,v)\cdot(v^T\cdot M \cdot y)\cdot v + X}.$$
Indeed, when $M$ is assigned to a matrix $A$ and $y$ to $b_j$, the vector $X$ will only be incremented with
entry $A_{ij}$ (computed in $v^T\cdot M \cdot y$) in position $i$ when $i>j$ due to $\mathsf{ssucc}(y,v)$.
%
%
% computes the vector which has zeroes in positions $1\ldots j$, and values $A_{ij}$ in positions $i>j$. Intuitively, the product $y^*\cdot Z_{<}\cdot v$ makes all positions up to $j$ equal zero, and $v^*\cdot A\cdot y$ extracts the element $A_{ij}$ of $A$ when $v$ is interpreted by the $i$th canonical vector, and $y$ by the $j$th one.



Using this expression, we can now compute $T_i$s by writing:
$$\red{A}{y} := I + f_/(\ccol{A}{y},-(y^*\cdot A\cdot y)\cdot \ones(y))y^*,$$
where $f_/$ is the division function. Notice that $f_/(\ccol{A}{y},-(y^*\cdot A\cdot y)\cdot \ones(y))$ in fact computes the vectors $c_i$ defined above. The Gaussian elimination can now be performed by the following expression:
$$
U(A) :=  \left( \initf{I}{v}{X}{\red{X\cdot A}{v}\cdot X} \right) \cdot A,
$$
which computes the $U$ from the $LU$ factorization of $A$, whenever $A$ can be $LU$ factorized without row interchange. Notice that the latter assumption is crucial, since it will ensure that the element of $A$ in position $ii$ is different from zero after each \texttt{reduce} step. As explained previously, the inner \texttt{for} loop in fact computes the product of matrices $T_{n}\cdots T_1$, which in fact amounts to the matrix $L^{-1}$ from the $LU$ factorization of $A$. Given that each $T_i$ is easily invertible, we can also recover $L$.

As stated previously, the construction above works under the assumption that no row pivoting is needed when performing the Gaussian eliminations, giving us:
\begin{proposition}\label{prop:gauss}
There is a \langfor expression $e(X)$ such that by interpreting $X$ with a matrix $A$ will result in the matrix $U$ from the $LU$ factorization of $A$ whenever $A$ is $LU$ factorizable.
\end{proposition}


If $A$ needs row interchange to be $LU$ factorizable (this is, $PA$ is $LU$ factorizable) we say that $A$ is $PALU$ factorizable, where the factorization is $PA=LU$. To accomplish this, we need something extra.

 \begin{proposition}\label{prop:palu}
There is a \langfor expression $e(X)$ such that by interpreting $X$ with a matrix $A$ will result in the matrix $U$ from the $PALU$ factorization of $A$ whenever $A$ is $PALU$ factorizable if and only if there exists a \langfor expression $e'(X)$ such that by interpreting $X$ with a vector $A$ outputs a canonical vector $e_k$ where $A_k$ is the first non zero entry of $A$.
\end{proposition}


Now, since $A^{-1}=U^{-1}L^{-1}$ and using that $U$ is upper triangular, we can obtain the determinant and inverse of $A$ and thus

\subsection{Determinant and inverse}
\thomas{I don't know how much of the proof goes here or if it's ok as it is.}


 \begin{proposition}\label{prop:determinant}
There is a \langfor expression $e(X)$ such that by interpreting $X$ with a matrix $A$ will result in the $1\times 1$ matrix with $\texttt{det}(A)$ in its only entry whenever $A$ is $LU$ factorizable.
\end{proposition}

\begin{proposition}\label{prop:inverse}
There is a \langfor expression $e(X)$ such that by interpreting $X$ with a matrix $A$ will result in the matrix $A^{-1}$ whenever $A$ is $LU$ factorizable.
\end{proposition}

