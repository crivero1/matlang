%!TEX root = ../main.tex
As already mentioned, our main motivation to introduce \texttt{for} loops is to be able
to express classical linear algebra algorithms in a natural way. We have seen that \langfor is
quite expressive as it can check for cliques, compute the transitive closure, and can even
leverage a successor relation on canonical vectors. The big question is how expressive \langfor
actually is. We will (partially) answer this in the next section by connecting \langfor with 
arithmetic circuits of polynomial degree. Through this connection, one can move back and forth between \langfor and arithmetic circuits, and as a consequence, anything computable by such a circuit can be
computed by \langfor as well. When it comes to specific linear algebra algorithms, the detour via circuits
can often be avoided. Indeed, in this section we illustrate that \langfor is able to
compute, in a natural way, the so-called LU decomposition of matrices. As a consequence, we show that \langfor is expressive enough to compute matrix inversion and the determinant. We recall that matrix inversion and determinant need to be explicitly added as separate operators in \lang~\cite{matlang,matlang-journal} and that the LARA language is unlikely to compute the inverse of a matrix under usual complexity-theoretic assumptions~\cite{BarceloH0S20}.
% In fact, many of the algorithms presented in \cite{num} translate naturally in \langfor.

\subsection{LU decomposition}
%
%
% The real power of \langfor comes from the fact that \texttt{for} loops allow us to express many classical linear algebra algorithms. Here we illustrate this by showing how to compute the
A lower-upper (LU) decomposition factors a matrix $A$ as the product of a lower triangular matrix $L$ and upper triangular matrix $U$. This decomposition, and more generally LU decomposition with pivoting,  underlies many linear algebra algorithms such as solving linear systems of equations, computing the inverse of a matrix and computing the determinant of a matrix, just to name a few. We next show that \langfor can compute these decompositions.

\smallskip
\noindent
\textbf{LU decomposition by Gaussian elimination.} LU decomposition can be seen as a matrix form of Gaussian elimination in which the columns of $A$
are reduced, one by one, to obtain the matrix $U$. The reduction of columns of $A$ is achieved
as follows. Consider the first column $[A_{11},\ldots,A_{n1}]^T$ of $A$ and  define 
$c_1 := [0, \alpha_{21},\ldots, \alpha_{n1}]^T$ 
with $\alpha_{j1} := -\frac{A_{j1}}{A_{11}}$. Let $T_1:=I+ c_1\cdot e_1^T$ and consider
$T_1\cdot A$. We note that the first column of $T_1\cdot A$ is now equal to $[A_{11},0,\ldots,0]^T$. That is,
all of its entries below the diagonal are zero. Additionally, the $j$th row of $T_1\cdot A$ is obtained by multiplying the first row of $A$ by $\alpha_{21}$ and adding it to the second row of $A$, as required. One then iteratively performs a similar computation, using a matrix $T_i:=I+c_i\cdot e_i^T$, where $c_i$ now depends on the $i$th column in $T_{i-1}\cdot\cdots\cdot T_1\cdot A$. As a consequence, $T_i\cdot T_{i-1}\cdots \cdot T_1\cdot A$ is upper triangular
in its first $i$ columns. At the end of this process, $T_n\cdot\cdots\cdot T_1\cdot A=U$ where $U$ is the desired upper triangular matrix.
Furthermore, it is easily verified that each $T_i$ is invertible and by defining $L:=T_1^{-1}\cdot\cdots\cdot T_n^{-1}$ one obtains a lower triangular matrix satisfying $A=L\cdot U$. Of course, the above procedure is only successful when the denominators used in the definition of the vectors $c_i$ are non-zero. When this is the case we call a matrix $A$ \textit{LU-factorizable}. 

In case when such a denominator is zero in one of the reduction steps, one can remedy this situation by \textit{row pivoting}. That is, when the $i$th entry of the
$i$th row in $T_{i-1}\cdot\cdots\cdot T_1\cdot A$ is zero, one replaces the $i$th row by  $j$th row in this matrix, with $j>i$, provided that $i$the entry of the $j$th row is non-zero. If no such row exists, this implies that all elements below the diagonal are zero already in column $i$ and one can proceed with the next column. One can formulate this in matrix terms by stating that there exists a permutation matrix $P$, which pivots rows, such that $P\cdot A=L\cdot U$. Any matrix $A$ is LU-factorizable \textit{with pivoting}.
 
 %
 % $LU$ factorization of a matrix using Gaussian eliminations, which is one of the most commonly used matrix algorithms. Recall that
% A matrix $A$ is said to be \textit{LU factorizable} if there exists matrices $T_1,\ldots, T_{n}$ where $T_i=E_{n}^{(i)}\cdots E_{i+1}^{(i)}$ for some \textit{elementary matrices} $E_{j}^{(i)}=I+\alpha_{ij} b_{i}\cdot b_{j}^{T}$ with $\alpha_{ij}\in \RR$, such that $T_{n}\cdots T_1\cdot A=U$ holds, where $U$ is an upper triangular matrix. The matrices $T_i$ are invertible and by defining $L=T_1^{-1}\cdots\cdots T_n^{-1}$ one obtains a lower triangular matrix $L$ such that $A=L\cdot U$. We recall that elementary matrices perform \textit{elementary row operations} such as, interchanging two rows, multiplying a row by an non-zero number, and multiply a row by a non-zero number and add the result to another row.
%
% Now, assume that $A$ is a square matrix which allows $LU$ factorization without row pivoting (we deal with this case later on). This means that after reducing the column $i$ of $A$ (i.e. we make all of the entries below the diagonal zero in the column $i$), we will end up with a matrix which has the element in position $i+1,i+1$ different from zero, for each $i$ strictly less than the dimension of $A$.
%
% To reduce the first column of $A$ it suffices to multiply $A$ from the left with the matrix $T_1 := I + c_1\cdot e_1^*$, where the vector $I$ is the identity matrix, $e_1$ is the first canonical vector, and $c_1$ is the vector
% \[
% c_1 :=
% \begin{bmatrix}
%     0 \\
%     \alpha_{21} \\
%     \vdots \\
%     \alpha_{n1}
% \end{bmatrix},
% \]
% % with $\alpha_{j1} := -\frac{A_{j1}}{A_{11}}$ is the number with which we need to multiply the first row of $A$ in order to reduce the row $j$ of $A$. More generally, if $A$ is reduced up to column $i-1$, reducing the $i$th column is achieved by computing $T_i \cdot A$, where $T_i := I + c_i\cdot e_i^*$, with $c_i$ being the vector with zeros up to position $i$, and with the value $-\frac{A_{ji}}{A_{ii}}$ in position $j > i$.To express the matrices $T_i$ in \langfor we will use the following expression:

\smallskip
\noindent
\textbf{Implementing LU decomposition in \langfor.} 
We first assume that the input matrices are LU-factorizable. We deal with general matrices later on.
To implement the above procedure, we need to compute the vector $c_i$ for each column $i$. We do this in two steps. First, we extract from our input matrix its $i$th column and set all its upper diagonal entries to zero. This can be achieved by the following expression:
$$\ccol{V}{y} := \ffor{v}{X}{\mathsf{succ}^+(y,v)\cdot(v^T\cdot V \cdot y)\cdot v + X}.$$
Indeed, when $V$ is assigned to a matrix $A$ and $y$ to $b_i$, we have that $X$ will be initially assigned
$A_0=\mathbf{0}$ and in consecutive iterations,  $A_j=A_{j-1}+ b_j^T\cdot A\cdot b_i$ if $j>i$ (because $\mathsf{succ}^+(b_i,b_j)=1$ if $j>i$) and $A_j=A_{j-1}$ otherwise (because $\mathsf{succ}^+(b_i,b_j)=0$ for $j\leq i$). It is now clear that the result of this evaluation is the desired column vector.
% vector $X$ will only be incremented with
% entry $A_{ij}$ (computed in $v^T\cdot M \cdot y$) in position $i$ when $i>j$ due to $\mathsf{ssucc}(y,v)$.
%
%
% computes the vector which has zeroes in positions $1\ldots j$, and values $A_{ij}$ in positions $i>j$. Intuitively, the product $y^*\cdot Z_{<}\cdot v$ makes all positions up to $j$ equal zero, and $v^*\cdot A\cdot y$ extracts the element $A_{ij}$ of $A$ when $v$ is interpreted by the $i$th canonical vector, and $y$ by the $j$th one.
Using $\ccol{V}{y}$, we can now compute $T_i$ by the following expression:
$$\red{V}{y} := e_{\mathsf{Id}}+ f_/(\ccol{V}{y},-(y^T\cdot V\cdot y)\cdot \ones(y))\cdot y^T,$$
where $f_/:\RR^2\to\RR:(x,y)\mapsto x/y$ is the division function. It should be clear that when $V$ is assigned to $A$ and $y$ to $b_i$, $f_/(\ccol{A}{b_i},-(b_i^T\cdot A\cdot b_i)\cdot \ones(b_i))$ is equal to the vector $c_i$ used in the definition of $T_i$. To perform the reduction steps for all columns, we consider
the expression:
$$
e_{U}(V) :=  \left( \initf{e_{\mathsf{Id}}}{y}{X}{\red{X\cdot V}{y}\cdot X} \right) \cdot V.
$$
That is, when $V$ is assigned $A$, $X$ will be initially $A_0=I$, and then
$A_i=\red{A_{i-1}\cdot A}{b_i}=T_i\cdot T_{i-1}\cdots T_1\cdot A$, as desired.
We show in the appendix that, because we can obtain the matrices $T_i$ in \langfor and that these
are easily invertible, we can also construct an expression $e_L(V)$ which evaluates to $L$ when $V$ is assigned to
$A$. We may thus conclude the following.
\begin{proposition}\label{prop:gauss}
There exists $\langforf{f_/}$ expressions $e_L(V)$ and $e_U(V)$ such that
$\sem{e_L}{\I}=L$ and $\sem{e_U}{\I}=U$ form an LU-decomposition of $A$,
where $\conc(V)=A$ and $A$ is LU-factorizable.\qed
% such that by interpreting $X$ with a matrix $A$ will result in the matrix $U$ from the $LU$ factorization of $A$ whenever $A$ is $LU$ factorizable.
\end{proposition}
We remark that the proposition holds when division is added as a function in $\Fun$
in \langfor. When row pivoting is needed, we can also obtain a permutation matrix
$P$ such that $P\cdot A=L\cdot U$ holds by means of an expression in \langfor, provided
that we additionally allow the function $f_{>0}$, as defined in Example~\ref{ex:programtc}. This function allows for computing the following
operation: Given a vector $v$, return $\mathsf{minnz}(v)=b_j$, the $j$th canonical vector, where $j$ is the first entry in $v$ that holds a non-zero value, or returns $\mathsf{minnz}(v)=\mathbf{0}$ if no such non-zero value exists. In fact, the ability to express
$\mathsf{minnz}(\cdot)$ is intrinsically tied to the expressibility of LU decompositions with pivoting:

%
% which computes the $U$ from the $LU$ factorization of $A$, whenever $A$ can be $LU$ factorized without row interchange. Notice that the latter assumption is crucial, since it will ensure that the element of $A$ in position $ii$ is different from zero after each \texttt{reduce} step. As explained previously, the inner \texttt{for} loop in fact computes the product of matrices $T_{n}\cdots T_1$, which in fact amounts to the matrix $L^{-1}$ from the $LU$ factorization of $A$. Given that each $T_i$ is easily invertible, we can also recover $L$.
%
% As stated previously, the construction above works under the assumption that no row pivoting is needed when performing the Gaussian eliminations, giving us:
% \begin{proposition}\label{prop:gauss}
% There is a \langfor expression $e(X)$ such that by interpreting $X$ with a matrix $A$ will result in the matrix $U$ from the $LU$ factorization of $A$ whenever $A$ is $LU$ factorizable.
% \end{proposition}

%
% If row pivoting is needed for $A$, i.e., such that $P\cdot A$ is LU-factorizable for a permutation matrix $P$,
% we can construct $P$ in \langfor, provided that we can express the following: Given a vector $v$, find $\mathsf{minnz}(v)=b_j$ where $j$ is the first entry in $v$ that holds a non-zero value.
% One can show this can be expressed in \langfor$(f>0)$. More generally:
 %
 % needs row interchange to be $LU$ factorizable (this is, $PA$ is $LU$ factorizable) we say that $A$ is $PALU$ factorizable, where the factorization is $PA=LU$. To accomplish this, we need something extra.

 \begin{proposition}\label{prop:palu}
There exist expressions $e_{L^{-1}P}(M)$ and $e_U(M)$ in $\langforf{\Fun}$  such that for
$L^{-1}\cdot P=\sem{e_{L^{-1}P}}{\I}$ and $U=\sem{e_U}{\I}$, $L^{-1}\cdot P\cdot A=U$ holds
if and only if $\Fun$ includes division $f_/$ and $\langforf{\Fun}$ can express the function $\mathsf{minnz}(\cdot)$.\qed
 %
 % such that by interpreting $X$ with a matrix $A$ will result in the matrix $U$ from the $PALU$ factorization of $A$ whenever $A$ is $PALU$ factorizable if and only if there exists a \langfor expression $e'(X)$ such that by interpreting $X$ with a vector $A$ outputs a canonical vector $e_k$ where $A_k$ is the first non zero entry of $A$.
\end{proposition}
As mentioned above, $\langforf{f_/,f_>0}$ suffices to compute LU decomposition with pivoting for arbitrary matrices.
Intuitively, by allowing $f_{>0}$ we introduce a limited form of \texttt{if-then-else} in \langfor, which is needed
to continue reducing columns only when the right pivot has been found.

\subsection{Determinant and inverse}
Other key linear algebra operations include the computation of the determinant and
the inverse of a matrix (if the matrix is invertible). When the input matrix
$A$ is LU-factorizable, then it is known that $\mathsf{det}(A)=\mathsf{det}(U)$.
Based on Proposition~\ref{prop:gauss}
and by recalling that the determinant of an upper triangular matrix is the product
of its diagonal entries, one can readily verify that there exists a $\langforf{f_/}$
expression $e_{\mathsf{det}}(V)$ such that $\sem{e_{\mathsf{det}}}{\I}=\mathsf{det}(A)$
when $\I$ assigns $A$ to $V$ and $A$ is LU-factorizable. Using a similar, but slightly
more complex argument, we can infer from Proposition~\ref{prop:palu} the following.
 \begin{proposition}\label{prop:determinant}
There is a $\langforf{f_/,f_{>0}}$ expression $e_{\mathsf{det}}(V)$ such 
that $\sem{e_{\mathsf{det}}}{\I}=\mathsf{det}(A)$ when $\I$ assigns $V$
to $A$.\qed
\end{proposition}

Furthermore, when $A$ is LU-factorizable, $A$ is invertible if and only if $\mathsf{det}(U)\neq 0$. More specifically,
in that case, $A^{-1}=U^{-1}\cdot L^{-1}$. We recall that $L^{-1}=T_n\cdot\cdots\cdot T_1$ and we argued already that this can be
computed in $\langforf{f_/}$. Due to $U$ being an upper triangular matrix with non-zero elements on its diagonal, it
is now readily verified that $U=(I-D^{-1}_U\cdot U')^{-1}\cdot D_U^{-1}$ with $D_U$ the diagonal matrix consisting of the diagonal entries of $U$ and $U'=U-D_U$. We remark that $D_U^{-1}$ can simply be obtained by inverting the (non-zero) elements on the diagonal by means of $f_/$. Furthermore, we observe that $(I-D^{-1}_U\cdot U')^{-1}=\sum_{i=0}^{n}(-D_U^{-1}U')^i$ which is something we can compute in \langfor as well. As a consequence:
\begin{proposition}\label{prop:inverse1}
There is a $\langforf{f_/}$ expression $e_{\mathsf{inv}}(V)$ such that 
$\sem{e_{\mathsf{inv}}}{\I}=A^{-1}$ when $\I$ assigns $V$
to $A$ and $A$ is invertible and LU-factorizable.
\qed
 % that by interpreting $X$ with a matrix $A$ will result in the matrix $A^{-1}$ whenever $A$ is $LU$ factorizable.
\end{proposition}
We can again generalize this result, by including $f_{>0}$ in the language, to arbitrary
matrices and by using that any matrix is LU-factorizable with pivoting. However, the results
in the next section (connecting \langfor with arithmetic circuits) imply that the determinant
and inverse of a matrix can already be defined in $\langforf{f_/}$. So instead of using LU decomposition with pivoting to treat the general matrix case, we provide an alternative solution
for computing the inverse.

More specifically, we rely on Csanky's algorithm for computing the inverse of a matrix~\cite{Csanky76}. In a nutshell, Csanky's algorithm relies on the characteristic
polynomial $p_A(x)=\mathsf{det}(xI-A)$ of a matrix. When expanded as a polynomial
$p_A(x)=\sum_{i=0}^{n} c_i x^i$ and it is known that $A^{-1}=\frac{-1}{\phantom{-1}c_n}\sum_{i=0}^{n-1}c_i A^{n-1-i}$
if $c_n\neq 0$. Furthermore, $c_0=1$, $c_n=(-1)^n\mathsf{det}(A)$ and the coefficients $c_i$ of $p_A(x)$
are known to satisfy the following system of equations:
$$
\left(\begin{matrix}
1 & 0 & 0 & \cdots & 0 & 0\\
S_1 & 2 & 0 & \cdots  &0 & 0\\
S_2 & S_1 & 3 & \cdots  &0 & 0\\
\vdots & \vdots & \vdots & \vdots & \vdots & 0\\
S_{n-1} & S_{n-2} & S_{n-3} & \cdots & S_1 & n\\
\end{matrix}\right)\cdot
\left(\begin{matrix}
c_1\\
c_2\\
c_3\\
\vdots\\
c_n\\
\end{matrix}\right)=\left(\begin{matrix}
S_1\\
S_2\\
S_3\\
\vdots\\
S_n\\
\end{matrix}\right),
$$
with $S_i=\mathsf{tr}(A^i)$. We show, in the appendix, that we can construct all ingredients of this system of equations in \langfor. By observing that the matrix in this system is a lower triangular matrix with non-zero elements on its diagonal, we have that this matrix is LU-factorizable. Proposition~\ref{prop:inverse1} then implies that we can compute the inverse of this matrix in 
$\langforf{f_/}$ and hence, we can solve the above system. That is, we can compute the vector $(c_1,\ldots,c_n)^T$ in \langfor. It now remains to construct $A^{-1}$,
as described above, in terms of the coefficients of $p_A(x)$ and powers of $A$. All this can be done
in \langfor. We can similarly extract the determinant of $A$ from $c_n$. We may thus conclude:
\begin{proposition}\label{prop:inverse}
There are $\langforf{f_/}$ expressions $e_{\mathsf{det}}(V)$ and $e_{\mathsf{inv}}(V)$ such that
$\sem{e_{\mathsf{det}}}{\I}=\mathsf{det}(A)$, and  
$\sem{e_{\mathsf{inv}}}{\I}=A^{-1}$ when $\I$ assigns $V$
to $A$ and $A$ is invertible.
\qed
 % that by interpreting $X$ with a matrix $A$ will result in the matrix $A^{-1}$ whenever $A$ is $LU$ factorizable.
\end{proposition}

%

% We next highlight two important consequence of the previous results. First,
% by observing that $\mathsf{det}(A)=\mathsf{det}(P)\cdot \mathsf{U}$ when
% $P\cdot A=L\cdot U$, we need to show that we can compute the determinant of
% the permutation matrix $P$ (which will be $+1$ or $-1$) and the determinant
% of $U$. For the latter, since $U$ is upper triangular $\mathsf{det}(U)=\prod_{i}U_{ii}$, which can easily be computed in \langfor.
% Moreover, $\mathsf{det}(P)=1$ when $P$ corresponds to an even permutation and
% $\mathsf{det}(P)=-1$ otherwise. One can determine in \langfor $\mathsf{det}(P)=1$
% or $\mathsf{det}(P)=-1$.

%
% that by interpreting $X$ with a matrix $A$ will result in the $1\times 1$ matrix with $\texttt{det}(A)$ in its only entry whenever $A$ is $LU$ factorizable.
%
% In case that $\mathsf{det}(A)\neq 0$ we can further obtain the inverse matrix $A^{-1}$ of $A$ from $P^T\cdot L^{-1}\cdot U$. Due to the simplicity of the lower triangular matrix $L$, one can show that \langfor can compute its inverse. Hence,
%
% Now, since $A^{-1}=U^{-1}L^{-1}$ and using that $U$ is upper triangular, we can obtain the determinant and inverse of $A$ and thus
% \thomas{I don't know how much of the proof goes here or if it's ok as it is.}

%
%  \begin{proposition}\label{prop:determinant}
% There is a \langfor expression $e(X)$ such that by interpreting $X$ with a matrix $A$ will result in the $1\times 1$ matrix with $\texttt{det}(A)$ in its only entry whenever $A$ is $LU$ factorizable.
% \end{proposition}

% \begin{proposition}\label{prop:inverse}
% There is a \langfor$(f_/,f_{>0})$ expression $e_{\mathsf{inv}}(V)$ such
% $\sem{e_{\mathsf{inv}}}{\I}=A^{-1}$ when $\I$ assigns $V$
% to $A$ and $A$ is invertible, and $\sem{e_{\mathsf{inv}}}{\I}=\mathbf{0}$ when
% $A$ is not invertible.\qed
%  % that by interpreting $X$ with a matrix $A$ will result in the matrix $A^{-1}$ whenever $A$ is $LU$ factorizable.
% \end{proposition}
%
% \floris{some final comments.}


