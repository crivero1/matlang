\newtheorem*{Undec}{Proposition~\ref{prop-undec}}
Let $e$ be a \langfor expression over a matrix schema $\mathcal{S}=(\mathcal{M},\textsf{size})$ and let $V_1,\ldots, V_k$ be
the variables of $e$, each of type $(\alpha,\alpha)$, $(1,\alpha)$, $(\alpha,1)$ or $(1,1)$. We know from Theorem~\ref{th-ml-to-circuits}
that there exists a uniform arithmetic circuit family $\{\Phi_n \mid n=1,2,\ldots\}$
such that $\sem{e}{\I}=\Phi_n(A_1,\ldots,A_k)$ for any instance $\I$ such that
$\mathcal{D}(\alpha)=n$ and $\conc(V_i)=A_i$ for $i=1,\ldots,k$. We are interested in deciding
whether there exists such a  uniform arithmetic circuit family $\{\Phi_n \mid n=1,2,\ldots\}$
of polynomial degree, i.e., such that $\mathsf{degree}(\Phi_n)=\mathcal{O}(p(n))$ for some polynomial $p(x)$. If such a circuit family exists, we call $e$ of polynomial degree.

\begin{Undec}
	Given a \langfor expression $e$ over a schema $\Sch$, it is undecidable to check whether $e$ is of polynomial degree.
\end{Undec}
\begin{proof}
We show undecidability based on the following undecidable language:
$$
\{ \langle M\rangle\mid \text{$M$ is a deterministic TM which halts on the empty input}\},
$$
where $\langle M\rangle$ is some string encoding of $M$.
Consider a TM $M$ described by $(Q,\Gamma=\{0,1\},q_0,q_m,\Delta)$
with $Q=\{q_1,\ldots,q_m\}$ its states, $q_1$ being the initial state and $q_m$ being
the halting state, $\Gamma$ is the tape alphabet, and $\Delta$ is a transition function
from $Q\times \Gamma\to Q\times\Gamma\times \{\leftarrow,\sqcup,\rightarrow\}$. The simulation
of linear space TM, as given in the proof of Proposition~\ref{prop:transducer} can be easily modified to
any TM $M$ provided that we limit the execution of $M$ to exactly $n$ steps. Let $e_M$ denote this expression. Similarly
as in the linear space TM simulation, we have vector variables $Q_1,\ldots,Q_m$ encoding the
states, a single relation $T$ encoding the tape and relation $H_T$ encoding the position
of the tape.  When an instance $\I$ assigns $n$ to $\alpha$, we have a tape of length $n$ at our disposal. This suffices if we let $M$ run for $n$ steps. We further observe that all vector variables can be assumed to be zero, initially.
This is because we do not have input. So, let $\I_n^0$ denote the instance which assigns vector variables to the $n$-dimensional zero vector.  Furthermore, by contrast to the linear space TM simulation, we use a single vector $v$ (instead of $k$ such vectors) to simulate $n$ steps of $M$. Finally, we modify the expression given in the proof of Proposition~\ref{prop:transducer} such $\sem{e_M}{\I_n^0}$  returns $1$ if $M$
halts in at most $n$ steps, and $0$ if $M$ did not halt yet after $n$ steps.

As a consequence, when $M$ does not halt, $\sem{e_M}{\I_n^0}=0$ for all $n\geq 0$. When $M$ halts, there will be an $n$ such that $\sem{e_M}{\I_n^0}=1$ It now suffices to consider the \langfor expression
$$
d_M:=e_M\cdot e_{\mathsf{exp}}
$$
where $e_{\texttt{exp}} = \ffor{v}{X=\ones(X)^T\cdot\ones(X)}{X\cdot X}$ such that
$e_{\texttt{exp}}(\I_n^0)=n^{2^n}$. Then, when $M$ does not halt we can clearly compute $d_M$ with a constant degree circuit ``0''
for any $n$, otherwise, the circuit needed will be of exponential degree
for at least one $n$, simply because no polynomial degree uniform  circuit family can compute $n^{2^n}$. In other words, deciding whether $d_M$ has polynomial degree coincides with deciding whether $M$ halts.
\end{proof}


%
% 3/ Based on e_M(n) we now consider the ML expression
%
%  d_M(n):=e_M(n) Pow(n)
%
% with Pow(n)=2^n (or any other ML expression which cannot
% be computed by a poly degree circuit). Then, d_M(n) will be of
% poly degree if M does not halt in n steps. And since we
% want to know whether it is poly degree for all n (uniform circuits)
% M should not halt. Hence, the poly degree problem cannot be decidable.
