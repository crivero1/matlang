%!TEX root = /Users/fgeerts/Documents/MLforloops/pods/main.tex

\floris{Many small details missing....}
Let $e$ be a MATLANG expression over a matrix schema $\mathcal{S}=(\mathcal{M},\textsf{size})$.
We define the size function of $\mathcal{M}$ as the sum $\sum_{V\in\mathcal{M}} \textsf{size}(V)$. Since
$\textsf{size}$ in $\mathcal{S}$ assigns either $1\times 1$, $\alpha\times 1$, $1\times\alpha$ or $\alpha\times\alpha$ to matrix variables in $\mathcal{M}$, the size of $\mathcal{M}$ is a polynomial $s(\alpha)$ in $\alpha$. 

We have seen that with any MATLANG expression $e$ over a matrix schema $\mathcal{S}$ we can associate a logspace computable function which on input $1^n$ returns an arithmetic circuit $\Phi_{s(n)}$ of size $s(n)$ such that for every instance $I=(\mathcal{D},\textsf{mat})$ over $\mathcal{S}$ with $\mathcal{D}(\alpha)=n$, $e(I)$ can be obtained from evaluating $\Phi_{s(n)}$ on $I$. 

We are here interested in deciding whether a given a MATLANG expression $\mathcal{S}$ whether we can
associate with it a uniform arithmetic circuit family $\{\Phi_{s(n)}\mid n=1,2,\}$ which has polynomial degree. More precisely, we want to decide membership in the following class
$$
\{ e\in \mathsf{MATLANG}\mid \text{$e$ can be evaluated by uniform polynomial degree circuits}\}.
$$
\floris{The above is the question right? That is, we do not want to know whether the circuits that we can generate are polynomially bounded. But want to know the existence of such a family...}

We show that this problem is undecidable. To see this, we use that the problem of deciding membership
in 
$$
\{ \langle M\rangle\mid \text{$M$ is a deterministic TM which halts on the empty input}\}
$$
is undecidable. Consider a TM $M$ described by $(Q,\Gamma=\{0,1\},q_0,q_m,\Delta)$
with $Q=\{q_1,\ldots,q_m\}$ its states, $q_1$ being the initial state and $q_m$ being
the halting state, $\Gamma$ is the tape alphabet, and $\Delta$ is a transition function
from $Q\times \Gamma\to Q\times\Gamma\times \{\leftarrow,\sqcup,\rightarrow\}$. The 
simulation by a MATLANG expression of a linear space TM can be easily modified to
any TM $M$ provided that we limit the execution of $M$ to exactly $n$ steps. Let $e_M$ denote this expression. Similarly
as in the linear space TM simulation, we have relation $Q_1,\ldots,Q_m$ encoding the
states, a single relation $T$ encoding the tape and relation $H_T$ encoding the position
of the tape. We note that we do not have any input and our tape has length $n$ when the matrix variables are assigned $n$ as size. By contrast to the linear space 
simulation, we also use a single vector $v$ (instead of $k$ such vectors) 
to simulate $n$ steps of $M$. We modify the expression such that it returns $1$ if $M$
halts in at most $n$ steps, and $0$ if $M$ did not halt yet after $n$ steps.

As a consequence, when $M$ does not halt, $e_M$ will always return $0$
for any $n$. When $M$ halts, there will be an $n$ such that $e_M$ returns $1$.
It now suffices to consider the matlang expression
$$
d_M:=e_M\cdot e_{\mathsf{exp}}
$$
where $e_{\mathsf{exp}}$ computes the functions $x^{2^n}$. 
\floris{this expression needs to be detailed....}
Then, when $M$ does not halt we can clearly compute $d_M$ with a constant circuit ``0''
for any $n$, otherwise, the circuit needed will be of exponential degree
for at least one $n$. 
\floris{Is the latter clear?}



%
% 3/ Based on e_M(n) we now consider the ML expression
%
%  d_M(n):=e_M(n) Pow(n)
%
% with Pow(n)=2^n (or any other ML expression which cannot
% be computed by a poly degree circuit). Then, d_M(n) will be of
% poly degree if M does not halt in n steps. And since we
% want to know whether it is poly degree for all n (uniform circuits)
% M should not halt. Hence, the poly degree problem cannot be decidable.
