\newtheorem*{ALU}{Proposition~\ref{prop:gauss}}

We start with LU-decomposition without pivoting. We recall proposition \ref{prop:gauss}:
\begin{ALU}
  There exists $\langforf{f_/}$ expressions $e_L(V)$ and $e_U(V)$ such that
  $\sem{e_L}{\I}=L$ and $\sem{e_U}{\I}=U$ form an LU-decomposition of $A$,
  where $\conc(V)=A$ and $A$ is LU-factorizable.
\end{ALU}
\begin{proof}
	Let $A$ be an LU-factorizable matrix. We already explained how the expression 
	$e_U(V)$ is obtained in the main body of the paper, i.e., 
	$$
	e_{U}(V) :=  \left( \initf{e_{\mathsf{Id}}}{y}{X}{\red{X\cdot V}{y}\cdot X} \right) \cdot V.
	$$
	We recall that $e_U(A)=T_n\cdot\cdots\cdot T_1\cdot A$ with $L^{-1}=T_n\cdot\cdots\cdot T_1$. Let
	$$
	e_{L^{-1}}(V) :=  \initf{e_{\mathsf{Id}}}{y}{X}{\red{X\cdot V}{y}\cdot X}.
	$$
such that	$$
	e_{\mathsf{U}}(V) :=  e_{L^{-1}}(V) \cdot V.
	$$
	%
%
% 	 We therefore focus on the
% 	expression $e_{L}(V)$.
%
% 	 We define
% $$
% e_{L^{-1}}(V) :=  \initf{e_{\mathsf{Id}}}{y}{X}{\red{X\cdot V}{y}\cdot X}.
% $$
% Note that, in the iteration, if $V$ is $A$, $X$ is initially $A_0=I$ then
% \begin{align*}
%   A_i&=\red{A_{i-1}\cdot A}{b_i}\cdot A_{i-1} \\
%   &=T_iA_{i-1} \\
%   &=T_iT_{i-1}\cdots T_1A_0 \\
%   &=T_iT_{i-1}\cdots T_1.
% \end{align*}
% Such that $T_nT_{n-1}\cdots T_1A=U$. Note that because of the definition of $\ccol{\cdot}{\cdot}$ inside $\red{\cdot}{\cdot}$, we always get $T_n=I$.
% So we have
% $$
% e_{\mathsf{U}}(V) :=  e_{L^{-1}}(V) \cdot V.
% $$
It now suffices to observe that, since $T_n=I$,
\begin{align*}
  L^{-1}&=(I-c_1\cdot b_1^T)\cdots (I-c_{n-1}\cdot  b_{n-1}^T) \\
  &=I-c_1\cdot b_1^T-\cdots - c_{n-1}\cdot b_{n-1}^T
\end{align*}
and hence,
\begin{align*}
  L&=(I+c_1\cdot b_1^T)\cdots (I+c_{n-1}\cdot b_{n-1}^T) \\
  &=I+c_1\cdot b_1^T+\cdots + c_{n-1}\cdot b_{n-1}^T.
\end{align*}
As a consequence, to obtain $L$ from $L^{-1}$ we just need to multiply every entry below the diagonal by $-1$. Since both  $L$ and $L^{-1}$ are lower triangular, this can done 
by computing $L=-1\times L^{-1} + 2\times I$. Translated into \langfor, this means that we can define
$$
e_{L}(V) :=  -1\times e_{L^{-1}}(V) + 2\times e_{\mathsf{Id}},
$$
which concludes the proof of the proposition.
\end{proof}
