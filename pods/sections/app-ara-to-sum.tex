\newtheorem*{ARATOSUM}{Proposition~\ref{prop:ara_to_sum}}

Here we present the proof of proposition \ref{prop:ara_to_sum}.

\begin{ARATOSUM}
  Let $\cR$ be a binary relational schema. For each $\mathsf{RA}_{K}^+$  expression $\arae$ over $\cR$  such that $|\cR(\arae)| = 2$, there exists a \langsum  expression $\Psi(\arae)$ over \lang \ schema $\text{Mat}(\cR)$ such that for any $K$-instance $\cJ$ with $\adom(\cJ) = \{d_1, \ldots, d_n\}$ over $\cR$,
	$$
	\ssem{\arae}{\cJ}((d_i, d_j))=\sem{\Psi(\arae)}{\text{Mat}(\cJ)}_{i,j}.
	$$
	Similarly for when $|\cR(\arae)| = 1$, or $|\cR(\arae)| = 0$ respectively.
\end{ARATOSUM}

\begin{proof}
Let $\cR$ be binary relational schema. For each $R\in \cR$ we associate a matrix variable 
$V_R$ such that, if $R$ is a binary relational signature, then $V_R$ represents a (square) matrix, 
if $R$ is unary, then $V_R$ represents a vector and if $|R|=0$ then $V_R$ represents a constant. Formally, 
fix a symbol $\arae \in \DD \setminus \{1\}$. Let $\text{Mat}(\cR)$ denote the \lang \ schema
$(\Mnam_\cR,\size_\cR)$ such that $\Mnam_\cR = \{ V_R \mid R \in \cR\}$ and $\size_\cR(V_R) = (\alpha, \alpha)$ 
whenever $|R| = 2$, $\size_\cR(V_R) = (\alpha, 1)$ whenever $|R|=1$ and $\size_\cR(V_R) = (1, 1)$ whenever $|R|=0$. 
Let $\cJ$ be the $K$-instance of $\cR$ and suppose that $\adom(\cJ) = \{d_1, \ldots, d_n\}$ is 
the active domain (with arbitrary order) of $\cJ$. 
Define the matrix instance $\text{Mat}(\cJ) = (\dom_\cJ,\conc_\cJ)$ such 
that $\dom_\cJ(\arae) = n$, $\conc_\cJ(V_R)_{i,j} = R^{\cJ}((d_i, d_j))$ whenever $|R|=2$, $\conc_\cJ(V_R)_{i} = R^{\cJ}((d_i))$ 
whenever $|R|=1$, 
\floris{This case relates to nullary relations. What does $R^{\cJ}$ mean?}
and $\conc_\cJ(V_R)_{1,1} = R^{\cJ}$ whenever $|R|=1$. 
Note that we consider the active domain of the whole $K$-instance.
The proof is by induction on the structure of expressions.

\floris{We need to introduce the vector variables $V_A$ for attributes $A$
explicitly in the induction?!}

\begin{itemize}
  \item If $\arae = R$ then $\Psi (\arae):=V_R$. Note that if $|R|=2$ then 
    $$\sem{\Psi(\arae)}{\text{Mat}(\cJ)}_{i,j}=\conc(V_R)_{i,j} = R^{\cJ}((d_i, d_j))=\ssem{\arae}{\cJ}((d_i, d_j)).$$ 
    If $|R|=1$ then 
    $$\sem{\Psi(\arae)}{\text{Mat}(\cJ)}_{i,j}=\conc(V_R)_{i,1} = R^{\cJ}((d_i))=\ssem{\arae}{\cJ}((d_i)).$$
    And if $|R|=0$ then 
	\floris{Not sure about this:}
    $\sem{\Psi(\arae)}{\text{Mat}(\cJ)}_{i,j}=\conc(V_R)_{1,1} = R^{\cJ}=\ssem{\arae}{\cJ}$.
  \item If $\arae=\arae_1\cup\arae_2$ then $\Psi(\arae):=\Psi(\arae_1)+\Psi(\arae_2)$.
  \item If $\arae = \pi_{Y}(\arae_1)$ for $Y\subseteq R(\arae_1)$. Let $\cV = \lbrace V_A:A\in R(\arae_1)\setminus Y \rbrace$. 
    Then
    $$
    \Psi(\arae) = \sum_{V\in\cV} \Psi(\arae_1) = \ssum V_{A_1}. \ssum \cdots \ssum V_{A_{l}}. \Psi(\arae_1).
    $$
\floris{What is $V_{\mathsf{Eq}_Y}$??}
  \item If $\arae = \sigma_Y(\arae_1)$ with $Y\subseteq R(\arae_1)$ then 
    $\Psi(\arae):=f_{\kprod}\left( \Psi(\arae_1), V_{\mathsf{Eq}_Y} \right)$.
  \item If $\arae = \rho_{X\rightarrow Y}(\arae_1)$ then $\Psi (\arae):= \Psi (\arae_1)\left[ V_A\gets V_B, A\in X, B\in Y, A \mapsto B \right]$.
  \item If $\arae = \arae_1\Join \arae_2$, where $R(\arae)=R(\arae_1)=R(\arae_2)$ then 
    $\Psi (\arae):=f_{\kprod}\left( \Psi (\arae_1), \Psi (\arae_2) \right)$
\end{itemize}

Note that for function application we only allow $\kprod$, 
which we can assume to have because we can do scalar multiplication and $\ssum$ (see section \ref{app:simp}).

\end{proof}
