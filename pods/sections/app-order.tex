We detail how order information on canonical vectors can be obtained in \langfor.
We provide explicit expressions for the operators mentioned in Section~\ref{sec:formatlang}
and furthermore, we also define expressions for operators that will be used in our proofs.
% The expression in this section are intended to be used over canonical vectors.

To begin with, we can easily obtain the last canonical vector using the expression 
$$
e_{\mathsf{max}} := \ffor{v}{X}{v}.
$$ 
In other words, we simply overwrite $X$ with the current canonical vector in each iteration.
Hence, at the end, $X$ is assigned to the last canonical vector.

%
% The fundamental property of the iteration that we use here is that the result variable $X$ initiated with the null matrix.
As already mentioned in the main body of the paper,
to define an order relation for canonical vectors, we notice that the following matrix:
\[
S_{\leq} = \begin{bmatrix}
    1 & 1 & \cdots &  1 \\
    0 & \ddots & \ddots & \vdots \\
    \hdotsfor{3} & 1 \\
    0 & \cdots & \cdots & 1 
\end{bmatrix}.
\]
has the property that for two canonical vectors $b_i$ and $b_j$ of the same dimension, 
$$b_i^T\cdot S_{\leq} \cdot b_j=\begin{cases}1 & \text{if $i\leq j$}\\
0 &\text{otherwise}.
\end{cases}
$$
We observe that $S_{\leq}$ can be expressed in \langfor as 
follows:
$$
S_{\leq}:=\ffor{v}{X}{X + \bigl((X\cdot e_{\mathsf{max}}) + v \bigr)\cdot v^T + v\cdot e^T_{\mathsf{max}}},
$$
where $e_{\mathsf{max}}$ is as defined above. 
The intuition behind this expression is that by using the last canonical vector $b_n$, as returned by $e_{\mathsf{max}}$, we have access to the last column of $X$ (via the product $X\cdot e_{\mathsf{max}}$). We use this column such that after the $i$-th iteration, this column contains the $i$-th column of $S_{\leq}$. This is done by incrementing $X$ with $v\cdot e_{\mathsf{max}}^T$.
To construct $S_{\leq}$, in the $i$-th iteration we further increment $X$ with 
(i)~the current last column in $X$ (via $X\cdot e_{\mathsf{max}}\cdot v^T$) which holds
the $(i-1)$-th column of $S_{\leq}$; and (ii)~the current canonical vector (via $v\cdot v^T$). Hence, after iteration $i$, $X$ contains the first $i$ columns of $S_{\leq}$ and holds the $i$th column of $S_{\leq}$ in its last column. It is now readily verified that $X=S_{\leq}$ after the $n$th iteration.
%
%
% The intuition behind this expression is that by using the last canonical vector $e_{\mathsf{max}}$, we have access to the final column of $X$
% (via the product $X\cdot e_{\mathsf{max}}$), to which we add the current canonical vector $v$, thus
% constructing $S_{\leq}$ by filling it column by column.

By defining 
$$
\mathsf{succ}(u,v) := u^T\cdot S_{\leq} \cdot v,
$$
we obtain an order relation that allows us to discern whether one canonical vector comes before 
the other in the order given by $S_{\leq}$. If we want a strict order, we can just use the matrix
$S_< := S_{\leq} - e_{\mathsf{Id}}$, where $e_{\mathsf{Id}}$ is an expression in \langfor which returns the identity matrix (of appropriate dimension). Given this, we define
$$\mathsf{succ}^+(u,v) := u^T\cdot S_{<} \cdot v.$$
from which we can also derive 
$$
\mathsf{max}(u):=u^T\cdot e_{\mathsf{max}}.
$$
which is an expression that returns the last canonical vector.

Interestingly, we can also define the \textit{previous} relation between canonical vectors. 
For this, we require the following matrix:
\[
\mathsf{Prev} = \begin{bmatrix}
    0 & 1 & \cdots &  0 \\
    0 & \ddots & \ddots & \vdots \\
    \hdotsfor{3} & 1 \\
    0 & \cdots & \cdots & 0
\end{bmatrix},
\]
Using this matrix, we have that for a canonical vector $b_i$:
\[
\mathsf{Prev}\cdot b_i=\begin{cases}
               b_{i-1}, \text{ if } i > 1. \\
              \mathbf{0}, \text{ if } i = 1.
            \end{cases}
\]
where $\mathbf{0}$ is a vector of zeros of the same type as $b_i$. Notice also that $\ones(u)^T\cdot \mathsf{Prev} \cdot u$ is equal to zero, for a canonical vector $u$, if and only if $u = b_1$ is the first canonical vector, and zero otherwise.
Therefore the expression $\mathsf{min}(u)$ is defined as $$\mathsf{min}(u) := 1 - \ones(u)^T\cdot \mathsf{Prev} \cdot u,$$ and, when evaluated over canonical vectors, will result in $1$ if and only if $u=b_1$ is the first canonical vector.
To define the first canonical vector in the order given by \texttt{for}, we can then write:
$$e_{\mathsf{min}} := \ffor{v}{X}{X + \mathsf{min}(v)\times v},$$
Finally, we show that $\mathsf{Prev}$ can be defined using the following \langfor expression:
$$e_{\mathsf{Prev}}:= \ffor{v}{X}{X + \bigl((1 - \mathsf{max}(v))\times v\cdot e_{\mathsf{max}}^T - (X\cdot e_{\mathsf{max}})\cdot e_{\mathsf{max}}^T + (X\cdot e_{\mathsf{max}})\cdot v^T\bigr)}.$$
Here, $X$ is initialized as $\mathbf{0}$ and thus in the first iteration we put
 $b_1$ in the last column of $X$ (note that $X\cdot e_{\mathsf{max}}$ is also zero in the first iteration). Next, in iteration two, we add a matrix that has the stored vector $X\cdot e_{\mathsf{max}}$ (the previous canonical vector) in the column indicated by $v$ (the current canonical vector) and $v-X\cdot e_{\mathsf{max}}$ in the last column, to replace the vector stored. As a consequence, $b_2$ is now stored in the last column. In the last iteration, we have $b_{n-1}$ already in the last column, so no further update of $X$ is required.
 
To get the \textit{next} relation we simply do $e_{\mathsf{Next}} = e_{\mathsf{Prev}}^T$. We have that for a canonical vector $b_i$:
\[
{\mathsf{Next}}\cdot b_i=\begin{cases}
               b_{i+1}, \text{ if } i < n. \\
              \mathbf{0}, \text{ if } i = n.
            \end{cases}
\]
In this way, we also can obtain the following operators for a canonical vector $v$: 
$$\mathsf{prev}(v):=e_{\mathsf{Prev}}\cdot v.$$
$$\mathsf{next}(v):=e_{\mathsf{Next}}\cdot v.$$
More generally, we define 
\begin{align*}
    e_{\mathsf{getPrevMatrix}}(v)&:=\sprod w.  \mathsf{succ}(w,v)\times e_{\mathsf{Prev}} + (1 - \mathsf{succ}(w,v))\times e_{\mathsf{Id}}\\
    e_{\mathsf{getNextMatrix}}(v)&:=\sprod w. \mathsf{succ}(w,v)\times e_{\mathsf{Next}} + (1 - \mathsf{succ}(w,v))\times e_{\mathsf{Id}}
\end{align*}
expressions that, when $v$ is interpreted as canonical vector $b_i$, output $\mathsf{Prev}^i$ and $\mathsf{Next}^i$ respectively.
Note that
\[
\mathsf{Prev}^j\cdot b_i=\begin{cases}
               b_{i-j}, \text{ if } i > j. \\
              \mathbf{0}, \text{ if } i \leq j.
            \end{cases}
\]
and
\[
\mathsf{Next}^j\cdot b_i=\begin{cases}
               b_{i+j}, \text{ if } i + j \leq n. \\
              \mathbf{0}, \text{ if } i + j > n.
            \end{cases}
\]
Finally, define
$$
e_{\mathsf{min}+i}:=\underbrace{e_{\mathsf{getNextMatrix}}(\ldots e_{\mathsf{getNextMatrix}}}_{i \text{ times}}(e_{\mathsf{min}}))
$$
and
$$
e_{\mathsf{max}-i}:=\underbrace{e_{\mathsf{getPrevMatrix}}(\ldots e_{\mathsf{getPrevMatrix}}}_{i \text{ times}}(e_{\mathsf{max}}))
$$
We note that some these expressions were already used in Section~\ref{app:def}.