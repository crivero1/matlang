%!TEX root = ../main.tex
%!TeX spellcheck = en_US

We start by recalling the matrix query language \lang, introduced in \cite{matlang-journal}, which serves as our starting point.

\smallskip
\noindent
\textbf{Syntax.}\,  Let $\Mvar = \{V_1, V_2, \ldots\}$ be a countable infinite set of \textit{matrix variables} and $\Fun=\bigcup_{k>1}\Fun_k$ with $\Fun_k$ a set of \textit{functions} of the  form $f:\RR^k \to \RR$, where $\RR$ denotes the set of real numbers. The syntax of $\lang$ expressions is defined by the following grammar\footnote{The original syntax also permits the operator $\llet$, which replaces every occurrence of $V$ in $e_2$ with the value of $e_1$. Since this is just syntactic sugar, we omit this operator. We also explicitly include matrix addition and scalar multiplication, although these can be simulated by pointwise function applications. Finally, we use transposition instead of conjugate transposition since we work with matrices over $\RR$.}:


\begin{tabular}{lcll}
$e$ & $::=$ & $V\in \Mvar$ & (matrix variable)\\
 & $|$ & $e^T$ & (transpose)\\ 
 & $|$ & $\ones(e)$ & (one-vector)\\ 
 & $|$ & $\diag(e)$ & (diagonalization of a vector)\\  
 & $|$ & $e_1 \cdot e_2$ & (matrix multiplication)\\   
 & $|$ & $e_1 + e_2$ & (matrix addition)\\   
 & $|$ & $e_1\times e_2$ & (scalar multiplication)\\
 & $|$ & $f(e_1,\ldots ,e_k)$ & (pointwise application of $f\in\Fun_k$).    
\end{tabular}
\vspace{1ex}

$\lang$ is parametrized by a collection of functions $\Fun$ but in the remainder of the paper we only make this dependence explicit, and write $\langf{\Fun}$, for some set $\Fun$ of functions, when these functions are crucial for some results to hold. When we simply write \lang, we mean that any function can be used, including not using any function at all.

When working with \lang\ expressions we assume that these expressions are given by their parse tree. Alternatively, we can assume the expression to be fully parenthesized. This allows us to disambiguate expressions of the form $e_1+ e_2\cdot e_3$, since the parse tree (or the parentheses) will tell us whether we are working with $(e_1) + (e_2\cdot e_3)$, or with $(e_1+e_2)\cdot e_3$. For the sake of brevity, throughout the paper we will also sometimes revert to usual operator precedence syntax, assuming matrix and scalar multiplication to have precedence over the sum (the precedence of the other operators is always unambiguous). This means that when we write $e_1 + e_2\cdot e_3$, we are actually referring to $(e_1) + (e_2\cdot e_3)$.

\smallskip
\noindent
\textbf{Schemas and typing.}\,
To define the semantics of \lang\ expressions we need a notion of schema and well-typedness of expressions. A \lang\ \textit{schema} $\Sch$ is a pair $\Sch=(\Mnam,\size)$, where $\Mnam\subset \Mvar$ is a finite set of matrix variables and $\size: \Mnam \mapsto \DD\times \DD$ is a function that maps each matrix variable in $\Mnam$ to a pair of \textit{size symbols}. The $\size$ function helps us determine whether certain matrix operations, such as matrix multiplication, can be performed for matrices adhering to a schema. 
We denote size symbols by Greek letters $\alpha,\beta,\gamma$. We also assume that $1\in \DD$. 
To help us determine whether a \lang\ expression can always be evaluated, we define the \textit{type} of an expression $e$, with respect to a schema $\Sch$, denoted by $\ttype(e)$, inductively as follows:
\begin{itemize}
\item $\ttype(V):= \size(V)$, for a matrix variable $V\in\Mnam$;
\item $\ttype(e^T):= (\beta,\alpha)$ if $\ttype(e)=(\alpha,\beta)$;
\item $\ttype(\ones(e)):= (\alpha,1)$ if $\ttype(e)=(\alpha,\beta)$;
\item $\ttype(\diag(e)):= (\alpha,\alpha)$, if $\ttype(e)=(\alpha,1)$;
\item $\ttype(e_1 \cdot e_2):= (\alpha,\gamma)$ if  $\ttype(e_1)=(\alpha,\beta)$ and $\ttype(e_2)=(\beta,\gamma)$;
\item $\ttype(e_1 + e_2):=(\alpha,\beta)$ if $\ttype(e_1)=\ttype(e_2)=(\alpha,\beta)$;
\item $\ttype(e_1\times e_2):=(\alpha,\beta)$ if $\ttype(e_1)=(1,1)$ and $\ttype(e_2)=(\alpha,\beta)$; and finally,
\item $\ttype(f(e_1,\ldots ,e_k)):= (\alpha,\beta)$ if $\ttype(e_1) = \cdots = \ttype(e_k) = (\alpha,\beta)$ and $f\in\Fun_k$.
\end{itemize}
We call an expression \textit{well-typed} according to the schema $\Sch$ if it has a defined type. 
A well-typed expression can be evaluated regardless of the actual sizes of the matrices assigned to matrix variables, as we describe next.

We will usually work with matrices with different sizes (i.e., ``rectangular matrices''), but sometimes it will be helpful to restrict to ``square matrices''. For this we say that $\Sch$ is a schema over square matrices if there exists $\alpha$ such that each matrix variable has type $(\alpha,\alpha),(\alpha,1),(1,\alpha)$, or $(1,1)$. This restriction will be useful when comparing the expressiveness of our proposed matrix query language with other formalisms (e.g., see Section~\ref{sec:restrict}). 

\smallskip
\noindent
\textbf{Semantics.}\, We use $\mtr{\RR}$ to denote the set of all real matrices and for 
$A\in\mtr{\RR}$, $\dim(A)\in\NN^2$ denotes its dimensions.
A (\lang) \textit{instance} $\I$ over a schema $\Sch=(\Mnam,\size)$ is a pair $\I = (\dom,\conc)$, where $\dom : \DD \mapsto \NN$ assigns a value to each size symbol (and thus in turn  dimensions to each matrix variable), and $\conc : \Mnam \mapsto \mtr{\RR}$ assigns a concrete matrix to each matrix variable $V\in \Mnam$, such that $\dim(\conc(V)) = \dom(\alpha)\times \dom(\beta)$ if $\size(V) = (\alpha,\beta)$. That is, an instance tells us the dimensions of each matrix variable, and also the concrete matrices assigned to the variable names in $\Mnam$. Notice that having the function $\dom$ allows us to specify when two matrix variables have to be assigned matrices of the same dimension. We assume that $\dom(1) = 1$, for every instance $\I$. If $e$ is a well-typed expression according to $\Sch$, then we denote by $\sem{e}{\I}$ the matrix obtained by evaluating $e$ over $\I$, and define it as follows\footnote{We also use easily definable operations as matrix minus ($-$) and assume the existence of some constants (such as $[1]$) in $\Mnam$. Additionally, several times we omit the $\times$ operator to weigh expressions.}:
\begin{itemize}
\item $\sem{V}{\I} := \conc(V)$, for $V\in \Mnam$;
\item $\sem{e^T}{\I} := \sem{e}{\I}^T$, where $A^T$ is the transpose of matrix $A$;
\item $\sem{\ones(e)}{\I}$ is a $n\times 1$ vector with $1$ as all of its entries, where $\dim(\sem{e}{\I})=(n,m)$;
\item $\sem{\diag(e)}{\I}$ is a diagonal matrix with the vector $\sem{e}{\I}$ on its main diagonal, and zero in every other position;
\item $\sem{e_1\cdot e_2}{\I} := \sem{e_1}{\I} \cdot \sem{e_2}{\I}$;
\item $\sem{e_1+ e_2}{\I} := \sem{e_1}{\I} + \sem{e_2}{\I}$;
\item $\sem{e_1\times e_2}{\I} := a\times \sem{e_2}{\I}$ with $\sem{e_1}{\I}=[a]$; and finally,
\item $\sem{f(e_1,\ldots ,e_k)}{\I}$ is a matrix $A$ of the same size as $\sem{e_1}{\I}$ and where entry $A_{ij}$ has the value $f(\sem{e_1}{\I}_{ij},\ldots ,\sem{e_k}{\I}_{ij})$.
\end{itemize}

\noindent
Although \lang\ forms a solid basis for a matrix query language, it is limited in expressive power. Indeed, \lang\ is subsumed by first order logic with aggregates that uses only three variables \cite{matlang-journal}. 
Hence, no \lang\ expression exists that can compute the transitive closure of a graph (represented by its adjacency matrix) or can compute the inverse of a matrix. Rather than extending \lang\ with specific linear algebra operators, such as matrix inversion, we  
next introduce a limited form of recursion in \lang.
