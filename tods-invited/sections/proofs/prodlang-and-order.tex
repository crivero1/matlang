% !TeX spellcheck = en_US
%!TEX root = ../main.tex

We now verify that \langmprod, extended with order and $f_{>0}$, 
can perform matrix inversion and compute the determinant. To this aim, we verify that all order
predicates in Section \ref{sec:formatlang:design} can be derived using $\ssum$, $\sprod$, $f_{>0}$ and 
$e_{S_{<}}$. Given this, it suffices to observe that Csanky's algorithm, as shown in Section~\ref{sec:queries:inverse}, only relies on expressions using $\ssum$ and $\sprod$ and order information on canonical vectors and $f_/$.
As consequence, our fragment can perform matrix inversion and compute the determinant.


It remains to show that if we have $e_{S_{<}}$, using $\ssum$ and $\sprod$ and $f_{>0}$ we can
can define all order predicates from Section~\ref{sec:formatlang:design}. We note that due to the restricted for-loops
in $\ssum$ and $\sprod$, we do not have access to the intermediate
result in the iterations and as such, it is unclear whether order information can be computed. This is why
we assume access to $e_{S_<}$.

We first remark that if we have $e_{S_{<}}$, we can also obtain
 $e_{S_{\leq}}$ by adding $e_{\mathsf{Id}}$. Hence,
we can compute $\mathsf{succ}$ and $\mathsf{succ}^+$ as well. Furthermore, 
\begin{align*}
  e_{\mathsf{min}}&:=\ssum v. \left[ \sprod w. \mathsf{succ}(w,v)\right] \times v. \\
  e_{\mathsf{max}}&:=\ssum v. \left[ \sprod w. \left( 1-\mathsf{succ}(w,v) \right) \right] \times v.
\end{align*}
Both expressions are only using $\ssum$ and $\sprod$ and $\mathsf{succ}$, so are in our fragment.
Furthermore, if we have $f_{>0}$ then we can define
$$
e_{\mathsf{Pred}}:= e_{S_{<}}- f_{>0}(e_{S_{<}}^2)
$$
Also, recall that  $e_{\mathsf{Next}}:=e_{\mathsf{Pred}}^T$. As a consequence, 
we can now define $\mathsf{prev}(v)$ and $\mathsf{next}(v)$ as in \ref{sec:formatlang:design}. Similarly,
it is readily verified that also $e_{\mathsf{getPrevMatrix}}(V)$,
$e_{\mathsf{getNextMatrix}}(V)$, $e_{\mathsf{min}+i}$ and $e_{\mathsf{max}+i}$ can be expressed
in our fragment.
