%!TEX root = ../../main.tex
\subsection{Proof of Proposition~\ref{prop:transducer2}}
A crucial ingredient for translating arithmetic circuits in \langfor is that \langfor
can express a number of circuit-related functions, as described in Section~\ref{subsubsec:simulate}.
We here show a more general result which, roughly speaking, says that any 
polynomial time Turing machine, working within linear space and producing linear space output, also
referred to as \textit{linear space machines}, 
can be simulated in \langfor. The circuit-related functions used in the evaluation algorithm in Section~\ref{subsec:actoformatlang}
are all of this form when considering arithmetic circuit families of logarithmic depth.

\subsubsection{Linear space machines}\label{subsubsec:linearspace}
We start by formally defining linear space machines. We consider  deterministic Turing Machines (TMs) $T$ 
consisting of $\ell$ read-only input tapes, denoted by $R_1,\ldots,R_\ell$, and
$s$ read-write working tapes, denote by $W_1,\ldots,W_s$. The TM $T$ has a set 
$Q$ of $m$ states, denoted by $q_1,\ldots,q_m$. We assume that $q_1$ is the initial state and $q_m$ is the accepting state.
The input and tape alphabet are $\Sigma=\{0,1\}$ and $\Gamma=\Sigma\cup\{\rhd,\lhd\}$, respectively. The special 
symbols $\rhd$ and $\lhd$ denote the beginning and end, respectively, of each of the tapes.
The transition function $\Delta$ of $T$ is defined as usual, i.e. 
$\Delta:Q\times \Gamma^{\ell+s} \to Q\times \Gamma^{s}\times \{\leftarrow,\sqcup,\rightarrow\}^{\ell+s}$ 
such that
$$
\Delta(q,(a_1,\ldots,a_{\ell},b_1, \ldots, b_s))=\bigl(q',(b_1',\ldots, b'_s),(\mathsf{d}_1,\ldots,\mathsf{d}_{\ell+s})\bigr)
$$
with $\mathsf{d}_i\in \{\leftarrow,\sqcup,\rightarrow\}$. The semantics is that when $T$ is in state $q$ and the $\ell+s$ 
heads on the tapes read symbols $a_1,\ldots,a_{\ell},b_1, \ldots, b_s$, respectively, then $T$ transitions to state $q'$,
writes $b_1', \ldots, b_s'$ on the work tapes at the position to which the work
tape's heads points at, and finally moves the heads on the tapes according 
$\mathsf{d}_1,\ldots,\mathsf{d}_{\ell+s}$. More specifically, $\leftarrow$  indicates a move to the left, 
$\rightarrow$ a move to the right, and finally, $\sqcup$ indicates that the head does not move.

We assume that $\Delta$ is defined such that it ensures that on none of the tapes, heads can move beyond 
the leftmost marker $\rhd$ and the rightmost marker $\lhd$. Furthermore, the tapes $R_1,\ldots,R_\ell$ are treated as read-only and $\Delta$ does not change the 
occurrences of $\rhd$ or $\lhd$ on the work tapes.

A \textit{configuration of} $T$ is defined in the usual way. That is, a configuration of a tape is of the form
$\rhd w_1qw_2\lhd$ with $w_1,w_2\in\Sigma^*$ and represents that the current tape content is 
$\rhd w_1w_2\lhd$, $T$ is in state $q$ and the head is positioned on the first symbol of $w_2$. 
A configuration of $T$ consists of configurations for all tapes. Given two configurations 
$c_1$ and $c_2$, we say that $c_1$ \textit{yields} $c_2$ if $c_2$ is the result of applying the transition 
function $\Delta$ of $T$ based on the information in $c_1$. As usual, we close this ``yields'' relation 
transitively.

Given $n > 0$ and $\ell$ input words $w_1,\ldots,w_\ell\in\Sigma^n$, we assume that the initial configuration of 
$T$ is given by
$$
\bigl(q_1 \!\rhd\! w_1\lhd,q_1\!\rhd\! w_2\lhd,\ldots, q_1\!\rhd\! w_\ell\lhd, q_1\!\rhd\! 0^n \lhd, \overset{\text{$s$-times}}{\ldots}, q_1\!\rhd\! 0^n \lhd \bigr)
$$ and an 
accepting configuration is assumed to be any configuration whose current state is $q_m$. We say that $T$ \textit{computes the function} $f=\bigcup_{n\geq 0} f_n:(\Sigma^n)^\ell\to\Sigma^n$ if for every
$w_1,\ldots,w_\ell\in\Sigma^*$, each of size $n$, the initial configuration yields (transitively) an accepting 
configuration such that the configuration on the first working tape $W_1$ is of the form $\rhd u_1 q_m u_2\lhd$ such that $f_n(w_1,\ldots,w_\ell) = u_1u_2$, namely, the output is the content on the first work tape. 
We assume that once $T$ reaches an accepting configuration it stays indefinitely in that configuration 
(i.e. it loops).
Finally, we say that $T$ runs in $\mathcal{O}(n^{k})$ time if for every $n$ and every $\ell$-input words of size $n$, it reaches an accepting configuration in less than $c\cdot n^{k}$ steps for some fix constant $c$.

%We further assume that $T$ only reaches an accepting configuration when all its input
%words have the same size. Furthermore, when all inputs have the same size, we assume that $T$ will reach an accepting 
%configuration. 


%We say that $T$ is a \textit{linear space machine} when it reaches an accepting configuration 
%on inputs of size $n$ by using $\mathcal{O}(n)$ space on its work tape and additionally needs 
%polynomially many steps to do so.

 A \textit{linear input-output function} is a function of the form 
$f=\bigcup_{n\geq 0} f_n:(\Sigma^n)^\ell\to\Sigma^n$. In other words, for every $\ell$ words of the same 
size $n$, $f$ returns a word of size $n$. We say that a linear input-output function is a 
\textit{linear space input-output function} if
there exists a linear space machine  $T$ that for every $n\geq 0$, on input $w_1,\ldots,w_\ell\in\Sigma^n$ 
the TM $T$ has
$f_n(w_1,\ldots,w_\ell)$ on its first work tape when (necessarily) reaching an accepting configuration.
We have similar notions in place for functions of the form $f=\bigcup_{n\geq 0} f_n:(\Sigma^n)^\ell\to\Sigma$.

\subsubsection{Proof of Proposition~\ref{prop:transducer2}}
We are now ready to precisely formulate  Proposition~\ref{prop:transducer2}.

\begin{proposition}
	For $\Sigma = \{0,1\}$ let $f=\bigcup_{n\geq 0}f_n:(\Sigma^n)^\ell\to \Sigma^n$ be a linear space input-ouput function 
	computed by a linear space  machine $T$ with $m$ states, $\ell$ input tapes, $s$ work tapes, and runs in $\mathcal{O}(n^{k-1})$ time on inputs of size $n$. 
	Then there exists (i)~a \langfor 
	schema $\mathcal{S}=(\mathcal{M},\textsf{size})$ where $\mathcal{M}$ contains matrix 
	variables
	% \footnote{We also need a finite number of auxiliary variables, these will be specified
	% in the proof.}
	$Q_1,\ldots,Q_m$, $R_1,\ldots,R_\ell$, $H_1,\ldots,H_\ell$, $W_1,\ldots,W_s$, $H_{W_1},\ldots,H_{W_s}$, $v_1,\ldots,v_{k}$  
	with $\mathsf{size}(V)=\alpha\times 1$ for all $V\in\mathcal{M}$; and (ii)~a \langfor 
	expression $e_f$ over $\mathcal{S}$ such that for the instance 
	$\I=(\mathcal{D},\textsf{mat})$ over $\mathcal{S}$ with $\mathcal{D}(\alpha)=n$ and 
	$\mathsf{mat}(R_i)=\mathsf{vec}(w_i)\in \mathbb{R}^n$, for $i\in[\ell]$, such that $\mathsf{vec}(w_i)$ is the $n\times 1$-vector 
	encoding the word $w_i\in\Sigma^n$, and all other matrix variables instantiated with the zero vector in $\mathbb{R}^n$,  we have that  $\mathsf{mat}(W_1)=\mathsf{vec}(f_n(w_1,\ldots,w_\ell))\in\mathbb{R}^n$ 
	after evaluating $e_f$ on~$\I$. We have a similar result for functions $f=\bigcup_{n\geq 0}f_n:(\Sigma^n)^\ell\to \Sigma$.
\end{proposition}

% For this proof only, we will denote the canonical vectors as
% $\mathbf{e}_1, \ldots, \mathbf{e}_n$, since $b$ will be used to represent a value on a position of a tape.
We only consider functions $f=\bigcup_{n\geq 0}f_n:(\Sigma^n)^\ell\to \Sigma^n$. The proof only requires
minimal modifications for functions  $f=\bigcup_{n\geq 0}f_n:(\Sigma^n)^\ell\to \Sigma$.

\begin{proof}
	The expression $e_f$ we construct will simulate the TM $T$. To have some more control on the time consumption of $T$, let us first assume that $n$ is large enough, say larger than $n\geq n_0$, 
	such that $T$ runs in $cn^{k-1}\leq n^{k}$ time for a constant $c$. We deal with $n<n_0$ separately.
 We split up the construction of $e_f$ in three
	parts: \textbf{(a)}~an expression $e_f^{\geq n_0}$, for dealing with $n\geq n_0$; \textbf{(b)}~an expression
	$e_f^{<n_0}$, for dealing with $n<n_0$; and finally \textbf{(c)}~the expression $e_f$ combining both these expressions. 

\medskip
\noindent
\underline{\textbf{(a)} Large $n$, i.e. $n\geq n_0$.}
    Let us first define $e_f^{\geq n_0}$. To simulate $T$ we need to encode states, tapes and head positions. For this, we assume that $\mathcal{M}$ contains matrix variables:
    $$
    Q_1,\ldots,Q_m, R_1,\ldots,R_\ell, H_1,\ldots,H_\ell, W_1,\ldots,W_s, H_{W_1},\ldots,H_{W_s}, v_1,\ldots,v_{k}
    $$
    which will store the states, tapes, and head positions. More specifically, the variables 
    $R_1,\ldots,R_\ell$ will hold the input vectors, and $W_1,\ldots,W_s$ will hold the contents of the work
    tapes.
	The vectors corresponding to the work tapes are initially set to the zero vector. 
    The vector for the input tape $R_i$ is set to $\mathsf{vec}(w_i)$, for $i\in[\ell]$.

\smallskip
    With each tape we associate a matrix variable encoding the position of the head. More specifically, 
    $H_1,\ldots,H_\ell$ correspond to the input tape heads, and
    $H_{W_1},\ldots, H_{W_s}$ to the work tape heads.
    All these vectors will be zero except 
    for a single position, indicating the positions in the corresponding tapes the heads point to. 
    For those positions $j$, $1<j<n$, the head vectors will carry value $1$.  When $j=1$ or $n$ and when 
    it concerns positions for the input tape, the head vectors can carry value $1$ or $2$. We need to treat 
    these border cases separately
    because we only have $n$ positions available to store the input words, whereas the actual input tapes 
    consist of $n+2$ symbols because of $\rhd$ and $\lhd$. So when, for example, $H_1$ has a $1$ in its first
    entry, we interpret it as if the head is pointing to the first symbol of the input word $w_1$. When $H_1$
    has a $2$ in its first position, we interpret it as if the head pointing to $\rhd$. The end marker $\lhd$ is
    dealt with in the same way, by using value $1$ or $2$ in the last position of $H_1$. We use this encoding
    for all input tapes, and also for the work tapes.
    

    To encode the states, we use the variables $Q_1,\ldots,Q_m$. We will ensure that when $T$ is in state 
    $q_i$ when
    $\mathsf{mat}(Q_i)=[1,0,\ldots,0]^T\in\mathbb{R}^n$, otherwise $\mathsf{mat}(Q_i)$ is the zero 
    vector in $\mathbb{R}^n$.	

    Finally, the variables $v_1,\ldots,v_{k}$ represent $k$ canonical vectors  which are used to iterate 
    in for-loops. By iterating over them, we can perform $n^{k}$ iterations, 
    which suffices for simulating the $\mathcal{O}(n^{k-1})$ steps used by $T$ to reach an accepting configuration.
    
    With these matrix variables in place, we start by defining $e_f^{\geq n_0}$. As a final remark, we note that during the evaluation of the formula, we need to force that all these matrix variables are either $\{0,1\}$ vectors (e.g., for variables encoding a tape), canonical vectors (e.g., for variables encoding a head tape), or vectors of the form $(2,0,\ldots,0)$ or $(0, \ldots, 0, 2)$ (e.g., for variables encoding $\rhd$ and $\lhd$ markers).
    We will force these restrictions on every formula below. 
    % We explain the expression
%     $e_f^{\geq N}$ first.

    We will use several expressions related to ordering information of canonical vectors, as explained in Section \ref{sec:formatlang:design}.
	In particular, we use $\mathsf{min}$ and $\mathsf{max}$, to test whether their input vector is the first, respectively, last canonical
	vector, expressions $e_{\mathsf{min}}$ and $e_{\mathsf{max}}$, which return the first and last canonical vector, respectively, and expressions $e_{\mathsf{Next}}$
	and $e_{\mathsf{Prev}}$, which return matrices that, when multiplied with a canonical vector, return the next, respectively previous, canonical vector.
	  % In our  expressions we use subexpressions which we defined before in . These subexpression
	  %     require some auxiliary variables, as detailed below. As a consequence, $e_f$ will be an expressions
	  %     defined over an extended schema $\mathcal{S}'$. Hence, the instance $\I$ in the statement of the Proposition
	  %     is  an instance $\I'$ of $\mathcal{S}'$ which
	  %     coincides with $\I$ on $\mathcal{S}$ and in which the auxiliary matrix variables are all instantiated with
	  %     zero vectors or matrices, depending on their size.
	  %
	  %     Now, we specify the finite auxiliary variables involved in the \langfor expression. These arise
	  %     when computing the following \langfor expressions defined
	  %
	  %     \begin{itemize}
	  %         \item $e_{\mathsf{Prev}}(z,Z,z',Z')$, and expression over auxiliary variables $z$, $z'$, $Z$ and $Z'$ with
	  %         $\mathsf{size}(z)=\mathsf{size}(z')=\mathsf{size}(Z)=\alpha\times 1$ and
	  %         $\mathsf{size}(Z')=\alpha\times\alpha$. On input $\I'$ with
	  %         $\mathsf{mat}(z)=\mathsf{mat}(z')=\mathsf{mat}(Z)$ the zero column vector of dimension $n$,
	  %         and $\mathsf{mat}(Z')$ the zero $n\times n$ matrix,
	  %         $\sem{e_{\mathsf{Prev}}}{\I'}$ returns the $n\times n$ matrix $\mathsf{Prev}$ such that
	  %         $$\mathsf{Prev}\cdot \mathbf{e}_i:=\begin{cases}
	  %         \mathbf{e}_{i-1} & \text{if $i>1$}\\
	  %         \mathbf{0} & \text{if $i=1$}.
	  %         \end{cases}
	  %         $$
	  %         % In other words, $e_{\mathsf{Prev}}$ defines a predecessor relation among canonical vectors of dimension $n$.
	  %         \item $e_{\mathsf{Next}}(z,Z,z',Z')$, and expression over auxiliary variables $z$, $z'$, $Z$ and $Z'$
	  %         with $\mathsf{size}(z)=\mathsf{size}(z')=\mathsf{size}(Z)=\alpha\times 1$ and
	  %         $\mathsf{size}(Z')=\alpha\times\alpha$. On input $\I'$ with
	  %         $\mathsf{mat}(z)=\mathsf{mat}(z')=\mathsf{mat}(Z)$ the zero column
	  %         vector of dimension $n$, and $\mathsf{mat}(Z')$ the zero $n\times n$ matrix,
	  %         $\sem{e_{\mathsf{Next}}}{\I'}$ returns the $n\times n$ matrix $\mathsf{Next}$ such that
	  %         $$\mathsf{Next}\cdot \mathbf{e}_i:=\begin{cases}
	  %         \mathbf{e}_{i+1} & \text{if $i<n$}\\
	  %         \mathbf{0} & \text{if $i=n$}.
	  %         \end{cases}
	  %         $$
	  %         \item $\textsf{min}(v,z,Z,z',Z)$ with auxiliary variables $z$, $z'$, $Z$ and $Z'$ as before,
	  %         and $v$ is one of the (vector) variables in $\mathcal{M}$. For an $n\times 1$ vector $\mathbf{v}$,
	  %         on input $\I'[v\gets \mathbf{v}]$	$$\sem{\mathsf{min}}{\I'[v\gets\mathbf{v}]}:=\begin{cases} 1 & \text{if $\mathbf{v}=\mathbf{e}_1$}\\
	  %             0 & \text{otherwise}.
	  %             \end{cases}$$
	  %
	  %         \item $\textsf{max}(v,z,Z,z',Z)$ with auxiliary variables $z$, $z'$, $Z$ and $Z'$ as before, and
	  %         and $v$ is one of the (vector) variables in $\mathcal{M}$. For an $n\times 1$ vector $\mathbf{v}$,
	  %         on input $\I'[v\gets \mathbf{v}]$
	  %
	  %         $$\sem{\mathsf{max}}{\I'[v\gets\mathbf{v}]}:=\begin{cases} 1 & \text{if $\mathbf{v}=\mathbf{e}_{n}$}\\
	  %             0 & \text{otherwise}.
	  %             \end{cases}$$
	  %         \item $e_{\mathsf{min}}(z,Z,z',Z',z'',Z'')$, an expressions with
	  %         auxiliary variables $z$, $z'$, $z''$, $Z$, $Z'$ and $Z''$ with
	  %         $\mathsf{size}(z)=\mathsf{size}(z')=\mathsf{size}(z'')=\mathsf{size}(Z)=\mathsf{size}(Z'')=\alpha\times 1$
	  %         and $\mathsf{size}(Z')=\alpha\times\alpha$. On input $\I'$ with
	  %         matrix variables instantiated with zero vectors (or matrix for $Z'$),
	  %         $\sem{e_{\mathsf{min}}}{\I'}=\mathbf{e}_1$.
	  %         \item $e_{\mathsf{max}}(z,Z,z',Z',z'',Z'')$, an expressions with
	  %         auxiliary variables $z$, $z'$, $z''$, $Z$, $Z'$ and $Z''$ with
	  %         $\mathsf{size}(z)=\mathsf{size}(z')=\mathsf{size}(z'')=\mathsf{size}(Z)=\mathsf{size}(Z'')=\alpha\times 1$
	  %         and $\mathsf{size}(Z')=\alpha\times\alpha$. On input $\I'$ with
	  %         matrix variables instantiated with zero vectors (or matrix for $Z'$),
	  %         $\sem{e_{\mathsf{max}}}{\I'}=\mathbf{e}_n$.
	  %     \end{itemize}
	  %     We thus see that we only need $z,z',z'',Z,Z',Z''$ as auxiliary variables and these can be re-used
	  %     whenever $e_f$ calls these functions. From now one, we omit the auxiliary variables from the description
	  %     of $e_f$.
	  %

Since we want to simulate $T$ we need to be able to check which 
    transitions of $T$ can be applied based on a current configuration. More precisely,
    suppose that we want to check whether $\Delta(q_i,(a_1,\ldots,a_{\ell},b_1,\ldots, b_s))$ is applicable, then we 
    need to check whether $T$ is in state $q_i$, we can do this by checking 
    $\mathsf{min}(Q_i)$, and whether the heads on the tapes read symbols $a_1,\ldots,a_{\ell},b_1, \ldots, b_s$. We 
    check the latter by the following expressions.
    First, we define the expressions 
    $$
    \mathsf{border}_{\rhd}(v) = 1/2 \times (e_{\mathsf{min}}^T \cdot v) \cdot (e_{\mathsf{min}}^T \cdot v - 1) \ \ \text{ and } \ \  \mathsf{border}_{\lhd}(v) = 1/2 \times (e_{\mathsf{max}}^T \cdot v) \cdot (e_{\mathsf{max}}^T \cdot v - 1)
    $$
    which output $1$ if $v$ is equal to $(2,0,\ldots,0)\in\mathbb{R}^n$ or $(0, \ldots, 0, 2)\in\mathbb{R}^n$, respectively, and $0$ if $v$ is any canonical vector.
    Then for the input tape $R_i$ we define
    $$
    \mathsf{test\_inp}^i_b:=\begin{cases}
    \bigl(1-\mathsf{border}_{\rhd}(H_i)\bigr)\cdot \bigl(1-\mathsf{border}_{\lhd}(H_i)\bigr)\cdot(1- R_i^T\cdot H_i) & \text{if $b=0$}\\
    \bigl(1-\mathsf{border}_{\rhd}(H_i)\bigr)\cdot \bigl(1-\mathsf{border}_{\lhd}(H_i)\bigr)\cdot(R_i^T\cdot H_i) & \text{if $b=1$}\\
    \mathsf{border}_{\rhd}(H_i) & \text{if $b=\rhd$}\\
    \mathsf{border}_{\lhd}(H_i) & \text{if $b=\lhd$},
    \end{cases}
    $$
    which returns $1$ if and only if either $b\in\{0,1\}$ is the value in $\mathsf{mat}(R_i)$ at the 
    position encoded by $\mathsf{mat}(H_i)$, or when $b=\rhd$ and $\mathsf{mat}(H_i)$ is the vector 
    $(2,0,\ldots,0)\in\mathbb{R}^n$, or when $b=\lhd$ and $\mathsf{mat}(H_i)$ is the vector 
    $(0,0,\ldots,2)\in\mathbb{R}^n$.
    Similarly, for the work tape $W_i$ and its head tape $H_{W_i}$ we can define an expression $\mathsf{test\_work}^i_b$ like above.
    We then combine all these expressions into a single expression for $q_i\in Q$, 
    $a_1,\ldots,a_\ell,b_1,\ldots, b_s\in\Gamma$:
    $$
    \mathsf{isconf}_{q_i,a_1,\ldots,a_\ell,b_1,\ldots, b_s}:=
    \mathsf{min}(Q_i)\cdot \left(\prod_{j=1}^{\ell} \mathsf{test\_inp}_{a_j}^j\right)
    \cdot\left(\prod_{j=1}^{s} \mathsf{test\_work}_{b_j}^j\right).
    $$
    This expression will return $1$ if and only if the vectors representing the tapes, 
    head positions and states are such that $Q_i$ is the first canonical vector 
    (and thus $T$ is in state $q_i$), the heads point to entries in the tape vectors storing 
    the symbols $a_1,\ldots,a_{\ell}, b_1, \ldots, b_s$ or they point to the first or last positions but have value $2$ (when the symbols are $\rhd$ or $\lhd$). 

    To ensure that at the beginning of the simulation of $T$ by $e_f^{\geq n_0}$ we correctly encode 
    that we are in the initial configuration, we thus need to initialize all vectors 
    $$\mathsf{mat}(H_1),\mathsf{mat}(H_2),\ldots, \mathsf{mat}(H_\ell), \mathsf{mat}(H_{W_1}),\ldots, \mathsf{mat}(H_{W_s})$$
    with the vector $(2,0,0,\ldots,0)\in\mathbb{R}$ since all heads read the symbol $\rhd$. Similarly, 
    we have to initialize $Q_1$ with the first canonical vector since $T$ is in the initial state $q_1$.

    We furthermore need to be able to correctly adjust head positions. We do this by means of the expressions $e_{\mathsf{Prev}}$ and $e_{\mathsf{Next}}$	described earlier.
    A consequence of our encoding is that we need to treat the border cases (corresponding to $\rhd$ and 
    $\lhd$) differently. More specifically, for the input tapes $R_i$ and heads $H_i$ we define 
    $$
    \mathsf{move\_inp}^i_{\mathsf{d}}:=
    \begin{cases}
    \bigl(2\cdot\mathsf{min}(H_i)\bigr)\times H_i &\!\!\!\!\!+ \bigl(1/2\cdot\mathsf{border}_{\lhd}(H_i)\bigr)\times H_i \\
    & \hspace{-4em}+\Bigl(\bigl(1-\mathsf{min}(H_i)\bigr)\cdot\bigl(1-\mathsf{border}_{\lhd}(H_i)\bigr)\Bigr)\times e_{\mathsf{Prev}}\cdot H_i \quad \text{ if $\mathsf{d}=\leftarrow$}\\
    \bigl(2\cdot \mathsf{max}(H_i)\bigr)\times H_i &\!\!\!\!\! + \bigl(1/2\cdot\mathsf{border}_{\rhd}(H_i)\bigr)\times H_i \\
    & \hspace{-4em}+ \Bigl( \bigl(1-\mathsf{max}(H_i)\bigr) \cdot \bigl(1-\mathsf{border}_{\rhd}(H_i)\bigr) \Bigr)\times e_{\mathsf{Next}}\cdot H_i  \quad\!\text{if $\mathsf{d}=\rightarrow$}\\
    H_i & \hspace{17.5em}\!\text{if $\mathsf{d}=\sqcup$}. 
    \end{cases}
    $$
    In other words, we shift to the previous (or next) canonical vector when $\mathsf{d}$ is $\leftarrow$ 
    or $\rightarrow$, respectively, unless we need to move to or from the position that will hold $\rhd$ 
    or $\lhd$. In those cases we readjust $\mathsf{mat}(H_i)$ (which will either $(1,0,\ldots,0)$, $(2,0,\ldots,0)$, 
    $(0,\ldots,0,1)$ or $(0,\ldots,0,2)$) by either dividing or multiplying with $2$. In this way we can 
    correctly infer whether or not the head points to the begin and end markers. For the $i$-th work tape we 
    proceed in a similar way and define $\mathsf{move\_work}^i_{\mathsf{d}}$ by replacing $H_{i}$ with $H_{W_i}$ in the expression above. 
 
    A final ingredient are expressions which update a work tape.
    To define these expression, we need the position and symbol to put on the tape. For the work tape $W_i$:
    $$
    \mathsf{write\_work}_b^i:=\begin{cases}
    W_i  & \text{if $b=\rhd$ or $b=\lhd$}\\
    (1-W_i^T\cdot H_{W_i})\times W_i + (W_i^T\cdot H_{W_i})\times (W_i-H_{W_i}) &\text{if $b=0$}\\
    (1-W_i^T\cdot H_{W_i})\times (W_i+H_{W_i}) + (W_i^T\cdot H_{W_i})\times W_i &\text{if $b=1$},
    \end{cases}
    $$
    We are now finally ready to define $e_f^{\geq n_0}$. Intuitively, we want update all vector variables
	$Q_1,\ldots,\allowbreak Q_m,\allowbreak H_1,\ldots,H_\ell,W_1,\ldots,W_s, H_{W_1},\ldots,H_{W_s}$
	simultaneously using expressions $e_{Q_1},\ldots,e_{Q_m},\allowbreak e_{H_1},\allowbreak\ldots,\allowbreak e_{H_\ell},\allowbreak e_{W_1},\ldots,e_{W_s},e_{H_{W_1}},\ldots,e_{H_{W_s}}$, respectively, and want to perform these updates $n^k$ times,
	which suffices to simulate all steps of the linear space machine. To this aim, we iterate over $k$ vectors $v_1,\ldots, v_k$.
	We succinctly write the expressions  $e_f^{\geq n_0}$ as follows:
    \begin{multline*}
    e_f^{\geq n_0}:= \mathsf{for\,} v_1,\ldots,v_{k},Q_1,\ldots,Q_m,H_1,\ldots,H_\ell,W_1,\ldots,W_s, H_{W_1},\ldots,H_{W_s}. \\
    (e_{Q_1},\ldots,e_{Q_m},e_{H_1},\ldots,e_{H_\ell},e_{W_1},\ldots,e_{W_s},e_{H_{W_1}},\ldots,e_{H_{W_s}}).
    \end{multline*}
	We provide the precise semantics of this expression in Section~\ref{subsubsec:generalloop} by casting it as a \langfor expression.
	Intuitively, each possible assignment of $v_1,\ldots,v_k$ to canonical vectors $b_{i_1}^n,\ldots, b_{i_k}^n$ corresponds to an index $\mathbf{i}=(i_1,\ldots,i_k)\in\{1,\ldots,n\}^k$
	and the order in which canonical vectors are considered imposes an ordering on these indices. We can regard these indices as ``timestamps''
	with $(1,1,\ldots,1)$ being the initial timestamp, and $(n,n,\ldots,n)$ the last timestamp.
		If we denote by $Q_j^{\mathbf{i}}$, for $j\in[m]$, $H_j^{\mathbf{i}}$, for $j\in[\ell]$, $W_j^{\mathbf{i}}$ and $H_{W_j}^{\mathbf{i}}$, for $j\in[s]$, the values of these vector variables at timestamp $\mathbf{i}$ during the evaluation of $e_f^{\geq n_0}$, then if $\mathbf{i}'$ is the next timestamp, we want to update them as follows:
	\begin{align*}
		Q_j^{\mathbf{i}'}&:=e_{Q_j}(Q_1^{\mathbf{i}},\ldots Q_m^{\mathbf{i}},H_1^{\mathbf{i}},\ldots,H_\ell^{\mathbf{i}},W_1^{\mathbf{i}},\ldots,W_s^{\mathbf{i}},H_{W_1}^{\mathbf{i}},\ldots,H_{W_s}^{\mathbf{i}},b_{i_1'}^n,\ldots,b_{i_k'}^n)\\
		H_j^{\mathbf{i}'}&:=e_{H_j}(Q_1^{\mathbf{i}},\ldots Q_m^{\mathbf{i}},H_1^{\mathbf{i}},\ldots,H_\ell^{\mathbf{i}},W_1^{\mathbf{i}},\ldots,W_s^{\mathbf{i}},H_{W_1}^{\mathbf{i}},\ldots,H_{W_s}^{\mathbf{i}},b_{i_1'}^n,\ldots,b_{i_k'}^n)\\
		W_j^{\mathbf{i}'}&:=e_{W_j}(Q_1^{\mathbf{i}},\ldots Q_m^{\mathbf{i}},H_1^{\mathbf{i}},\ldots,H_\ell^{\mathbf{i}},W_1^{\mathbf{i}},\ldots,W_s^{\mathbf{i}},H_{W_1}^{\mathbf{i}},\ldots,H_{W_s}^{\mathbf{i}},b_{i_1'}^n,\ldots,b_{i_k'}^n)\\
				H_{W_j}^{\mathbf{i}'}&:=e_{H_{W_j}}(Q_1^{\mathbf{i}},\ldots Q_m^{\mathbf{i}},H_1^{\mathbf{i}},\ldots,H_\ell^{\mathbf{i}},W_1^{\mathbf{i}},\ldots,W_s^{\mathbf{i}},H_{W_1}^{\mathbf{i}},\ldots,H_{W_s}^{\mathbf{i}},b_{i_1'}^n,\ldots,b_{i_k'}^n)\\
O^{\mathbf{i}'}&:=e_{O}(Q_1^{\mathbf{i}},\ldots Q_m^{\mathbf{i}},H_1^{\mathbf{i}},\ldots,H_\ell^{\mathbf{i}},W_1^{\mathbf{i}},\ldots,W_s^{\mathbf{i}},H_{W_1}^{\mathbf{i}},\ldots,H_{W_s}^{\mathbf{i}},b_{i_1'}^n,\ldots,b_{i_k'}^n)\\
H_{O}^{\mathbf{i}'}&:=e_{H_O}(Q_1^{\mathbf{i}},\ldots Q_m^{\mathbf{i}},H_1^{\mathbf{i}},\ldots,H_\ell^{\mathbf{i}},W_1^{\mathbf{i}},\ldots,W_s^{\mathbf{i}},H_{W_1}^{\mathbf{i}},\ldots,H_{W_s}^{\mathbf{i}},b_{i_1'}^n,\ldots,b_{i_k'}^n).
	\end{align*}
And this precisely what $e_f^{\geq n_0}$ does. Furthermore,	it ensures that  $W_1^{(n,n,\ldots,n)}$ is such that it encodes the result $f_n(w_1,\ldots,w_\ell)$. We remark that we
always perform $n^k$ iterations which is fine because we assume that once the Turing machine ends in an accepting configuration,
it stays there (it loops). We also recall that we assume that an accepting configuration is always reached. It remains to
explain the expressions used to do the updates. We do this by using all expressions defined earlier, for moving heads and updating
tapes, in combination with a check of what configuration is currently present. Furthermore, we need to ensure that at the initial timestamp,
when each vector variable $v_i$ is assigned to $b_1^n$, we initialize states and heads correctly. As such, we include a check
$\prod_{j=1}^{k} \textsf{min}(v_i)$ to detect whether or not we are at the initial timestamp. More precisely, the update expressions
are given as follows, where we use $\star$ to mark irrelevant information in the transitions: 
    \begin{align*}   
        e_{Q_1}&:=\left(\prod_{j=1}^{k} \textsf{min}(v_i)\right)\times e_{\mathsf{min}}
        + \sum_{\substack{(q_i,a_1,\ldots,a_\ell,b_1, \ldots, b_s)\\
        \Delta(q_i,a_1,\ldots,a_\ell,b_1, \ldots, b_s)=(q_1,\star)}} \!\!\!\!\!\!\!\!\! \mathsf{isconf}_{q_i,a_1,\ldots,a_\ell,b_1, \ldots, b_s}\times e_{\mathsf{min}} \\
        e_{Q_j}&:=\sum_{\substack{(q_i,a_1,\ldots,a_\ell,b_1, \ldots, b_s)\\
        \Delta(q_i,a_1,\ldots,a_\ell,b_1, \ldots, b_s)=(q_j,\star)}} \!\!\!\!\!\!\!\!\! \mathsf{isconf}_{q_i,a_1,\ldots,a_\ell,b_1, \ldots, b_s}\times e_{\mathsf{min}}
        \quad \text{for $j\neq 1$}\displaybreak\\
        e_{H_i}&:=2\left(\prod_{j=1}^{k} \textsf{min}(v_i)\right)\times e_{\mathsf{min}}
        +\sum_{\substack{(q,a_1,\ldots,a_\ell,b_1, \ldots, b_s) \\  \Delta(q,a_1,\ldots,a_\ell,b_1, \ldots, b_s)=(\star,\mathsf{d_i},\star)}}\!\!\!\!\!\!\!\!\! \mathsf{isconf}_{q,a_1,\ldots,a_\ell,b_1, \ldots, b_s}\times\mathsf{move\_inp}^i_{\mathsf{d}_i}\\
        e_{H_{W_i}}&:=2\left(\prod_{j=1}^{k} \textsf{min}(v_i)\right)\times e_{\mathsf{min}}
    +\sum_{\substack{(q,a_1,\ldots,a_\ell,b_1, \ldots, b_s) \\ \displaybreak\\
    \Delta(q,a_1,\ldots,a_\ell,b_1, \ldots, b_s)=(\star,\mathsf{d_{\ell+i}},\star)}}\!\!\!\!\!\!\!\!\! \mathsf{isconf}_{q,a_1,\ldots,a_\ell,b_1, \ldots, b_s}\times\mathsf{move\_work}_{\mathsf{d}_{\ell+i}}^i\\
 	    e_{W_i}&:=\sum_{\substack{(q,a_1,\ldots,a_\ell,b_1, \ldots, b_s)\displaybreak\\
        \Delta(q,a_1,\ldots,a_\ell,b_1, \ldots, b_s)=(\star,b'_i,\star)}}\!\!\!\!\!\!\!\!\! \mathsf{isconf}_{q,a_1,\ldots,a_\ell,b_1, \ldots, b_s}\times\mathsf{write\_work}_{b_i'}^i.
    \end{align*}
With these update expressions at hand, the correctness of $e_f^{\geq n_0}$ should be clear from the construction (one can formally verify this by
    induction on the number of iterations). 
	
\medskip
\noindent
\underline{\textbf{(b)} Small $n$, i.e. $n< n_0$.}	
We next consider the case when $n<n_0$. For each $n<n_0$ and every possible input words
    $w_1,\ldots,w_\ell$ of size $n$, we define a \langfor expression which checks whether
    $\mathsf{mat}(R_i)=\mathsf{vec}(w_i)$ for each $i\in[\ell]$. This can be easily done since $n$ 
    can be regarded as a constant. For example, to check whether $\mathsf{mat}(R_i)=[0,1,1]^T$ we simply write
    $$
    (1- R_i^T\cdot e_{\mathsf{min}})\cdot (R_i^T\cdot e_{\mathsf{Next}}\cdot e_{\mathsf{min}})\cdot (1- R_i^T\cdot e_{\mathsf{Next}}\cdot e_{\mathsf{Next}}\cdot e_{\mathsf{min}})\times (1- e_{\ones}(R_i)^T\cdot e_{\mathsf{Next}}\cdot e_{\mathsf{Next}}\cdot e_{\mathsf{Next}}\cdot e_{\mathsf{min}})
    $$
    which will evaluate to $1$ if and only if $\mathsf{mat}(R_i)=[0,1,1]^T$. We note that the final factor is in 
    place to check that the dimension of $\mathsf{mat}(R_i)$ is three.
    We denote by
    $e_{n,w}^i$ the expression which evaluates to $1$ if and only if $\mathsf{mat}(R_i)=\mathsf{vec}(w)$
    for $|w|=n$.
    We can similarly
    write any word $w$ of fixed size in the matrix variable $W_1$. For example, suppose that $w=101$
    then we write 
    $$
    W_1+ e_{\mathsf{min}}+  e_{\mathsf{Next}}\cdot e_{\mathsf{Next}}\cdot e_{\mathsf{min}}.
    $$
    We let $e_{n,w}$ be the expression which writes $w$ of size $|w|=n$ in matrix variable $W_1$.
    Then, the expression
    $$
    e_{n,w_1,\ldots,w_n,w}:=(e_{n,w_1}^1\cdot\cdots\cdot e_{nw_{\ell}}^\ell)\times e_{n,w}
    $$
    will write $w$ in $W_1$ if and only if $\mathsf{mat}(R_i)=\mathsf{vec}(w_i)$ for $i\in[\ell]$.
    We now simply take the disjunction over all words 
    $w_1,\ldots,w_\ell\in\Sigma^n$ and $w=f_n(w_1,\ldots,w_\ell)\in\Sigma^n$:
    $$
    e_n:=\sum_{w_1,\ldots,w_\ell\in\Sigma^n} e_{n,w_1,\ldots,w_\ell,f_n(w_1,\ldots,w_\ell)},
    $$
    which correctly writes $f_n(w_1,\ldots,w_\ell)$ in $W_1$. For $e_f^{<n_0}$ it now remains
to take a further disjunction for $n=0,\ldots, n_0-1$. That is,
    $
    e_f^{<n_0}:=\sum_{n=0}^{n_0-1} e_n
    $.
    Since every possible input is covered and only a unique expression 
    $e_{n,w_1,\ldots,w_\ell,f_n(w_1,\ldots,w_\ell)}$ will be triggered, $e_f^{<n_0}$ will correctly
    evaluate $f$ on input words smaller than $n_0$.

\medskip
\noindent
\underline{\textbf{(c)} The final expression $e_f$.}	
 The desired final expression $e_f$ is now given by
    $$
    e_f:=e_f^{<n_0} + \bigl(e_{\ones}(R_1)^T\cdot\underbrace{e_{\mathsf{Next}}\cdot\cdots\cdot e_{\mathsf{Next}}}_{\text{$n_0$ times}}\bigr)\times e_f^{\geq n_0},
    $$
where we note that $e_{\ones}(R_1)^T\cdot\underbrace{e_{\mathsf{Next}}\cdot\cdots\cdot e_{\mathsf{Next}}}_{\text{$n_0$ times}}$ evaluates to $1$ if
$n\geq n_0$ and to $0$ otherwise.
\end{proof}

\subsubsection{Generalized iteration expressions}\label{subsubsec:generalloop}
The expression $e_f^{\geq n_0}$ is given in terms of an expression of the form 
\begin{align*}
    \texttt{for}\, v_1,\ldots, v_k,X_1,\ldots, X_\ell \texttt{.}\, \Big( &e_1(X_1,\ldots,X_\ell,v_1,\ldots, v_k), \\
    &\hspace{1em}e_2(X_1,\ldots, X_\ell,v_1,\ldots, v_k), \ldots, e_\ell(X_1,\ldots,X_\ell,v_1,\ldots, v_k) \Big),
\end{align*}
where $X_1,X_2,\ldots,X_\ell$ are vector variables. Intuitively, each $X_i$ is updated according to expression $e_i$.
We now give the formal semantics of such expressions by defining it as a \langfor expression, for which we have defined the precise semantics in Section~\ref{sec:formatlang}.
More precisely, the expression above is simply a shorthand notation for the \langfor expression
\begin{align*}
  &\ffor{v_1}{X_1}{} \\
  &\hspace{1em}\initf{X_1}{v_2}{X_2}{} \\
  &\hspace{2em}\initf{X_2}{v_3}{X_3}{} \\
  &\hspace{8em}\ddots \\
  &\hspace{4em}\initf{X_{k-1}}{v_k}{X_k}{ e'(X_k,v_1,\ldots, v_k)}
\end{align*}
with $e'(X,v_1,\ldots,v_k)$ the expression
$$
\sum_{i=1}^{\ell} e_i(X\cdot e_{\min}, X\cdot e_{\mathsf{min}+1},\ldots,X\cdot e_{\mathsf{min}+(\ell-1)},v_1,\ldots,v_k)\cdot \bigl(
e_{\mathsf{diag}}(e_{\mathbf{1}}(X^T))\cdot e_{\mathsf{min}+(i-1)}\bigr)^T,
%
% e_1(X\cdot e_{\mathsf{min}},v_1,\ldots,v_k)\cdot (e_{\diag}(e_{\ones}(X^T))\cdot e_{\mathsf{min}})^T \\
% &+ e_2(X\cdot e_{\mathsf{min} + 1},v_1,\ldots,v_k)\cdot (e_{\diag}(e_{\ones}(X^T))\cdot e_{\mathsf{min} + 1})^T \\
% &+ \ldots + e_\ell(X\cdot e_{\mathsf{max}},v_1,\ldots,v_k)\cdot (e_{\diag}(e_{\ones}(X^T))\cdot e_{\mathsf{max}})^T
$$
where $e_{\mathsf{min}+\mathsf{i}}$, for $i=0,1,\ldots,n-1$ returns the $(i+1)$th canonical vector. Here, if the $v_i$'s have
size $\alpha\times 1$ then $X$ has size
$\alpha\times\beta$ and we assume on instances that $\dom(\beta)=\ell$. The $i$th column of  $X$ thus corresponds to the vector
variable $X_i$ and to evaluate $e_i$ we thus first extract all columns separately, by means of $X\cdot e_{\mathsf{min}+\mathsf{i}}$, and feed all of these into $e_i$.
We remark that the canonical vectors used are $b_{j}^{\dom(\beta)}$, i.e. of dimension $\ell\times 1$.
After evaluating $e_i$, which returns a vector of size $\alpha\times 1$, we place this vector back in the $i$the column of $X$ by multiplying with the appropriate
row vector corresponding to column $i$. This is done by the last factor in the expression $e'$. Furthermore, $e_{\mathsf{diag}}(e_{\mathbf{1}}(X^T))$ is included to ensure that
$e_{\mathsf{min} + \mathsf{(i-1)}}$ again evaluates to the canonical vector $b_{i}^{\dom(\beta)}$ of dimension $\ell\times 1$.
% %
% %
% %
% %  We sometimes will want to iterate over $k$ canonical vectors. We define the following shorthand notation:
% %
% % \begin{align*}
% %   \ffor{v_1,\ldots, v_k}{X}{e(X,v_1,\ldots, v_n)}:= &\ffor{v_1}{X_1}{X_1 +} \\
% %   &\hspace{1em}\initf{X_1}{v_2}{X_2}{X_2 + } \\
% %   &\hspace{2em}\initf{X_2}{v_3}{X_3}{X_3 + } \\
% %   &\hspace{8em}\ddots \\
% %   &\hspace{4em}\initf{X_{k-1}}{v_k}{X_k}{ e(X_k,v_1,\ldots, v_k)}.
% % \end{align*}
% %
% % To reference $\ell$ different vector variables $X_1,\ldots,X_\ell$ in every iteration and update them in different ways we define:
% % \begin{multline*}
% % \ffor{v}{X_1,\ldots, X_\ell}{\left( e_1(X_1,v), e_2(X_2,v), \ldots, e_l(X_\ell,v) \right)} := \\
% % \ffor{v}{X}{e_1(X\cdot e_{\mathsf{min}},v)\cdot (e_{\diag}(e_{\ones}(X^T))\cdot e_{\mathsf{min}})^T +\\ e_2(X\cdot e_{\mathsf{min} + 1},v)\cdot (e_{\diag}(e_{\ones}(X^T))\cdot e_{\mathsf{min} + 2})^T + \ldots \\
% % + e_\ell(X\cdot e_{\mathsf{max}},v)\cdot (e_{\diag}(e_{\ones}(X^T))\cdot e_{\mathsf{max}})^T} \\
% % \end{multline*}
%
% We note that for the latter expression to be semantically correct $v$ has to be of type $\gamma\times 1$,
% both $X_i$ and $e_i$ for $ i=1,\ldots,\ell$ have to be of type $\alpha\times 1$,
% and $X$ has to be of type $\alpha\times\beta$, where $\dom(\beta)=\ell$. Here
% we use $e_{\diag}(e_{\ones}(X^T))$ to compute the $\beta\times\beta$ identity and ensure the typing of the
% $e_{\mathsf{min} + i}$.
% When evaluated on an instance $\I$,
% $e_{\mathsf{min}}, e_{\mathsf{min} + i}$ evaluate to $b_1^{\dom(\beta)}$ and $b_{1+i}^{\dom(\beta)}$,
% respectively, and we show their defining expressions in section \ref{sec:formatlang:design}.
% Similarly for $e_{\mathsf{max}}=b_n^{\dom(\beta)}$.
% The combinations of both previous operators results in:
%
% \begin{align*}
%     \texttt{for}\, v_1,\ldots, v_k,X_1,\ldots, X_\ell \texttt{.}\, \Big( &e_1(X_1,v_1,\ldots, v_k), \\
%     &\hspace{1em}e_2(X_2,v_1,\ldots, v_k), \ldots, e_\ell(X_\ell,v_1,\ldots, v_k) \Big) := \\
%     &\hspace{8em}\ffor{v_1,\ldots, v_k}{X}{e'(X,v_1,\ldots, v_k)} \\
% \end{align*}
% where
% \begin{align*}
% e'(X,v_1,\ldots,v_k):=&e_1(X\cdot e_{\mathsf{min}},v_1,\ldots,v_k)\cdot (e_{\diag}(e_{\ones}(X^T))\cdot e_{\mathsf{min}})^T \\
% &+ e_2(X\cdot e_{\mathsf{min} + 1},v_1,\ldots,v_k)\cdot (e_{\diag}(e_{\ones}(X^T))\cdot e_{\mathsf{min} + 1})^T \\
% &+ \ldots + e_\ell(X\cdot e_{\mathsf{max}},v_1,\ldots,v_k)\cdot (e_{\diag}(e_{\ones}(X^T))\cdot e_{\mathsf{max}})^T
% \end{align*}
% It is clear that this expression iterates over $k$ canonical vectors and references $\ell$ independent vectors updating each of them in their particular way.
