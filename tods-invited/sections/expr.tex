% !TeX spellcheck = en_US
%!TEX root = ../main.tex
We next explore the expressive power of $\langfor$ in comparison to arithmetic circuits.
Given that arithmetic circuits capture most standard linear algebra algorithms~\cite{Raz02,ShpilkaY10},
they seem a natural candidate for comparison. 
%
Intuitively, an arithmetic circuit is similar to a boolean circuit,
%\cite{aroraB2009}
except that it has gates computing sums and products, 
instead of computing disjunctions and conjunctions, and processes elements of $\RR$, 
instead of boolean values. 


A first observation is that in order to connect \langfor to arithmetic circuits, we need a 
notion of uniformity of such circuits. After all, a \langfor expression can take matrices 
of arbitrary dimensions as input and we want to avoid having (completely) different circuits
for each dimension.
%
To handle inputs of different sizes, we thus consider a notion of uniform families of arithmetic
circuits, defined -- as usual -- via a Turing machine generating a description of the circuit for 
each input size $n$.
%
The main result of this section is that any function $f$ which operates on matrices, and is computed 
by a uniform family of arithmetic circuits of bounded degree, can also be computed by 
a \langfor expression, and vice versa.

In order to keep the notation light, we will focus on \langfor schemas over square matrices where 
each matrix variable has type $(\alpha,\alpha),(\alpha,1),(1,\alpha)$, or $(1,1)$.
Our results hold without these restrictions as well, however. 
%
In what follows, we write $\langfor$ to denote $\langforf{\emptyset}$, i.e. the fragment of 
our language with no additional pointwise functions. 
%
We begin by defining circuits in Section~\ref{subsect:ac}, and then show how circuit families can be
simulated by $\langfor$ in Section~\ref{subsec:actoformatlang}. In Section~\ref{subsec:formatlangtoac} 
we associate circuits to $\langfor$ expressions. We consider extensions of $\langfor$ with additional
operators, and the division operator in particular, in Section~\ref{subsec:additionalop}.
\textcolor{red}{Added some more subsec..to be filled in.}

\subsection{Arithmetic circuits}\label{subsect:ac}
We first recall the definition of arithmetic circuits~\cite{allender}. 
%
An \textit{arithmetic circuit} $\Phi$ over a set $X=\{x_1,\ldots,x_n\}$ of input variables 
is a directed acyclic labeled graph. The vertices of $\Phi$ are called \textit{gates} and 
denoted by $g_1,\ldots,g_m$; the edges in $\Phi$ are called \textit{wires}. We denote by 
$\mathsf{gates}(\Phi)$ the number of gates in $\Phi$ and by $\mathsf{wires}(\Phi)$ the number
of wires in $\Phi$.
%
The \textit{children} of a gate $g$ correspond to all gates $g'$ such that $(g,g')$ is a wire. 
The \textit{parents} of $g$ correspond to all gates $g'$ such that $(g',g)$ is a wire. The
\textit{in-degree}, or a \textit{fan-in}, of a gate $g$ refers to its number of children, and 
the \textit{out-degree} to its number of parents. We do not assume any restriction on the 
in-degree of a gate, and will thus consider circuits with unbounded fan-in. 
Gates with in-degree $0$ are called \textit{input gates} and are labeled by either a variable 
in $X$ or a constant $0$ or $1$. All other gates are labeled by either $+$ or $\times$, and 
are referred to as \textit{sum gates} or \textit{product gates}, respectively.
Gates with out-degree $0$ are called \textit{output gates}.
 % When talking about arithmetic circuits, one usually focuses on circuits with $n$ input gates and a single output gate.
The \textit{size} of $\Phi$, denoted by $|\Phi|$, is its number of gates and wires. 
%
The \textit{depth} of $\Phi$, denoted
by $\mathsf{depth}(\Phi)$, is the length of the longest directed path from any its output gates to any of the input gates. The \textit{degree} of a gate is defined inductively: an input gate has degree~1, a sum gate has a degree equal to the maximum of degrees of its children, and a product gate has a degree equal to the sum of the degrees of its children. 

When talking about arithmetic circuits, one usually focuses on circuits with $n$ input gates and a single output gate. In this case, 
% When $\Phi$ has a single output gate,
the \textit{degree} of $\Phi$, denoted by $\mathsf{degree}(\Phi)$, is defined as the degree of its output gate. Furthermore, 
% If $\Phi$ has a single output gate and its input gates take values from $\RR$, then
$\Phi$ can then be seen to correspond to a polynomial in $\RR[X]$ in a natural way. The {degree} of $\Phi$ then equals the degree of the polynomial corresponding to $\Phi$. Finally, if $a_1,\ldots ,a_n$ are values in $\RR$, then 
the \textit{result of the circuit} on this input is the value computed by the corresponding polynomial, denoted by $\Phi(a_1,\ldots ,a_n)$.

In order to handle inputs of different sizes, we use the notion of uniform circuit families. An \textit{arithmetic circuit family} is a set of arithmetic circuits $\{\Phi_n\mid n=1,2,\ldots\}$ where $\Phi_n$ has $n$ input variables and a single output gate. An arithmetic circuit family is \textit{uniform} if there exists a \logspace-Turing machine,
which on input $1^n$, returns an encoding of the arithmetic circuit $\Phi_n$ for each $n$.
We observe that uniform arithmetic circuit families are necessarily of polynomial size. 
Another important parameter is the circuit depth. A circuit family is of logarithmic depth, whenever $\mathsf{depth}(\Phi_n)\in \mathcal{O}(\log\, n)$, for each $n$. We  now show that \langfor subsumes uniform arithmetic circuit families that are of logarithmic depth. 


\subsection{From arithmetic circuits to \langfor}\label{subsec:actoformatlang}
This section is dedicated to showing how \langfor expressions can be connected to arithmetic circuits, as formalized
in the following theorem.

\begin{theorem}
\label{th-circuits-ml}
For any uniform arithmetic circuit family $\{\Phi_n\mid n=1,2,\ldots\}$ of logarithmic depth there is a \langfor schema $\Sch$ and an expression $e_\Phi$ using a matrix variable $v$, with $\ttype(v)=(\alpha,1)$ and $\ttype(e_\Phi) = (1,1)$, such that for any input values $a_1,\ldots ,a_n$: 
\begin{quote} If $\I = (\dom,\conc)$ is a \lang\ instance such that $\dom(\alpha) = n$ and $\conc(v) = [a_1 \ldots a_n]^T$,
 then $\sem{e_\Phi}{\I} = \Phi_n(a_1,\ldots ,a_n)$.
\end{quote}
\end{theorem}
It is important to note that the expression $e_\Phi$ does not change depending on the input size, meaning that it is uniform in the same sense as the circuit family being generated by a single Turing machine. The different input sizes for a \langfor instance are handled by the typing mechanism of the language. 

The proof of Theorem~\ref{th-circuits-ml} is as follows: First, in
Section~\ref{subsubsec:eval}, we describe a stack-based evaluation algorithm for arithmetic circuits. 
Then,  we show how to simulate that evaluation algorithm in \langfor in Section~\ref{subsubsec:simulate}.

\subsubsection{Evaluation algorithm for arithmetic circuits}\label{subsubsec:eval}
We start by describing a stack-based evaluation algorithm for arithmetic circuits based on a depth-first top-down traversal of the circuit. As we will see later on, this algorithm lends itself well to be simulated in \langfor. 

Let $\Phi_n$ be a circuit with $n$ input gates. We assume an ordering on the children of gates and that there are only variable input gates or input gates with value $1$ ($1$-gates). Indeed, we can prune away input gates with value $0$ in a preprocessing step. We store the values of the input gates in an $n\times 1$ vector $A$, where $A[i]$ denotes the value
of the $i$th input gate. To evaluate $\Phi_n$ on $A$ we will maintain two stacks: A \textit{gate stack}
 $\cG$ and a \textit{value stack} $\cV$. For a stack $S$, we have the standard operation: $\push{S}{s}$: pushes $s$ into $S$,
$\pop{S}$: pops the top element,  $\getsize{S}$: returns the length of the stack, and
$\gettop{S}$: returns the top element in the stack. We refer to the $i$th element on $S$ by $S[i]$.

The evaluation algorithm works for any arithmetic circuit. The logarithmic depth assumption in Theorem~\ref{th-circuits-ml} is only needed for simulating the evaluation algorithm in \langfor.

Algorithm~\ref{alg:eval_code} depicts the evaluation algorithm, consisting of three functions:
$\textsc{Evaluate}(\Phi_n,A)$, the main algorithm; $\textsc{Extend}(\cG,\cV,A)$, in charge of creating paths from input gates to the output gate; and $\textsc{Aggregate}(\cG,\cV)$, in charge of computing a partial of evaluation of $\Phi_n$ based on the paths considered so far and in charge of initiating the exploration of the next path during the traversal of $\Phi_n$.
The algorithm, during its execution, ensures that 
$\cG$ always contains a path from a gate in $\Phi_n$ to the output gate and that all paths are visited in a depth-first left-to-right traversal of $\Phi_n$. Moreover, $\cV$ will hold values of gates stored in $\cG$ based on the paths explored so far.


More precisely, $\textsc{Evaluate}(\Phi_n,A)$ starts with empty stacks $\cG$ and $\cV$ (line 2) and then pushes the output
gate to $\cG$ (line 3) while leaving $\cV$ empty. It then alternates between calls to $\textsc{Extend}(\cG,\cV,A)$ and $\textsc{Aggregate}(\cG,\cV)$ based on whether $\getsize{\cG} = \getsize{\cV}+1$ or $\getsize{\cG} = \getsize{\cV}$, respectively (lines 5--8). Since, initially, $\getsize{\cG}=1$ and $\getsize{\cV}=0$, $\textsc{Extend}(\cG,\cV,A)$ is
called first.
 \begin{algorithm}[t]
  \caption{Stack-based evaluation algorithm for arithmetic circuits}
 \label{alg:eval_code}
 \begin{algorithmic}[1]
 \Function{Evaluate}{$\Phi_n,A$}\Comment{Input: Circuit $\Phi_n$, $ n\times 1$ vector $A$.}
   \State $\cG\gets\emptyset$,$\cV\gets\emptyset$
    \State $\push{\cG}{\getroot}$ \Comment{$\getroot$: outputs the root gate of the circuit}
     \While{$\getsize{\cG}\neq 1$ or $\getsize{\cV}\neq 1$}
         \If{$\getsize{\cG}\neq \getsize{V}$}
             \State $(\cG,\cV) := \textsc{Initialize}(\cG,\cV,A)$
         \Else
             \State $(\cG,\cV):= \textsc{Aggregate}(\cG,\cV)$
         \EndIf
     \EndWhile
     \State \textbf{return} $\gettop{\cV}$
 \EndFunction
 \Function{Extend}{$\cG, \cV, A$}\Comment{Always called when $\getsize{\cG} =  \getsize{\cV} + 1$}
     \If{$\isplus{\gettop{\cG}}$} \Comment{$\isplus{g}$ checks whether $g$ is a $+$-gate}
         \State $\push{\cV}{0}$
         \State $\push{\cG}{\getfirst{\gettop{\cG}}}$ \Comment{$\getfirst{g}$ outputs the first child of $g$}
     \ElsIf{$\isprod{\gettop{\cG}}$} \Comment{$\isprod{g}$ checks whether $g$ is a $\times$-gate}
         \State $\push{\cV}{1}$
         \State $\push{\cG}{\getfirst{\gettop{\cG}}}$
     \ElsIf{$\isone{\gettop{\cG}}$} \Comment{$\isone{g}$ checks whether $g$ is a $1$-gate}
         \State $\push{\cV}{1}$
     \ElsIf{$\isinput{\gettop{\cG}}$} \Comment{$\isinput{g}$ checks whether $g$ is an input gate}
         \State $\push{\cV}{A\left[ \getinput{\gettop{\cG}} \right]}$ \Comment{$\getinput{g}$ outputs $A[i]$ if $g$ is the $i$th input}
     \EndIf
     \State \textbf{return} $\cG, \cV$
 \EndFunction
 \Function{Aggregate}{$\cG, \cV$}\Comment{Always called when $\getsize{\cG} =  \getsize{\cV}$}
     \State $g = \pop{\cG}$
     \State $v = \pop{\cV}$
     \If{$\isplus{\gettop{\cG}}$}
         \State $\gettop{\cV} = \gettop{\cV} + v$
     \ElsIf{$\isprod{\gettop{\cG}}$}
         \State $\gettop{\cV} = \gettop{\cV} \cdot v$
     \EndIf
     \If{$\isnotlast{\gettop{\cG}}{g}$} \Comment{$\isnotlast{g}{g'}$ is true if $g'$ is not the last child of $g$}
         \State $\push{\cG}{\nextgate{\gettop{\cG}}{g}}$ \Comment{$\nextgate{g}{g'}$ returns  next child of $g$ after $g'$.}
     \EndIf
     \State \textbf{return} $\cG, \cV$
 \EndFunction
  \end{algorithmic}
 \end{algorithm}
As mentioned,  $\textsc{Extend}(\cG,\cV,A)$ is in charge of pushing gates and values to the stacks corresponding the current visited path and it does so until $\getsize{\cG} = \getsize{\cV}$ is satisfied (line 5). In particular,
$\textsc{Extend}(\cG,\cV,A)$ checks the type of the current top gate $g=\gettop{\cG}$ in $\cG$ and does the following.
If $g$ is a $+$-gate (lines 11--13) then it pushes $0$ to $\cV$, and if $g$ is a $\times$-gate (lines 14--16) then it pushes $1$ to $\cV$.
In both cases, the first child of $g$ is also pushed to $\cG$ ensuring that $\getsize{\cG} = \getsize{\cV}+1$. If $\cV[i]$ corresponded to the value of gate $\cG[i+1]$ computed so far, then this invariant remains to hold after these steps.

However, when $\textsc{Extend}(\cG,\cV,A)$ detects a $1$-gate (lines 17--18) or an input gate (lines 19-20), it ensures  $\getsize{\cG} = \getsize{\cV}$
by pushing the value of the input gate to $\cV$. As such, we have that $\cV[i]$ now holds the value of $\cG[i]$ computed so far.
Importantly, $\textsc{Evaluate}(\Phi_n,A)$ then switches to $\textsc{Aggregate}(\cG,\cV)$ (line 7). 

The function $\textsc{Aggregate}(\cG,\cV)$ 
will pop both stacks (lines 23--24) retrieving gate $g$ and value $v$. Recall that $v$ is the value of gate $g$ computed so far (since  $\getsize{\cG} = \getsize{\cV}$) and we now need to pass $v$ to the value $v'$ of the parent  $g'$ of $g$. The gate $g'$ and value $v'$  are now the top elements in the stacks.
Depending on the type of $g'$, $\textsc{Aggregate}(\cG,\cV)$ either adds $v$ to $v'$, in case $g'$ is a $+$-gate (lines 25-26), or multiplies $v$ with $v'$, in case $g'$ is a $\times$-gate (lines 27--28). At this point,
$\getsize{\cG} = \getsize{\cV}$ and $\cV[i]$ correctly holds the value of evaluation of $\cG[i]$. Furthermore, if $g$ was the last child
of $g'$, then we are done with $g'$ as well, and $\textsc{Aggregate}(\cG,\cV)$ continues popping the stacks. 
In contrast,
if $g$ was not the last child of $g'$, to get the correct value of $g'$ we need to continue the exploration of different branches
below $g'$. The function $\textsc{Aggregate}(\cG,\cV)$ does this by pushing the next child $g''$ of $g'$ (the one after $g$) to $\cG$ (lines 29--30), resulting in $\getsize{\cG} = \getsize{\cV}+1$. This implies that $\textsc{Evaluate}(\Phi_n,A)$ switches again to $\textsc{Extend}(\cG,\cV,A)$, which in turn will extend the path leading to $g''$, as before. We thus ensure
that the entire circuit below a gate is evaluated before it is popped from $\cG$. And moreover, $\cV[i]$ corresponds to the value of gate $\cG[i+1]$ when $\getsize{\cG} = \getsize{\cV}+1$ and $\cV[i]$ corresponds to the value of gate $\cG{i}$ when
 $\getsize{\cG} = \getsize{\cV}$. 
 
 The algorithm terminates when $\textsc{Aggregate}(\cG,\cV)$ can pop all but one of the elements from $\cG$ and $\cV$, resulting in $\getsize{\cG} = \getsize{\cV}=1$. In this case,  $\textsc{Evaluate}(\cG,\cV,A)$ stops executing $\textsc{Extend}(\cG,\cV,A)$ and $\textsc{Aggregate}(\cG,\cV)$ (line 9) and returns the top element of $\cV$,
 which indeed holds the result of evaluating $\Phi_n$ on $A$.
%
%
%  in which case $\cV[i]$ holds the result of the partial evaluation of $\Phi_n$ on $A$ of gate $\cG[i]$ based on the visited part of $\Phi_n$ so far. Then, the function $\textsc{Aggregate}(\cG,\cV)$ is called on line 8 in
% Algorithm~\ref{alg:eval_code}. It keeps popping gates and values from the stack, meaning going up in the circuit, updating the values according the type of gate under consideration. It maintains $\getsize{\cG} = \getsize{\cV}$ unless the popped gate was not the last child of the current top gate in $\cG$. In this case,  the next child of the top gate is pushed onto $\cG$, meaning that we start exploring the next path in our traversal of $\Phi_n$. Moreover, this results in
% $\getsize{\cG} = \getsize{\cV}+1$ and hence $\textsc{Initialize}(\cG,\cV,A)$ is triggered again. The process then continues until in the end, $\getsize{\cG} = \getsize{\cV}=1$ (line 4). We then have that $\gettop{\cV}$ holds the result of the output gate of $\Phi_n$ after visiting all paths and is returned (line 9).
%
% Furthermore, when an input gate is reached and pushed on $\cG$, it is guaranteed that
%  $\getsize{\cG} = \getsize{\cV}$ and that $\cV[i]$ holds the result of the partial evaluation of $\Phi_n$ on $A$ of gate
%  $\cG[i]$. At any other point in time $\getsize{\cG} = \getsize{\cV} + 1$ and $\cV[i]$ corresponds to the value of $\cG[i+1]$.
%  The function $\textsc{Initialize}(\cG,\cV,A)$ on line 6 is executed until  $\getsize{\cG} = \getsize{\cV}$ holds.
%
%
% So, when $\getsize{\cG} = \getsize{\cV}$ holds,
%
% At the same,  at any point in time, either $\getsize{\cG} = \getsize{\cV}$ or  $\getsize{\cG} = \getsize{\cV} + 1$ holds. If $\getsize{\cG} = \getsize{\cV}$ holds, then $\cV[i]$ is the value that the gate $\cG[i]$ outputs, as the result of the current partial evaluation of $\Phi_n$. In particular, in this case,  $\gettop{\cV}$ is the value of evaluating the circuit in gate $\gettop{\cG}$.
%
%     The $Evaluate$ algorithm gives us the output of the circuit. Note that after each iteration it either holds that $\getsize{\cG} =  \getsize{\cV} + 1$ or $\getsize{\cG} =  \getsize{\cV}$. Furthermore, when we start we have $\getsize{\cG}=1$ and $\getsize{\cV}=0$. The condition $\getsize{\cG}= 1$ and $\getsize{\cV}=1$ holds only when we have traversed all the circuit, and the value in $\gettop{\cV}$ is the value that the root of the circuit outputs after its computation.
%
%
% that there are two configuration
%     During the evaluation algorithm there will be two possible configurations of $\cG$ and $\cV$.
%
%     \begin{enumerate}
%         \item $\getsize{\cG} = \getsize{\cV} + 1$: this means that $\gettop{\cG}$ is a gate that we visit for the first time and we need to initialize its value.
%
%         \item $\getsize{\cG} = \getsize{\cV}$: here $\gettop{\cV}$ is the value of evaluating the circuit in gate $\gettop{\cG}$. Therefore, we need to aggregate the value $\gettop{\cV}$ to the parent gate of $g$.
%     \end{enumerate}

%
% We first explain the intuition behind the algorithm \textsc{Evaluate} by an example.
%
% \begin{example}  Let us consider the following simple circuit $\Phi_4$ computing  $f(a_1,a_2,a_3,a_4)=a_1a_2 +a_3a_4$.
%     \begin{center}
%     \begin{tikzpicture}[level distance=0.5cm,
%     level 1/.style={sibling distance=1.7cm},
%     level 2/.style={sibling distance=1cm},
%     every node/.style = {
%         shape=circle,
%         draw, scale=0.8,
%         align=center,
%         top color=white,
%         bottom color=white
%         }
%     ]
%     \node {$+$}
%         child { node {$\times$} edge from parent [->]
%         child { node {$a_1$} }
%         child { node { $a_2$} }
%         }
%         child { node {$\times$} edge from parent [->]
%         child { node {$a_3$} }
%         child { node {$a_4$} }
%         };
%     \end{tikzpicture}
%     \end{center}
% The idea is to traverse the circuit top down in a depth first search way. For example, in the circuit
%
%
% 	 $f(a_1,a_2,a_3,a_4)=a_1a_2 +a_3a_4$ above, we would initialize the output gate value as $0$ because it is a $+$ gate, so $\cG=\lbrace +\rbrace$, $\cV=\lbrace 0\rbrace$. Then stack the left $\times$ gate to $\cG$, stack its initial value (i.e. $1$) to $\cV$. Now stack $a_1$ to $\cG$ and its value (i.e. $a_1$) to $\cV$. Since we are on an input gate we pop the gate and value pair off of $\cG$ and $\cV$ respectively, aggregate $a_1$ to $\gettop{\cV}$ and continue by stacking the $a_2$ gate to $\cG$. We pop $a_2$ off of $\cV$ (and its gate off of $\cG$) and aggregate its value to $\gettop{\cV}$. We pop and aggregate the value of the left $\times$ gate to $\gettop{\cV}$ (the root value). Then continue with the right $\times$ gate branch similarly.
%
% % \begin{figure}[h]
% {	\footnotesize
% \begin{tabular}{p{4.5mm}|p{2mm}p{2mm}|p{1mm}p{2mm}|}
% & $\cG$&  & $\cV$ & \\\cline{2-5}
% & $\emptyset$ & (0) &  $\emptyset$ & (0)\\\cline{2-5}
% \textsc{Eval} & $g_{000}$ & (1) & $\emptyset$ & (0)\\\cline{2-5}
% \textsc{Ini} & $g_{001}$ &(2) & $0$& (1)\\
%  & $g_{000}$  & & & \\\cline{2-5}
%  \textsc{Ini} & $g_{010}$ & (3) & 1 &(2)\\
%  &  $g_{001}$ & & $0$& \\
%   & $g_{000}$  & & & \\\cline{2-5}
%  \textsc{Ini} & $g_{010}$ & (3) & $a_1$ &(3)\\
%  &  $g_{001}$ & & $1$& \\
%   & $g_{000}$  & & 0 & \\\cline{2-5}
% \end{tabular}
% \begin{tabular}{p{4.5mm}|p{2mm}p{2mm}|p{2.2mm}p{2mm}|}
% & $\cG$&  & $\cV$ & \\\cline{2-5}
%  \textsc{Agg} & $g_{011}$ & (3) & $a_1$ &(2)\\
%  &  $g_{001}$ & & $0$ & \\
%  & $g_{000}$ & & & \\\cline{2-5}
%  \textsc{Ini} & $g_{011}$ & (3) & $a_2$ &(3)\\
%  &  $g_{001}$ & & $a_1$ & \\
%  & $g_{000}$ & & 0 & \\\cline{2-5}
%  \textsc{Agg} & $g_{001}$ & (2) & $a_1a_2$ &(2)\\
%  & $g_{000}$ & & 0 & \\\cline{2-5}
%  \textsc{Agg} & $g_{100}$ & (2) & $a_1a_2$ &(1)\\
%  & $g_{000}$ & &  & \\\cline{2-5}
% \end{tabular}
% \begin{tabular}{p{4.5mm}|p{2mm}p{2mm}|p{1mm}p{2mm}|}
% & $\cG$&  & $\cV$ & \\\cline{2-5}
%  \textsc{Ini} & $g_{101}$ & (3) & $1$ &(2)\\
%  &  $g_{100}$ & & $a_1a_2$ & \\
%  & $g_{000}$ & & & \\\cline{2-5}
%  \textsc{Ini} & $g_{101}$ & (3) & $a_3$ &(3)\\
%  &  $g_{100}$ & & $1$ & \\
%  & $g_{000}$ & & $a_1a_2$& \\\cline{2-5}
%  \textsc{Agg} & $g_{111}$ & (3) & $a_3$ &(3)\\
%  &  $g_{100}$ & & $a_1a_2$ & \\
%  & $g_{000}$ & & &  \\\cline{2-5}
%  \multicolumn{5}{c}{}\\
%  \end{tabular}
% \begin{tabular}{p{4.5mm}|p{2mm}p{2mm}|p{13mm}p{2mm}|}
% & $\cG$&  & $\cV$ & \\\cline{2-5}
%  \textsc{Ini} & $g_{111}$ & (3) & $a_4$ &(3)\\
%  &  $g_{100}$ & & $a_3$ & \\
%  & $g_{000}$ & & $a_1a_2$ & \\\cline{2-5}
%  \textsc{Agg} & $g_{100}$ & (2) & $a_3a_4$ &(2)\\
%  &  $g_{000}$ & & $a_1a_2$ & \\\cline{2-5}
%  \textsc{Agg} & $g_{000}$ & (1) & $a_1a_2+a_3a_4$ &(1) \\\cline{2-5}
%  \multicolumn{5}{c}{}\\
%   \multicolumn{5}{c}{}\\ \multicolumn{5}{c}{}\\
%  \multicolumn{5}{c}{}
%  \end{tabular}}
% \end{example}
%%
%
% On input $\Phi_n$ and $A$ it
% starts
 
%  \begin{algorithm}[t]
%  \caption{Extend}\label{alg:init_code}
%  \begin{algorithmic}[1]
%  \Function{Extend}{$\cG, \cV, A$}\Comment{The stacks and input. Here, $\getsize{\cG} =  \getsize{\cV} + 1$}
%      \If{$\isplus{\gettop{\cG}}$} \Comment{$\isplus{g}$ checks whether $g$ is a $+$-gate}
%          \State $\push{\cV}{0}$
%          \State $\push{\cG}{\getfirst{\gettop{\cG}}}$ \Comment{$\getfirst{g}$ outputs the first child of $g$}
%      \ElsIf{$\isprod{\gettop{\cG}}$} \Comment{$\isprod{g}$ checks whether $g$ is a $\times$-gate}
%          \State $\push{\cV}{1}$
%          \State $\push{\cG}{\getfirst{\gettop{\cG}}}$
%      \ElsIf{$\isone{\gettop{\cG}}$} \Comment{$\isone{g}$ checks whether $g$ is a $1$-gate}
%          \State $\push{\cV}{1}$
%      \ElsIf{$\isinput{\gettop{\cG}}$} \Comment{$\isinput{g}$ checks whether $g$ is an input gate}
%          \State $\push{\cV}{A\left[ \getinput{\gettop{\cG}} \right]}$ \Comment{$\getinput{g}$ outputs $A[i]$ if $g$ is the $i$th input}
%      \EndIf
%      \State \textbf{return} $\cG, \cV$
%  \EndFunction
%  \end{algorithmic}
%  \end{algorithm}
%
%  \begin{algorithm}[t]
%  \caption{Aggregate}\label{alg:agg_code}
%  \begin{algorithmic}[1]
%  \Function{Aggregate}{$\cG, \cV$}\Comment{Here, $\getsize{\cG} =  \getsize{\cV}$}
%      \State $g = \pop{\cG}$
%      \State $v = \pop{\cV}$
%      \If{$\isplus{\gettop{\cG}}$}
%          \State $\gettop{\cV} = \gettop{\cV} + v$
%      \ElsIf{$\isprod{\gettop{\cG}}$}
%          \State $\gettop{\cV} = \gettop{\cV} \cdot v$
%      \EndIf
%      \If{$\isnotlast{\gettop{\cG}}{g}$} \Comment{$\isnotlast{g}{g'}$ is true if $g'$ is not the last child of $g$}
%          \State $\push{\cG}{\nextgate{\gettop{\cG}}{g}}$ \Comment{$\nextgate{g}{g'}$ returns  next child of $g$ after $g'$.}
%      \EndIf
%      \State \textbf{return} $\cG, \cV$
%  \EndFunction
%  \end{algorithmic}
%  \end{algorithm}
%
%  At any point in time, the gate stack $\cG$ contains the path from a gate to the output gate that is currently explored.
%  In the value stack $\cV$ contains
%
%
%   The property that holds during the simulation is that the value in $\cV[i]$ is the value that $\cG[i]$ currently outputs. The algorithm ends with $\cG=\left[ g_{\texttt{root}}\right]$ and $\cV=\left[ v_{\texttt{root}}\right]$ after traversing the circuit, and returns $v_{\texttt{root}}$
%
% 

    % We assume the circuit has input gates, $+, \times$-gates and allow constant $1$-gate.


   %  For the pseudo-code, we supply ourselves with the following functions:
   %
   %  \begin{itemize}
   %      % \item[--] $\isplus{g}$: true if and only if $g$ is a $+$-gate.
   %      % \item[--] $\isprod{g}$: true if and only if $g$ is a $\times$-gate.
   %      % \item[--] $\isone{g}$: true if and only if $g$ is a $1$-gate.
   %      % \item[--] $\isinput{g}$: true if and only if $g$ is an input gate.
   %      % \item[--] $\getfirst{g}$: outputs the first child of $g$.
   %      % \item[--] $\getinput{g}$: outputs $A[i]$ when $g$ is the $i$-th input.
   %      % \item[--] $\isnotlast{g_1}{g_2}$: true if and only if $g_2$ is not the last child gate of $g_1$.
   %      % \item[--] $\nextgate{g_1}{g_2}$: outputs the next child gate of $g_1$ after $g_2$.
   %    %      \item[--] $\getroot$: outputs the root gate of the circuit.
   %  \end{itemize}
   %
   %  % The corresponding $\lbrace 0,1 \rbrace^n\rightarrow\lbrace 0,1 \rbrace^n$ functions are:
   % %
   % %  \begin{itemize}
   % %      \item[--] $\isplus{g}$: $1$ if and only if $g$ is a $+$-gate.
   % %      \item[--] $\isprod{g}$: $1$ if and only if $g$ is a $\times$-gate.
   % %      \item[--] $\isone{g}$: $1$ if and only if $g$ is a $1$-gate.
   % %      \item[--] $\isinput{g}$: $1$ if and only if $g$ is an input gate.
   % %      \item[--] $\getfirst{g}$: outputs the $\texttt{id}$ of the first child of $g$.
   % %      \item[--] $\getinput{g}$: outputs canonical vector $b_i$, where the $i$-th input gate of $\Phi_n$ is encoded by $g$.
   % %      \item[--] $\isnotlast{g_1}{g_2}$: $1$ if and only if $g_2$ is not the last child gate of $g_1$.
   % %      \item[--] $\nextgate{g_1}{g_2}$: outputs the $\texttt{id}$ of the next child gate of $g_1$ after $g_2$.
   % %      \item[--] $\getroot$: outputs the $\texttt{id}$ of the root gate of the circuit.
   % %  \end{itemize}
   % %
   % %  The previous functions are all definable by an $L$-transducer and can be defined from the $L$-transducer of $f$. Then, by proposition \ref{prop:transducer}, for each of these functions there is a \langfor expression that simulates them.
   %
   %  Now, we give the pseudo-code of the top-down evaluation. We define the functions $Initialize$ (algorithm \ref{alg:init_code}), $Aggregate$ (algorithm \ref{alg:agg_code}) and $Evaluate$ (algorithm \ref{alg:eval_code}). The main algorithm is $Evaluate$.
 




\subsubsection{\langfor simulation of algorithm {\normalfont \textsc{Evaluate}}}\label{subsubsec:simulate}
We next face the challenging task to simulate $\textsc{Evaluate}(\cG,\cV,A)$ in \langfor. It is here that the
assumption that  $\{\Phi_n\mid n=1,2,\ldots\}$ is a uniform arithmetic circuit family of logarithmic depth is crucial.
Indeed, this implies that every gate in $\Phi_n$ has an $\texttt{id}\in\lbrace 0,1 \rbrace^n$. In the following, when
when we write $g$ for a gate, we mean the $\texttt{id}$ encoding $g$. We can use these identifiers to impose an
ordering on gates, as needed for the evaluation algorithm. 

In order to simulate $\textsc{Evaluate}(\Phi_n,A)$ we also need to simulate the following
functions: $\isplus{g}$, $\isprod{g}$, $\isone{g}$, $\isinput{g}$, $\getfirst{g}$, $\getinput{g}$, $\isnotlast{g_1}{g_2}$,
\allowbreak $\nextgate{g_1}{g_2}$ and $\getroot$, all of which are explained in Algorithm~\ref{alg:eval_code}. 
We first explain how these are simulated in \langfor.

Using the identifiers of gates, we can regard these as functions $f:(\lbrace 0,1 \rbrace^{n})^\ell\rightarrow\lbrace 0,1 \rbrace^n$, in case they return some gate (identifier) or $f:(\lbrace 0,1 \rbrace^{n})^\ell\rightarrow\lbrace 0,1 \rbrace$,
in case they return some boolean value. Now, given the \logspace-Turing machine for $\{\Phi_n\mid n=1,2,\ldots\}$, it is readily verified that there exists a \textit{linear
space machine} $T$ such that for every $n\geq 0$, and $w_1,\ldots,w_\ell\in\{0,1\}$, $T(w_1,\ldots,w_\ell)$ has $f(w_1,\ldots,w_\ell)$ on its output tape. We defer the precise definition of linear space machines to the appendix.
Important here is that such machines and the functions that they compute  can be simulated in \langfor. We defer the
proof the appendix.

\begin{proposition} \label{prop:transducer2}
Let $f=\bigcup_{n\geq 0}f_n:(\{0,1\}^n)^\ell\to \{0,1\}^n$ be a function computable by a linear space machine
$T$.
 % $m$ states, $\ell$ input tapes
% , which consumes
% $\mathcal{O}(n)$ space and runs in $\mathcal{O}(n^{k-1})$ time on inputs of size $n$.
There exists (i)~a \langfor  
schema $\mathcal{S}=(\mathcal{M},\textsf{size})$ where $\mathcal{M}$ contains variables
$I_1,\ldots,I_\ell$ and $O$ all of size $\alpha\times 1$; and (ii)~a \langfor 
expression $e_f$ over $\mathcal{S}$ such that for the instance 
$\I=(\mathcal{D},\textsf{mat})$ over $\mathcal{S}$ with $\mathcal{D}(\alpha)=n$ and 
$\mathsf{mat}(R_i)=\mathsf{vec}(w_i)\in \mathbb{R}^n$,  for $i\in[\ell]$ and words $w_1,\ldots,w_\ell\in\Sigma^n$ and such that $\mathsf{vec}(w_i)$ is the $n\times 1$-vector 
encoding the word $w_i$, we have that  $\mathsf{mat}(O)=\mathsf{vec}(f_n(w_1,\ldots,w_\ell))\in\mathbb{R}^n$ 
after evaluating $e_f$ on $\I$. Similarly, if $f=\bigcup_{n\geq 0}f_n:(\{0,1\}^n)^\ell\to \{0,1\}$, then we have
such an expression $e_f$ such that $\mathsf{mat}(O)=f(w_1,\ldots,w_\ell)\in \mathbb{R}$ where $O$ is now of size $1\times 1$.\qed
\end{proposition}

In other words, when we represent the reverse binary identifiers of gates as $n\times 1$ vectors, all of the above-mentioned functions have a \langfor counterpart that either return again an $n\times 1$ vector, representing a gate, or a value $0$ or $1$, representing true or false. In what follows, we  abuse notation and identify the functions with their \langfor counterparts. For example, $\getfirst{e}$ expects a $n\times 1$ binary vector $b$ as input and returns again an $n\times 1$ binary vector encoding the first child of the gate with $b$ as identifier.

%
% The previous functions are all definable by an $L$-transducer and can be defined from the $L$-transducer of $f$. Then, by proposition \ref{prop:transducer}, for each of these functions there is a \langfor expression that simulates them.
%
%  of logarithmic depth. This implies that each we label each gate with an identifier in $\{0,1\}^n$ and can assume an ordering
% on the children of gates (e.g., lexicographically). We further assume  We will maintain two
%
%     Let $\Phi_n$ be a circuit with $n$ input gates and such that it can be computed by a $L-$uniform arithmetic circuit of log-depth. Each gate of the circuit that encodes $f$ has an $\texttt{id}\in\lbrace 0,1 \rbrace^n$. From now on, when we write $g$ for a gate of the circuit, we mean the $\texttt{id}$ encoding $g$.


We now turn our attention to simulating $\textsc{Evaluate}(\Phi_n,A)$. We will use a matrix variable $X$
in which we store all necessary information for the simulation. We split up the simulation in two parts:
first we assume that is $n$ is large enough such that we have specific bounds for the number of gates, wires
and depth of $\Phi_n$; second we deal with small $n$.

\paragraph{$\rhd$  Large $n$.}
 To simulate the two stacks $\cG$ and $\cV$ we use a matrix $X$ of dimensions $n \times n$, where
    \begin{itemize}
        \item Columns $1$ to $n-3$ contain binary identifiers of the gates in the stack $\cG$;
        \item Column $n-2$ is the stack of values where $X[1, n-2]$ is the bottom of the stack;
        \item Column $n-1$ will store a canonical vector that marks the top of stack $\cG$;
        \item Column $n$ will store a canonical vector that marks the top of stack $\cV$.
    \end{itemize}
\begin{example}
We illustrate the matrix $X$ encoding $\cG$ and $\cV$ in case that
$\getsize{\cG} = \getsize{\cV}$ (matrices below on the left) and  when $\getsize{\cG} = \getsize{\cV}+1$ (matrices below on the right), respectively. We also depict vectors $V$, $G_{\textsl{top}}$ and $V_{\textsl{top}}$
representing the contents of the value stack $\cV$ and the positions of the
top elements in the gate stack $\cG$ and value stack $\cV$, respectively. The matrix
$G$ (shown on the right) has the same dimensions as $X$ and encodes the stack $\cG$.
    %
    % If we have $j$ gates in the stack and currently $\getsize{\cG}=\getsize{\cV}$ then $X$ would look like:
$$ \underbrace{\left[\begin{smallmatrix}
        \texttt{id}_1 & v_1 & 0 & 0 \\
        \texttt{id}_2 & v_2 & 0 & 0 \\
        \vphantom{\int\limits^x}\smash{\vdots} &\vphantom{\int\limits^x}\smash{\vdots} & \vphantom{\int\limits^x}\smash{\vdots} & \vphantom{\int\limits^x}\smash{\vdots} \\
        \texttt{id}_{j-1} & v_{j-1} & 0 & 0 \\
        \texttt{id}_j & v_j & 1 & 1 \\
        0 & 0 & 0 & 0 \\
     \vphantom{\int\limits^x}\smash{\vdots} &\vphantom{\int\limits^x}\smash{\vdots} & \vphantom{\int\limits^x}\smash{\vdots} & \vphantom{\int\limits^x}\smash{\vdots} \\
        0 & 0 & 0 & 0
    \end{smallmatrix}\right]}_{X}, \underbrace{\left[\begin{smallmatrix}
        \vphantom{\texttt{id}_1}v_1  \\
        \vphantom{\texttt{id}_1}v_2 \\
        \vphantom{\int\limits^x}\smash{\vdots} \\
       \vphantom{\texttt{id}_1} v_{j-1} \\
      \vphantom{\texttt{id}_1}  v_j \\
        0 \\
       \vphantom{\int\limits^x}\smash{\vdots} \\
        0 
    \end{smallmatrix}\right]}_{V}, 
    \underbrace{\left[\begin{smallmatrix}
        \vphantom{\texttt{id}_1}0  \\
       \vphantom{\texttt{id}_1} 0 \\
        \vphantom{\int\limits^x}\smash{\vdots}   \\
       \vphantom{\texttt{id}_1} 0 \\
       \vphantom{\texttt{id}_1} 1 \\
       \vphantom{\texttt{id}_1} 0 \\
       \vphantom{\int\limits^x}\smash{\vdots} \\
       \vphantom{\texttt{id}_1} 0 
    \end{smallmatrix}\right]}_{G_{\textsl{top}}}, 
     \underbrace{\left[\begin{smallmatrix}
        \vphantom{\texttt{id}_1} 0  \\
         \vphantom{\texttt{id}_1}0 \\
         \vphantom{\int\limits^x}\smash{\vdots}  \\
        \vphantom{\texttt{id}_1} 0 \\
         \vphantom{\texttt{id}_1} 1 \\
         \vphantom{\texttt{id}_1}0 \\
         \vphantom{\int\limits^x}\smash{\vdots} \\
        \vphantom{\texttt{id}_1} 0 
    \end{smallmatrix}\right]}_{V_{\textsl{top}}} \text{ or }
	 \underbrace{\left[\begin{smallmatrix}
        \texttt{id}_1 & v_1 & 0 & 0 \\
        \texttt{id}_2 & v_2 & 0 & 0 \\
       \vphantom{\int\limits^x}\smash{\vdots} &\vphantom{\int\limits^x}\smash{\vdots} & \vphantom{\int\limits^x}\smash{\vdots} & \vphantom{\int\limits^x}\smash{\vdots} \\
        \texttt{id}_{j-1} & v_{j-1} & 0 & 1 \\
        \texttt{id}_j & 0 & 1 & 0 \\
        0 & 0 & 0 & 0 \\
    \vphantom{\int\limits^x}\smash{\vdots} &\vphantom{\int\limits^x}\smash{\vdots} & \vphantom{\int\limits^x}\smash{\vdots} & \vphantom{\int\limits^x}\smash{\vdots} \\
        0 & 0 & 0 & 0
    \end{smallmatrix}\right]}_{X}\underbrace{\left[\begin{smallmatrix}
        \vphantom{\texttt{id}_1}v_1  \\
        \vphantom{\texttt{id}_1}v_2 \\
         \vphantom{\int\limits^x}\smash{\vdots}  \\
       \vphantom{\texttt{id}_1} v_{j-1} \\
      \vphantom{\texttt{id}_1}  0 \\
        0 \\
       \vphantom{\int\limits^x}\smash{\vdots} \\
        0 
    \end{smallmatrix}\right]}_{V}, 
    \underbrace{\left[\begin{smallmatrix}
        \vphantom{\texttt{id}_1}0  \\
       \vphantom{\texttt{id}_1} 0 \\
        \vphantom{\int\limits^x}\smash{\vdots}  \\
       \vphantom{\texttt{id}_1} 0 \\
       \vphantom{\texttt{id}_1} 1 \\
       \vphantom{\texttt{id}_1} 0 \\
        \vphantom{\int\limits^x}\smash{\vdots} \\
       \vphantom{\texttt{id}_1} 0 
    \end{smallmatrix}\right]}_{G_{\textsl{top}}}, 
     \underbrace{\left[\begin{smallmatrix}
        \vphantom{\texttt{id}_1} 0  \\
         \vphantom{\texttt{id}_1}0 \\
      \vphantom{\int\limits^x}\smash{\vdots}   \\
        \vphantom{\texttt{id}_1} 1 \\
         \vphantom{\texttt{id}_1} 0 \\
         \vphantom{\texttt{id}_1}0 \\
         \vphantom{\int\limits^x}\smash{\vdots} \\
        \vphantom{\texttt{id}_1} 0 
    \end{smallmatrix}\right]}_{V_{\textsl{top}}} 
		\text{ and } \underbrace{\left[\begin{smallmatrix}
	        \texttt{id}_1 & 0 & 0 & 0 \\
	        \texttt{id}_2 & 0 & 0 & 0\\
	    \vphantom{\int\limits^x}\smash{\vdots} &\vphantom{\int\limits^x}\smash{\vdots} & \vphantom{\int\limits^x}\smash{\vdots} & \vphantom{\int\limits^x}\smash{\vdots} \\
	        \texttt{id}_{j-1} & 0 & 0 & 0\\
	        \texttt{id}_j & 0 & 0 & 0\\
	        0 & 0 & 0 & 0 \\
	    \vphantom{\int\limits^x}\smash{\vdots} &\vphantom{\int\limits^x}\smash{\vdots} & \vphantom{\int\limits^x}\smash{\vdots} & \vphantom{\int\limits^x}\smash{\vdots} \\
	        0 & 0 & 0 & 0
	    \end{smallmatrix}\right]}_{G},
   $$
Here, the gate identifiers $\texttt{id}_j$ consist of
 $n-3$ zero/one entries, so they take up $n-3$ columns.\qed
\end{example}

In order to ensure that the $n\times n$ matrix $X$ is large enough to simulate $\textsc{Evaluate}(\cG,\cV,A)$, we consider large enough $n$, i.e. $n\geq n_0$, such that there exists constants $k,k'\in\mathbb{N}$ satisfying: for any $n\geq n_0$ the circuits $\Phi_n$ satisfy the following:
$$        \mathsf{gates}(\Phi_n) \leq n^k-3 \quad  \mathsf{depth}(\Phi_n)\leq k' \ceil{\log(n)} \leq n       
		% \leq k \ceil{\log (n)} \leq n-3
		 \text{ and }
   2\mathsf{wires}(\Phi_n)\leq n^k  
$$
The first two conditions imply that we need at most $k\log (n)$ bits to store the $id$ of a gate and that we have sufficiently many rows to
represent the stacks (which are of size at most the depth of the circuit). The third condition ensures that $n^k$ is an upper bound on the number
of steps required to run $\textsc{Evaluate}(\cG,\cV,A)$, since during its execution maximally $2\mathsf{wire}(\Phi_n)$ will be traversed.
We note that
$n_0$, $k$ and $k'$ exist because $\{\Phi_n\mid n=1,2, \}$ is  uniform arithmetic circuit family of logarithmic depth. From now on, we assume
that $n\geq n_0$. 


% %
% % $$
% % we need at most $k\log (n)$ bits to store the $id$ of a gate.
% % $$
% % we assume that the circuits $\Phi_n$ we consider
% % are we have constants $k$ and $k'$ such that
% %
% %
% %     Since $n\geq n_0$, $(\star)$ holds and thus we never use more than $n-3$ bits to encode an $\texttt{id}$. Also, $j\leq n$ given that we never keep more gates than the depth of the tree. As a consequence, we never keep more than $n$ values either.
% %
% %  Let $n_0\in\mathbb{N}$ be big enough for ($\star$) to hold and let $n\geq k$. Hence, the number of gates (values) is bounded by $n^k$ and we need $k\log (n)$ bits to encode the id of each gate.
% %
% %
% %
% %     An important detail is that the $\texttt{ids}$ of the gates are encoded as $\texttt{id}_r000$ for it to have dimension $n$, where $\texttt{id}_r$ is the corresponding binary number in reverse.
% %
% %  Let $n^k$ be a polynomial such that the number of wires, $W(n)$ is smaller than $n^k$ for $n$ big enough. Furthermore, we assume that $2W(n)\leq n^k$. We need this because the \langfor simulation of the circuit is in a depth first search way, so $2W(n)$ wires will be traversed.
% %     Then we have that:
%     \begin{itemize}
%         \item the number of gates is bounded by $n^k$.
%         \item we need at most $k\log (n)$ bits to store the $id$ of a gate.
%         \item the depth of the circuit is at most $k'\log (n)$ for some $k'$.
%     \end{itemize}
%
%     So, let $n_0$ and $k$ such that $\forall n\geq n_0:$

    % \begin{align*}[right=\empheqrbrace (\star)]
    % \end{align*}

%     We know $n_0$ and $k$ exist. Let $n\geq n_0$.
% 	% Towards the end, we will deal with the case when $n<n_0$.
% %
% %     Let $g$ be a gate. The children of $g$ are denoted by $g_1,\ldots, g_l$.
% %% \begin{center}
% %     \begin{tikzpicture}[level distance=1.5cm,
% %     level 1/.style={sibling distance=1.5cm},
% %     every node/.style = {
% %         shape=circle,
% %         draw,
% %         align=center,
% %         top color=white,
% %         bottom color=white
% %         }]
% %     \node {\( g \)}
% %         child {node { \( g_1 \) }}
% %         child {node { \( \cdots \) }}
% %         child {node { \( g_l \) }};
% %     \end{tikzpicture}
% %     \end{center}
% %
% %     For example, a circuit that encodes the function $f(a_1,a_2,a_3,a_4)=a_1a_2 +a_3a_4$ is
% %
% %     \begin{center}
% %     \begin{tikzpicture}[level distance=1.5cm,
% %     level 1/.style={sibling distance=3cm},
% %     level 2/.style={sibling distance=1.5cm},
% %     every node/.style = {
% %         shape=circle,
% %         draw,
% %         align=center,
% %         top color=white,
% %         bottom color=white
% %         }
% %     ]
% %     \node { \( + \) }
% %         child { node { \( \times \) }
% %         child { node { \( a_1 \) } }
% %         child {node { \( a_2 \) } }
% %         }
% %         child { node { \( \times \) }
% %         child { node { \( a_3 \) } }
% %         child { node { \( a_4 \) } }
% %         };
% %     \end{tikzpicture}
% %     \end{center}
% %
% %     We can simulate the polynomial $x^2+xy$ by doing $f(A)$ where $A=[x \hspace{1ex} x \hspace{1ex} x \hspace{1ex} y]^T$. The main idea is to traverse the circuit top down in a depth first search way and store visited gates in a stack and its corresponding current values in another stack, and aggregate in the iterations according to the gate type.
%
%
%     Next, we show how to encode this algorithm in $\langfor$.
    %
    % We make a series of definitions to make the notation more clear. Refer to section \ref{sec:formatlang:design} for more information
    % about these expressions.
    %
    % Let $b_i$ be the $i$-th canonical vector. $\mathsf{Next}$ and $\mathsf{Prev}$ denote the successor and predecessor matrices respectively, such that
    %
    % \[
    %             \mathsf{Next}\cdot b_i=\begin{cases}
    %             b_{i+1} \text{ if } i\leq n \\
    %             \mathbf{0} \text{ otherwise }
    %             \end{cases}
    % \]
    %
    % \[
    %             \mathsf{Prev}\cdot b_i=\begin{cases}
    %             b_{i-1} \text{ if } i\geq n \\
    %             \mathbf{0} \text{ otherwise }
    %             \end{cases}
    % \]
    %
    % We write expressions $e_{\mathsf{min}}$ for the first canonical vector and $e_{\mathsf{max}}$ for the last canonical vector. For any $i$ we write
    % \begin{align*}
    %     e_{\mathsf{min}+\mathsf{i}} &= \mathsf{Next}^i\cdot e_{\mathsf{min}} \\
    %     e_{\mathsf{max}+\mathsf{i}} &= \mathsf{Prev}^i\cdot e_{\mathsf{max}}
    % \end{align*}
    %
    % We use the extra $\lbrace 0,1 \rbrace^n\rightarrow\lbrace 0,1 \rbrace^n$ functions that have a $\langfor$ translation:
    %
    % \[
    %             \mathsf{min}(e)=\begin{cases}
    %             1 \text{ if } e=e_{\mathsf{min}} \\
    %             0 \text{ otherwise }
    %             \end{cases}
    % \]
    %
    % \[
    %             \mathsf{max}(e)=\begin{cases}
    %             1 \text{ if } e=e_{\mathsf{max}} \\
    %             0 \text{ otherwise }
    %             \end{cases}
    % \]
    %
    % \[
    %             \mathsf{succ}(b_i,b_j)=\begin{cases}
    %             1 \text{ if } i\leq j \\
    %             0 \text{ otherwise }
    %             \end{cases}
    % \]
    %
    % When used in $\langfor$ these functions output $[0]$ and $[1]$.
	
We are now ready to start simulating $\textsc{Evaluate}(\cG,\cV,A)$. Intuitively, we will iterate over $k$ canonical vectors and perform
$n^k$ iterative updates to the matrix $X$, starting from the initial matrix $X$ only storing the output gate id and value $1$ in column $n-1$
(indicating the top of the stack $\cG$). After performing the updates, we end up with a matrix $X$ such that $X[1,n-2]$ holds the desired
evaluation of $\Phi_n$ on $A$. As stated in Theorem~\ref{th-circuits-ml}, we use the vector variable $v$ to encode the input values in $A$, i.e. we consider
an instance satisfying $\conc(v) = [a_1 \ldots a_n]^T$. The resulting \langfor expressions $e_\Phi$ relies on several other expressions. Indeed, alongside
the expressions that extract gate information (see Proposition~\ref{prop:transducer2} above), we also use \langfor expressions that leverage the order on canonical vectors, as explained in Section~\ref{sec:formatlang:design}. 

In particular, we use  $e_{\mathsf{min}+\mathsf{i}}$ and  $e_{\mathsf{max}+\mathsf{i}}$, for $i=0,\ldots,n-1$,
to obtain the $i$th and $(n-i)$th canonical vector, respectively, and also $\mathsf{Next}$ and $\mathsf{Prev}$, expressions that return matrices that allow to
move from one canonical vector to the next, respectively, previous, one. In addition, $\mathsf{min}$, $\mathsf{max}$ and $\mathsf{succ}$ are expressions returning
$1$ when their input vectors are the first canonical vector, the last canonical vectors, and consecutive canonical vectors, respectively. Given these, we use
the following shorthand notations:
\begin{gather*}\allowdisplaybreaks
   e_{V}:=e_{\mathsf{max}-2} \quad e_{G_{top}}:=e_{\mathsf{max}-1} \quad     e_{V_{top}}:=e_{\mathsf{max}} \\
   \Iden{b_i}:=\ssum v. \mathsf{succ}(v,b_i)\cdot (v\cdot v^T)  \quad   V_{top}:= X\cdot e_{V_{top}} \\
        V:= \Iden{V_{top}} \cdot X \cdot e_V \quad
        G_{top} :=X\cdot e_{G_{top}} \text{ and }
        G := \Iden{G_{top}}\cdot X \cdot \Iden{e_{\mathsf{max}-3}}
 \end{gather*}

    % For a canonical vector, let $$.$$ This matrix has ones in the diagonal up to position $i$ marked by $e_{i}$. We define the following sub-matrices of $X$:
%     \begin{align*}
%
%     \end{align*}
%
%     % For example, if we are in a step where $\getsize{\cG}=\getsize{\cV} + 1$ then
%     %
%     % \[
%     % X = \begin{bmatrix}
%     %     \texttt{id}_1 & v_1 & 0 & 0 \\
%     %     \texttt{id}_2 & v_2 & 0 & 0 \\
%     %     \vdots & \vdots & \vdots & \vdots \\
%     %     \texttt{id}_{j-1} & v_{j-1} & 0 & 1 \\
%     %     \texttt{id}_j & 0 & 1 & 0 \\
%     %     0 & 0 & 0 & 0 \\
%     %     \vdots & \vdots & \vdots & \vdots \\
%     %     0 & 0 & 0 & 0
%     % \end{bmatrix},
%    $$ G = \begin{bmatrix}
%         \texttt{id}_1 & 0 & 0 & 0 \\
%         \texttt{id}_2 & 0 & 0 & 0\\
%         \vdots & \vdots & \vdots & \vdots  \\
%         \texttt{id}_{j-1} & 0 & 0 & 0\\
%         \texttt{id}_j & 0 & 0 & 0\\
%         0 & 0 & 0 & 0 \\
%         \vdots & \vdots & \vdots & \vdots \\
%         0 & 0 & 0 & 0
%     \end{bmatrix},$$
% $$ % V = \begin{bmatrix}
%     %     v_1  \\
%     %     v_2 \\
%     %     \vdots   \\
%     %     v_{j-1} \\
%     %     0 \\
%     %     0 \\
%     %     \vdots \\
%     %     0
%     % \end{bmatrix},
%     % G_{top} = \begin{bmatrix}
%     %     0  \\
%     %     0 \\
%     %     \vdots   \\
%     %     0 \\
%     %     1 \\
%     %     0 \\
%     %     \vdots \\
%     %     0
%     % \end{bmatrix},
%     % V_{top} = \begin{bmatrix}
%     %     0  \\
%     %     0 \\
%     %     \vdots   \\
%     %     1 \\
%     %     0 \\
%     %     0 \\
%     %     \vdots \\
%     %     0
%     % \end{bmatrix}
%     % \]

    % Here, $V$ is a vector encoding the stack of values in $X$ and $G$ is a matrix encoding the stack of gates in $X$.
    % Note that what is \textit{over} the top of the stacks is always set to zero due to $\Iden{G_{top}}$ and $\Iden{V_{top}}$.
    % Also, note that $G$ is of the same size as $X$. We sometimes omit the zeroes due to simplicity.

    %
    % The $Evaluate$ method (algorithm \ref{alg:eval_code}) is defined as follows:

We simulate $\textsc{Evaluate}(\cG,\cV,A)$ by the expression
$e_{\Phi}:= e_{\mathsf{min}}^T\cdot e_{\mathsf{iterate}}\cdot e_{\max -2}$ with
    \begin{align*}\allowdisplaybreaks
		e_{\mathsf{iterate}}&:= \ffor{v_1}{X} X+\\
		& \qquad\ffor{v_2}{X_2=X} X_2+ \\
		& \qquad \quad \ffor{v_3}{X_3=X_2} X_3+ \cdots\\
		& \qquad \qquad\quad \ffor{v_k}{X_{k}=X_{k-1}} \Biggl(\Bigl(\bigl( \sprod_{i=1}^k \mathsf{min}(v_i)\bigr) \times e_{\mathsf{start}}\Bigr) + 
        \biggl( \bigl(1- \sprod_{i=1}^k \mathsf{min}(v_i)\bigr)\times{}\\
        &\hspace{0.3em} \biggl( \bigl(1 -\mathsf{min}(G_{\mathsf{top}})\cdot \mathsf{min}(V_{\mathsf{top}})\bigr) \times \Bigl(\bigl( 1 - G_{\mathsf{top}}^T\cdot V_{\mathsf{top}}\bigr) \times e_{\mathsf{extend}} + 
        \bigl(G_{\mathsf{top}}^T\cdot V_{\mathsf{top}}\bigr) \times e_{\mathsf{aggregate}}\Bigl)  \\
        &\hspace{8cm}+\bigl(\mathsf{min}(G_{\mathsf{top}})\cdot \mathsf{min}(V_{\mathsf{top}})\bigr)\times X\biggr)\Biggr).
	   \end{align*}
In the expression $e_{\mathsf{iterate}}$, when all $v_1,\ldots,v_k$ are initially equal to $b_1$, we are at ``timestamp'' $0$ and start the evaluation by 
calling expression $e_{\mathsf{start}}$ which encodes line 3 of the algorithm. This case is determined by the indicator function
$ \sprod_{i=1}^k \mathsf{min}(v_i)$. For all subsequent assignments of $v_1,\ldots,v_k$, when $1- \sprod_{i=1}^k \mathsf{min}(v_i)$ is equal to $1$, and thus all
later timestamps, we check whether or not we reach the end of the algorithm. This happens with both $G_{\mathsf{top}}$ and $V_{\mathsf{top}}$ are equal to $b_1$
and thus $\size(\cG)=\size(\cV)=1$. This is checked by $G_{\mathsf{top}}^T\cdot V_{\mathsf{top}}$.
In that case, we stop updating $X$ (last summand in expression $e_{\mathsf{iterate}}$). Otherwise, when $1-G_{\mathsf{top}}^T\cdot V_{\mathsf{top}}$ is $1$, we check whether $\size(\cG)=\size(\cV)$ or $\size(\cG)\neq \size(\cV)$. In the former case, we call expression $e_{\mathsf{aggregate}}$, which encodes the function $\textsc{Aggregate}(\cG,\cV)$, in the latter case we call expression $e_{\mathsf{extend}}$, which encodes function $\textsc{Extend}(\cG,\cV,A)$. These two cases are checked by the indicator functions $G_{\mathsf{top}}^T\cdot V_{\mathsf{top}}$ and  $1 - G_{\mathsf{top}}^T\cdot V_{\mathsf{top}}$, respectively. As mentioned before, $e_\Phi$ retrieves entry $X[1,\max-2]$ from the value stack column in $X$. 
\begin{figure}
	$$
\underbrace{\left[\begin{smallmatrix}
	       \texttt{id}_1 & 0 & 1 & 0\,\, \\
	       \vphantom{\texttt{id}_1} 0 & 0 & 0 & 0 \\
	        \vphantom{\int\limits^x}\smash{\vdots} &\vphantom{\int\limits^x}\smash{\vdots} & \vphantom{\int\limits^x}\smash{\vdots} & \vphantom{\int\limits^x}\smash{\vdots} \\
	        \vphantom{\texttt{id}_1} 0 & 0 & 0 & 0 \\
			 \vphantom{\texttt{id}_1}   0 & 0 & 0 & 0 \\
 	        \vphantom{\texttt{id}_1} 0 & 0 & 0 & 0 \\
 			 \vphantom{\texttt{id}_1}   0 & 0 & 0 & 0 \\
          \vphantom{\int\limits^x}\smash{\vdots} &\vphantom{\int\limits^x}\smash{\vdots} & \vphantom{\int\limits^x}\smash{\vdots} & \vphantom{\int\limits^x}\smash{\vdots} \\
			   0 & 0 & 0 & 0
	       \end{smallmatrix}\right]}_{e_{\mathsf{start}}}
		       \underbrace{\left[\begin{smallmatrix}
        \texttt{id}_1 & v_1 & 0 & 0 \, \\
        \texttt{id}_2 & v_2 & 0 & 0 \\
        \vphantom{\int\limits^x}\smash{\vdots} &\vphantom{\int\limits^x}\smash{\vdots} & \vphantom{\int\limits^x}\smash{\vdots} & \vphantom{\int\limits^x}\smash{\vdots} \\
        \texttt{id}_{j-1} & v_{j-1} & 0 & 1 \\
        \texttt{id}_j & 0 & 1 & 0 \\
       \vphantom{\texttt{id}_1} 0 & 0 & 0 & 0 \\
        0 & 0 & 0 & 0 \\
        \vphantom{\int\limits^x}\smash{\vdots} &\vphantom{\int\limits^x}\smash{\vdots} & \vphantom{\int\limits^x}\smash{\vdots} & \vphantom{\int\limits^x}\smash{\vdots} \\
        0 & 0 & 0 & 0
    \end{smallmatrix}\right]\mapsto
	\left[\begin{smallmatrix}
        \texttt{id}_1 & v_1 & 0 & 0 \\
        \texttt{id}_2 & v_2 & 0 & 0 \\
       \vphantom{\int\limits^x}\smash{\vdots} &\vphantom{\int\limits^x}\smash{\vdots} & \vphantom{\int\limits^x}\smash{\vdots} & \vphantom{\int\limits^x}\smash{\vdots} \\
        \texttt{id}_{j-1} & v_{j-1} & 0 & 0 \\
        \texttt{id}_j & v_j & 0 & 1 \\
        \texttt{id}_{j+1} & 0 & 1 & 0 \\
        0 & 0 & 0 & 0 \\
         \vphantom{\int\limits^x}\smash{\vdots} &\vphantom{\int\limits^x}\smash{\vdots} & \vphantom{\int\limits^x}\smash{\vdots} & \vphantom{\int\limits^x}\smash{\vdots} \\
        0 & 0 & 0 & 0
    \end{smallmatrix}\right]}_{\text{Result of $e_{\mathsf{ext\_plus}}$}}
		       \underbrace{\left[\begin{smallmatrix}
        \texttt{id}_1 & v_1 & 0 & 0 \, \\
        \texttt{id}_2 & v_2 & 0 & 0 \\
        \vphantom{\int\limits^x}\smash{\vdots} &\vphantom{\int\limits^x}\smash{\vdots} & \vphantom{\int\limits^x}\smash{\vdots} & \vphantom{\int\limits^x}\smash{\vdots} \\
        \texttt{id}_{j-1} & v_{j-1} & 0 & 1 \\
        \texttt{id}_j & 0 & 1 & 0 \\
       \vphantom{\texttt{id}_1} 0 & 0 & 0 & 0 \\
        0 & 0 & 0 & 0 \\
        \vphantom{\int\limits^x}\smash{\vdots} &\vphantom{\int\limits^x}\smash{\vdots} & \vphantom{\int\limits^x}\smash{\vdots} & \vphantom{\int\limits^x}\smash{\vdots} \\
        0 & 0 & 0 & 0
    \end{smallmatrix}\right]\mapsto
	\left[\begin{smallmatrix}
        \texttt{id}_1 & v_1 & 0 & 0 \\
        \texttt{id}_2 & v_2 & 0 & 0 \\
       \vphantom{\int\limits^x}\smash{\vdots} &\vphantom{\int\limits^x}\smash{\vdots} & \vphantom{\int\limits^x}\smash{\vdots} & \vphantom{\int\limits^x}\smash{\vdots} \\
        \texttt{id}_{j-1} & v_{j-1} & 0 & 0 \\
        \texttt{id}_j & v_j & 0 & 1 \\
        \texttt{id}_{j+1} & 1 & 1 & 0 \\
        0 & 0 & 0 & 0 \\
         \vphantom{\int\limits^x}\smash{\vdots} &\vphantom{\int\limits^x}\smash{\vdots} & \vphantom{\int\limits^x}\smash{\vdots} & \vphantom{\int\limits^x}\smash{\vdots} \\
        0 & 0 & 0 & 0
    \end{smallmatrix}\right]}_{\text{Result of $e_{\mathsf{ext\_prod}}$}}$$
$$ \underbrace{\left[\begin{smallmatrix}
        \texttt{id}_1 & v_1 & 0 & 0 \\
        \texttt{id}_2 & v_2 & 0 & 0 \\
          \vphantom{\int\limits^x}\smash{\vdots} &\vphantom{\int\limits^x}\smash{\vdots} & \vphantom{\int\limits^x}\smash{\vdots} & \vphantom{\int\limits^x}\smash{\vdots} \\
        \texttt{id}_{j-1} & v_{j-1} & 0 & 1 \\
        \texttt{id}_j & 0 & 1 & 0 \\
       \vphantom{\texttt{id}_1}  0 & 0 & 0 & 0 \\
	    \vphantom{\texttt{id}_1}  0 & 0 & 0 & 0 \\
        \vphantom{\int\limits^x}\smash{\vdots} &\vphantom{\int\limits^x}\smash{\vdots} & \vphantom{\int\limits^x}\smash{\vdots} & \vphantom{\int\limits^x}\smash{\vdots} \\
        0 & 0 & 0 & 0
    \end{smallmatrix}\right]\mapsto
    \left[\begin{smallmatrix}
        \texttt{id}_1 & v_1 & 0 & 0 \\
        \texttt{id}_2 & v_2 & 0 & 0 \\
        \vphantom{\int\limits^x}\smash{\vdots} &\vphantom{\int\limits^x}\smash{\vdots} & \vphantom{\int\limits^x}\smash{\vdots} & \vphantom{\int\limits^x}\smash{\vdots} \\
        \texttt{id}_{j-1} & v_{j-1} & 0 & 0 \\
        \texttt{id}_j & 1 & 1 & 1 \\
       \vphantom{\texttt{id}_1}  0 & 0 & 0 & 0 \\
	    \vphantom{\texttt{id}_1}  0 & 0 & 0 & 0 \\
         \vphantom{\int\limits^x}\smash{\vdots} &\vphantom{\int\limits^x}\smash{\vdots} & \vphantom{\int\limits^x}\smash{\vdots} & \vphantom{\int\limits^x}\smash{\vdots} \\
        0 & 0 & 0 & 0
    \end{smallmatrix}\right]}_{\text{Result of $e_{\mathsf{ext\_one}}$}}
	\underbrace{\left[\begin{smallmatrix}
	        \texttt{id}_1 & v_1 & 0 & 0 \\
	        \texttt{id}_2 & v_2 & 0 & 0 \\
	          \vphantom{\int\limits^x}\smash{\vdots} &\vphantom{\int\limits^x}\smash{\vdots} & \vphantom{\int\limits^x}\smash{\vdots} & \vphantom{\int\limits^x}\smash{\vdots} \\
	        \texttt{id}_{j-1} & v_{j-1} & 0 & 1 \\
	        \texttt{id}_j & 0 & 1 & 0 \\
	       \vphantom{\texttt{id}_1}  0 & 0 & 0 & 0 \\
		    \vphantom{\texttt{id}_1}  0 & 0 & 0 & 0 \\
	        \vphantom{\int\limits^x}\smash{\vdots} &\vphantom{\int\limits^x}\smash{\vdots} & \vphantom{\int\limits^x}\smash{\vdots} & \vphantom{\int\limits^x}\smash{\vdots} \\
	        0 & 0 & 0 & 0
	    \end{smallmatrix}\right]\mapsto
	    \left[\begin{smallmatrix}
	        \texttt{id}_1 & v_1 & 0 & 0 \\
	        \texttt{id}_2 & v_2 & 0 & 0 \\
	        \vphantom{\int\limits^x}\smash{\vdots} &\vphantom{\int\limits^x}\smash{\vdots} & \vphantom{\int\limits^x}\smash{\vdots} & \vphantom{\int\limits^x}\smash{\vdots} \\
	        \texttt{id}_{j-1} & v_{j-1} & 0 & 0 \\
	        \texttt{id}_j & v_j & 1 & 1 \\
	       \vphantom{\texttt{id}_1}  0 & 0 & 0 & 0 \\
		    \vphantom{\texttt{id}_1}  0 & 0 & 0 & 0 \\
	         \vphantom{\int\limits^x}\smash{\vdots} &\vphantom{\int\limits^x}\smash{\vdots} & \vphantom{\int\limits^x}\smash{\vdots} & \vphantom{\int\limits^x}\smash{\vdots} \\
	        0 & 0 & 0 & 0
	    \end{smallmatrix}\right]}_{\text{Result of $e_{\mathsf{ext\_input}}$}}
	$$
\caption{Result of updating $X$ by means of the \langfor expressions encoding the initialization by \textsc{Evaluate} and the different cases in \textsc{Extend}.}\label{fig:langforevalalg1}
\end{figure}
It now remains us to explain the expressions $e_{\mathsf{start}}$, $e_{\mathsf{extend}}$ and $e_{\mathsf{aggregate}}$.

    %
    %
    %
    %
    % Note that the $ \texttt{for}$-expression does the evaluation. The final output is in $X[1,max-2]$, we extract this value by multiplying the final result as $e_{\mathsf{min}}^T\cdot [\texttt{for}(\ldots )]\cdot e_{V}$.
Expression $\varphi_{\mathsf{start}}$ is used to set to matrix $X$ in the initial state, conform to line 3 in \textsc{Evaluate}. More precisely,
    % To set the initial state (algorithm \ref{alg:eval_code} line 2) we define the $\langfor$ expressio
$$e_{\mathsf{start}}:=e_{\mathsf{min}}\cdot \getroot^T + e_{\mathsf{min}}\cdot e_{G_{\mathsf{top}}}^T.$$
In other words, it will put the id of the output gate on the first row of $X$ and also sets the $n-1$ entry to $1$,
as depicted in Figure~\ref{fig:langforevalalg1} (left). To simulate $\textsc{Extend}$ we define
$$
e_{\mathsf{extend}}:=e_{\mathsf{ext\_plus}}+e_{\mathsf{ext\_prod}}+e_{\mathsf{ext\_one}}+e_{\mathsf{ext\_input}},
$$
where
    \begin{align*}\allowdisplaybreaks
      e_{\mathsf{ext\_plus}}&:= \texttt{isplus}\bigl(G^T\cdot G_{\mathsf{top}}\bigr)\times \Bigl( G + 
       \mathsf{Next}\cdot G_{\mathsf{top}} \cdot \texttt{getfirst}\bigl(G^T\cdot G_{top}\bigr)^T  +{} \\
        &\hspace{4cm}\mathsf{Next}\cdot G_{\mathsf{top}}\cdot e_{G_{\mathsf{top}}}^T +V\cdot e_{V}^T + \mathsf{Next}\cdot V_{top}\cdot e_{V_{top}}^T \Bigr) \\
  \displaybreak[0]      e_{\mathsf{ext\_prod}}&:= \texttt{isprod}\bigl(G^T\cdot G_{\mathsf{top}}\bigr)\times \Bigl(G +
        \mathsf{Next}\cdot G_{\mathsf{top}} \cdot \texttt{getfirst}\bigl(G^T\cdot G_{\mathsf{top}}\bigr)^T + {}\\
        &\hspace{2cm}\mathsf{Next}\cdot G_{\mathsf{top}}\cdot e_{G_{\mathsf{top}}}^T +(V + \mathsf{Next}\cdot V_{\mathsf{top}})\cdot e_{V}^T + \mathsf{Next}\cdot V_{\mathsf{top}}\cdot e_{V_{\mathsf{top}}}^T \Bigr) \\
        e_{\mathsf{ext\_const}}&:= \texttt{isone}\bigl(G^T\cdot G_{\mathsf{top}}\bigr)\times \Bigl(G + (V + \mathsf{Next}\cdot V_{\mathsf{top}})\cdot e_{V}^T + \mathsf{Next}\cdot V_{\mathsf{top}}\cdot e_{V_{\mathsf{top}}}^T \Bigr) \\
        e_{\mathsf{ext\_input}}&:= \texttt{isinput}\bigl(G^T\cdot G_{\mathsf{top}}\bigr)\times \\
		&\biggl(G + 
        \Bigl(V + \bigl( v^T \cdot \mathsf{Next}\cdot V_{\mathsf{top}} \cdot \texttt{getinput}\bigl(G^T\cdot G_{\mathsf{top}}\bigr)^T \bigr)\Bigr)\cdot e_{V}^T + \mathsf{Next}\cdot V_{\mathsf{top}}\cdot e_{V_{\mathsf{top}}}^T \biggr).
    \end{align*} 
Here, $e_{\mathsf{ext\_plus}}$, $e_{\mathsf{ext\_prod}}$, $e_{\mathsf{ext\_one}}$ and $e_{\mathsf{ext\_input}}$ correspond to the different cases in \textsc{Extend}, and lines 11--13, 14--16, 17--18, and 19-20 in Algorithm~\ref{alg:eval_code}, respectively. We have depicted their effect on the matrix $X$ in Figure~\ref{fig:langforevalalg1} (middle, right). We remark that only $e_{\mathsf{ext\_one}}$
use the input variable $v$ which holds the input values in $A$ of the circuit. 

We next unfold these expressions and explain their workings.
Here, $G^T\cdot G_{top}$ is to get the current id in the top of the stack. In INIT${\_}$PLUS we get the current stack $G$, we add $\mathsf{Next}\cdot G_{top} \cdot \getfirst{G^T\cdot G_{top}}^T$ which is an $n\times n$ matrix with the first child of $G^T\cdot G_{top}$ in the next row. Then $\mathsf{Next}\cdot G_{top}\cdot e_{G_{top}}^T$ adds $\mathsf{Next}\cdot G_{top}$ to the $n-1$ column to mark the gate we added as the top. Next, we do the same with the values by adding $V\cdot e_{V} + \mathsf{Next}\cdot V_{top}\cdot e_{V_{top}}^T$.

    %
    %
    % For the initialize step, we define the $\langfor$ expressions: INIT${\_}$PLUS (algorithm \ref{alg:init_code}, lines 2, 3, 4), INIT${\_}$PROD (algorithm \ref{alg:init_code}, lines 5, 6, 7), CONST (algorithm \ref{alg:init_code}, lines 8, 9) and INPUT (algorithm \ref{alg:init_code}, lines 10, 11):

 
    % Where $v$ is the matrix variable stated in the theorem, the one associated with the input $A$ of the circuit.

    % The $\langfor$ expression equivalent to algorithm \ref{alg:init_code} is $$\text{INIT}:=\text{INIT{\_}PLUS}+\text{INIT{\_}PROD}+\text{CONST}+\text{INPUT}.$$
    %
    % The idea is to return the matrix for the next iteration. Recall that here $\getsize{\cG}=\getsize{\cV} + 1$. So, when the operation is INPUT or CONST, if we start with
   %
   %  \[
   %  \begin{bmatrix}
   %      \texttt{id}_1 & v_1 & 0 & 0 \\
   %      \texttt{id}_2 & v_2 & 0 & 0 \\
   %      \vdots & \vdots & \vdots & \vdots \\
   %      \texttt{id}_{j-1} & v_{j-1} & 0 & 1 \\
   %      \texttt{id}_j & 0 & 1 & 0 \\
   %      0 & 0 & 0 & 0 \\
   %      \vdots & \vdots & \vdots & \vdots \\
   %      0 & 0 & 0 & 0
   %  \end{bmatrix}, \text{ then we return }
   %  \begin{bmatrix}
   %      \texttt{id}_1 & v_1 & 0 & 0 \\
   %      \texttt{id}_2 & v_2 & 0 & 0 \\
   %      \vdots & \vdots & \vdots & \vdots \\
   %      \texttt{id}_{j-1} & v_{j-1} & 0 & 0 \\
   %      \texttt{id}_j & v_j & 1 & 1 \\
   %      0 & 0 & 0 & 0 \\
   %      \vdots & \vdots & \vdots & \vdots \\
   %      0 & 0 & 0 & 0
   %  \end{bmatrix}.
   %  \]
   %
   %  When the operation is INIT{\_}PLUS or INIT{\_}PROD, if we start with
   %
   %  \[
   %  \begin{bmatrix}
   %      \texttt{id}_1 & v_1 & 0 & 0 \\
   %      \texttt{id}_2 & v_2 & 0 & 0 \\
   %      \vdots & \vdots & \vdots & \vdots \\
   %      \texttt{id}_{j-1} & v_{j-1} & 0 & 1 \\
   %      \texttt{id}_j & 0 & 1 & 0 \\
   %      0 & 0 & 0 & 0 \\
   %      0 & 0 & 0 & 0 \\
   %      \vdots & \vdots & \vdots & \vdots \\
   %      0 & 0 & 0 & 0
   %  \end{bmatrix}, \text{ then we return }
   %  \begin{bmatrix}
   %      \texttt{id}_1 & v_1 & 0 & 0 \\
   %      \texttt{id}_2 & v_2 & 0 & 0 \\
   %      \vdots & \vdots & \vdots & \vdots \\
   %      \texttt{id}_{j-1} & v_{j-1} & 0 & 0 \\
   %      \texttt{id}_j & v_j & 0 & 1 \\
   %      \texttt{id}_{j+1} & 0 & 1 & 0 \\
   %      0 & 0 & 0 & 0 \\
   %      \vdots & \vdots & \vdots & \vdots \\
   %      0 & 0 & 0 & 0
   %  \end{bmatrix}.
   %  \]


    The $\langfor$ expression equivalent to algorithm \ref{alg:agg_code} is 
	$$
	e_{\mathsf{aggregate}}:=e_{\mathsf{pop}}+e_{\mathsf{agg\_plus}}+e_{\mathsf{agg\_prod}}+e_{\mathsf{agg\_not\_last}}+e_{\mathsf{agg\_last}},
	$$
	% $$\text{AGG}:=\text{POP} + \text{AGG{\_}PLUS}+\text{AGG{\_}PROD}+\text{IS{\_}NOT{\_}LAST}+\text{IS{\_}LAST}.$$
	%
	%
	%     For the aggregate expression (algorithm \ref{alg:agg_code}) we do the following. Let $$\pondIden{b_i}{c}=\ssum v. (v^T\cdot b_i)\cdot c\cdot v\cdot v^T + (1-v^T\cdot b_i)\cdot v \cdot v^T,$$ namely, it is the identity with $c$ in position $(i,i)$.
	%
	%     We define the expressions: AGG${\_}$PLUS (algorithm \ref{alg:agg_code}, lines 4, 5), AGG${\_}$PROD (algorithm \ref{alg:agg_code}, lines 6, 7),  IS${\_}$NOT${\_}$LAST (algorithm \ref{alg:agg_code}, lines 8, 9), IS${\_}$LAST and POP:
where
    \begin{align*}\allowdisplaybreaks
        e_{\mathsf{pop}} &:= \Iden{\mathsf{Prev}\cdot G_{\mathsf{top}}}\cdot G + \mathsf{Prev}\cdot V_{\mathsf{top}}\cdot e_{V_{\mathsf{top}}}^T  \\
        e_{\mathsf{agg\_plus}}&:= \texttt{isplus}\bigl(G^T \cdot P \cdot G_{\mathsf{top}}\bigr) \times \biggl(\Bigl(\Iden{\mathsf{Prev}\cdot V_{\mathsf{top}}} \cdot V + \bigl( V^T \cdot V_{\mathsf{top}} \bigr)\cdot \bigl(\mathsf{Prev}\cdot V_{\mathsf{top}}\bigr)\Bigr) \cdot e_{V}^T \biggr) \\
        e_{\mathsf{agg\_prod}}&:= \isprod{G^T \cdot P \cdot G_{\mathsf{top}}} \times \biggl( \pondIden{\mathsf{Prev}\cdot V_{\mathsf{top}}}{V^T \cdot V_{\mathsf{top}}} \cdot \Iden{\mathsf{Prev}\cdot V_{top}} \cdot V  \cdot e_{V}^T \biggr) \\
e_{\mathsf{agg\_not\_last}} &:= \texttt{isnotlast}\bigl(G^T \cdot P \cdot G_{\mathsf{top}},G^T \cdot G_{\mathsf{top}}\bigr) \\
&\hspace{3cm}\times \biggl(G_{\mathsf{top}} \cdot \texttt{nextgate}\bigl(G^T \cdot \mathsf{Prev}\cdot G_{\mathsf{top}}, ,G^T \cdot G_{\mathsf{top}}\bigr)^T + G_{\mathsf{top}}\cdot e_{G_{\mathsf{top}}}^T \biggr) \\
 e_{\mathsf{agg\_last}}&:= \Bigl( 1 - \texttt{isnotlast}\bigl(G^T \cdot P \cdot G_{\mathsf{top}},G^T \cdot G_{\mathsf{top}}\bigr)\Bigr)\times \Bigl(\mathsf{Prev}\cdot G_{\mathsf{top}} \cdot e_{G_{\mathsf{top}}}^T \Bigr).
    \end{align*}

   
  
\begin{figure}
	$$
% \underbrace{\left[\begin{smallmatrix}
% 	       \texttt{id}_1 & 0 & 1 & 0\,\, \\
% 	       \vphantom{\texttt{id}_1} 0 & 0 & 0 & 0 \\
% 	        \vphantom{\int\limits^x}\smash{\vdots} &\vphantom{\int\limits^x}\smash{\vdots} & \vphantom{\int\limits^x}\smash{\vdots} & \vphantom{\int\limits^x}\smash{\vdots} \\
% 	        \vphantom{\texttt{id}_1} 0 & 0 & 0 & 0 \\
% 			 \vphantom{\texttt{id}_1}   0 & 0 & 0 & 0 \\
%  	        \vphantom{\texttt{id}_1} 0 & 0 & 0 & 0 \\
%  			 \vphantom{\texttt{id}_1}   0 & 0 & 0 & 0 \\
%           \vphantom{\int\limits^x}\smash{\vdots} &\vphantom{\int\limits^x}\smash{\vdots} & \vphantom{\int\limits^x}\smash{\vdots} & \vphantom{\int\limits^x}\smash{\vdots} \\
% 			   0 & 0 & 0 & 0
% 	       \end{smallmatrix}\right]}_{e_{\mathsf{pop}}}
		       \underbrace{\left[\begin{smallmatrix}
        \texttt{id}_1 & v_1 & 0 & 0 \, \\
        \texttt{id}_2 & v_2 & 0 & 0 \\
        \vphantom{\int\limits^x}\smash{\vdots} &\vphantom{\int\limits^x}\smash{\vdots} & \vphantom{\int\limits^x}\smash{\vdots} & \vphantom{\int\limits^x}\smash{\vdots} \\
        \texttt{id}_{j-1} & v_{j-1} & 0 & 0 \\
        \texttt{id}_j & v_j & 1 & 1 \\
            0 & 0 & 0 & 0 \\
        \vphantom{\int\limits^x}\smash{\vdots} &\vphantom{\int\limits^x}\smash{\vdots} & \vphantom{\int\limits^x}\smash{\vdots} & \vphantom{\int\limits^x}\smash{\vdots} \\
        0 & 0 & 0 & 0
    \end{smallmatrix}\right]\mapsto
	\left[\begin{smallmatrix}
        \texttt{id}_1 & v_1 & 0 & 0 \\
        \texttt{id}_2 & v_2 & 0 & 0 \\
       \vphantom{\int\limits^x}\smash{\vdots} &\vphantom{\int\limits^x}\smash{\vdots} & \vphantom{\int\limits^x}\smash{\vdots} & \vphantom{\int\limits^x}\smash{\vdots} \\
        \texttt{id}_{j-1} & v_{j-1}+v_j & 1 & 1 \\
       \vphantom{\texttt{id}_1} 0 & 0 & 0 & 0 \\
        0 & 0 & 0 & 0 \\
         \vphantom{\int\limits^x}\smash{\vdots} &\vphantom{\int\limits^x}\smash{\vdots} & \vphantom{\int\limits^x}\smash{\vdots} & \vphantom{\int\limits^x}\smash{\vdots} \\
        0 & 0 & 0 & 0
    \end{smallmatrix}\right]}_{\text{Result of $e_{\mathsf{agg\_plus}}$}}
		   		       \underbrace{\left[\begin{smallmatrix}
		           \texttt{id}_1 & v_1 & 0 & 0 \, \\
		           \texttt{id}_2 & v_2 & 0 & 0 \\
		           \vphantom{\int\limits^x}\smash{\vdots} &\vphantom{\int\limits^x}\smash{\vdots} & \vphantom{\int\limits^x}\smash{\vdots} & \vphantom{\int\limits^x}\smash{\vdots} \\
		           \texttt{id}_{j-1} & v_{j-1} & 0 & 0 \\
		           \texttt{id}_j & v_j & 1 & 1 \\
		               0 & 0 & 0 & 0 \\
		           \vphantom{\int\limits^x}\smash{\vdots} &\vphantom{\int\limits^x}\smash{\vdots} & \vphantom{\int\limits^x}\smash{\vdots} & \vphantom{\int\limits^x}\smash{\vdots} \\
		           0 & 0 & 0 & 0
		       \end{smallmatrix}\right]\mapsto
		   	\left[\begin{smallmatrix}
		           \texttt{id}_1 & v_1 & 0 & 0 \\
		           \texttt{id}_2 & v_2 & 0 & 0 \\
		          \vphantom{\int\limits^x}\smash{\vdots} &\vphantom{\int\limits^x}\smash{\vdots} & \vphantom{\int\limits^x}\smash{\vdots} & \vphantom{\int\limits^x}\smash{\vdots} \\
		           \texttt{id}_{j-1} & v_{j-1}v_j & 1 & 1 \\
		          \vphantom{\texttt{id}_1} 0 & 0 & 0 & 0 \\
		           0 & 0 & 0 & 0 \\
		            \vphantom{\int\limits^x}\smash{\vdots} &\vphantom{\int\limits^x}\smash{\vdots} & \vphantom{\int\limits^x}\smash{\vdots} & \vphantom{\int\limits^x}\smash{\vdots} \\
		           0 & 0 & 0 & 0
		       \end{smallmatrix}\right]}_{\text{Result of $e_{\mathsf{agg\_prod}}$}}$$
$$ \underbrace{\left[\begin{smallmatrix}
        \texttt{id}_1 & v_1 & 0 & 0 \\
        \texttt{id}_2 & v_2 & 0 & 0 \\
          \vphantom{\int\limits^x}\smash{\vdots} &\vphantom{\int\limits^x}\smash{\vdots} & \vphantom{\int\limits^x}\smash{\vdots} & \vphantom{\int\limits^x}\smash{\vdots} \\
        \texttt{id}_{j-1} & v_{j-1} & 0 & 0 \\
        \texttt{id}_j & v_j & 1 & 1 \\
       	    \vphantom{\texttt{id}_1}  0 & 0 & 0 & 0 \\
        \vphantom{\int\limits^x}\smash{\vdots} &\vphantom{\int\limits^x}\smash{\vdots} & \vphantom{\int\limits^x}\smash{\vdots} & \vphantom{\int\limits^x}\smash{\vdots} \\
        0 & 0 & 0 & 0
    \end{smallmatrix}\right]\mapsto
    \left[\begin{smallmatrix}
        \texttt{id}_1 & v_1 & 0 & 0 \\
        \texttt{id}_2 & v_2 & 0 & 0 \\
        \vphantom{\int\limits^x}\smash{\vdots} &\vphantom{\int\limits^x}\smash{\vdots} & \vphantom{\int\limits^x}\smash{\vdots} & \vphantom{\int\limits^x}\smash{\vdots} \\
        \texttt{id}_{j-1} & v_{j-1} & 0 & 1 \\
        \texttt{id}_{j+1} & 0 & 1 & 0 \\
	    \vphantom{\texttt{id}_1}  0 & 0 & 0 & 0 \\
         \vphantom{\int\limits^x}\smash{\vdots} &\vphantom{\int\limits^x}\smash{\vdots} & \vphantom{\int\limits^x}\smash{\vdots} & \vphantom{\int\limits^x}\smash{\vdots} \\
        0 & 0 & 0 & 0
    \end{smallmatrix}\right]}_{\text{Result of $e_{\mathsf{agg\_not\_last}}$}}
	\underbrace{\left[\begin{smallmatrix}
	        \texttt{id}_1 & v_1 & 0 & 0 \\
	        \texttt{id}_2 & v_2 & 0 & 0 \\
	          \vphantom{\int\limits^x}\smash{\vdots} &\vphantom{\int\limits^x}\smash{\vdots} & \vphantom{\int\limits^x}\smash{\vdots} & \vphantom{\int\limits^x}\smash{\vdots} \\
	        \texttt{id}_{j-1} & v_{j-1} & 0 & 0 \\
	        \texttt{id}_j & v_j & 1 & 1 \\
	         \vphantom{\texttt{id}_1}  0 & 0 & 0 & 0 \\
	        \vphantom{\int\limits^x}\smash{\vdots} &\vphantom{\int\limits^x}\smash{\vdots} & \vphantom{\int\limits^x}\smash{\vdots} & \vphantom{\int\limits^x}\smash{\vdots} \\
	        0 & 0 & 0 & 0
	    \end{smallmatrix}\right]\mapsto
	    \left[\begin{smallmatrix}
	        \texttt{id}_1 & v_1 & 0 & 0 \\
	        \texttt{id}_2 & v_2 & 0 & 0 \\
	        \vphantom{\int\limits^x}\smash{\vdots} &\vphantom{\int\limits^x}\smash{\vdots} & \vphantom{\int\limits^x}\smash{\vdots} & \vphantom{\int\limits^x}\smash{\vdots} \\
	        \texttt{id}_{j-1} & v_{j-1} & 1 & 1 \\
	       \vphantom{\texttt{id}_1}  0 & 0 & 0 & 0 \\
		    \vphantom{\texttt{id}_1}  0 & 0 & 0 & 0 \\
	         \vphantom{\int\limits^x}\smash{\vdots} &\vphantom{\int\limits^x}\smash{\vdots} & \vphantom{\int\limits^x}\smash{\vdots} & \vphantom{\int\limits^x}\smash{\vdots} \\
	        0 & 0 & 0 & 0
	    \end{smallmatrix}\right]}_{\text{Result of $e_{\mathsf{agg\_last}}$}}
	$$
\caption{Result of updating $X$ by means of the \langfor expressions encoding the initialization by \textsc{Evaluate} and the different cases in \textsc{Extend}.}\label{fig:langforevalalg2}
\end{figure}

\paragraph{$\rhd$  Small $n$.}

    Finally, we need to take care of all $n<n_0$, where $(\star)$ does not necessarily hold. For any $i$, let: $$\text{Eval}[i,A]:= \text{ the } 1\times 1 \text{ matrix with the value of the polynomial } \Phi_n(A) \text{ when } n=i.$$

    Then we define: 
    $$
    \Phi_n(a_1,\ldots, a_i)=\ssum_{i=0}^{n_0-1}\left( e_{\mathsf{min}+\mathsf{i}}^T\cdot (e_{\mathsf{diag}}(e_{\ones}(v))\cdot e_{\mathsf{max}}) \right) \times \text{EVAL}[i,v] + \left( (\mathsf{Next}^{n_0}\cdot e_{\mathsf{min}})^T\cdot e_{\ones} (v) \right)\times \text{EVAL}[v].
    $$ 
    Above, $(e_{\mathsf{min}+\mathsf{i}}^T\cdot e_{\mathsf{max}})$ checks if the dimension is equal to $i$ (we multiply
    by the $n\times n$ identity $e_{\mathsf{diag}}(e_{\ones}(v))$ to ensure typing), 
    and $(\mathsf{Next}^{n_0}\cdot e_{\mathsf{min}})^T\cdot e_{\ones} (e_{\mathsf{min}})$ checks if the 
    dimension is greater or equal than $n_0$.



\subsection{From \langfor to circuits}\label{subsec:formatlangtoac}

Now that we know that arithmetic circuits can be simulated using \langfor expressions, it is natural to ask whether the same holds in the other direction. That is, we are asking whether for each \langfor expression $e$ over some schema $\Sch$ there is a uniform family of arithmetic circuits computing precisely the same result depending on the input size. 

In order to handle the fact that \langfor\ expressions can produce any matrix, and not just a single value, as their output, we need to consider circuits which have multiple output gates. Similarly, we need to encode matrix inputs of a \langfor\ expression in our circuits. We will write $\Phi(A_1,\ldots ,A_k)$, where $\Phi$ is an arithmetic circuit with multiple output gates, and each $A_i$ is a matrix of dimensions $\alpha_i\times \beta_i$, with $\alpha_i,\beta_i \in \{n,1\}$ to denote the input matrices for a circuit $\Phi$. We will also write $\texttt{type}(\Phi)=(\alpha,\beta)$, with $\alpha,\beta\in \{n,1\}$, to denote the size of the output matrix for $\Phi$. We call such circuits \textit{arithmetic circuits over matrices}. When $\{\Phi_n\mid n=1,2,\ldots\}$ is a uniform family of arithmetic circuits over matrices, we will assume that the Turing machine for generating $\Phi_n$ also gives us the information about how to access a position of each input matrix, and how to access the positions of the output matrix, as is usually done when handling matrices with arithmetic circuits \cite{Raz02}. The notion of degree is extended to be the sum of the degrees of all the output gates. With this  at hand, we can now show the following result.

\begin{theorem}
\label{th-ml-to-circuits}
Let $e$ be a \langfor expression over a schema $\Sch$, and let $V_1,\ldots ,V_k$ be the variables of $e$ such that $\ttype(V_i)\in \{(\alpha,\alpha), (\alpha,1), (1,\alpha), (1,1)\}$. Then there exists a uniform arithmetic circuit family over matrices $\Phi_n(A_1,\ldots ,A_k)$ such that:
\begin{itemize}
\item For any instance $\I = (\dom,\conc)$ such that $\dom(\alpha) = n$ and $\conc(V_i) = A_i$ it holds that:
\item $\sem{e}{\I} = \Phi_n(A_1,\ldots ,A_k)$.
\end{itemize}
\end{theorem}


It is not difficult to see that the proof of Theorem \ref{th-circuits-ml} can also be extended to support arithmetic circuits over matrices. In order to identify the class of functions computed by \langfor expressions, we need to impose one final restriction: than on the degree of an expression. 
In particular, we will be interested in \langfor expressions of polynomial degree. Formally, an expression $e$ is of polynomial degree, whenever there is an equivalent circuit family for $e$ of a polynomial degree.  
For example, all \langfor expressions seen so far have polynomial degree.
With this definition, we can now identify the class of functions for which arithmetic circuits and \langfor formulas are equivalent. This is the main technical contribution of the paper. 

\begin{corollary}
\label{th-equivalence}
Let $f$ be a function with input matrices $A_1,\ldots ,A_k$ of dimensions $\alpha\times \beta$, with $\alpha,\beta \in \{n,1\}$. Then, $f$ is computed by a uniform circuit family over matrices of polynomial degree if and only if there is a \langfor expression of polynomial degree for $f$. 
\end{corollary}

Note that this result crucially depends on the fact that expressions in \langfor are of polynomial degree.
Some \langfor expressions are easily seen to produce results which are not polynomial.
An example of such an expression is, for instance, $e_{\texttt{exp}} = \ffor{v}{X=A}{X\cdot X}$, over a schema $\Sch$ with $\ttype(v)= (\gamma,1)$, and $\ttype(X)=(1,1)$.
Over an instance which assigns $n$ to $\gamma$ this expression computes the function $a^{2^n}$, for $A=[a]$.
Therefore, a natural question to ask then is whether we can determine the degree of a \langfor expression.
Unfortunately, as we show in the following proposition this question is in fact undecidable.

Let $e$ be a \langfor expression over a matrix schema $\mathcal{S}=(\mathcal{M},\textsf{size})$ and let $V_1,\ldots, V_k$ be
the variables of $e$, each of type $(\alpha,\alpha)$, $(1,\alpha)$, $(\alpha,1)$ or $(1,1)$. We know from Theorem~\ref{th-ml-to-circuits}
that there exists a uniform arithmetic circuit family $\{\Phi_n \mid n=1,2,\ldots\}$
such that $\sem{e}{\I}=\Phi_n(A_1,\ldots,A_k)$ for any instance $\I$ such that
$\mathcal{D}(\alpha)=n$ and $\conc(V_i)=A_i$ for $i=1,\ldots,k$. We are interested in deciding
whether there exists such a  uniform arithmetic circuit family $\{\Phi_n \mid n=1,2,\ldots\}$
of polynomial degree, i.e., such that $\mathsf{degree}(\Phi_n)=\mathcal{O}(p(n))$ for some polynomial $p(x)$. If such a circuit family exists, we call $e$ of polynomial degree.

\begin{proposition}
\label{prop-undec}
Given a \langfor expression $e$ over a schema $\Sch$, it is undecidable to check whether $e$ is of polynomial degree.
\end{proposition}


Of course, one might wonder whether it is possible to define a syntactic subclass of \langfor expressions that are of polynomial degree and can still express many important linear algebra algorithms. We identify one such class in Section \ref{ss:sumML}, called \langsum, and in fact show that this class is powerful enough to capture relational algebra on (binary) $K$-relations. 

\subsection{BBBB}
% % Now we can prove theorem \ref{th-circuits-ml}
%
% \begin{proof}

    % For a stack $S$, the operations are standard:
    %
    % \begin{itemize}
    %     \item $\push{S}{s}$: pushes $s$ into $S$.
    %     \item $\pop{S}$: pops the top element.
    %     \item $\getsize{S}$: the length of the stack.
    %     \item $\gettop{S}$: the top element in the stack.
    % \end{itemize}
    %
    % For the pseudo-code, $\cG$ and $\cV$ denote stacks of gates and values, respectively. The property that holds during the simulation is that the value in $\cV[i]$ is the value that $\cG[i]$ currently outputs. The algorithm ends with $\cG=\left[ g_{\texttt{root}}\right]$ and $\cV=\left[ v_{\texttt{root}}\right]$ after traversing the circuit, and returns $v_{\texttt{root}}$
    %
    % During the evaluation algorithm there will be two possible configurations of $\cG$ and $\cV$.
    %
    % \begin{enumerate}
    %     \item $\getsize{\cG} = \getsize{\cV} + 1$: this means that $\gettop{\cG}$ is a gate that we visit for the first time and we need to initialize its value.
    %
    %     \item $\getsize{\cG} = \getsize{\cV}$: here $\gettop{\cV}$ is the value of evaluating the circuit in gate $\gettop{\cG}$. Therefore, we need to aggregate the value $\gettop{\cV}$ to the parent gate of $g$.
    % \end{enumerate}
    %
    % We assume the circuit has input gates, $+, \times$-gates and allow constant $1$-gate.
    %
    % The idea is to traverse the circuit top down in a depth first search way. For example, in the circuit $f(a_1,a_2,a_3,a_4)=a_1a_2 +a_3a_4$ above, we would initialize the output gate value as $0$ because it is a $+$ gate, so $\cG=\lbrace +\rbrace$, $\cV=\lbrace 0\rbrace$. Then stack the left $\times$ gate to $\cG$, stack its initial value (i.e. $1$) to $\cV$. Now stack $a_1$ to $\cG$ and its value (i.e. $a_1$) to $\cV$. Since we are on an input gate we pop the gate and value pair off of $\cG$ and $\cV$ respectively, aggregate $a_1$ to $\gettop{\cV}$ and continue by stacking the $a_2$ gate to $\cG$. We pop $a_2$ off of $\cV$ (and its gate off of $\cG$) and aggregate its value to $\gettop{\cV}$. We pop and aggregate the value of the left $\times$ gate to $\gettop{\cV}$ (the root value). Then continue with the right $\times$ gate branch similarly.


%
% \textit{Proof sketch.} The proof of this Theorem, which is the deepest technical result of the paper, depends crucially on two facts: (i) that any polynomial time Turing machine working within linear space and producing linear size output, can be simulated via a \langfor\ expression; and (ii) that evaluating an arithmetic circuit $\Phi_n$ can be done using two stacks of  depth $n$.
%
% Evaluating  $\Phi_n$ on input $(a_1,\ldots ,a_n)$ can be done in a depth-first manner by maintaining  two stacks: the gates-stack that tracks the current gate being evaluated, and the values-stack that stores the value that is being computed for this gate. The idea behind having two stacks is that whenever the number of items on the gates-stack is higher by one than the number of items on the values-stack, we know that we are processing a fresh gate, and we have to initialize its current value (to 0 if it is a sum gate, and to 1 if it is a product gate), and push it to the values-stack. We then proceed by processing the children of the head of the gates-stack one by one, and aggregate the results using sum if we are working with a sum gate, and by using product otherwise.
%
% In order  to access the information about the gate we are processing (such as whether it is a sum or a product gate, the list of its children, etc.) we use the uniformity of our circuit family. Namely, we know that we can generate the circuit $\Phi_n$ with a \logspace-Turing machine $M_\Phi$ by running it on the input $1^n$. Using this machine, we can in fact compute all the information needed to run the two-stack algorithms described above. For instance, we can construct a \logspace\ machine that checks, given two gates $g_1$ and $g_2$, whether $g_2$ is a child of $g_1$. Similarly, we can construct a machine that, given $g_1$ and $g_2$ tells us whether $g_2$ is the final child of $g_1$, or the one that produces the following child of $g_1$ (according to the ordering given by the machine $M_\Phi$). Defining these machines based on $M_\Phi$ is similar to the algorithm for the composition of two \logspace\ transducers, and is commonly used to evaluate arithmetic circuits \citep{allender}.
%
% To simulate the circuit evaluation algorithm that uses two stacks, in \langfor we can use a binary matrix of size $n\times n$, where $n$ is the number of inputs. The idea here is that  the gates-stack corresponds to the first $n-3$ columns of the matrix, with each gate being encoded as a binary number in positions $1,\ldots,n-3$ of a row. The remaining three columns are reserved for the values-stack, the number of elements on the gates stack, and the number of elements on the values stack, respectively. The number of elements is encoded as a canonical vector of size $n$. Here we crucially depend on the fact that the circuit is of logarithmic depth, and therefore the size of the two stacks is bounded by $n$ (apart from the portion before the asymptotic bound kicks-in, which can be hard-coded into the expression $e_\Phi$). Similarly, given that the circuits are of polynomial size, we can assume that gate ids can be encoded into $n-3$ bits.
%
% This matrix is then updated in the same way as the two-stack algorithm. It processes gates one by one, and using the successor relation for canonical vectors determines whether we have more elements on the gates stack. In this case, a new value is added to the values stack ($0$ if the gate is a sum gate, and $1$ otherwise), and the process continues. Information about the next child, last child, or input value, are obtained using the expression which simulates the Turing machine generating this data about the circuit (the machines used never produce an output longer than their input). Given that the size of the circuit is polynomial, say $n^k$, we can initialize the matrix with the output gate only, and run the simulation of the two-stack algorithm for $n^k$ steps (by iterating $k$ times over size $n$ canonical vectors). After this, the value in position  $(1,n-2)$ (the top of the values stack) holds the final results.



\smallskip
While Theorem \ref{th-circuits-ml} gives us an idea on how to simulate arithmetic circuits, it does not tell us which classes of functions over real numbers can be computed by \langfor expressions. In order to answer this question, we note that arithmetic circuits can be used to compute functions over real numbers. Formally, a circuit family $\{\Phi_n\mid n=1,2,\ldots\}$ computes a function $f:\bigcup_{n\geq 1} \mathbb{R}^n\mapsto\mathbb{R}$, if for any $a_1,\ldots a_n\in \mathbb{R}$ it holds that $\Phi_n(a_1,\ldots ,a_n) = f(a_1,\ldots ,a_n)$. To make the connection with \langfor\!, we need to look at circuit families of bounded degree. 

A circuit family $\{\Phi_n\mid n=1,2,\ldots\}$ is said to be of \textit{polynomial degree} if $\mathsf{degree}(\Phi_n)\in O(p(n))$, for some polynomial $p(n)$. Note that polynomial size circuit families are not necessarily of polynomial degree. An easy corollary of Theorem \ref{th-circuits-ml} tells us that all functions computed by uniform family of circuits of polynomial degree and logarithmic depth can be simulated using \langfor expressions. However, we can actually drop the restriction on circuit depth due to the result of Valiant et. al.~\cite{valiant1981fast} and Allender et. al. \cite{AllenderJMV98} which says that any function computed by a uniform circuit family of polynomial degree (and polynomial depth), can also be computed by a uniform circuit family of logarithmic depth. Using this fact, we can conclude the following:


\begin{corollary}
\label{cor-circ-ml}
For any function $f$ computed by a uniform family of arithmetic circuits of polynomial degree, there is an equivalent \langfor formula $e_f$.
\end{corollary}

Note that there is nothing special about circuits that have a single output, and both Theorem \ref{th-circuits-ml} and Corollary \ref{cor-circ-ml} also hold for functions  $f:\bigcup_{n\geq 1} \mathbb{R}^n\mapsto\mathbb{R}^{s(n)}$, where $s$ is a polynomial. Namely, in this case, we can assume that circuits for $f$ have multiple output gates, and that the depth reduction procedure of \cite{AllenderJMV98} is carried out for each output gate separately. Similarly, the construction underlying the proof of Theorem \ref{th-circuits-ml} can be performed for each output gate independently, and later composed into a single output vector.

\subsection{Supporting additional operators}\label{subsec:additionalop}
The equivalence of \langfor and arithmetic circuits we proved above assumes that circuits can only use the sum and product gates (note that even without the sum and the product function, \langfor\ can simulate these operations via matrix sum/product). However, both arithmetic circuits and expressions in $\langfor$ can be allowed to use a multitude of functions over $\RR$. The most natural addition to the set of functions is the division operator, which is crucially needed in many linear algebra algorithms, such as, for instance, Gaussian elimination, or $LU$ decomposition (recall Proposition \ref{prop:gauss}).
Interestingly, the equivalence in this case still holds, mainly due to a surprising result which shows that (almost all) divisions can in fact be removed for arithmetic circuits which allow sum, product, and division gates \cite{allender}.

More precisely, in \cite{strassen1973vermeidung,borodin1982fast,kaltofen1988greatest} it was shown that for any function of the form $f = g/h$, where $g$ and $h$ are relatively prime polynomials of degree $d$, if $f$ is computed by an arithmetic circuit of size $s$, then both $g$ and $h$ can be computed by a circuit whose size is polynomial in $s + d$. Given that we can postpone the division without affecting the final result, this, in essence, tells us that division can be eliminated (pushed to the top of the circuit), and we can work with sum-product circuits instead. The degree of a circuit for $f$, can then be defined as the maximum of degrees of circuits for $g$ and $h$. Given this fact, we can again use the depth reduction procedure of \cite{AllenderJMV98}, and extend Corollary~\ref{th-equivalence} to circuits with division.
\begin{corollary}
\label{cor-division}
Let $f$ be a function taking as its input matrices $A_1,\ldots ,A_k$ of dimensions $\alpha\times \beta$, with $\alpha,\beta \in \{n,1\}$. Then, $f$ is computed by a uniform circuit family over matrices of polynomial degree that allows divisions, if and only if there is a $\langforf{f_/}$ expression of polynomial degree for $f$.
\end{corollary}

An interesting line of future work here is to see which additional functions can be added to arithmetic circuits and \langfor formulas, in order to preserve their equivalence. Note that this will crucially depend on the fact that these functions have to allow the depth reduction of \cite{AllenderJMV98} in order to be supported.
