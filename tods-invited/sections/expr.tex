% !TeX spellcheck = en_US
%!TEX root = ../main.tex
In this section we explore the expressive power of $\langfor.$ Given that arithmetic circuits \cite{allender} capture most standard linear algebra algorithms \cite{Raz02,ShpilkaY10}, they seem as a natural candidate for comparison. Intuitively, an arithmetic circuit is similar to a boolean circuit \cite{aroraB2009}, except that it has gates computing the sum and the product function, and processes elements of $\RR$ instead of boolean values. To connect \langfor to arithmetic circuits we need a notion of uniformity of such circuits. After all, a \langfor expression can take matrices of arbitrary dimensions as input and we want to avoid having  different circuits for each dimension. To handle inputs of different sizes, we thus consider a notion of uniform families of arithmetic circuits, defined via a Turing machine generating a description of the circuit for each input size $n$.

What we show in the remainder of this section is that any function $f$ which operates on matrices, and is computed by a uniform family of arithmetic circuits of bounded degree, can also be computed by a \langfor expression, and vice versa. In order to keep the notation light, we will focus on 
 \langfor schemas over ``square matrices'' where each variable has type $(\alpha,\alpha),(\alpha,1),(1,\alpha)$, or $(1,1)$, although all of our results hold without these restrictions as well. In what follows, we will write $\langfor$ to denote $\langforf{\emptyset}$, i.e. the fragment of our language with no additional pointwise functions. We begin by defining circuits and then show how circuit families can be simulated by $\langfor.$

\subsection{From arithmetic circuits to \langfor}
Let us first recall the definition of arithmetic circuits. 
An \textit{arithmetic circuit} $\Phi$ over a set $X=\{x_1,\ldots,x_n\}$ of input variables is a directed
acyclic labeled graph. The vertices of $\Phi$ are called \textit{gates} and denoted by $g_1,\ldots,g_m$;
the edges in $\Phi$ are called \textit{wires}. The children of a gate $g$ correspond to all gates
$g'$ such that $(g,g')$ is an edge. The parents of $g$ correspond to all gates $g'$ 
such that $(g',g)$ is an edge. The \textit{in-degree}, or a \textit{fan-in}, of a gate $g$ refers to its number of children, and 
the \textit{out-degree} to its number of parents. We will not assume any restriction on the in-degree of a gate, and will thus consider circuits with unbounded fan-in. Gates with in-degree $0$ are called \textit{input gates}
and are labeled by either a variable in $X$ or a constant $0$ or $1$. All other gates
are labeled by either $+$ or $\times$, and are referred to as \textit{sum gates} or \textit{product gates}, respectively.
Gates with out-degree $0$ are called \textit{output gates}. When talking about arithmetic circuits, one usually focuses on circuits with $n$ input gates and a single output gate.\looseness=-1

The \textit{size} of $\Phi$, denoted by $|\Phi|$, is its number of gates and wires. The \textit{depth} of $\Phi$, denoted
by $\mathsf{depth}(\Phi)$, is the length of the longest directed path from any of its output gates to any of the input gates. The \textit{degree} of a gate is defined inductively: an input gate has degree~1, a sum gate has a degree equal to the maximum of degrees of its children, and a product gate has a degree equal to the sum of the degrees of its children. When $\Phi$ has a single output gate, the \textit{degree} of $\Phi$, denoted by $\mathsf{degree}(\Phi)$, is defined as the degree of its output gate. If $\Phi$ has a single output gate and its input gates take values from $\RR$, then $\Phi$ corresponds to a polynomial in $\RR[X]$ in a natural way. In this case, the {degree} of $\Phi$ equals the degree of the polynomial corresponding to $\Phi$.
If $a_1,\ldots ,a_n$ are values in $\RR$, then 
the result of the circuit on this input is the value computed by the corresponding polynomial, denoted by $\Phi(a_1,\ldots ,a_k)$.

In order to handle inputs of different sizes, we use the notion of uniform circuit families. An \textit{arithmetic circuit family} is a set of arithmetic circuits $\{\Phi_n\mid n=1,2,\ldots\}$ where $\Phi_n$ has $n$ input variables and a single output gate. An arithmetic circuit family is \textit{uniform} if there exists a \logspace-Turing machine,
which on input $1^n$, returns an encoding of the arithmetic circuit $\Phi_n$ for each $n$.
We observe that uniform arithmetic circuit families are necessarily of polynomial size. 
Another important parameter is the circuit depth. A circuit family is of logarithmic depth, whenever $\mathsf{depth}(\Phi_n)\in \mathcal{O}(\log\, n)$. We  now show that \langfor subsumes uniform arithmetic circuit families that are of logarithmic depth. 

\begin{theorem}
\label{th-circuits-ml}
For any uniform arithmetic circuit family $\{\Phi_n\mid n=1,2,\ldots\}$ of logarithmic depth there is a \langfor schema $\Sch$ and an expression $e_\Phi$ using a matrix variable $v$, with $\ttype(v)=(\alpha,1)$ and $\ttype(e) = (1,1)$, such that for any input values $a_1,\ldots ,a_n$: 
\begin{itemize}
\item If $\I = (\dom,\conc)$ is a \lang\ instance such that $\dom(\alpha) = n$ and $\conc(v) = [a_1 \ldots a_n]^T$.
\item Then $\sem{e_\Phi}{\I} = \Phi_n(a_1,\ldots ,a_n)$.
\end{itemize}
\end{theorem}
It is important to note that the expression $e_\Phi$ does not change depending on the input size, meaning that it is uniform in the same sense as the circuit family being generated by a single Turing machine. The different input sizes for a \langfor instance are handled by the typing mechanism of the language. 

\textit{Proof sketch.} The proof of this Theorem, which is the deepest technical result of the paper, depends crucially on two facts: (i) that any polynomial time Turing machine working within linear space and producing linear size output, can be simulated via a \langfor\ expression; and (ii) that evaluating an arithmetic circuit $\Phi_n$ can be done using two stacks of  depth $n$.

Evaluating  $\Phi_n$ on input $(a_1,\ldots ,a_n)$ can be done in a depth-first manner by maintaining  two stacks: the gates-stack that tracks the current gate being evaluated, and the values-stack that stores the value that is being computed for this gate. The idea behind having two stacks is that whenever the number of items on the gates-stack is higher by one than the number of items on the values-stack, we know that we are processing a fresh gate, and we have to initialize its current value (to 0 if it is a sum gate, and to 1 if it is a product gate), and push it to the values-stack. We then proceed by processing the children of the head of the gates-stack one by one, and aggregate the results using sum if we are working with a sum gate, and by using product otherwise. 

In order  to access the information about the gate we are processing (such as whether it is a sum or a product gate, the list of its children, etc.) we use the uniformity of our circuit family. Namely, we know that we can generate the circuit $\Phi_n$ with a \logspace-Turing machine $M_\Phi$ by running it on the input $1^n$. Using this machine, we can in fact compute all the information needed to run the two-stack algorithms described above. For instance, we can construct a \logspace\ machine that checks, given two gates $g_1$ and $g_2$, whether $g_2$ is a child of $g_1$. Similarly, we can construct a machine that, given $g_1$ and $g_2$ tells us whether $g_2$ is the final child of $g_1$, or the one that produces the following child of $g_1$ (according to the ordering given by the machine $M_\Phi$). Defining these machines based on $M_\Phi$ is similar to the algorithm for the composition of two \logspace\ transducers, and is commonly used to evaluate arithmetic circuits \citep{allender}. 

To simulate the circuit evaluation algorithm that uses two stacks, in \langfor we can use a binary matrix of size $n\times n$, where $n$ is the number of inputs. The idea here is that  the gates-stack corresponds to the first $n-3$ columns of the matrix, with each gate being encoded as a binary number in positions $1,\ldots,n-3$ of a row. The remaining three columns are reserved for the values-stack, the number of elements on the gates stack, and the number of elements on the values stack, respectively. The number of elements is encoded as a canonical vector of size $n$. Here we crucially depend on the fact that the circuit is of logarithmic depth, and therefore the size of the two stacks is bounded by $n$ (apart from the portion before the asymptotic bound kicks-in, which can be hard-coded into the expression $e_\Phi$). Similarly, given that the circuits are of polynomial size, we can assume that gate ids can be encoded into $n-3$ bits.

This matrix is then updated in the same way as the two-stack algorithm. It processes gates one by one, and using the successor relation for canonical vectors determines whether we have more elements on the gates stack. In this case, a new value is added to the values stack ($0$ if the gate is a sum gate, and $1$ otherwise), and the process continues. Information about the next child, last child, or input value, are obtained using the expression which simulates the Turing machine generating this data about the circuit (the machines used never produce an output longer than their input). Given that the size of the circuit is polynomial, say $n^k$, we can initialize the matrix with the output gate only, and run the simulation of the two-stack algorithm for $n^k$ steps (by iterating $k$ times over size $n$ canonical vectors). After this, the value in position  $(1,n-2)$ (the top of the values stack) holds the final results.

\textit{Syntactic Sugar.} We sometimes will want to iterate over $k$ canonical vectors. We define the following shorthand notation:

\begin{align*}
  \ffor{v_1,\ldots, v_k}{X}{e(X,v_1,\ldots, v_n)}\coloneqq  &\ffor{v_1}{X_1}{X_1 +} \\
  &\hspace{1em}\initf{X_1}{v_2}{X_2}{X_2 + } \\
  &\hspace{2em}\initf{X_2}{v_3}{X_3}{X_3 + } \\
  &\hspace{8em}\ddots \\
  &\hspace{4em}\initf{X_{k-1}}{v_k}{X_k}{ e(X_k,v_1,\ldots, v_k)}.
\end{align*}

To reference $\ell$ different vector variables $X_1,\ldots,X_\ell$ in every iteration and update them in different ways we define:
\begin{multline*}
\ffor{v}{X_1,\ldots, X_\ell}{\left( e_1(X_1,v), e_2(X_2,v), \ldots, e_l(X_\ell,v) \right)} \coloneqq  \\
\ffor{v}{X}{e_1(X\cdot e_{\mathsf{min}},v)\cdot (e_{\diag}(e_{\ones}(X^T))\cdot e_{\mathsf{min}})^T +\\ e_2(X\cdot e_{\mathsf{min} + 1},v)\cdot (e_{\diag}(e_{\ones}(X^T))\cdot e_{\mathsf{min} + 2})^T + \ldots \\
+ e_\ell(X\cdot e_{\mathsf{max}},v)\cdot (e_{\diag}(e_{\ones}(X^T))\cdot e_{\mathsf{max}})^T} \\
\end{multline*}

We note that for the latter expression to be semantically correct $v$ has to be of type $\gamma\times 1$, 
both $X_i$ and $e_i$ for $ i=1,\ldots,\ell$ have to be of type $\alpha\times 1$, 
and $X$ has to be of type $\alpha\times\beta$, where $\dom(\beta)=\ell$. Here
we use $e_{\diag}(e_{\ones}(X^T))$ to compute the $\beta\times\beta$ identity and ensure the typing of the
$e_{\mathsf{min} + i}$.
When evaluated on an instance $\I$,
$e_{\mathsf{min}}, e_{\mathsf{min} + i}$ evaluate to $b_1^{\dom(\beta)}$ and $b_{1+i}^{\dom(\beta)}$, 
respectively, and we show their defining expressions in section \ref{sec:formatlang:design}.
Similarly for $e_{\mathsf{max}}=b_n^{\dom(\beta)}$.
The combinations of both previous operators results in:

\begin{align*}
    \texttt{for}\, v_1,\ldots, v_k,X_1,\ldots, X_\ell \texttt{.}\, \Big( &e_1(X_1,v_1,\ldots, v_k), \\
    &\hspace{1em}e_2(X_2,v_1,\ldots, v_k), \ldots, e_\ell(X_\ell,v_1,\ldots, v_k) \Big) \coloneqq  \\
    &\hspace{8em}\ffor{v_1,\ldots, v_k}{X}{e'(X,v_1,\ldots, v_k)} \\
\end{align*}
where 
\begin{align*}
e'(X,v_1,\ldots,v_k)\coloneqq &e_1(X\cdot e_{\mathsf{min}},v_1,\ldots,v_k)\cdot (e_{\diag}(e_{\ones}(X^T))\cdot e_{\mathsf{min}})^T \\
&+ e_2(X\cdot e_{\mathsf{min} + 1},v_1,\ldots,v_k)\cdot (e_{\diag}(e_{\ones}(X^T))\cdot e_{\mathsf{min} + 1})^T \\
&+ \ldots + e_\ell(X\cdot e_{\mathsf{max}},v_1,\ldots,v_k)\cdot (e_{\diag}(e_{\ones}(X^T))\cdot e_{\mathsf{max}})^T
\end{align*}
It is clear that this expression iterates over $k$ canonical vectors and references $\ell$ independent vectors updating each of them in their particular way.

%!TEX root = ../../main.tex
\subsection{Proof of Proposition~\ref{prop:transducer2}}
A crucial ingredient for translating arithmetic circuits in \langfor is that \langfor
can express a number of circuit-related functions, as described in Section~\ref{subsubsec:simulate}.
We here show a more general result which, roughly speaking, says that any 
polynomial time Turing machine, working within linear space and producing linear space output, also
referred to as \textit{linear space machines}, 
can be simulated in \langfor. The circuit-related functions used in the evaluation algorithm in Section~\ref{subsec:actoformatlang}
are all of this form when considering arithmetic circuit families of logarithmic depth.

\subsubsection{Linear space machines}\label{subsubsec:linearspace}
We start by formally defining linear space machines. We consider  deterministic Turing Machines (TMs) $T$ 
consisting of $\ell$ read-only input tapes, denoted by $R_1,\ldots,R_\ell$, and
$s$ read-write working tapes, denote by $W_1,\ldots,W_s$. The TM $T$ has a set 
$Q$ of $m$ states, denoted by $q_1,\ldots,q_m$. We assume that $q_1$ is the initial state and $q_m$ is the accepting state.
The input and tape alphabet are $\Sigma=\{0,1\}$ and $\Gamma=\Sigma\cup\{\rhd,\lhd\}$, respectively. The special 
symbols $\rhd$ and $\lhd$ denote the beginning and end, respectively, of each of the tapes.
The transition function $\Delta$ of $T$ is defined as usual, i.e. 
$\Delta:Q\times \Gamma^{\ell+s} \to Q\times \Gamma^{s}\times \{\leftarrow,\sqcup,\rightarrow\}^{\ell+s}$ 
such that
$$
\Delta(q,(a_1,\ldots,a_{\ell},b_1, \ldots, b_s))=\bigl(q',(b_1',\ldots, b'_s),(\mathsf{d}_1,\ldots,\mathsf{d}_{\ell+s})\bigr)
$$
with $\mathsf{d}_i\in \{\leftarrow,\sqcup,\rightarrow\}$. The semantics is that when $T$ is in state $q$ and the $\ell+s$ 
heads on the tapes read symbols $a_1,\ldots,a_{\ell},b_1, \ldots, b_s$, respectively, then $T$ transitions to state $q'$,
writes $b_1', \ldots, b_s'$ on the work tapes at the position to which the work
tape's heads points at, and finally moves the heads on the tapes according 
$\mathsf{d}_1,\ldots,\mathsf{d}_{\ell+s}$. More specifically, $\leftarrow$  indicates a move to the left, 
$\rightarrow$ a move to the right, and finally, $\sqcup$ indicates that the head does not move.

We assume that $\Delta$ is defined such that it ensures that on none of the tapes, heads can move beyond 
the leftmost marker $\rhd$ and the rightmost marker $\lhd$. Furthermore, the tapes $R_1,\ldots,R_\ell$ are treated as read-only and $\Delta$ does not change the 
occurrences of $\rhd$ or $\lhd$ on the work tapes.

A \textit{configuration of} $T$ is defined in the usual way. That is, a configuration of a tape is of the form
$\rhd w_1qw_2\lhd$ with $w_1,w_2\in\Sigma^*$ and represents that the current tape content is 
$\rhd w_1w_2\lhd$, $T$ is in state $q$ and the head is positioned on the first symbol of $w_2$. 
A configuration of $T$ consists of configurations for all tapes. Given two configurations 
$c_1$ and $c_2$, we say that $c_1$ \textit{yields} $c_2$ if $c_2$ is the result of applying the transition 
function $\Delta$ of $T$ based on the information in $c_1$. As usual, we close this ``yields'' relation 
transitively.

Given $n > 0$ and $\ell$ input words $w_1,\ldots,w_\ell\in\Sigma^n$, we assume that the initial configuration of 
$T$ is given by
$$
\bigl(q_1 \!\rhd\! w_1\lhd,q_1\!\rhd\! w_2\lhd,\ldots, q_1\!\rhd\! w_\ell\lhd, q_1\!\rhd\! 0^n \lhd, \overset{\text{$s$-times}}{\ldots}, q_1\!\rhd\! 0^n \lhd \bigr)
$$ and an 
accepting configuration is assumed to be any configuration whose current state is $q_m$. We say that $T$ \textit{computes the function} $f=\bigcup_{n\geq 0} f_n:(\Sigma^n)^\ell\to\Sigma^n$ if for every
$w_1,\ldots,w_\ell\in\Sigma^*$, each of size $n$, the initial configuration yields (transitively) an accepting 
configuration such that the configuration on the first working tape $W_1$ is of the form $\rhd u_1 q_m u_2\lhd$ such that $f_n(w_1,\ldots,w_\ell) = u_1u_2$, namely, the output is the content on the first work tape. 
We assume that once $T$ reaches an accepting configuration it stays indefinitely in that configuration 
(i.e. it loops).
Finally, we say that $T$ runs in $\mathcal{O}(n^{k})$ time if for every $n$ and every $\ell$-input words of size $n$, it reaches an accepting configuration in less than $c\cdot n^{k}$ steps for some fix constant $c$.

%We further assume that $T$ only reaches an accepting configuration when all its input
%words have the same size. Furthermore, when all inputs have the same size, we assume that $T$ will reach an accepting 
%configuration. 


%We say that $T$ is a \textit{linear space machine} when it reaches an accepting configuration 
%on inputs of size $n$ by using $\mathcal{O}(n)$ space on its work tape and additionally needs 
%polynomially many steps to do so.

 A \textit{linear input-output function} is a function of the form 
$f=\bigcup_{n\geq 0} f_n:(\Sigma^n)^\ell\to\Sigma^n$. In other words, for every $\ell$ words of the same 
size $n$, $f$ returns a word of size $n$. We say that a linear input-output function is a 
\textit{linear space input-output function} if
there exists a linear space machine  $T$ that for every $n\geq 0$, on input $w_1,\ldots,w_\ell\in\Sigma^n$ 
the TM $T$ has
$f_n(w_1,\ldots,w_\ell)$ on its first work tape when (necessarily) reaching an accepting configuration.
We have similar notions in place for functions of the form $f=\bigcup_{n\geq 0} f_n:(\Sigma^n)^\ell\to\Sigma$.

\subsubsection{Proof of Proposition~\ref{prop:transducer2}}
We are now ready to precisely formulate  Proposition~\ref{prop:transducer2}.

\begin{proposition}
	For $\Sigma = \{0,1\}$ let $f=\bigcup_{n\geq 0}f_n:(\Sigma^n)^\ell\to \Sigma^n$ be a linear space input-ouput function 
	computed by a linear space  machine $T$ with $m$ states, $\ell$ input tapes, $s$ work tapes, and runs in $\mathcal{O}(n^{k-1})$ time on inputs of size $n$. 
	Then there exists (i)~a \langfor 
	schema $\mathcal{S}=(\mathcal{M},\textsf{size})$ where $\mathcal{M}$ contains matrix 
	variables
	% \footnote{We also need a finite number of auxiliary variables, these will be specified
	% in the proof.}
	$Q_1,\ldots,Q_m$, $R_1,\ldots,R_\ell$, $H_1,\ldots,H_\ell$, $W_1,\ldots,W_s$, $H_{W_1},\ldots,H_{W_s}$, $v_1,\ldots,v_{k}$  
	with $\mathsf{size}(V)=\alpha\times 1$ for all $V\in\mathcal{M}$; and (ii)~a \langfor 
	expression $e_f$ over $\mathcal{S}$ such that for the instance 
	$\I=(\mathcal{D},\textsf{mat})$ over $\mathcal{S}$ with $\mathcal{D}(\alpha)=n$ and 
	$\mathsf{mat}(R_i)=\mathsf{vec}(w_i)\in \mathbb{R}^n$, for $i\in[\ell]$, such that $\mathsf{vec}(w_i)$ is the $n\times 1$-vector 
	encoding the word $w_i\in\Sigma^n$, and all other matrix variables instantiated with the zero vector in $\mathbb{R}^n$,  we have that  $\mathsf{mat}(W_1)=\mathsf{vec}(f_n(w_1,\ldots,w_\ell))\in\mathbb{R}^n$ 
	after evaluating $e_f$ on~$\I$. We have a similar result for functions $f=\bigcup_{n\geq 0}f_n:(\Sigma^n)^\ell\to \Sigma$.
\end{proposition}

% For this proof only, we will denote the canonical vectors as
% $\mathbf{e}_1, \ldots, \mathbf{e}_n$, since $b$ will be used to represent a value on a position of a tape.
We only consider functions $f=\bigcup_{n\geq 0}f_n:(\Sigma^n)^\ell\to \Sigma^n$. The proof only requires
minimal modifications for functions  $f=\bigcup_{n\geq 0}f_n:(\Sigma^n)^\ell\to \Sigma$.

\begin{proof}
	The expression $e_f$ we construct will simulate the TM $T$. To have some more control on the time consumption of $T$, let us first assume that $n$ is large enough, say larger than $n\geq n_0$, 
	such that $T$ runs in $cn^{k-1}\leq n^{k}$ time for a constant $c$. We deal with $n<n_0$ separately.
 We split up the construction of $e_f$ in three
	parts: \textbf{(a)}~an expression $e_f^{\geq n_0}$, for dealing with $n\geq n_0$; \textbf{(b)}~an expression
	$e_f^{<n_0}$, for dealing with $n<n_0$; and finally \textbf{(c)}~the expression $e_f$ combining both these expressions. 

\medskip
\noindent
\underline{\textbf{(a)} Large $n$, i.e. $n\geq n_0$.}
    Let us first define $e_f^{\geq n_0}$. To simulate $T$ we need to encode states, tapes and head positions. For this, we assume that $\mathcal{M}$ contains matrix variables:
    $$
    Q_1,\ldots,Q_m, R_1,\ldots,R_\ell, H_1,\ldots,H_\ell, W_1,\ldots,W_s, H_{W_1},\ldots,H_{W_s}, v_1,\ldots,v_{k}
    $$
    which will store the states, tapes, and head positions. More specifically, the variables 
    $R_1,\ldots,R_\ell$ will hold the input vectors, and $W_1,\ldots,W_s$ will hold the contents of the work
    tapes.
	The vectors corresponding to the work tapes are initially set to the zero vector. 
    The vector for the input tape $R_i$ is set to $\mathsf{vec}(w_i)$, for $i\in[\ell]$.

\smallskip
    With each tape we associate a matrix variable encoding the position of the head. More specifically, 
    $H_1,\ldots,H_\ell$ correspond to the input tape heads, and
    $H_{W_1},\ldots, H_{W_s}$ to the work tape heads.
    All these vectors will be zero except 
    for a single position, indicating the positions in the corresponding tapes the heads point to. 
    For those positions $j$, $1<j<n$, the head vectors will carry value $1$.  When $j=1$ or $n$ and when 
    it concerns positions for the input tape, the head vectors can carry value $1$ or $2$. We need to treat 
    these border cases separately
    because we only have $n$ positions available to store the input words, whereas the actual input tapes 
    consist of $n+2$ symbols because of $\rhd$ and $\lhd$. So when, for example, $H_1$ has a $1$ in its first
    entry, we interpret it as if the head is pointing to the first symbol of the input word $w_1$. When $H_1$
    has a $2$ in its first position, we interpret it as if the head pointing to $\rhd$. The end marker $\lhd$ is
    dealt with in the same way, by using value $1$ or $2$ in the last position of $H_1$. We use this encoding
    for all input tapes, and also for the work tapes.
    

    To encode the states, we use the variables $Q_1,\ldots,Q_m$. We will ensure that when $T$ is in state 
    $q_i$ when
    $\mathsf{mat}(Q_i)=[1,0,\ldots,0]^T\in\mathbb{R}^n$, otherwise $\mathsf{mat}(Q_i)$ is the zero 
    vector in $\mathbb{R}^n$.	

    Finally, the variables $v_1,\ldots,v_{k}$ represent $k$ canonical vectors  which are used to iterate 
    in for-loops. By iterating over them, we can perform $n^{k}$ iterations, 
    which suffices for simulating the $\mathcal{O}(n^{k-1})$ steps used by $T$ to reach an accepting configuration.
    
    With these matrix variables in place, we start by defining $e_f^{\geq n_0}$. As a final remark, we note that during the evaluation of the formula, we need to force that all these matrix variables are either $\{0,1\}$ vectors (e.g., for variables encoding a tape), canonical vectors (e.g., for variables encoding a head tape), or vectors of the form $(2,0,\ldots,0)$ or $(0, \ldots, 0, 2)$ (e.g., for variables encoding $\rhd$ and $\lhd$ markers).
    We will force these restrictions on every formula below. 
    % We explain the expression
%     $e_f^{\geq N}$ first.

    We will use several expressions related to ordering information of canonical vectors, as explained in Section \ref{sec:formatlang:design}.
	In particular, we use $\mathsf{min}$ and $\mathsf{max}$, to test whether their input vector is the first, respectively, last canonical
	vector, expressions $e_{\mathsf{min}}$ and $e_{\mathsf{max}}$, which return the first and last canonical vector, respectively, and expressions $e_{\mathsf{Next}}$
	and $e_{\mathsf{Prev}}$, which return matrices that, when multiplied with a canonical vector, return the next, respectively previous, canonical vector.
	  % In our  expressions we use subexpressions which we defined before in . These subexpression
	  %     require some auxiliary variables, as detailed below. As a consequence, $e_f$ will be an expressions
	  %     defined over an extended schema $\mathcal{S}'$. Hence, the instance $\I$ in the statement of the Proposition
	  %     is  an instance $\I'$ of $\mathcal{S}'$ which
	  %     coincides with $\I$ on $\mathcal{S}$ and in which the auxiliary matrix variables are all instantiated with
	  %     zero vectors or matrices, depending on their size.
	  %
	  %     Now, we specify the finite auxiliary variables involved in the \langfor expression. These arise
	  %     when computing the following \langfor expressions defined
	  %
	  %     \begin{itemize}
	  %         \item $e_{\mathsf{Prev}}(z,Z,z',Z')$, and expression over auxiliary variables $z$, $z'$, $Z$ and $Z'$ with
	  %         $\mathsf{size}(z)=\mathsf{size}(z')=\mathsf{size}(Z)=\alpha\times 1$ and
	  %         $\mathsf{size}(Z')=\alpha\times\alpha$. On input $\I'$ with
	  %         $\mathsf{mat}(z)=\mathsf{mat}(z')=\mathsf{mat}(Z)$ the zero column vector of dimension $n$,
	  %         and $\mathsf{mat}(Z')$ the zero $n\times n$ matrix,
	  %         $\sem{e_{\mathsf{Prev}}}{\I'}$ returns the $n\times n$ matrix $\mathsf{Prev}$ such that
	  %         $$\mathsf{Prev}\cdot \mathbf{e}_i:=\begin{cases}
	  %         \mathbf{e}_{i-1} & \text{if $i>1$}\\
	  %         \mathbf{0} & \text{if $i=1$}.
	  %         \end{cases}
	  %         $$
	  %         % In other words, $e_{\mathsf{Prev}}$ defines a predecessor relation among canonical vectors of dimension $n$.
	  %         \item $e_{\mathsf{Next}}(z,Z,z',Z')$, and expression over auxiliary variables $z$, $z'$, $Z$ and $Z'$
	  %         with $\mathsf{size}(z)=\mathsf{size}(z')=\mathsf{size}(Z)=\alpha\times 1$ and
	  %         $\mathsf{size}(Z')=\alpha\times\alpha$. On input $\I'$ with
	  %         $\mathsf{mat}(z)=\mathsf{mat}(z')=\mathsf{mat}(Z)$ the zero column
	  %         vector of dimension $n$, and $\mathsf{mat}(Z')$ the zero $n\times n$ matrix,
	  %         $\sem{e_{\mathsf{Next}}}{\I'}$ returns the $n\times n$ matrix $\mathsf{Next}$ such that
	  %         $$\mathsf{Next}\cdot \mathbf{e}_i:=\begin{cases}
	  %         \mathbf{e}_{i+1} & \text{if $i<n$}\\
	  %         \mathbf{0} & \text{if $i=n$}.
	  %         \end{cases}
	  %         $$
	  %         \item $\textsf{min}(v,z,Z,z',Z)$ with auxiliary variables $z$, $z'$, $Z$ and $Z'$ as before,
	  %         and $v$ is one of the (vector) variables in $\mathcal{M}$. For an $n\times 1$ vector $\mathbf{v}$,
	  %         on input $\I'[v\gets \mathbf{v}]$	$$\sem{\mathsf{min}}{\I'[v\gets\mathbf{v}]}:=\begin{cases} 1 & \text{if $\mathbf{v}=\mathbf{e}_1$}\\
	  %             0 & \text{otherwise}.
	  %             \end{cases}$$
	  %
	  %         \item $\textsf{max}(v,z,Z,z',Z)$ with auxiliary variables $z$, $z'$, $Z$ and $Z'$ as before, and
	  %         and $v$ is one of the (vector) variables in $\mathcal{M}$. For an $n\times 1$ vector $\mathbf{v}$,
	  %         on input $\I'[v\gets \mathbf{v}]$
	  %
	  %         $$\sem{\mathsf{max}}{\I'[v\gets\mathbf{v}]}:=\begin{cases} 1 & \text{if $\mathbf{v}=\mathbf{e}_{n}$}\\
	  %             0 & \text{otherwise}.
	  %             \end{cases}$$
	  %         \item $e_{\mathsf{min}}(z,Z,z',Z',z'',Z'')$, an expressions with
	  %         auxiliary variables $z$, $z'$, $z''$, $Z$, $Z'$ and $Z''$ with
	  %         $\mathsf{size}(z)=\mathsf{size}(z')=\mathsf{size}(z'')=\mathsf{size}(Z)=\mathsf{size}(Z'')=\alpha\times 1$
	  %         and $\mathsf{size}(Z')=\alpha\times\alpha$. On input $\I'$ with
	  %         matrix variables instantiated with zero vectors (or matrix for $Z'$),
	  %         $\sem{e_{\mathsf{min}}}{\I'}=\mathbf{e}_1$.
	  %         \item $e_{\mathsf{max}}(z,Z,z',Z',z'',Z'')$, an expressions with
	  %         auxiliary variables $z$, $z'$, $z''$, $Z$, $Z'$ and $Z''$ with
	  %         $\mathsf{size}(z)=\mathsf{size}(z')=\mathsf{size}(z'')=\mathsf{size}(Z)=\mathsf{size}(Z'')=\alpha\times 1$
	  %         and $\mathsf{size}(Z')=\alpha\times\alpha$. On input $\I'$ with
	  %         matrix variables instantiated with zero vectors (or matrix for $Z'$),
	  %         $\sem{e_{\mathsf{max}}}{\I'}=\mathbf{e}_n$.
	  %     \end{itemize}
	  %     We thus see that we only need $z,z',z'',Z,Z',Z''$ as auxiliary variables and these can be re-used
	  %     whenever $e_f$ calls these functions. From now one, we omit the auxiliary variables from the description
	  %     of $e_f$.
	  %

Since we want to simulate $T$ we need to be able to check which 
    transitions of $T$ can be applied based on a current configuration. More precisely,
    suppose that we want to check whether $\Delta(q_i,(a_1,\ldots,a_{\ell},b_1,\ldots, b_s))$ is applicable, then we 
    need to check whether $T$ is in state $q_i$, we can do this by checking 
    $\mathsf{min}(Q_i)$, and whether the heads on the tapes read symbols $a_1,\ldots,a_{\ell},b_1, \ldots, b_s$. We 
    check the latter by the following expressions.
    First, we define the expressions 
    $$
    \mathsf{border}_{\rhd}(v) = 1/2 \times (e_{\mathsf{min}}^T \cdot v) \cdot (e_{\mathsf{min}}^T \cdot v - 1) \ \ \text{ and } \ \  \mathsf{border}_{\lhd}(v) = 1/2 \times (e_{\mathsf{max}}^T \cdot v) \cdot (e_{\mathsf{max}}^T \cdot v - 1)
    $$
    which output $1$ if $v$ is equal to $(2,0,\ldots,0)\in\mathbb{R}^n$ or $(0, \ldots, 0, 2)\in\mathbb{R}^n$, respectively, and $0$ if $v$ is any canonical vector.
    Then for the input tape $R_i$ we define
    $$
    \mathsf{test\_inp}^i_b:=\begin{cases}
    \bigl(1-\mathsf{border}_{\rhd}(H_i)\bigr)\cdot \bigl(1-\mathsf{border}_{\lhd}(H_i)\bigr)\cdot(1- R_i^T\cdot H_i) & \text{if $b=0$}\\
    \bigl(1-\mathsf{border}_{\rhd}(H_i)\bigr)\cdot \bigl(1-\mathsf{border}_{\lhd}(H_i)\bigr)\cdot(R_i^T\cdot H_i) & \text{if $b=1$}\\
    \mathsf{border}_{\rhd}(H_i) & \text{if $b=\rhd$}\\
    \mathsf{border}_{\lhd}(H_i) & \text{if $b=\lhd$},
    \end{cases}
    $$
    which returns $1$ if and only if either $b\in\{0,1\}$ is the value in $\mathsf{mat}(R_i)$ at the 
    position encoded by $\mathsf{mat}(H_i)$, or when $b=\rhd$ and $\mathsf{mat}(H_i)$ is the vector 
    $(2,0,\ldots,0)\in\mathbb{R}^n$, or when $b=\lhd$ and $\mathsf{mat}(H_i)$ is the vector 
    $(0,0,\ldots,2)\in\mathbb{R}^n$.
    Similarly, for the work tape $W_i$ and its head tape $H_{W_i}$ we can define an expression $\mathsf{test\_work}^i_b$ like above.
    We then combine all these expressions into a single expression for $q_i\in Q$, 
    $a_1,\ldots,a_\ell,b_1,\ldots, b_s\in\Gamma$:
    $$
    \mathsf{isconf}_{q_i,a_1,\ldots,a_\ell,b_1,\ldots, b_s}:=
    \mathsf{min}(Q_i)\cdot \left(\prod_{j=1}^{\ell} \mathsf{test\_inp}_{a_j}^j\right)
    \cdot\left(\prod_{j=1}^{s} \mathsf{test\_work}_{b_j}^j\right).
    $$
    This expression will return $1$ if and only if the vectors representing the tapes, 
    head positions and states are such that $Q_i$ is the first canonical vector 
    (and thus $T$ is in state $q_i$), the heads point to entries in the tape vectors storing 
    the symbols $a_1,\ldots,a_{\ell}, b_1, \ldots, b_s$ or they point to the first or last positions but have value $2$ (when the symbols are $\rhd$ or $\lhd$). 

    To ensure that at the beginning of the simulation of $T$ by $e_f^{\geq n_0}$ we correctly encode 
    that we are in the initial configuration, we thus need to initialize all vectors 
    $$\mathsf{mat}(H_1),\mathsf{mat}(H_2),\ldots, \mathsf{mat}(H_\ell), \mathsf{mat}(H_{W_1}),\ldots, \mathsf{mat}(H_{W_s})$$
    with the vector $(2,0,0,\ldots,0)\in\mathbb{R}$ since all heads read the symbol $\rhd$. Similarly, 
    we have to initialize $Q_1$ with the first canonical vector since $T$ is in the initial state $q_1$.

    We furthermore need to be able to correctly adjust head positions. We do this by means of the expressions $e_{\mathsf{Prev}}$ and $e_{\mathsf{Next}}$	described earlier.
    A consequence of our encoding is that we need to treat the border cases (corresponding to $\rhd$ and 
    $\lhd$) differently. More specifically, for the input tapes $R_i$ and heads $H_i$ we define 
    $$
    \mathsf{move\_inp}^i_{\mathsf{d}}:=
    \begin{cases}
    \bigl(2\cdot\mathsf{min}(H_i)\bigr)\times H_i &\!\!\!\!\!+ \bigl(1/2\cdot\mathsf{border}_{\lhd}(H_i)\bigr)\times H_i \\
    & \hspace{-4em}+\Bigl(\bigl(1-\mathsf{min}(H_i)\bigr)\cdot\bigl(1-\mathsf{border}_{\lhd}(H_i)\bigr)\Bigr)\times e_{\mathsf{Prev}}\cdot H_i \quad \text{ if $\mathsf{d}=\leftarrow$}\\
    \bigl(2\cdot \mathsf{max}(H_i)\bigr)\times H_i &\!\!\!\!\! + \bigl(1/2\cdot\mathsf{border}_{\rhd}(H_i)\bigr)\times H_i \\
    & \hspace{-4em}+ \Bigl( \bigl(1-\mathsf{max}(H_i)\bigr) \cdot \bigl(1-\mathsf{border}_{\rhd}(H_i)\bigr) \Bigr)\times e_{\mathsf{Next}}\cdot H_i  \quad\!\text{if $\mathsf{d}=\rightarrow$}\\
    H_i & \hspace{17.5em}\!\text{if $\mathsf{d}=\sqcup$}. 
    \end{cases}
    $$
    In other words, we shift to the previous (or next) canonical vector when $\mathsf{d}$ is $\leftarrow$ 
    or $\rightarrow$, respectively, unless we need to move to or from the position that will hold $\rhd$ 
    or $\lhd$. In those cases we readjust $\mathsf{mat}(H_i)$ (which will either $(1,0,\ldots,0)$, $(2,0,\ldots,0)$, 
    $(0,\ldots,0,1)$ or $(0,\ldots,0,2)$) by either dividing or multiplying with $2$. In this way we can 
    correctly infer whether or not the head points to the begin and end markers. For the $i$-th work tape we 
    proceed in a similar way and define $\mathsf{move\_work}^i_{\mathsf{d}}$ by replacing $H_{i}$ with $H_{W_i}$ in the expression above. 
 
    A final ingredient are expressions which update a work tape.
    To define these expression, we need the position and symbol to put on the tape. For the work tape $W_i$:
    $$
    \mathsf{write\_work}_b^i:=\begin{cases}
    W_i  & \text{if $b=\rhd$ or $b=\lhd$}\\
    (1-W_i^T\cdot H_{W_i})\times W_i + (W_i^T\cdot H_{W_i})\times (W_i-H_{W_i}) &\text{if $b=0$}\\
    (1-W_i^T\cdot H_{W_i})\times (W_i+H_{W_i}) + (W_i^T\cdot H_{W_i})\times W_i &\text{if $b=1$},
    \end{cases}
    $$
    We are now finally ready to define $e_f^{\geq n_0}$. Intuitively, we want update all vector variables
	$Q_1,\ldots,\allowbreak Q_m,\allowbreak H_1,\ldots,H_\ell,W_1,\ldots,W_s, H_{W_1},\ldots,H_{W_s}$
	simultaneously using expressions $e_{Q_1},\ldots,e_{Q_m},\allowbreak e_{H_1},\allowbreak\ldots,\allowbreak e_{H_\ell},\allowbreak e_{W_1},\ldots,e_{W_s},e_{H_{W_1}},\ldots,e_{H_{W_s}}$, respectively, and want to perform these updates $n^k$ times,
	which suffices to simulate all steps of the linear space machine. To this aim, we iterate over $k$ vectors $v_1,\ldots, v_k$.
	We succinctly write the expressions  $e_f^{\geq n_0}$ as follows:
    \begin{multline*}
    e_f^{\geq n_0}:= \mathsf{for\,} v_1,\ldots,v_{k},Q_1,\ldots,Q_m,H_1,\ldots,H_\ell,W_1,\ldots,W_s, H_{W_1},\ldots,H_{W_s}. \\
    (e_{Q_1},\ldots,e_{Q_m},e_{H_1},\ldots,e_{H_\ell},e_{W_1},\ldots,e_{W_s},e_{H_{W_1}},\ldots,e_{H_{W_s}}).
    \end{multline*}
	We provide the precise semantics of this expression in Section~\ref{subsubsec:generalloop} by casting it as a \langfor expression.
	Intuitively, each possible assignment of $v_1,\ldots,v_k$ to canonical vectors $b_{i_1}^n,\ldots, b_{i_k}^n$ corresponds to an index $\mathbf{i}=(i_1,\ldots,i_k)\in\{1,\ldots,n\}^k$
	and the order in which canonical vectors are considered imposes an ordering on these indices. We can regard these indices as ``timestamps''
	with $(1,1,\ldots,1)$ being the initial timestamp, and $(n,n,\ldots,n)$ the last timestamp.
		If we denote by $Q_j^{\mathbf{i}}$, for $j\in[m]$, $H_j^{\mathbf{i}}$, for $j\in[\ell]$, $W_j^{\mathbf{i}}$ and $H_{W_j}^{\mathbf{i}}$, for $j\in[s]$, the values of these vector variables at timestamp $\mathbf{i}$ during the evaluation of $e_f^{\geq n_0}$, then if $\mathbf{i}'$ is the next timestamp, we want to update them as follows:
	\begin{align*}
		Q_j^{\mathbf{i}'}&:=e_{Q_j}(Q_1^{\mathbf{i}},\ldots Q_m^{\mathbf{i}},H_1^{\mathbf{i}},\ldots,H_\ell^{\mathbf{i}},W_1^{\mathbf{i}},\ldots,W_s^{\mathbf{i}},H_{W_1}^{\mathbf{i}},\ldots,H_{W_s}^{\mathbf{i}},b_{i_1'}^n,\ldots,b_{i_k'}^n)\\
		H_j^{\mathbf{i}'}&:=e_{H_j}(Q_1^{\mathbf{i}},\ldots Q_m^{\mathbf{i}},H_1^{\mathbf{i}},\ldots,H_\ell^{\mathbf{i}},W_1^{\mathbf{i}},\ldots,W_s^{\mathbf{i}},H_{W_1}^{\mathbf{i}},\ldots,H_{W_s}^{\mathbf{i}},b_{i_1'}^n,\ldots,b_{i_k'}^n)\\
		W_j^{\mathbf{i}'}&:=e_{W_j}(Q_1^{\mathbf{i}},\ldots Q_m^{\mathbf{i}},H_1^{\mathbf{i}},\ldots,H_\ell^{\mathbf{i}},W_1^{\mathbf{i}},\ldots,W_s^{\mathbf{i}},H_{W_1}^{\mathbf{i}},\ldots,H_{W_s}^{\mathbf{i}},b_{i_1'}^n,\ldots,b_{i_k'}^n)\\
				H_{W_j}^{\mathbf{i}'}&:=e_{H_{W_j}}(Q_1^{\mathbf{i}},\ldots Q_m^{\mathbf{i}},H_1^{\mathbf{i}},\ldots,H_\ell^{\mathbf{i}},W_1^{\mathbf{i}},\ldots,W_s^{\mathbf{i}},H_{W_1}^{\mathbf{i}},\ldots,H_{W_s}^{\mathbf{i}},b_{i_1'}^n,\ldots,b_{i_k'}^n)\\
O^{\mathbf{i}'}&:=e_{O}(Q_1^{\mathbf{i}},\ldots Q_m^{\mathbf{i}},H_1^{\mathbf{i}},\ldots,H_\ell^{\mathbf{i}},W_1^{\mathbf{i}},\ldots,W_s^{\mathbf{i}},H_{W_1}^{\mathbf{i}},\ldots,H_{W_s}^{\mathbf{i}},b_{i_1'}^n,\ldots,b_{i_k'}^n)\\
H_{O}^{\mathbf{i}'}&:=e_{H_O}(Q_1^{\mathbf{i}},\ldots Q_m^{\mathbf{i}},H_1^{\mathbf{i}},\ldots,H_\ell^{\mathbf{i}},W_1^{\mathbf{i}},\ldots,W_s^{\mathbf{i}},H_{W_1}^{\mathbf{i}},\ldots,H_{W_s}^{\mathbf{i}},b_{i_1'}^n,\ldots,b_{i_k'}^n).
	\end{align*}
And this precisely what $e_f^{\geq n_0}$ does. Furthermore,	it ensures that  $W_1^{(n,n,\ldots,n)}$ is such that it encodes the result $f_n(w_1,\ldots,w_\ell)$. We remark that we
always perform $n^k$ iterations which is fine because we assume that once the Turing machine ends in an accepting configuration,
it stays there (it loops). We also recall that we assume that an accepting configuration is always reached. It remains to
explain the expressions used to do the updates. We do this by using all expressions defined earlier, for moving heads and updating
tapes, in combination with a check of what configuration is currently present. Furthermore, we need to ensure that at the initial timestamp,
when each vector variable $v_i$ is assigned to $b_1^n$, we initialize states and heads correctly. As such, we include a check
$\prod_{j=1}^{k} \textsf{min}(v_i)$ to detect whether or not we are at the initial timestamp. More precisely, the update expressions
are given as follows, where we use $\star$ to mark irrelevant information in the transitions: 
    \begin{align*}   
        e_{Q_1}&:=\left(\prod_{j=1}^{k} \textsf{min}(v_i)\right)\times e_{\mathsf{min}}
        + \sum_{\substack{(q_i,a_1,\ldots,a_\ell,b_1, \ldots, b_s)\\
        \Delta(q_i,a_1,\ldots,a_\ell,b_1, \ldots, b_s)=(q_1,\star)}} \!\!\!\!\!\!\!\!\! \mathsf{isconf}_{q_i,a_1,\ldots,a_\ell,b_1, \ldots, b_s}\times e_{\mathsf{min}} \\
        e_{Q_j}&:=\sum_{\substack{(q_i,a_1,\ldots,a_\ell,b_1, \ldots, b_s)\\
        \Delta(q_i,a_1,\ldots,a_\ell,b_1, \ldots, b_s)=(q_j,\star)}} \!\!\!\!\!\!\!\!\! \mathsf{isconf}_{q_i,a_1,\ldots,a_\ell,b_1, \ldots, b_s}\times e_{\mathsf{min}}
        \quad \text{for $j\neq 1$}\displaybreak\\
        e_{H_i}&:=2\left(\prod_{j=1}^{k} \textsf{min}(v_i)\right)\times e_{\mathsf{min}}
        +\sum_{\substack{(q,a_1,\ldots,a_\ell,b_1, \ldots, b_s) \\  \Delta(q,a_1,\ldots,a_\ell,b_1, \ldots, b_s)=(\star,\mathsf{d_i},\star)}}\!\!\!\!\!\!\!\!\! \mathsf{isconf}_{q,a_1,\ldots,a_\ell,b_1, \ldots, b_s}\times\mathsf{move\_inp}^i_{\mathsf{d}_i}\\
        e_{H_{W_i}}&:=2\left(\prod_{j=1}^{k} \textsf{min}(v_i)\right)\times e_{\mathsf{min}}
    +\sum_{\substack{(q,a_1,\ldots,a_\ell,b_1, \ldots, b_s) \\ \displaybreak\\
    \Delta(q,a_1,\ldots,a_\ell,b_1, \ldots, b_s)=(\star,\mathsf{d_{\ell+i}},\star)}}\!\!\!\!\!\!\!\!\! \mathsf{isconf}_{q,a_1,\ldots,a_\ell,b_1, \ldots, b_s}\times\mathsf{move\_work}_{\mathsf{d}_{\ell+i}}^i\\
 	    e_{W_i}&:=\sum_{\substack{(q,a_1,\ldots,a_\ell,b_1, \ldots, b_s)\displaybreak\\
        \Delta(q,a_1,\ldots,a_\ell,b_1, \ldots, b_s)=(\star,b'_i,\star)}}\!\!\!\!\!\!\!\!\! \mathsf{isconf}_{q,a_1,\ldots,a_\ell,b_1, \ldots, b_s}\times\mathsf{write\_work}_{b_i'}^i.
    \end{align*}
With these update expressions at hand, the correctness of $e_f^{\geq n_0}$ should be clear from the construction (one can formally verify this by
    induction on the number of iterations). 
	
\medskip
\noindent
\underline{\textbf{(b)} Small $n$, i.e. $n< n_0$.}	
We next consider the case when $n<n_0$. For each $n<n_0$ and every possible input words
    $w_1,\ldots,w_\ell$ of size $n$, we define a \langfor expression which checks whether
    $\mathsf{mat}(R_i)=\mathsf{vec}(w_i)$ for each $i\in[\ell]$. This can be easily done since $n$ 
    can be regarded as a constant. For example, to check whether $\mathsf{mat}(R_i)=[0,1,1]^T$ we simply write
    $$
    (1- R_i^T\cdot e_{\mathsf{min}})\cdot (R_i^T\cdot e_{\mathsf{Next}}\cdot e_{\mathsf{min}})\cdot (1- R_i^T\cdot e_{\mathsf{Next}}\cdot e_{\mathsf{Next}}\cdot e_{\mathsf{min}})\times (1- e_{\ones}(R_i)^T\cdot e_{\mathsf{Next}}\cdot e_{\mathsf{Next}}\cdot e_{\mathsf{Next}}\cdot e_{\mathsf{min}})
    $$
    which will evaluate to $1$ if and only if $\mathsf{mat}(R_i)=[0,1,1]^T$. We note that the final factor is in 
    place to check that the dimension of $\mathsf{mat}(R_i)$ is three.
    We denote by
    $e_{n,w}^i$ the expression which evaluates to $1$ if and only if $\mathsf{mat}(R_i)=\mathsf{vec}(w)$
    for $|w|=n$.
    We can similarly
    write any word $w$ of fixed size in the matrix variable $W_1$. For example, suppose that $w=101$
    then we write 
    $$
    W_1+ e_{\mathsf{min}}+  e_{\mathsf{Next}}\cdot e_{\mathsf{Next}}\cdot e_{\mathsf{min}}.
    $$
    We let $e_{n,w}$ be the expression which writes $w$ of size $|w|=n$ in matrix variable $W_1$.
    Then, the expression
    $$
    e_{n,w_1,\ldots,w_n,w}:=(e_{n,w_1}^1\cdot\cdots\cdot e_{nw_{\ell}}^\ell)\times e_{n,w}
    $$
    will write $w$ in $W_1$ if and only if $\mathsf{mat}(R_i)=\mathsf{vec}(w_i)$ for $i\in[\ell]$.
    We now simply take the disjunction over all words 
    $w_1,\ldots,w_\ell\in\Sigma^n$ and $w=f_n(w_1,\ldots,w_\ell)\in\Sigma^n$:
    $$
    e_n:=\sum_{w_1,\ldots,w_\ell\in\Sigma^n} e_{n,w_1,\ldots,w_\ell,f_n(w_1,\ldots,w_\ell)},
    $$
    which correctly writes $f_n(w_1,\ldots,w_\ell)$ in $W_1$. For $e_f^{<n_0}$ it now remains
to take a further disjunction for $n=0,\ldots, n_0-1$. That is,
    $
    e_f^{<n_0}:=\sum_{n=0}^{n_0-1} e_n
    $.
    Since every possible input is covered and only a unique expression 
    $e_{n,w_1,\ldots,w_\ell,f_n(w_1,\ldots,w_\ell)}$ will be triggered, $e_f^{<n_0}$ will correctly
    evaluate $f$ on input words smaller than $n_0$.

\medskip
\noindent
\underline{\textbf{(c)} The final expression $e_f$.}	
 The desired final expression $e_f$ is now given by
    $$
    e_f:=e_f^{<n_0} + \bigl(e_{\ones}(R_1)^T\cdot\underbrace{e_{\mathsf{Next}}\cdot\cdots\cdot e_{\mathsf{Next}}}_{\text{$n_0$ times}}\bigr)\times e_f^{\geq n_0},
    $$
where we note that $e_{\ones}(R_1)^T\cdot\underbrace{e_{\mathsf{Next}}\cdot\cdots\cdot e_{\mathsf{Next}}}_{\text{$n_0$ times}}$ evaluates to $1$ if
$n\geq n_0$ and to $0$ otherwise.
\end{proof}

\subsubsection{Generalized iteration expressions}\label{subsubsec:generalloop}
The expression $e_f^{\geq n_0}$ is given in terms of an expression of the form 
\begin{align*}
    \texttt{for}\, v_1,\ldots, v_k,X_1,\ldots, X_\ell \texttt{.}\, \Big( &e_1(X_1,\ldots,X_\ell,v_1,\ldots, v_k), \\
    &\hspace{1em}e_2(X_1,\ldots, X_\ell,v_1,\ldots, v_k), \ldots, e_\ell(X_1,\ldots,X_\ell,v_1,\ldots, v_k) \Big),
\end{align*}
where $X_1,X_2,\ldots,X_\ell$ are vector variables. Intuitively, each $X_i$ is updated according to expression $e_i$.
We now give the formal semantics of such expressions by defining it as a \langfor expression, for which we have defined the precise semantics in Section~\ref{sec:formatlang}.
More precisely, the expression above is simply a shorthand notation for the \langfor expression
\begin{align*}
  &\ffor{v_1}{X_1}{} \\
  &\hspace{1em}\initf{X_1}{v_2}{X_2}{} \\
  &\hspace{2em}\initf{X_2}{v_3}{X_3}{} \\
  &\hspace{8em}\ddots \\
  &\hspace{4em}\initf{X_{k-1}}{v_k}{X_k}{ e'(X_k,v_1,\ldots, v_k)}
\end{align*}
with $e'(X,v_1,\ldots,v_k)$ the expression
$$
\sum_{i=1}^{\ell} e_i(X\cdot e_{\min}, X\cdot e_{\mathsf{min}+1},\ldots,X\cdot e_{\mathsf{min}+(\ell-1)},v_1,\ldots,v_k)\cdot \bigl(
e_{\mathsf{diag}}(e_{\mathbf{1}}(X^T))\cdot e_{\mathsf{min}+(i-1)}\bigr)^T,
%
% e_1(X\cdot e_{\mathsf{min}},v_1,\ldots,v_k)\cdot (e_{\diag}(e_{\ones}(X^T))\cdot e_{\mathsf{min}})^T \\
% &+ e_2(X\cdot e_{\mathsf{min} + 1},v_1,\ldots,v_k)\cdot (e_{\diag}(e_{\ones}(X^T))\cdot e_{\mathsf{min} + 1})^T \\
% &+ \ldots + e_\ell(X\cdot e_{\mathsf{max}},v_1,\ldots,v_k)\cdot (e_{\diag}(e_{\ones}(X^T))\cdot e_{\mathsf{max}})^T
$$
where $e_{\mathsf{min}+\mathsf{i}}$, for $i=0,1,\ldots,n-1$ returns the $(i+1)$th canonical vector. Here, if the $v_i$'s have
size $\alpha\times 1$ then $X$ has size
$\alpha\times\beta$ and we assume on instances that $\dom(\beta)=\ell$. The $i$th column of  $X$ thus corresponds to the vector
variable $X_i$ and to evaluate $e_i$ we thus first extract all columns separately, by means of $X\cdot e_{\mathsf{min}+\mathsf{i}}$, and feed all of these into $e_i$.
We remark that the canonical vectors used are $b_{j}^{\dom(\beta)}$, i.e. of dimension $\ell\times 1$.
After evaluating $e_i$, which returns a vector of size $\alpha\times 1$, we place this vector back in the $i$the column of $X$ by multiplying with the appropriate
row vector corresponding to column $i$. This is done by the last factor in the expression $e'$. Furthermore, $e_{\mathsf{diag}}(e_{\mathbf{1}}(X^T))$ is included to ensure that
$e_{\mathsf{min} + \mathsf{(i-1)}}$ again evaluates to the canonical vector $b_{i}^{\dom(\beta)}$ of dimension $\ell\times 1$.
% %
% %
% %
% %  We sometimes will want to iterate over $k$ canonical vectors. We define the following shorthand notation:
% %
% % \begin{align*}
% %   \ffor{v_1,\ldots, v_k}{X}{e(X,v_1,\ldots, v_n)}:= &\ffor{v_1}{X_1}{X_1 +} \\
% %   &\hspace{1em}\initf{X_1}{v_2}{X_2}{X_2 + } \\
% %   &\hspace{2em}\initf{X_2}{v_3}{X_3}{X_3 + } \\
% %   &\hspace{8em}\ddots \\
% %   &\hspace{4em}\initf{X_{k-1}}{v_k}{X_k}{ e(X_k,v_1,\ldots, v_k)}.
% % \end{align*}
% %
% % To reference $\ell$ different vector variables $X_1,\ldots,X_\ell$ in every iteration and update them in different ways we define:
% % \begin{multline*}
% % \ffor{v}{X_1,\ldots, X_\ell}{\left( e_1(X_1,v), e_2(X_2,v), \ldots, e_l(X_\ell,v) \right)} := \\
% % \ffor{v}{X}{e_1(X\cdot e_{\mathsf{min}},v)\cdot (e_{\diag}(e_{\ones}(X^T))\cdot e_{\mathsf{min}})^T +\\ e_2(X\cdot e_{\mathsf{min} + 1},v)\cdot (e_{\diag}(e_{\ones}(X^T))\cdot e_{\mathsf{min} + 2})^T + \ldots \\
% % + e_\ell(X\cdot e_{\mathsf{max}},v)\cdot (e_{\diag}(e_{\ones}(X^T))\cdot e_{\mathsf{max}})^T} \\
% % \end{multline*}
%
% We note that for the latter expression to be semantically correct $v$ has to be of type $\gamma\times 1$,
% both $X_i$ and $e_i$ for $ i=1,\ldots,\ell$ have to be of type $\alpha\times 1$,
% and $X$ has to be of type $\alpha\times\beta$, where $\dom(\beta)=\ell$. Here
% we use $e_{\diag}(e_{\ones}(X^T))$ to compute the $\beta\times\beta$ identity and ensure the typing of the
% $e_{\mathsf{min} + i}$.
% When evaluated on an instance $\I$,
% $e_{\mathsf{min}}, e_{\mathsf{min} + i}$ evaluate to $b_1^{\dom(\beta)}$ and $b_{1+i}^{\dom(\beta)}$,
% respectively, and we show their defining expressions in section \ref{sec:formatlang:design}.
% Similarly for $e_{\mathsf{max}}=b_n^{\dom(\beta)}$.
% The combinations of both previous operators results in:
%
% \begin{align*}
%     \texttt{for}\, v_1,\ldots, v_k,X_1,\ldots, X_\ell \texttt{.}\, \Big( &e_1(X_1,v_1,\ldots, v_k), \\
%     &\hspace{1em}e_2(X_2,v_1,\ldots, v_k), \ldots, e_\ell(X_\ell,v_1,\ldots, v_k) \Big) := \\
%     &\hspace{8em}\ffor{v_1,\ldots, v_k}{X}{e'(X,v_1,\ldots, v_k)} \\
% \end{align*}
% where
% \begin{align*}
% e'(X,v_1,\ldots,v_k):=&e_1(X\cdot e_{\mathsf{min}},v_1,\ldots,v_k)\cdot (e_{\diag}(e_{\ones}(X^T))\cdot e_{\mathsf{min}})^T \\
% &+ e_2(X\cdot e_{\mathsf{min} + 1},v_1,\ldots,v_k)\cdot (e_{\diag}(e_{\ones}(X^T))\cdot e_{\mathsf{min} + 1})^T \\
% &+ \ldots + e_\ell(X\cdot e_{\mathsf{max}},v_1,\ldots,v_k)\cdot (e_{\diag}(e_{\ones}(X^T))\cdot e_{\mathsf{max}})^T
% \end{align*}
% It is clear that this expression iterates over $k$ canonical vectors and references $\ell$ independent vectors updating each of them in their particular way.


% Now we can prove theorem \ref{th-circuits-ml}
%
% \begin{proof}

    % For a stack $S$, the operations are standard:
    %
    % \begin{itemize}
    %     \item $\push{S}{s}$: pushes $s$ into $S$.
    %     \item $\pop{S}$: pops the top element.
    %     \item $\getsize{S}$: the length of the stack.
    %     \item $\gettop{S}$: the top element in the stack.
    % \end{itemize}
    %
    % For the pseudo-code, $\cG$ and $\cV$ denote stacks of gates and values, respectively. The property that holds during the simulation is that the value in $\cV[i]$ is the value that $\cG[i]$ currently outputs. The algorithm ends with $\cG=\left[ g_{\texttt{root}}\right]$ and $\cV=\left[ v_{\texttt{root}}\right]$ after traversing the circuit, and returns $v_{\texttt{root}}$
    %
    % During the evaluation algorithm there will be two possible configurations of $\cG$ and $\cV$.
    %
    % \begin{enumerate}
    %     \item $\getsize{\cG} = \getsize{\cV} + 1$: this means that $\gettop{\cG}$ is a gate that we visit for the first time and we need to initialize its value.
    %
    %     \item $\getsize{\cG} = \getsize{\cV}$: here $\gettop{\cV}$ is the value of evaluating the circuit in gate $\gettop{\cG}$. Therefore, we need to aggregate the value $\gettop{\cV}$ to the parent gate of $g$.
    % \end{enumerate}
    %
    % We assume the circuit has input gates, $+, \times$-gates and allow constant $1$-gate.
    %
    % The idea is to traverse the circuit top down in a depth first search way. For example, in the circuit $f(a_1,a_2,a_3,a_4)=a_1a_2 +a_3a_4$ above, we would initialize the output gate value as $0$ because it is a $+$ gate, so $\cG=\lbrace +\rbrace$, $\cV=\lbrace 0\rbrace$. Then stack the left $\times$ gate to $\cG$, stack its initial value (i.e. $1$) to $\cV$. Now stack $a_1$ to $\cG$ and its value (i.e. $a_1$) to $\cV$. Since we are on an input gate we pop the gate and value pair off of $\cG$ and $\cV$ respectively, aggregate $a_1$ to $\gettop{\cV}$ and continue by stacking the $a_2$ gate to $\cG$. We pop $a_2$ off of $\cV$ (and its gate off of $\cG$) and aggregate its value to $\gettop{\cV}$. We pop and aggregate the value of the left $\times$ gate to $\gettop{\cV}$ (the root value). Then continue with the right $\times$ gate branch similarly.



\smallskip
While Theorem \ref{th-circuits-ml} gives us an idea on how to simulate arithmetic circuits, it does not tell us which classes of functions over real numbers can be computed by \langfor expressions. In order to answer this question, we note that arithmetic circuits can be used to compute functions over real numbers. Formally, a circuit family $\{\Phi_n\mid n=1,2,\ldots\}$ computes a function $f:\bigcup_{n\geq 1} \mathbb{R}^n\mapsto\mathbb{R}$, if for any $a_1,\ldots a_n\in \mathbb{R}$ it holds that $\Phi_n(a_1,\ldots ,a_n) = f(a_1,\ldots ,a_n)$. To make the connection with \langfor\!, we need to look at circuit families of bounded degree. 

A circuit family $\{\Phi_n\mid n=1,2,\ldots\}$ is said to be of \textit{polynomial degree} if $\mathsf{degree}(\Phi_n)\in O(p(n))$, for some polynomial $p(n)$. Note that polynomial size circuit families are not necessarily of polynomial degree. An easy corollary of Theorem \ref{th-circuits-ml} tells us that all functions computed by uniform family of circuits of polynomial degree and logarithmic depth can be simulated using \langfor expressions. However, we can actually drop the restriction on circuit depth due to the result of Valiant et. al.~\cite{valiant1981fast} and Allender et. al. \cite{AllenderJMV98} which says that any function computed by a uniform circuit family of polynomial degree (and polynomial depth), can also be computed by a uniform circuit family of logarithmic depth. Using this fact, we can conclude the following:


\begin{corollary}
\label{cor-circ-ml}
For any function $f$ computed by a uniform family of arithmetic circuits of polynomial degree, there is an equivalent \langfor formula $e_f$.
\end{corollary}

Note that there is nothing special about circuits that have a single output, and both Theorem \ref{th-circuits-ml} and Corollary \ref{cor-circ-ml} also hold for functions  $f:\bigcup_{n\geq 1} \mathbb{R}^n\mapsto\mathbb{R}^{s(n)}$, where $s$ is a polynomial. Namely, in this case, we can assume that circuits for $f$ have multiple output gates, and that the depth reduction procedure of \cite{AllenderJMV98} is carried out for each output gate separately. Similarly, the construction underlying the proof of Theorem \ref{th-circuits-ml} can be performed for each output gate independently, and later composed into a single output vector.

\subsection{From \langfor to circuits}

Now that we know that arithmetic circuits can be simulated using \langfor expressions, it is natural to ask whether the same holds in the other direction. That is, we are asking whether for each \langfor expression $e$ over some schema $\Sch$ there is a uniform family of arithmetic circuits computing precisely the same result depending on the input size. 

In order to handle the fact that \langfor\ expressions can produce any matrix, and not just a single value, as their output, we need to consider circuits which have multiple output gates. Similarly, we need to encode matrix inputs of a \langfor\ expression in our circuits. We will write $\Phi(A_1,\ldots ,A_k)$, where $\Phi$ is an arithmetic circuit with multiple output gates, and each $A_i$ is a matrix of dimensions $\alpha_i\times \beta_i$, with $\alpha_i,\beta_i \in \{n,1\}$ to denote the input matrices for a circuit $\Phi$. We will also write $\texttt{type}(\Phi)=(\alpha,\beta)$, with $\alpha,\beta\in \{n,1\}$, to denote the size of the output matrix for $\Phi$. We call such circuits \textit{arithmetic circuits over matrices}. When $\{\Phi_n\mid n=1,2,\ldots\}$ is a uniform family of arithmetic circuits over matrices, we will assume that the Turing machine for generating $\Phi_n$ also gives us the information about how to access a position of each input matrix, and how to access the positions of the output matrix, as is usually done when handling matrices with arithmetic circuits \cite{Raz02}. The notion of degree is extended to be the sum of the degrees of all the output gates. With this  at hand, we can now show the following result.

\begin{theorem}
\label{th-ml-to-circuits}
Let $e$ be a \langfor expression over a schema $\Sch$, and let $V_1,\ldots ,V_k$ be the variables of $e$ such that $\ttype(V_i)\in \{(\alpha,\alpha), (\alpha,1), (1,\alpha), (1,1)\}$. Then there exists a uniform arithmetic circuit family over matrices $\Phi_n(A_1,\ldots ,A_k)$ such that:
\begin{itemize}
\item For any instance $\I = (\dom,\conc)$ such that $\dom(\alpha) = n$ and $\conc(V_i) = A_i$ it holds that:
\item $\sem{e}{\I} = \Phi_n(A_1,\ldots ,A_k)$.
\end{itemize}
\end{theorem}

%!TEX root = ../../main.tex
We here show how to convert \langfor expressions to arithmetic circuits over matrices.
Such circuits take matrices as input and can also return matrices by means of multiple
output gates. The translation from \langfor to circuits is done inductively, on the structure
of \langfor expressions, and we inductively build the corresponding circuit. To this aim, we keep track of
the type of circuits, which basically indicates the size of the output matrix. More specifically,
for a circuit $\Phi$ we write $\ttype(\Phi)=(\alpha,\beta)$, with $\alpha,\beta\in \{n,1\}$, to denote the size of 
the output matrix of $\Phi$. 
%
% We denote such circuits by $\Phi(A_1,\ldots ,A_k)$, where $\Phi$ is an arithmetic circuit
% with multiple output gates, and each  $A_i$ is a matrix of dimensions $\alpha_i\times \beta_i$,
% with $\alpha_i,\beta_i \in \{n,1\}$ to denote the input matrices for a circuit $\Phi$.
% When $\{\Phi_n\mid n=1,2,\ldots\}$ is a uniform family of
% arithmetic circuits over matrices, we will assume that the Turing machine for generating $\Phi_n$ also
% gives us the information about how to access a position of each input matrix, and how to access the
% positions of the output matrix, as is usually done when handling matrices with arithmetic
% circuits \cite{Raz02}.
%
% We will also write $\ttype(\Phi)=(\alpha,\beta)$, with
% $\alpha,\beta\in \{n,1\}$, to denote the size of the output matrix for $\Phi$.
We describe circuits by specifying circuits for each output gate. We refer to
these circuits by $\Phi_{n}[i,j]$ when $\ttype(\Phi)=(n,n)$, 
$\Phi_{n}[i,1]$ when $\ttype(\Phi)=(n,1)$, $\Phi_{n}[1,j]$ when $\ttype(\Phi)=(1,n)$ and 
$\Phi_{n}$ when $\ttype(\Phi)=(1,1)$. The indices $i$ and $j$ always range over $\{1,2,\ldots,n\}$.


% The notion of degree is extended to be the sum of the degrees of all
% the output gates. The former will b. Also, when we write $a \oplus b$ we mean
As will become clear shortly, we translate addition and multiplication in \langfor
expressions in terms of addition ($\oplus$) and multiplication ($\otimes$) of circuits.
That is, given a set $I$ of output gates of a circuit $\Phi$, we define 
$\bigoplus_{g\in I} \Phi[g]$ as the circuit consisting a $+$-gate with children
$\Phi(g)$, for $g\in I$. Similarly, $\bigotimes_{g\in I} \Phi(g)$ is a circuit
consisting of a $\times$-gate with children $\Phi(g)$, for $g\in I$. When 
we write $\Phi[g]=\Phi_1[g_1]\oplus \Phi_2[g_2]$ we mean that the output gate
$g$ in $\Phi$ is the top-most $+$-gate in the circuit $\Phi_1[g_1]\oplus \Phi_2[g_2]$, and 
similarly for $\Phi[g]=\Phi_1[g_1]\otimes \Phi_2[g_2]$.


% We also write
% $\Phi(g)\oplus \Phi(g')$
%
%
% More precisely, if we h
% %
% % \begin{center}
% % \begin{tikzpicture}[level distance=1.5cm,
% %   level 1/.style={sibling distance=1.5cm},
% %   every node/.style = {
% %   	shape=circle,
% %     draw,
% %     align=center,
% %     top color=white,
% %     bottom color=white
% %     }]
% %   \node {\( + \)}
% %     child {node { \( a \) }}
% %     child {node { \( b \) }};
% % \end{tikzpicture}
% % \end{center}
% % When we write $\bigoplus_{l=1}^n a_l$ we mean
%
% \begin{center}
% \begin{tikzpicture}[level distance=1.5cm,
%   level 1/.style={sibling distance=1.5cm},
%   every node/.style = {
%   	shape=circle,
%     draw,
%     align=center,
%     top color=white,
%     bottom color=white
%     }]
%   \node {\( + \)}
%     child {node { \( a_1 \) }}
%     child {node { \( \cdots \) }}
%     child {node { \( a_n \) }};
% \end{tikzpicture}
% \end{center}
% Same with $\otimes$. Now we prove the statement.
%

\medskip
Let $e$ be a \langfor expression. We define the circuit $\Phi_n^e$ and its type $\ttype(\Phi_n^e)$ inductively on the structure of $e$.

\begin{itemize}
	\item
	If $e=V$ then $\Phi_n^e$ consists of gates corresponding to an input matrix $A$. More specifically:
\begin{itemize}
	\item If $\ttype(V)=(1,1)$ then $\ttype(\Phi^e_n)=(1,1)$ and $\Phi_n^e$ consists of single input gate, which is also
	the output gate.
	% $\Phi^e_n[1,1]$ has  one input/output gate.
	\item If $\ttype(V)=(1,\alpha)$ then $\ttype(\Phi^e_n)=(1,n)$ and $\Phi^e_n$ has $n$ input/output gates.
  \item If $\ttype(V)=(\alpha,1)$ then $\ttype(\Phi^e_n)=(n,1)$ and $\Phi^e_n$ has $n$ input/output gates.
	\item If $\ttype(V)=(\alpha,\alpha)$ then $\ttype(\Phi^e_n)=(n,n)$ and $\Phi^e_n$ has $n^2$ input/output gates. 
\end{itemize}
\item
If $e=e_1^T$ then $\Phi^e_n$ is $\Phi^{e_1}_n$ but in which the output gates of $\Phi_n^{e_1}$ are rearranged, as follows:
\begin{itemize}
	\item If $\ttype(\Phi^{e_1}_n)=(1, 1)$ then $\type(\Phi^e_n)=(1,1)$ and $\Phi_n^e\coloneqq \Phi_n^{e_1}$.
	\item If $\ttype(\Phi^{e_1}_n)=(1, n)$ then $\type(\Phi^e_n)=(n,1)$ and $\Phi^e_n[i,1]\coloneqq \Phi^{e_1}_n[1,i]$. 
  \item If $\ttype(\Phi^{e_1}_n)=(n, 1)$ then $\type(\Phi^e_n)=(1,n)$ and $\Phi^e_n[1,j]\coloneqq \Phi^{e_1}_n[j,1]$. 
  \item If $\ttype(\Phi^{e_1}_n)=(n, n)$ then $\type(\Phi^e_n)=(n,n)$ and $\Phi^e_n[i,j]\coloneqq \Phi^{e_1}_n[j,i]$. 
\end{itemize}
\item
If $e={\ones}(e_1)$ with $\ttype(\Phi^{e_1}_n)=(\alpha,\beta)$ then $\ttype(\Phi^{e}_n)=(\alpha,1)$ and we just
create a circuit consisting of $\alpha$ $1$-gates, which are also the output gates. We write this as
$\Phi^e_n[i,1]\coloneqq 1$.
\item
If $e=e_1 + e_2$ then $\Phi_n^e$ consists of $\Phi_n^{e_1}$ and $\Phi_n^{e_2}$ with a new layer connecting the output gates of $\Phi_n^{e_1}$ and $\Phi_n^{e_2}$ by means
of $+$-gates, as follows:
\begin{itemize}
	\item If $\ttype(\Phi^{e_1}_n)=\ttype(\Phi^{e_2}_n)=(1, 1)$  then $\ttype(\Phi^{e}_n)=(1, 1)$ and $\Phi^e_n\coloneqq \Phi^{e_1}_n \oplus \Phi^{e_2}_n$.
  \item If $\ttype(\Phi^{e_1}_n)=\ttype(\Phi^{e_2}_n)=(1, n)$  then $\ttype(\Phi^{e}_n)=(1, n)$ and $\Phi^e_n[1,j]\coloneqq \Phi^{e_1}_n[1,j] \oplus \Phi^{e_2}_n[1,j]$.
  \item If $\ttype(\Phi^{e_1}_n)=\ttype(\Phi^{e_2}_n)=(n, 1)$  then $\ttype(\Phi^{e}_n)=(n, 1)$ and $\Phi^e_n[i,1]\coloneqq \Phi^{e_1}_n[i,1] \oplus \Phi^{e_2}_n[i,1]$.
  \item If $\ttype(\Phi^{e_1}_n)=\ttype(\Phi^{e_2}_n)=(n, n)$  then $\ttype(\Phi^{e}_n)=(n, n)$ and $\Phi^e_n[i,j]\coloneqq \Phi^{e_1}_n[i,j] \oplus \Phi^{e_2}_n[i,j]$.
\end{itemize}
% \end{itemize}
% If $e=f(e_1, \ldots, e_k)$ we have two cases
%
% \begin{itemize}
%   \item When $f$ is the function $f_{\odot}$ (recall that this function is definable in $\langf{\emptyset}$ by Lemma \ref{lm-prod-sum}) then
%   \begin{itemize}
%     \item If $\ttype(\Phi^{e_1}_n)=\ldots =\ttype(\Phi^{e_k}_n)=(1, 1)$ then $\Phi^e_n\coloneqq \bigotimes_{l=1}^k \Phi^{e_l}_n$.
%     \item If $\ttype(\Phi^{e_1}_n)=\ldots =\ttype(\Phi^{e_k}_n)=(1, n)$ then $\Phi^e_n[1,j]\coloneqq \bigotimes_{l=1}^k \Phi^{e_l}_n[1,j]$.
%     \item If $\ttype(\Phi^{e_1}_n)=\ldots =\ttype(\Phi^{e_k}_n)=(n, 1)$ then $\Phi^e_n[i,1]\coloneqq \bigotimes_{l=1}^k \Phi^{e_l}_n[i,1]$.
%     \item If $\ttype(\Phi^{e_1}_n)=\ldots =\ttype(\Phi^{e_k}_n)=(n, n)$ then $\Phi^e_n[i,j]\coloneqq \bigotimes_{l=1}^k \Phi^{e_l}_n[i,j]$.
%   \end{itemize}
% 	\item When $f$ is any function, we prove the case when $\ttype(\Phi^{e_1}_n)=\ldots =\ttype(\Phi^{e_k}_n)=(1, 1)$
%   (only case necessary, as discussed in section \ref{sec:queries:simp}). Here $\Phi^e_n$ is
%
% \begin{center}
% \begin{tikzpicture}[level distance=1.5cm,
%   level 1/.style={sibling distance=1.5cm},
%   every node/.style = {
%   	shape=circle,
%     draw,
%     align=center,
%     top color=white,
%     bottom color=white
%     }]
%   \node {\( f \)}
%     child {node { \( \Phi^{e_1}_n \) }}
%     child {node { \( \cdots \) }}
%     child {node { \( \Phi^{e_k}_n \) }};
% \end{tikzpicture}
% \end{center}
%
% Note that since for the context of this result we only consider $\langfor = \langf{\emptyset}$, this case is not strictly necessary, modulo for $f_\odot,f_\oplus$ due to Lemma \ref{lm-prod-sum}. However, if we extend the circuits with the same functions allowed in \langfor, then our inductive construction still goes through, as just illustrated.
%
% \end{itemize}
\item 
If $e=e_1\cdot e_2$ then $\Phi_n^e$ consists of $\Phi_n^{e_1}$ and $\Phi_n^{e_2}$ in which we connect the output gates of $\Phi_n^{e_1}$ and $\Phi_n^{e_2}$
by means of $+$-gates and $\times$-gates in order to perform matrix multiplication, as follows:

\begin{itemize}
	\item If $\ttype(\Phi^{e_1}_n)=(1,1)$ and $\ttype(\Phi^{e_2}_n)=(1, 1)$ then $\ttype(\Phi^{e}_n)=(1, 1)$ and $\Phi^{e}_n\coloneqq \Phi^{e_1}_n \otimes \Phi^{e_2}_n$.
  \item If $\ttype(\Phi^{e_1}_n)=(1,1)$ and $\ttype(\Phi^{e_2}_n)=(1, n)$ then $\ttype(\Phi^{e}_n)=(1, n)$ and $\Phi^{e}_n[1,j]\coloneqq \Phi^{e_1}_n \otimes \Phi^{e_2}_n[1,j]$.
  \item If $\ttype(\Phi^{e_1}_n)=(n,1)$ and $\ttype(\Phi^{e_2}_n)=(1, 1)$ then $\ttype(\Phi^{e}_n)=(n, 1)$ and $\Phi^{e}_n[i,1]\coloneqq \Phi^{e_1}_n[i,1] \otimes \Phi^{e_2}_n$.
  \item If $\ttype(\Phi^{e_1}_n)=(n,1)$ and $\ttype(\Phi^{e_2}_n)=(1, n)$ then $\ttype(\Phi^{e}_n)=(n, n)$ and $\Phi^{e}_n[i,j]\coloneqq \Phi^{e_1}_n[i,1] \otimes \Phi^{e_2}_n[1,j]$.
  \item If $\ttype(\Phi^{e_1}_n)=(1,n)$ and $\ttype(\Phi^{e_2}_n)=(n, 1)$ then $\ttype(\Phi^{e}_n)=(1, 1)$ and $\Phi^{e}_n\coloneqq \bigoplus_{k=1}^n \left( \Phi^{e_1}_n[1,k] \otimes \Phi^{e_2}_n[k,1] \right)$.
  \item If $\ttype(\Phi^{e_1}_n)=(1,n)$ and $\ttype(\Phi^{e_2}_n)=(n, n)$ then $\ttype(\Phi^{e}_n)=(1, n)$ and $\Phi^{e}_n[1,j]\coloneqq \bigoplus_{k=1}^n \left( \Phi^{e_1}_n[1,k] \otimes \Phi^{e_2}_n[k,j] \right)$.
  \item If $\ttype(\Phi^{e_1}_n)=(n,n)$ and $\ttype(\Phi^{e_2}_n)=(n, 1)$ then $\ttype(\Phi^{e}_n)=(n, 1)$ and $\Phi^{e}_n[i,1]\coloneqq \bigoplus_{k=1}^n \left( \Phi^{e_1}_n[i,k] \otimes \Phi^{e_2}_n[k,1] \right)$.
  \item If $\ttype(\Phi^{e_1}_n)=(n,n)$ and $\ttype(\Phi^{e_2}_n)=(n, n)$ then $\ttype(\Phi^{e}_n)=(n, n)$ and $\Phi^{e}_n[i,j]\coloneqq \bigoplus_{k=1}^n \left( \Phi^{e_1}_n[i,k] \otimes \Phi^{e_2}_n[k,j] \right)$.
\end{itemize}
\item If $e=e_1\times e_2$ then $\Phi_n$ uses $\times$-gates to connect every output gate of $\Phi_n^{e_2}$ with the output gate $\Phi^e_n[1,1]$ of $\Phi_n^{e_1}$, as follows:
\begin{itemize}
	\item If $\ttype(\Phi^{e_2}_n)=(1, 1)$  then  $\Phi^e_n\coloneqq \Phi^{e_1}_n \otimes \Phi^{e_2}_n$.
  \item If $\ttype(\Phi^{e_2}_n)=(1, n)$  then  $\Phi^e_n[1,j]\coloneqq \Phi^{e_1} \otimes \Phi^{e_2}_n[1,j]$.
  \item If $\ttype(\Phi^{e_2}_n)=(n, 1)$  then  $\Phi^e_n[i,1]\coloneqq \Phi^{e_1}_n \otimes \Phi^{e_2}_n[i,1]$.
  \item If $\ttype(\Phi^{e_2}_n)=(n, n)$ then $\Phi^e_n[i,j]\coloneqq \Phi^{e_1}_n \otimes \Phi^{e_2}_n[i,j]$.
\end{itemize}

\item Finally, if $e=\ffor{X}{v}e_1(X, v)$ then  we first  define $\Phi^{\mathbf{0}}$ 
as the ``zero matrix circuit'' of type  $\ttype(\Phi^{\mathbf{0}})=(1,1)$ if $\ttype(\Phi^{e_1}_n)=(1,1)$ and 
$\ttype(\Phi^{\mathbf{0}})=(n,n)$ if $\ttype(\Phi^{e_1}_n)=(n,n)$.  This circuit consists entirely out of $0$-gates,
that is, $\Phi^{\mathbf{0}}\coloneqq 0$ if $\ttype(\Phi^{\mathbf{0}})=(1,1)$ and
$\Phi^{\mathbf{0}}[i,j]\coloneqq 0$ if $\ttype(\Phi^{\mathbf{0}})=(n,n)$.
Furthermore, for each $i=1,\ldots, n$, we define a ``canonical vector circuit'' $\Phi^{v_i}$ as the circuit such that $\ttype(\Phi^{v_i})=(n,1)$ and $\Phi^{v_i}[i,1]\coloneqq 1$ and $0$ otherwise. So, $\Phi^{v_i}$ consists of a single $1$-gate and $(n-1)$ $0$-gates.
The circuit $\Phi_n^e$ now simply corresponds to the following composition of circuits:
$$\Phi^{e}_n\coloneqq \Phi^{e_1}_n\left( \Phi^{e_1}_n \left( \cdots \left( \Phi^{e_1}_n\left( \Phi^{\mathbf{0}}, \Phi^{v_1}\right), \Phi^{v_2}\right)\cdots, \Phi^{v_{n-1}} \right), \Phi^{v_n} \right).$$
Here $\Phi_n^{e_1}$ is a circuit, inductively computed, that takes a matrix $A$ and canonical vector $b$ as input, and returns a matrix $\Phi_n^{e_1}(A,b)$ of the same type as $A$. In the expression above we initially replace $\Phi_n^{e_1}$ by substituting $A$ and $b$ by $\Phi^{\mathbf{0}}$ and $\Phi^{v_1}$, resulting in the circuit 
$\Phi_n^{e_1}(\Phi^{\mathbf{0}},\Phi^{v_1})$. In the next step, we consider the circuit $\Phi_n^{e_1}$ by substituting $A$ and $b$ by $\Phi_n^{e_1}(\Phi^{\mathbf{0}},\Phi^{v_1})$ and 
$\Phi^{v_2}$, and so on, until the last canonical vector circuit $\Phi^{v_n}$ is considered.
\end{itemize}

The correctness of  $\Phi_n^e$ should be clear from the construction. That is, let $e$ be a \langfor expression over  a schema $\Sch$, and  let $V_1,\ldots ,V_k$ be the variables of $e$ such that $\ttype(V_i)\in \{(\alpha,\alpha),\allowbreak (\alpha,1), (1,\alpha), (1,1)\}$. Then 
for any instance $\I = (\dom,\conc)$ such that $\dom(\alpha) = n$ and $\conc(V_i) = A_i$ it holds that $\sem{e}{\I} = \Phi_n^e(A_1,\ldots ,A_k)$.

% \floris{There was some remark before about the polynomial depth of the obtained circuits (see commented latex). Is this needed?
% The description below is a bit high-level, not sure whether more details are needed. Please check.}
 % Note that every circuit adds a constant number of layers except when $e=\ffor{X}{v}e'(\cI, X, v)$. This means that the depth still is polynomial. When $e=\ffor{X}{v}e'(\cI, X, v)$ we have that the depth of the circuit is $n\cdot p(n)$, where the depth of $e'(\cI, X, v)$ is $p(n)$, so it also remains polynomial.
%
Theorem~\ref{th-ml-to-circuits} also requires showing that $\{\Phi_n^e \mid n=1,2,\ldots\}$ is a uniform arithmetic circuit family over matrices.
We do not provide the precise \logspace-Turing machines that describe these circuits which also need to provide information about how to access a position of each input matrix, and how to access the positions of the output matrix. We remark that the inductive construction of the circuits only depends on $e$ and $n$ and,
except for  $e=\ffor{X}{v}e_1(X, v)$, only simple modifications are made to the previously constructed circuits (at most two additional layers of gates are added).
The corresponding  \logspace-Turing machines can be devised easily. For  $e=\ffor{X}{v}e_1(X, v)$, we compose circuits and to see that uniformity is preserved by such compositions, a proof analogous to the standard proof that the composition of \logspace-functions is again a \logspace-function  \cite{aroraB2009} can be used. It is important to observe that for $e=\ffor{X}{v}e_1(X, v)$, we only need to keep track of where we are in the evaluation (i.e. which $v_i$ we are processing), and not of all the previous results, given that the resulting matrix is updated in a fixed order.

A final comment is that we can extend the above construction even when \langfor expressions using function applications $f$ are considered. We only need to extend
the notion of circuits such that ``$f$-gates'' can be used, where the semantics of $f$-gates is the application of the function $f$ on the values of circuits
corresponding the $f$-gate's children.

% %
% % However, the inductive construction of $\Phi_n^e$ given above only depends on the expression $e$ and
% % the dimension $n$, and in most cases,  except when $e=\ffor{X}{v}e'(X, v)$, we add at most two layers of new gates. An inductive construction of the \logspace\
% % Turing machine is thus possible.
% %
% % For $e=\ffor{X}{v}e'(X, v)$
% % we have that the depth of the circuit is $n\cdot p(n)$, where the depth of $e'(X, v)$ is $p(n)$
% %
% %
% %
% %
% % Note that every circuit adds a constant number of layer
% % This means that the depth still is polynomial. When $e=\ffor{X}{v}e'(X, v)$
% % we have that the depth of the circuit is $n\cdot p(n)$, where the depth of $e'(X, v)$ is $p(n)$,
% % so it also remains polynomial.
% %
% % Here, we do not need to translate scalar multiplication
% % because it can be simulated using the $\mathsf{ones}$ operator and $f_{\kprod}$ (see section \ref{sec:queries:simp}).
%
% Finally, we remark that when composing the circuits the fact that uniformity is preserved (i.e. the resulting circuit can be generated by a \logspace\ machine) is proved analogously as when composing two \logspace\ transducers \cite{aroraB2009}. The only more involved case is treating for-loop construction, however, notice here that we only need to keep track of where we are in the evaluation (i.e. which $v_i$ we are processing), and not of all the previous results, given that they update the resulting matrix in a fixed order.
% \end{proof}

% As a consequence, blah blah....

% In general, uniform arithmetic circuit family $\{\Phi_n^e\mid n=1,2,\ldots\}$ is not necessarily of polynomial degree. Indeed, consider it suffices to consider the MATLANG expression which computes
% $f_n(x)=x^{2^n}$. 
% As uniform arithmetic circuit families of polynomial degree have nice properties, e.g., they can be assumed to be of logarithmic depth, we next want to zoom in such families. In what follows we therefore limit ourselves to MATLANG expressions $e$ such that $\{\Phi_n^e\mid n=1,2,\ldots\}$ is a family of polynomial degree arithmetic circuits.

% \begin{itemize}
% 	\item Checking whether a MATLANG expression $e$ corresponds to a family of of polynomial degree arithmetic circuits is undecidable. 
% \end{itemize}

% Let $e$ be a ``nice'' MATLANG expression, i.e.,  $\{\Phi_n^e\mid n=1,2,\ldots\}$ is a family of polynomial degree arithmetic circuits which can be assumed to be logarithmic depth. 
% \begin{itemize}
% 	\item There exists a MATLANG expression that can
% \end{itemize}

% Let $e$ be a \langfor expression. If $e=V$ we have
% \begin{itemize}
% 	\item When $V$ is $1\times 1$ then $\Phi^e_1$ has the one input/output gate (\textit{not necessary, covered in second item}).
% 	\item When $V$ is $n\times 1$ or $1\times n$ then $\Phi^e_n$ has $n$ input/output gates.
% 	\item When $V$ is $n\times n$ then $\Phi^e_n$ has $n^2$ input/output gates. Here, $V_{ij}=\Phi_n(V)\left[ j+n(i-1)\right]$ and $i,j=1,\ldots, n$ (entries listed row by row).
% \end{itemize}

% If $e=e'^T$ then $\Phi^e_n=\Phi^{e'}_n$. 

% If $e=e_1 + e_2$ we have

% \begin{itemize}
% 	\item When $e$ is $1\times 1$ then $\Phi^e_1$ is $\Phi^{e_1}_1 \oplus \Phi^{e_2}_1$.
% 	\item When $e$ is $n\times 1$ or $1\times n$ then $\Phi^e_n$ has $n$ output gates, where gate $k$ is $\Phi^{e_1}_n[k] \oplus \Phi^{e_2}_n[k]$.
% 	\item When $V$ is $n\times n$ then $\Phi^e_1$ has $n^2$ output gates, where gate $k$ is $\Phi^{e_1}_n[k] \oplus \Phi^{e_2}_n[k]$.
% \end{itemize}

% If $e=f(e_1, \ldots, e_k)$ we have

% \begin{itemize}
% 	\item When $e$ is $1\times 1$ (only case necessary) then $\Phi^e_1$ is 
	
% \begin{center}
% \begin{tikzpicture}[level distance=1.5cm,
%   level 1/.style={sibling distance=1.5cm},
%   every node/.style = {
%   	shape=circle,
%     draw,
%     align=center,
%     top color=white,
%     bottom color=white
%     }]
%   \node {\( f \)}
%     child {node { \( \Phi^{e_1}_1 \) }}
%     child {node { \( \cdots \) }}
%     child {node { \( \Phi^{e_k}_1 \) }};
% \end{tikzpicture}
% \end{center}

% \end{itemize}

% If $e=e_1\cdot e_2$ we have

% \begin{itemize}
% 	\item When $e_1,e_2$ are $1\times 1$ then $\Phi^e_1$ is $\Phi^{e_1}_1 \otimes \Phi^{e_2}_1$.
% 	\item When $e_1$ is $1\times 1$ and $e_2$ is $1\times n$ then $\Phi^e_n$ has $n$ output gates, where output gate $i$ is $\Phi^{e_1}_1 \otimes \Phi^{e_2}_n[i]$.
% 	\item When $e_1$ is $n\times 1$ and $e_2$ is $1\times 1$ then $\Phi^e_n$ has $n$ output gates, where output gate $i$ is $\Phi^{e_1}_n[i] \otimes \Phi^{e_2}_1$.
% 	\item When $e_1$ is $n\times 1$ and $e_2$ is $1\times n$ then $\Phi^e_n$ has $n^2$ output gates, where output gate $k=j + n(i-1)$ is $\Phi^{e_1}_n[i] \otimes \Phi^{e_2}_n[j]$. Note that $k=1,\ldots, n^2$ and $i,j=1,\ldots, n$.
% 	\item When $e_1$ is $1\times n$ and $e_2$ is $n\times 1$ then $\Phi^e_n$ has one output gate $$\bigoplus_{l=1}^n \left( \Phi^{e_1}_n[l] \otimes \Phi^{e_2}_n[l] \right).$$
% 	\item When $e_1$ is $1\times n$ and $e_2$ is $n\times n$ then $\Phi^e_n$ has $n$ output gates, where gate $j$ is $$\bigoplus_{i=1}^n \left( \Phi^{e_1}_n[j] \otimes \Phi^{e_2}_n[j + n(i-1)] \right).$$
% 	\item When $e_1$ is $n\times n$ and $e_2$ is $n\times 1$ then $\Phi^e_n$ has $n$ output gates, where gate $i$ is $$\bigoplus_{j=1}^n \left( \Phi^{e_1}_n[j+n(i-1)] \otimes \Phi^{e_2}_n[j] \right).$$
% 	\item When $e_1$ is $n\times n$ and $e_2$ is $n\times n$ then $\Phi^e_n$ has $n^2$ output gates, where gate $k=j+n(i-1)$ is $$\bigoplus_{l=1}^n \left( \Phi^{e_1}_n[i] \otimes \Phi^{e_2}_n[j+n(l-1)] \right).$$ Note that $k=1,\ldots, n^2$ and $i,j=1,\ldots, n$.
% \end{itemize}

% If $e=\ffor{X}{v}e'(\cI, X, v)$ then $$\Phi^{e}_n=\Phi^{e'}_n\left( \cI, \Phi^{e'}_n \left( \cI, \cdots \Phi^{e'}_n\left( \cI, \Phi^{e'}_n\left( \cI, 0, v_1\right), v_2\right)\cdots, v_{n-1} \right), v_n \right).$$

%

It is not difficult to see that the proof of Theorem \ref{th-circuits-ml} can also be extended to support arithmetic circuits over matrices. In order to identify the class of functions computed by \langfor expressions, we need to impose one final restriction: than on the degree of an expression. 
In particular, we will be interested in \langfor expressions of polynomial degree. Formally, an expression $e$ is of polynomial degree, whenever there is an equivalent circuit family for $e$ of a polynomial degree.  
For example, all \langfor expressions seen so far have polynomial degree.
With this definition, we can now identify the class of functions for which arithmetic circuits and \langfor formulas are equivalent. This is the main technical contribution of the paper. 

\begin{corollary}
\label{th-equivalence}
Let $f$ be a function with input matrices $A_1,\ldots ,A_k$ of dimensions $\alpha\times \beta$, with $\alpha,\beta \in \{n,1\}$. Then, $f$ is computed by a uniform circuit family over matrices of polynomial degree if and only if there is a \langfor expression of polynomial degree for $f$. 
\end{corollary}

Note that this result crucially depends on the fact that expressions in \langfor are of polynomial degree.
Some \langfor expressions are easily seen to produce results which are not polynomial.
An example of such an expression is, for instance, $e_{\texttt{exp}} = \ffor{v}{X=A}{X\cdot X}$, over a schema $\Sch$ with $\ttype(v)= (\gamma,1)$, and $\ttype(X)=(1,1)$.
Over an instance which assigns $n$ to $\gamma$ this expression computes the function $a^{2^n}$, for $A=[a]$.
Therefore, a natural question to ask then is whether we can determine the degree of a \langfor expression.
Unfortunately, as we show in the following proposition this question is in fact undecidable.

Let $e$ be a \langfor expression over a matrix schema $\mathcal{S}=(\mathcal{M},\textsf{size})$ and let $V_1,\ldots, V_k$ be
the variables of $e$, each of type $(\alpha,\alpha)$, $(1,\alpha)$, $(\alpha,1)$ or $(1,1)$. We know from Theorem~\ref{th-ml-to-circuits}
that there exists a uniform arithmetic circuit family $\{\Phi_n \mid n=1,2,\ldots\}$
such that $\sem{e}{\I}=\Phi_n(A_1,\ldots,A_k)$ for any instance $\I$ such that
$\mathcal{D}(\alpha)=n$ and $\conc(V_i)=A_i$ for $i=1,\ldots,k$. We are interested in deciding
whether there exists such a  uniform arithmetic circuit family $\{\Phi_n \mid n=1,2,\ldots\}$
of polynomial degree, i.e., such that $\mathsf{degree}(\Phi_n)=\mathcal{O}(p(n))$ for some polynomial $p(x)$. If such a circuit family exists, we call $e$ of polynomial degree.

\begin{proposition}
\label{prop-undec}
Given a \langfor expression $e$ over a schema $\Sch$, it is undecidable to check whether $e$ is of polynomial degree.
\end{proposition}

We here show that it undecidable to check whether a \langfor expression is of polynomial degree.
We show undecidability based on the following undecidable language:
    $$
    \{ \langle M\rangle\mid \text{$M$ is a deterministic TM which halts on the empty input}\},
    $$
    where $\langle M\rangle$ is some string encoding of $M$.
    Consider a TM $M$ described by $(Q,\Gamma=\{0,1\},q_0,q_m,\Delta)$
    with $Q=\{q_1,\ldots,q_m\}$ its states, $q_1$ being the initial state and $q_m$ being
    the halting state, $\Gamma$ is the tape alphabet, and $\Delta$ is a transition function
    from $Q\times \Gamma\to Q\times\Gamma\times \{\leftarrow,\sqcup,\rightarrow\}$. The simulation
    of linear space TMs in \langfor, as given in the proof of Proposition~\ref{prop:transducer2}, can be easily modified to
    any TM $M$ provided that we limit the execution of $M$ to exactly $n$ steps. Let $e_M$ denote this expression. 
	
	Similarly
    as in the linear space TM simulation, we have vector variables $Q_1,\ldots,Q_m$ encoding the
    states, a single relation $T$ encoding the tape and relation $H_T$ encoding the position
    of the tape.  When an instance $\I$ assigns $n$ to $\alpha$, we have a tape of length $n$ at our disposal. This suffices if we let $M$ run for $n$ steps. We further observe that all vector variables can be assumed to be zero, initially.
    This is because $M$ we do not have any input. 
	
	So, let $\I_n^0$ denote the instance which assigns the vector variables to the $n$-dimensional zero vector.  Furthermore, by contrast to the linear space TM simulation, we use a single vector $v$ (instead of $k$ such vectors) to simulate $n$ steps of $M$. Finally, we modify the expression given in the proof of Proposition~\ref{prop:transducer2} such $\sem{e_M}{\I_n^0}$  returns $1$ if $M$
    halts in at most $n$ steps, and $0$ if $M$ did not halt yet after $n$ steps.
    As a consequence, when $M$ does not halt, $\sem{e_M}{\I_n^0}=0$ for all $n\geq 0$. When $M$ halts, there will be an $n$ such that $\sem{e_M}{\I_n^0}=1$. It now suffices to consider the \langfor expression
    $$
    d_M:=e_M\cdot e_{\mathsf{exp}}
    $$
    where $e_{\texttt{exp}} = \ffor{v}{X=e_{\mathbf{1}}(X)^T\cdot e_{\mathbf{1}}(X)}{X\cdot X}$ such that
    $e_{\texttt{exp}}(\I_n^0)=n^{2^n}$. Then, when $M$ does not halt we can clearly compute $\sem{d_M}{\I_n^0}$ with a constant degree circuit ``0''
    for any $n$, otherwise, the circuit needed will be of exponential degree
    for at least one $n$, simply because no polynomial degree uniform  circuit family can compute $\sem{d_M}{\I_n^0}=n^{2^n}$. In other words, deciding whether $d_M$ has polynomial degree coincides with deciding whether $M$ halts.


Of course, one might wonder whether it is possible to define a syntactic subclass of \langfor expressions that are of polynomial degree and can still express many important linear algebra algorithms. We identify one such class in Section \ref{ss:sumML}, called \langsum, and in fact show that this class is powerful enough to capture relational algebra on (binary) $K$-relations. 

\subsection{Supporting additional operators}
The equivalence of \langfor and arithmetic circuits we proved above assumes that circuits can only use the sum and product gates (note that even without the sum and the product function, \langfor\ can simulate these operations via matrix sum/product). However, both arithmetic circuits and expressions in $\langfor$ can be allowed to use a multitude of functions over $\RR$. The most natural addition to the set of functions is the division operator, which is crucially needed in many linear algebra algorithms, such as, for instance, Gaussian elimination, or $LU$ decomposition (recall Proposition \ref{prop:gauss}).
Interestingly, the equivalence in this case still holds, mainly due to a surprising result which shows that (almost all) divisions can in fact be removed for arithmetic circuits which allow sum, product, and division gates \cite{allender}.

More precisely, in \cite{strassen1973vermeidung,borodin1982fast,kaltofen1988greatest} it was shown that for any function of the form $f = g/h$, where $g$ and $h$ are relatively prime polynomials of degree $d$, if $f$ is computed by an arithmetic circuit of size $s$, then both $g$ and $h$ can be computed by a circuit whose size is polynomial in $s + d$. Given that we can postpone the division without affecting the final result, this, in essence, tells us that division can be eliminated (pushed to the top of the circuit), and we can work with sum-product circuits instead. The degree of a circuit for $f$, can then be defined as the maximum of degrees of circuits for $g$ and $h$. Given this fact, we can again use the depth reduction procedure of \cite{AllenderJMV98}, and extend Corollary~\ref{th-equivalence} to circuits with division.
\begin{corollary}
\label{cor-division}
Let $f$ be a function taking as its input matrices $A_1,\ldots ,A_k$ of dimensions $\alpha\times \beta$, with $\alpha,\beta \in \{n,1\}$. Then, $f$ is computed by a uniform circuit family over matrices of polynomial degree that allows divisions, if and only if there is a $\langforf{f_/}$ expression of polynomial degree for $f$.
\end{corollary}

An interesting line of future work here is to see which additional functions can be added to arithmetic circuits and \langfor formulas, in order to preserve their equivalence. Note that this will crucially depend on the fact that these functions have to allow the depth reduction of \cite{AllenderJMV98} in order to be supported.
