An arithmetic circuit $\Phi$ over a set $X=\{x_1,\ldots,x_n\}$ of variables is a directed
acyclic labelled graph. The vertices of $\Phi$ are called gates and denoted by $g_1,\ldots,g_m$;
the edges in $\Phi$ are called wires. The children of a gate $g$ correspond to all gates
$g'$ such that $(g,g')$ is an edge. The parents of $g$ correspond to all gates $g'$ 
such that $(g,g')$ is an edge. The in-degree of a gate $g$ refers to its number of children,
the out-degree to its number of parents. Gates with in-degree $0$ are called input gates
and are labelled by either a variable in $X$ or a constant $0$ or $1$. Every other gate
is labeled by either $+$ or $\times$ and are referred to sum or product gates, respectively.
Gates with out-degree $0$ are called output gates.

\floris{Restrictions on fan-in?}

The size of $\Phi$, denoted by $|\Phi|$, is its number of gates. The depth of a gate $g$, denoted
by $\mathsf{depth}(g)$, is the length of the longest directed path reaching $g$. The depth of $\Phi$
is the maximal depth of a gate in $\Phi$. An arithmetic circuit $\Phi$ corresponds to a polynomial in $\mathbb{N}[X]$ in a natural way. The degree of $\Phi$ is the degree of the polynomial corresponding to $\Phi$.

An arithmetic circuit family is a set of arithmetic circuits $\{\Phi_n\mid n=1,2,\ldots\}$ where $\Phi_n$ has $n$ input variables. An arithmetic circuit family is uniform if there exists a logspace-computable function
which on input $1^n$ returns an encoding of an arithmetic circuits $C_n$ for each $n=1,2$. We observe that
uniform arithmetic circuit families are necessarily of polynomial size.
