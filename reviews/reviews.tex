\documentclass[a4paper]{article}


%\usepackage[utf8]{inputenc}
%\usepackage{fullpage}
%\usepackage[english]{babel}
\usepackage{amsmath,amsthm,amssymb,amsfonts,mathrsfs,latexsym,stmaryrd}
%\usepackage{parskip}

%\usepackage{tikz}
%\usetikzlibrary{automata}
%\usetikzlibrary{arrows}
%\usetikzlibrary{trees}
%\usetikzlibrary{shapes}
%\usetikzlibrary{fit}
%\usetikzlibrary{calc}
%\usetikzlibrary{positioning}

\usepackage[textwidth=2cm,textsize=small]{todonotes}
\newcommand{\domagoj}[1]{\todo[inline, color=blue!15]{{\bf Domagoj:} #1}}
\newcommand{\cristian}[1]{\todo[inline, color=orange!30]{{\bf Cristian:} #1}}
\newcommand{\thomas}[1]{\todo[inline, color=green!30]{{\bf Thomas:} #1}}
\newcommand{\floris}[1]{\todo[inline, color=red!30]{{\bf Floris:} #1}}

\newcommand{\DONE}{\textcolor{green}{\textbf{(DONE)}} \ }

\newcommand{\TODO}{\textcolor{red}{\textbf{(TODO)}} \ }

\newcommand{\OMIT}{\textcolor{blue}{\textbf{(OMITED)}} \ }

%\newcommand{\as}{~\mathtt{AS}~}
%\newcommand{\FILTER}{~\mathtt{FILTER}~}

\begin{document}
	
\section*{REVIEWER 1}
    
At least two proposals to unify relational algebra and linear algebra have been proposed in the community: LARA and MATLANG.
Both languages have their shortcomings in that they can not express certain matrix operations. As matrix operations can always be 
added to both languages, a basic question that arises is what are the basis operations that need to be added to support a large 
class of matrix operations.  Rather than focussing on specific matrix operations/algorithms this paper takes a more principled 
approach and defines an extension of MATLANG that captures a robust class of linear algebra algorithms, those defined by arithmetic 
circuit families. The extension of FOR-MATLANG is defined in terms of for-loops that iterate over the canonical vectors to simulate 
bounded iteration (say, over the dimension of a matrix). The extension is both elegant and effective (as the main technical result 
shows). In addition, fragments of FOR-MATLANG are distinguished corresponding to more well-known fragments of (extensions of) the 
relational algebra.

The present paper contains all necessary ingredients for a solid PODS paper. The problem is well-motivated (what is a 
core language for linear algebra), the solution is elegant (looping extension taking the characteristics of matrices into 
account) and effective (capture a robust class of linear algebra algorithms). Furthermore, meaningful restrictions of the 
language are provided as well. Strong contribution for PODS.

\TODO On a different note, for-loops where considered before in the community as an extension of the relational algebra: 
Frank Neven, Martin Otto, Jerzy Tyszkiewicz, Jan Van den Bussche: 
Adding For-Loops to First-Order Logic. Inf. Comput. 168(2): 156-186 (2001). Not in the context of matrix operations but 
as an alternative to other looping constructs like while loops.

\cristian{To discuss.}
\thomas{I did a quick read of the document. They add the for loop operator to relational algebra, and to first order logic. They focus on whether the nesting of the for loops adds any expressive power. An interesting question but not addressed in our paper since the presence of the operator suffices to gain expressive power. Moreover, we restrict the operator to understand its expressiveness. Maybe we could add this to the document? Specifically, at the beggining of section 3.}

\floris{I added ``For-loop extensions of standard first-order logic are considered in [30] but with a different semantics as ours and in which no aggregation is supported''}
\newpage

\section*{REVIEWER 2}

Building on recent work on matrix query languages, the authors
propose an extension of the language MATLANG by a for-loop. They show
that many natural linear algebra constructions can be expressed in
this language.

The main technical result is that the new language has the same
expressiveness as logspace-uniform arithmetic circuits of polynomial degree.
This is a very nice and strong result,

The authors also consider various, somewhat more tractable,
restrictions of the language.

As the authors point out, the for-loop implicitly introduces an order
on the index space of the matrices, because iteration is over the
vectors of a standard basis. For a matrix language, this is,
natural, though still a bit unsatisfactory, in a relational context
it may be regarded slightly problematic. I think this is outweighed
by the expressivenness one gains, but it may have been the reason why
people have avoided such constructions in earlier papers.

I think this is a strong paper that should be accepted.

\newpage

\section*{REVIEWER 3}

\subsection*{Summary}

The paper introduces an extension of the query language over matrices MATLANG, which the authors call for-MATLANG, where for-loops, 
implemented via the use of canonical vectors, are allowed.
The main premise is that plain MATLANG is not able to express important linear algebra algorithms, 
due to the lack of a mean to iterate over the dimensions of the input matrices.
Such algorithms include computing the determinant, inverse, LU-decompositions, etc.

The authors introduce the syntax and semantics of for-MATLANG and show how the above linear algebra tasks, 
as well as some graph algortihms, can be implemented in the new language.
The linear algebra algorithms require an external division function $f_/$ to be provided.
Then, the authors embark in a more systematic expressibility analysis, by comparing the functions over matrices that are 
computable by for-MATLANG (without any additional external function) and uniform families of arithmetic circuits (which 
capture most standard linear algebra algorithms).
The equivalence is shown to hold, when restricting circuits (and for-MATLANG expressions) to have polynomial degree.

Then, the authors focus on some sub-fragments of for-MATLANG, specifically the one where for-loops are only allowed to compute 
incremental summations of an expression, called sum-MATLANG. This sub-fragment is then extended to also allow incremental 
computations of the Hadamard product or the standard matrix product of an expression, leading to the languages FO-MATLANG and 
prod-MATLANG, respectively.
They show that sum-MATLANG (when properly extended to arbitrary semirings K) is equivalent to positive relational algebra over 
binary K-relations.
Regarding FO-MATLANG, they show that when focusing on FO-MATLANG expressions where only vectors and square matrices all of the 
same size are allowed, FO-MATLANG is equivalnt to weighted-logics over binary signatures.
Finally, they show that prod-MATLANG strictly subsumes FO-MATLANG, and when extended with ordering information (i.e., having 
access to the upper triangular matrix $S_\leq$ with all non-zero entries equal to 1) and the external function $f_/$, the resulting language is expressive enough to compute the determinant and the inverse of a matrix. Transitive closure of an adjacency matrix is also computable, given the external function $f_{>0}$ checking whether a given real number is non-zero, is provided.
No expressibility equivalences are established for prod-MATLANG.

\subsection*{Review}

Overall, the paper is technically solid, with non-trivial results regarding the expressivity of the new language.
The modification of MATLANG via for-loops is quite natural and not very "invasive". The connection with arithmetic circuits 
provides a good formal argument on the capacity of the new language of expressing linear algebra constructs.
I also find the sum-, FO-, and prod- fragments resonable limitations of the language. The results are correct from what I can tell.

On the other hand, I believe that on the motivations and presentation side the paper is not on par with the high standards of a 
PODS paper.

\begin{enumerate}
	 
\item \OMIT One of the premises of the paper is to find the atomic operations to be introduced in MATLANG, that allow to "define standard 
linear algebra algorithms".
I totally get the point, and it is clear that continuously adding ad hoc operators to the language is not the way to go. 
However, if the explicitly provided premise is only to define standard algorithms, it is also fair to ask what other advantages 
the new language has. Implementing "simple" procedures such as 4-clique and especially the PLU-decomposition seem to require 
some not-so-natural expressions (and in some cases, like PLU, required to dig into some details/properties of the algorithm to 
implement it). One can argue that syntactic sugar can be added to aid in implementing more complex algorithms. So, I would have 
liked to see some discussion on this and a comparison between the new language and some of the languages mentioned in the related 
work section.\thomas{I don't know how to properly mention that it is \textbf{a} way to go, maybo not \textbf{the} way to go, that is to be determined in the future. The main result is strong enough to consider the setting, as other reviewers mention.}
\domagoj{This is a dumb comment by the reviewer. I would not reply.}
\cristian{I agree here to don't do anything. This is a PODS paper. One would expect a follow-up paper that does this.}
\floris{It is not a dumb comment. The language introduced is to study expressive power in a formal way, and is not intended
to serve as ``the'' language for linear algebra. It is like fix point logics, you can use them to express things but it is not
really a user-friendly language. Sure we don't want to comment?}
\item \OMIT Space for the above discussion could be gained by pushing some more details in the appendix, such as part of the 
definition of$ RA_K^+$ (e.g., the semantics is somehow obvious, and can be deferred to the appendix).

\cristian{I marked it as ommited, given that it is related with the previous comment.}

\item \OMIT Similarly, in the equivalence between FO-MATLANG (with all square matrices of the same size) and WL, the authors could 
have spent some words in describing why such a fragment on restricted square matrices (all of the same size) is 
interesting/useful. I also would have given some more intuition of the 4-clique expression in Example 3.3.\thomas{I would maybe define the schema as a "fixed dimension schema" (accepting names suggestion). We already mention that it is for a "crisp translation between both languages". I don't know how to elaborate more, we actually state the problem and say we leave it for future work.}

\cristian{Well, it is connected with the previous comment. Here I don't believe that this is necessary. FO-MATLANG is a natural syntactical restriction of For-MATLANG. So, no motivation is needed. I marked it as ommited.}

\item \OMIT I also find that the paper is at times sloppy and not very formal, even at the level of the definition of the syntax and 
semantics of (for-)MATLANG.
\cristian{To general the comment to do something.}

\item \DONE In particular, the semantics of (for-)MATLANG expressions is not well-defined, as no 
parenthesization in the syntax, or priority among the operators is specified. Hence the semantics of expressions 
like $A+B \cdot C$ is ambiguous. \thomas{Working on this. Accepting suggestions.}
\domagoj{Actually the semantics is fine. We assume that the thing is fully parenthesised in a sense (i.e. when we write $e_1 + e_2$ we assume to know what $e_1,e_2$ are). I added a paragraph on this, but I find it strange that PODS reviewers forgot how sytax definition works.}
\cristian{Agree.}
\floris{Very strange comment indeed.}

\item \OMIT Moreover, the notion of expressibility is never formally defined, especially when talking 
about expressing specific constructs like the 4-clique. Without clarifying this, one can (I know this is a bit extreme) e.g., 
express 4-clique by encoding the adjacency matrix as a natural number (in binary), and use a function f from F, that given the 
encoding of the matrix, checks if a 4-clique exists, placing 4-clique in $MATLANG[{f}]$.\thomas{I think he didn't understand our approach for functions. What he says is perfectly reasonable and does not go against any of our results. For example, corollary 6.2 includes this case.}
\cristian{Agree.}
\floris{He just wants to point out that, if not familiar with expressibility, you could question what's the point of all this. Just
instantiate a matrix with the solution and you're done. See also comment below.}

\item \OMIT Similar issues occur in other expressibility results such as Propositions 4.1, 4.2, 4.3 etc., where the instance I 
is never fully qualified. In this way, one can make the instance I map a dummy matrix M variable to the result of the function 
to be computed, and make the for-MATLANG expression be just M, making all such claims trivially true.
I believe such claims should be provided with a spirit similar to Theorem 5.3, where the variables of the expression, 
and the shape of the instance I are clearly defined.\thomas{Working on this.}
\cristian{No way. I disagree that the quantification is needed. It is obvious from the context. I marked it as ommited.}
\floris{The reviewer just asks to say that the expressions work for any matrix. Anyways.}

\item \OMIT  Another instance of this is Corollary 5.2, where the term "equivalent" is used, without formally defining it. Considering 
that expressibility and equivalence results are the main task of the paper, I would make these notions and results more formal.
\domagoj{The conditions are the same as in 5.1. Perhaps mention this, but nothing more is needed.}
\cristian{Again, there is a matter of context that the reviewer is not considered. I disagree to modify this. I marked it as ommited.}

\item \DONE Also, the use of symbols and operators not formally part of the language, such as the minus symbol, literal matrices (i.e., 
constants), etc., might deserve a footnote, mentioning that they are just syntactic sugar, and easily definable.

\item \OMIT So, I would suggest to improve presentation in the above regard, in particular w.r.t. the expressibility results.
\cristian{I disagree. So I marked it as ommited.}

\item \DONE In Figure 1, the equivalences shown are a bit misleading as the paper does not show that FO-ML $\equiv$ WL in general, but 
only under some assumptions (square matrices schemas). The same applies to Proposition 6.7, where such crucial assumptions 
are specified outside (before) the claim.\thomas{Maybe define something like $\text{FO-ML}_{\mathcal{S}}$ to denote FO-ML restricted to a fixed dimension schema. And something similar with weighted logics, as $WL^2$ to represent that the vocabulary has arity at most two.}
\cristian{I edit the legend of the figure and the content to say ``òver square matrices''. The crucial assumptions outside Prop. 6.7 are ok and I keep it.}

\item \DONE Finally, although I can seen in the paper where 4Clique and DP are shown to lie outside of the lower language, i.e. 4Clique 
in sum-ML/ML and DP in FO-ML / sum-ML, I could not find any mention (even in the appendix) that Inv and Det are actually 
in prod-ML + $S_\leq$  but not in FO-ML. This should be clarified.\thomas{Added this comment on this on section 6.3: We do not know if the inversion and determinant algorithms
are expressible in FO-MATLANG.}

\end{enumerate}

\subsection*{Minor comments}


\begin{enumerate} \setcounter{enumi}{12}
\item \DONE pg 2, "uniform arithmentic circuits" -- do you mean "uniform family"?
\item \DONE pg 2, "and show that two natural fragment" -- fragments
\item \DONE pg 3, paragraph "Semantics": the schema $\cal S$ should be qualified as $(M,size)$. Same on the right column
\item \DONE pg 3, before Example 3.1: the notation $\mathcal{I} [v := b^n_i, X := A_{i-1}]$ is not defined for multiple variable assignments.
\item \DONE pg 4, when discussing how to implement arbitrary for-loop initializations, the authors conclude stating that "the result of 
this evaluation is equal to $[| \text{for } v,X=e_0 . e|](I)$, but it appears to me that for this to be true, the instance I must also be 
updated, to make v have one more element, so to "sacrifice" the first one for the initialization phase, and then use the rest as 
expected. Moreover, due to this, the expression e must also change in the expression $(1-min(v)) \cdot e(v,X)$, to "filter out" the 
first element in v, for the type to be well-defined.\thomas{Working on this.}

\cristian{I don't see what is the problem here. I checked it and it is correct. To discuss.}
\thomas{You're right, I think the reviewer didn't get that the instance is updated in each iteration of the for}

\item \DONE pg 5, Section 4.1, the symbol I is used as the identity matrix, but sometimes the symbol $e_Id$ is used, please be uniform.\thomas{A small check doesn't hurt.}
\item \DONE pg 5, right column: "provided that ithe entry" -- the ith entry.
\item \DONE pg 5, definition of $f_/$. You might want to add a footnote to specify how to define the function when y is 0 (as any definition 
would be fine).\thomas{I added a footnote that says we map division by zero to zero. If you prefer to say it in a parenthesis after the definition I'm ok with it, feel free to change it}
\item \OMIT pg 7, Theorem 5.1: As discussed above, fully qualify any variables $e_\Phi$ uses, and how $\cal I$ is defined.\thomas{Working on this.}
\cristian{Similar comments than before, for me the statement is correct. One can understand from the context where $e$ and  $\cal I$ comes from. I marked it as ommited. }

\item \DONE pg 8, degree of an expression: circuit families have no notion of degree, circuits do. So definint the degree of an expression 
as the "minimum" degree of a family does not make much sense. I would just define when an expression is of polynomial degree, 
as it is the only notion needed later on.
\domagoj{I did as the reviewer suggested. Works well enough for our purposes.}
\cristian{I checked it and I agree with the reviewer. Also agree with the change of Domagoj.}
\item \DONE pg 8, right columns "expression are easily seen" -- expressions
\item \DONE pg 9, title of Section 6.1: matlang -- MATLANG
\item \DONE pg 11, "schema of square matrices" is misleading, as it makes it seem that the expressions allow arbitrary square matrices, 
of different dimensions, which is not the case. I would find an alternative name.\thomas{I propose the name: fixed dimension schema.}
\cristian{I defined this formaly in Section 2, when we define schemas and typing. I spent one paragraph and it should be pretty clear what does this mean in the future. The name is good enough if it is clarified from the beginning.}

\item \DONE pg 12, Section 6.3: "can expressed" -- can be expressed.
\end{enumerate}

\newpage

\section*{REVIEWER 4}

\subsection*{Summary}

The paper analyzes the expressive power of the language MATLANG with the
additional operation to allow loops over all canonical vectors of a given
dimension. The language MATLANG was presented two years ago with the goal to
provide a query language for matrices. The restriction of loops to canonical
vectors ensures that all loops are bounded linearly in the input size.
Especially, the query language will not be Turing complete.

The main result is that MATLANG+for is equally expressive as uniform
arithmetic circuits. This result is very technical and by no means trivial.
It does not seem that hard to convert a given arithmetic circuit into
a formula. However, one has to deal with a family of circuits that is given by
a logspace turing machine. This turing machine is given a size (in unary) and
then computes a circuit for the given size. Thus the actual reduction is from
logspace Turing machines to MATLANG formulas. I do not see how this
technicallity can be avoided.

The paper demonstrates the use of for-loops by providing MATLANG-for
expressions for important linear algebra algorithms (lower-upper decomposition,
inversion of matrices, computing determinants). It also discusses syntactic
restrictions of the introduced loops and shows that a restricted class of
for-loops is equally expressive as relational algebra over binary schemas.

This paper entirely focuses on exprissivity and leaves efficient evaluation to
future work.

The prrofs are non-trivial and they look sound to me. However, it did not verify
all results. Adding a reasonably powerful looping mechanism to the language is   
a natural extension. From a theoretical point of view with the conenctions to
arithmetic circuits this seems to be the strongest of the MATLANG papers up to
now and should therefore be accepted.

\subsection*{Comments}


\begin{enumerate}
	\item  \OMIT You have to state the operator priorities of MATLANG and for-MATLANG.\thomas{Working on this. Accepting suggestions.}
\domagoj{This is the same as above. We should just state that we assume the thing to be fully parenthesised, or given via a parse tree.}
\cristian{Agree with Domagoj. This comment is not needed.}

\item  \DONE I do not understand why an Instance has to assign values to the size symbols.
All sizes are implicit by the assigned matrices. I would rather have I: M $\rightarrow$ Mat[R]
just assign matrices and then require that there exist a function D: Symb $\rightarrow$ N
such that each assigned matrix has the correct dimensions. This should simplify some
statements.\thomas{Thinking about this.}
\domagoj{Added a comment about this after defining instances.}

\item  \DONE I also notices that you are very sloppy when it comes to formally defining I. In most
of the examples, I is not defined at all. This should be changed. Probably it would
help to state that in the simple case with only one matrix symbol you write [[e]](A) to
denote [[e]](I), where I is the function that assigns A to the only matrix symbol. Then
you can get rid of I in the examples altogether.\thomas{I only changed it in the for-MATLANG expressions, I left matrix expressions with $I$. Again, please give a small check if you can.}

\item  \OMIT You point out that introducing loops also introduces order. Whether there is an order
available or not is certainly an important aspect for query languages. It might be
interesting in future work to look at the fragment of for-MATLANG, where the outcome
of the loop may not depend on the order in which the canonical vectors are processed
by the loop.

\cristian{Agree. Thanks!}

\item  \DONE When introducing the restricted variants you should point out that the variable X is
not allowed inside the expression e. Otherwise sum-MATLANG will turn out to be much
more powerful than you want it to be.
\cristian{This is clear from the definition.}

\item  \TODO What is your reasoning behind the name FO-MATLANG? FO is usually associated with
first order logic and the name can lead to confusion. Maybe you can think of a
better alternative or explain your choice of the name.\thomas{Candidate: sp-MATLANG}

\item  \OMIT You should use the $\colon$ latex operator instead of : in function declarations. The : is defined
as an operator in latex and produces incorrect spacing.\thomas{Testing and working on this.}

\cristian{I disagree. I prefer the : over the $\colon$. So, I marked it as ommited. }

\item  \DONE After Prop. 6.4: necessary -- necessarily
\end{enumerate}

\end{document}
